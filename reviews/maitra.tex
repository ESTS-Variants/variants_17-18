%%%%%%%%%%%%%%
%% METADATA %%
%%%%%%%%%%%%%%

\contributor{
% your name(s)
Dipanjan Maitra}

\contribution{Ariadne Nunes, Joana Moura and Marta Pacheco Pinto, eds. \emph{Genetic Translation Studies: Conflict and Collaboration in Liminal Spaces}}

\begin{review}
\renewcommand*{\pagemark}{}

%%%%%%%%%%%%%%%%%%%%%%%%%%%%%%%%%%%%%%
%% DESCRIPTION OF THE REVIEWED BOOK %%
%%%%%%%%%%%%%%%%%%%%%%%%%%%%%%%%%%%%%%

\begin{reviewed}
Review of \thecontribution. London and New York, Ny.: Bloomsbury Academic, 2021. 256 pp. ISBN: 978–1–350–14681–5.
\end{reviewed}

%%%%%%%%%%%%%%%%%%%%%%%%%%%%%
%% YOUR REVIEW STARTS HERE %%
%%%%%%%%%%%%%%%%%%%%%%%%%%%%%

% remove asterisk (*) if you want to number your sections
% add a title for your section in between the {curly brackets} if you need one
\section*{} 
``Enhancing textual awareness is one of the major roles translation can
play in the nexus between genetic criticism and translation'', noted
Dirk Van Hulle (\citeyear[43]{van_hulle_translation_2015}). It is precisely this ``nexus'' between
genetic criticism and translation studies which forms the basis of
\emph{Genetic Translation Studies: Conflict and Collaboration in Liminal
Spaces}, edited by Ariadne Nunes, Joana Moura and Marta Pacheco Pinto.
Genetic criticism attempts to isolate itself from other modes of textual
criticism by resisting a teleological impulse towards a \emph{final},
published text. Instead, a genetic study of \emph{avant-textes} focuses
on the writerly process, its twists and turns, unraveling, to use a
Borgesian metaphor, a text's multiple ``forking paths'', its myriad
possibilities. The ``independence'' of the genetic process from the
published text sees a parallel in translation studies where the
independence of the translated text from its ``original'' ur-text has
become a talking point. Perhaps if one sees genetic criticism as a
description of the genesis of a text, then translation studies could be
seen as concerned with the post-text and its future reincarnations. In
an oft-quoted essay by Cordingley and Montini, the authors thus
postulate that ``Genetic criticism has given little attention to the
post-text while at the same time much attention in recent translation
studies has been directed towards establishing the independence of the
translated text'' (\citeyear[5]{cordingley_genetic_2015}). The significance of incorporating genetic
approaches to translation studies is revisited powerfully in
\emph{Genetic Translation Studies}.

Conceived initially as an international conference on \emph{Unexpected
Intersections: Translation Studies and Genetic Criticism} under the
aegis of the University of Lisbon in 2017, the book includes fourteen
essays grouped under three broad tenets: (1) genetic approaches to
translation and collaboration, (2) translators' stories and testimonies,
and (3) translators at work. Expectedly, several of the essays focus on
case studies of Portuguese authors but the anthology does a commendable
job in mapping a wide linguistic canvas with essays on translation
projects in several major European languages. These include translated
texts from French to English (Esa Christine Hartmann on Robert
Fitzgerald's translation of Saint-John Perse's poem \emph{Chronique};
\citeyear{perse_chronique_1961}), English to Polish (Ewa Kołodziejczyk on Czesław Miłosz's
translation of ``Negro spirituals''; \cite{Miłosz_negro_1948,Miłosz_wiersze_1948}), from English to French
(Patrick Hersant on Maurice Coindreau's translations of American novels;
e.g. Steinbeck \citeyear{steinbeck_souris_1939} and Capote \citeyear{capote_harpe_1952}), French to German (Joana Moura on
Peter Handke's translation of René Char's poems from \emph{Le nu perdu};
\citedate{char_ruckkehr_1984}) and even from Sanskrit to Portuguese (Marta Pacheco Pinto and
Ariadne Nunes on the Portuguese Orientalist Vasconcelos Abreu's efforts
at translating the \emph{Panchatantra}).

João Dionísio's opening piece lays out the challenges (or, as he calls
it, ``latency'') of genetic translation studies in Portugal. For him,
this is a result of the absence of ``a self-aware interpretive
community'' (28). Despite these obstacles, his two main examples, a
fifteenth-century treatise by King Duarte of Portugal on Latin
translations and a Fernando Pessoa poem which contains transcreations of
a poem by the Victorian poet Alfred Austin, both showcase how genetic
studies of translations challenge any unitary understanding of
authorship. Dionísio ably demonstrates how a genetic analysis of these
authors separated by centuries can still reveal authorship as a complex,
multifarious process, involving external agents or a ``network of
interacting and negotiating subjects'' (39). Laura Ivaska's meticulously
researched article on a Finnish translation of Nikos Kazantzakis' novel
\emph{The Fratricides} (\citeyear{kazantzakis_veljesviha_1967}), which used a French translation for its
``best text'' is a further elaboration on this network of interacting
and negotiating subjects.

If translators are crucial ``intermediaries'' in introducing ``foreign''
texts into a literary field (to echo Pascale \citeauthor{casanova_world_2004}), then perhaps one
of the more intriguing case studies in the volume is of a translation
that is not even complete: Marta Pacheco Pinto and Ariadne Nunes'
account of Guilherme de Vasconcelos Abreu's Orientalist projects
(1842--1907), in what would have been the first Portuguese translation
of the ancient Sanskrit text of fables \emph{Panchatantra} (200 B.C.
ca.). Pacheco Pinto and Nunes analyze Abreu's extant notes for the
translation in two unpublished volumes and demonstrate how even an
unfinished, unpublished project such as Abreu's can shed light on the
debates that informed Orientalist scholarship in its early days in
Portugal. What is only tantalizingly hinted at here and therefore opens
up avenues for future scholarship, are larger questions about print
culture and the communications circuits that connected Portuguese
Orientalists like Abreu with books published in Calcutta via the India
Office in London (221).

This question of continual mediation in translation studies becomes even
more apparent as we introspect the increasing proximity of digital and
computational technologies and genetic criticism. Understandably,
several essays in the collection refer to ongoing and future digital
humanities projects. To give one example, Elsa Pereira examines the
prospects of a digital edition of Pedro Homem de Melo's poetry. Pereira
argues that because of the complexity of de Melo's genetic dossier –– a
mixture of allographic translations (translations of de Melo's poetry by
translators other than the poet), self-translations and what Pereira
terms ``alloglottic rewriting'' involving several stages of rewritings
of his own poems in French –– a multi-layered, para-genetic edition is
the need of the hour. For a ``multi-level'' edition of this kind, a
digital edition becomes a necessity.

What these accounts highlight more acutely, however, is that, despite
new scholarship and groundbreaking theorizations in the field, the
translators' role still lacks visibility and due recognition. Marisa
Mourinha quotes Gregory Rabassa –– easily one of the undisputed
authorities in the field, due to his much-venerated translations of
Gabriel García Márquez (\citeyear{marquez_one_1970}, \citeyear{marquez_autumn_1975} and \citeyear{marquez_chronicle_1983}) –– from an interview
released during the central years of the Latin American Boom, arguing
that ``It is time for at least one definitive statement about
translation: it is impossible to make a handsome living out of it''
(\citedate[37--38]{rabassa_if_1974}; qtd. 149). As Mourinha notes, Rabassa's statement may
still ring true today, in a field that is still dominated by academics.
This is why, perhaps, one of the more memorable sections of this
collection consists of chapters which chart out translators'
testimonies, self-reflections and correspondence, as well as the very
real physical labor of translation, as in Ivančić and Zepter's chapter
(123--34). This is also where the genetic dossier becomes a treasure
trove of personal anecdotes, scholarly journeys and where textuality
touches on the experiential. They highlight for us moments when, as
Emily Apter would note in a slightly different context, translators
decide to go on, ward off challenges –– philological, socio-economic,
philosophic (``the singularity of untranslatable alterity'') –– and
decide to translate \emph{quand même} (\citedate[104]{apter_translation_2006}).


\begin{flushleft}
    % use smallcaps for author names
    \renewcommand*{\mkbibnamefamily}[1]{\textsc{#1}}
    \renewcommand*{\mkbibnamegiven}[1]{\textsc{#1}} 
\printbibliography
\end{flushleft}

\end{review}