\documentclass{article}
%%%% CLASS OPTIONS 

\KOMAoptions{
    fontsize=10pt,              % set default font size
    DIV=calc,
    titlepage=false,
    paper=150mm:220mm,
    twoside=true, 
    twocolumn=false,
    toc=chapterentryfill,       % for dots: chapterentrydotfill
    parskip=false,              % space between paragraphs. "full" gives more space; "false" uses indentation instead
    headings=small,
    bibliography=leveldown,     % turns the Bibliography into a \section rather than a \chapter (so it appears on the same page)
}

%%%% PAGE SIZE

\usepackage[
    top=23mm,
    left=20mm,
    height=173mm,
    width=109mm,
    ]{geometry}

\setlength{\marginparwidth}{1.25cm} % sets up acceptable margin for \todonotes package (see preamble/packages.tex).

%%%% PACKAGES

\usepackage[dvipsnames]{xcolor}
\usepackage[unicode]{hyperref}  % hyperlinks
\usepackage{booktabs}           % professional-quality tables
\usepackage{nicefrac}           % compact symbols for 1/2, etc.
%\usepackage{microtype}          % microtypography
\usepackage{lipsum}             % lorem ipsum at the ready
\usepackage{graphicx}           % for figures
\usepackage{footmisc}           % makes symbol footnotes possible
\usepackage{ragged2e}
\usepackage{changepage}         % detect odd/even pages
\usepackage{array}
\usepackage{float}              % get figures etc. to stay where they are with [H]
\usepackage{subfigure}          % \subfigures witin a \begin{figure}
\usepackage{longtable}          % allows for tables that stretch over multiple pages
\setlength{\marginparwidth}{2cm}
\usepackage[textsize=footnotesize]{todonotes} % enables \todo's for editors
\usepackage{etoolbox}           % supplies commands like \AtBeginEnvironment and \atEndEnvironment
\usepackage{ifdraft}            % switches on proofreading options in the draft mode
\usepackage{rotating}           % provides sidewaysfigure environment
\usepackage{media9}             % allows for video in the pdf
\usepackage{xurl}               % allows URLs to (line)break ANYWHERE

%%%% ENCODING

\usepackage[full]{textcomp}                   % allows \textrightarrow etc.

% LANGUAGES

\usepackage{polyglossia}
\setmainlanguage{english} % Continue using english for rest of the document

% If necessary, the following lets you use \texthindi. Note, however, that BibLaTeX does not support it and will report a 'warning'.
 \setotherlanguages{hindi} 
 \newfontfamily\hindifont{Noto Sans Devanagari}[Script=Devanagari]

% biblatex
\usepackage[
    authordate,
    backend=biber,
    natbib=true,
    maxcitenames=2,
    ]{biblatex-chicago}
\usepackage{csquotes}

% special characters  
\usepackage{textalpha}                  % allows for greek characters in text 

%%%% FONTS

% Palatino font options
\usepackage{unicode-math}
\setmainfont{TeX Gyre Pagella}
\let\circ\undefined
\let\diamond\undefined
\let\bullet\undefined
\let\emptyset\undefined
\let\owns\undefined
\setmathfont{TeX Gyre Pagella Math}
\let\ocirc\undefined
\let\widecheck\undefined

\addtokomafont{disposition}{\rmfamily}  % Palatino for titles etc.
\setkomafont{descriptionlabel}{         % font for description lists    
\usekomafont{captionlabel}\bfseries     % Palatino bold
}
\setkomafont{caption}{\footnotesize}    % smaller font size for captions


\usepackage{mathabx}                    % allows for nicer looking \cup, \curvearrowbotright, etc. !!IMPORTANT!! These are math symbols and should be surrounded by $dollar signs$
\usepackage[normalem]{ulem}                       % allows for strikethrough with \sout etc.
\usepackage{anyfontsize}                          % fixes font scaling issue

%%%% ToC

% No (sub)sections in TOC
\setcounter{tocdepth}{0}                

% Redefines chapter title formatting
\makeatletter                               
\def\@makechapterhead#1{
  \vspace*{50\p@}%
  {\parindent \z@ \normalfont
    \interlinepenalty\@M
    \Large\raggedright #1\par\nobreak%
    \vskip 40\p@%
  }}
\makeatother
% a bit more space between titles and page numbers in TOC

\makeatletter   
\renewcommand\@pnumwidth{2.5em} 
\makeatother

%%%% CONTRIBUTOR

% Title and Author of individual contributions
\makeatletter
% paper/review author = contributor
\newcommand\contributor[1]{\renewcommand\@contributor{#1}}
\newcommand\@contributor{}
\newcommand\thecontributor{\@contributor} 
% paper/review title = contribution
\newcommand\contribution[1]{\renewcommand\@contribution{#1}}
\newcommand\@contribution{}
\newcommand\thecontribution{\@contribution}
% short contributor for running header
\newcommand\shortcontributor[1]{\renewcommand\@shortcontributor{#1}}
\newcommand\@shortcontributor{}
\newcommand\theshortcontributor{\@shortcontributor} 
% short title for running header
\newcommand\shortcontribution[1]{\renewcommand\@shortcontribution{#1}}
\newcommand\@shortcontribution{}
\newcommand\theshortcontribution{\@shortcontribution}
\makeatother

%%%% COPYRIGHT

% choose copyright license
\usepackage[               
    type={CC},
    modifier={by},
    version={4.0},
]{doclicense}

% define \copyrightstatement for ease of use
\newcommand{\copyrightstatement}{
         \doclicenseIcon \ \theyear. 
         \doclicenseLongText            % includes a link
}

%%%% ENVIRONMENTS
% Environments
\AtBeginEnvironment{quote}{\footnotesize\vskip 1em}
\AtEndEnvironment{quote}{\vskip 1em}

\setkomafont{caption}{\footnotesize}

% Preface
\newenvironment{preface}{
    \newrefsection
    \phantomsection
    \cleardoublepage
    \addcontentsline{toc}{part}{\thecontribution}
    % enable running title
    \pagestyle{preface}
    % \chapter*{Editors' Preface}    
    % reset the section counter for each paper
    \setcounter{section}{0}  
    % no running title on first page, page number center bottom instead
    \thispagestyle{chaptertitlepage}
}{}
\AtEndEnvironment{preface}{%
    % safeguard section numbering
    \renewcommand{\thesubsection}{\thesection.\arabic{subsection}}  
    %last page running header fix
    \protect\thispagestyle{preface}
}
% Essays
\newenvironment{paper}{
    \newrefsection
    \phantomsection
    % start every new paper on an uneven page 
    \cleardoublepage
    % enable running title
    \pagestyle{fancy}
    % change section numbering FROM [\chapter].[\section].[\subsection] TO [\section].[\subsection] ETC.
    \renewcommand{\thesection}{\arabic{section}}
    % mark chapter % add author + title to the TOC
    \chapter[\normalfont\textbf{\emph{\thecontributor}}: \thecontribution]{\vspace{-4em}\Large\normalfont\thecontribution\linebreak\normalsize\begin{flushright}\emph{\thecontributor}\end{flushright}}    
    % reset the section counter for each paper
    \setcounter{section}{0}  
    % reset the figure counter for each paper
    \renewcommand\thefigure{\arabic{figure}}    
    % reset the table counter for each paper
    \renewcommand\thetable{\arabic{table}} 
    % no running title on first page, page number center bottom instead, include copyright statement
    \thispagestyle{contributiontitlepage}
    % formatting for the bibliography

}{}
\AtBeginEnvironment{paper}{
    % keeps running title from the first page:
    \renewcommand*{\pagemark}{}%                            
}
\AtEndEnvironment{paper}{
    % safeguard section numbering
    \renewcommand{\thesubsection}{\thesection.\arabic{subsection}}  
    % last page running header fix
    \protect\thispagestyle{fancy}%                              
}
% Reviews
\newenvironment{review}{
    \newrefsection
    \phantomsection
    % start every new paper on an uneven page 
    \cleardoublepage
    % enable running title
    \pagestyle{reviews}
    % change section numbering FROM [\chapter].[\section].[\subsection] TO [\section].[\subsection] ETC.
    \renewcommand{\thesection}{\arabic{section}} 
    % mark chapter % add author + title to the TOC
    \chapter[\normalfont\textbf{\emph{\thecontributor}}: \thecontribution]{}    % reset the section counter for each paper
    \setcounter{section}{0}  
    % no running title on first page, page number center bottom instead, include copyright statement
    \thispagestyle{contributiontitlepage}
    % formatting for the bibliography
}{}
\AtBeginEnvironment{review}{
% keeps running title from the first page
    \renewcommand*{\pagemark}{}%                                   
}
\AtEndEnvironment{review}{
    % author name(s)
    \begin{flushright}\emph{\thecontributor}\end{flushright}
    % safeguard section numbering
    \renewcommand{\thesubsection}{\thesection.\arabic{subsection}} 
    % last page running header fix
    \protect\thispagestyle{reviews}                           
}

% Abstract
\newenvironment{abstract}{% 
\setlength{\parindent}{0pt} \begin{adjustwidth}{2em}{2em}\footnotesize\emph{\abstractname}: }{%
\vskip 1em\end{adjustwidth}
}{}

% Keywords
\newenvironment{keywords}{
\setlength{\parindent}{0pt} \begin{adjustwidth}{2em}{2em}\footnotesize\emph{Keywords}: }{%
\vskip 1em\end{adjustwidth}
}{}

% Review Abstract
\newenvironment{reviewed}{% 
\setlength{\parindent}{0pt}
    \begin{adjustwidth}{2em}{2em}\footnotesize}{%
\vskip 1em\end{adjustwidth}
}{}

% Motto
\newenvironment{motto}{% 
\setlength{\parindent}{0pt} \small\raggedleft}{%
\vskip 2em
}{}

% Example
\newcounter{example}[chapter]
\newenvironment{example}[1][]{\refstepcounter{example}\begin{quote} \rmfamily}{\begin{flushright}(Example~\theexample)\end{flushright}\end{quote}}

%%%% SECTIONOPTIONS

% command for centering section headings
\newcommand{\centerheading}[1]{   
    \hspace*{\fill}#1\hspace*{\fill}
}

% Remove "Part #." from \part titles
% KOMA default: \newcommand*{\partformat}{\partname~\thepart\autodot}
\renewcommand*{\partformat}{} 

% No dots after figure or table numbers
\renewcommand*{\figureformat}{\figurename~\thefigure}
\renewcommand*{\tableformat}{\tablename~\thetable}

% paragraph handling
\setparsizes%
    {1em}% indent
    {0pt}% maximum space between paragraphs
    {0pt plus 1fil}% last line not justified
    

% In the "Authors" section, author names are put in the \paragraph{} headings. To reduce the space after these  headings, the default {-1em} has been changed to {-.4em} below.
\makeatletter
\renewcommand\paragraph{\@startsection {paragraph}{4}{\z@ }{3.25ex \@plus 1ex \@minus .2ex}{-.4em}{\normalfont \normalsize \bfseries }
}
\makeatother

% add the following (uncommented) in environments where you want to count paragraph numbers in the margin
%    \renewcommand*{\paragraphformat}{%
%    \makebox[-4pt][r]{\footnotesize\theparagraph\autodot\enskip}
%    }
%    \renewcommand{\theparagraph}{\arabic{paragraph}}
%    \setcounter{paragraph}{0}
%    \setcounter{secnumdepth}{4}
    
%%%% HEADERFOOTER

% running title
\RequirePackage{fancyhdr}
% cuts off running titles that are too long
%\RequirePackage{truncate}
% makes header as wide as geometry (SET SAME AS \TEXTWIDTH!)
\setlength{\headwidth}{109mm} 
% LO = Left Odd
\fancyhead[LO]{\small\emph{\theshortcontributor} \hspace*{.5em} \theshortcontribution} 
% RE = Right Even
\fancyhead[RE]{\scshape{\small\theissue}}
% LE = Left Even
\fancyhead[LE]{\small\thepage}            
% RE = Right Odd
\fancyhead[RO]{\small\thepage}    
\fancyfoot{}
% no line under running title; cannot be \@z but needs to be 0pt
\renewcommand{\headrulewidth}{0 pt} 

% special style for authors pages
\fancypagestyle{authors}{
    \fancyhead[LO]{\small\textit{Authors}} 
    \fancyhead[LE]{\small\thepage}            
    \fancyhead[RE]{\scshape{\small\theissue}}
    \fancyhead[RO]{\small\thepage}            
    \fancyfoot{}
}

% special style for book reviews
\fancypagestyle{reviews}{
    \fancyhead[LO]{\small\textit{Book Reviews}} 
    \fancyhead[LE]{\small\thepage}            
    \fancyhead[RE]{\scshape{\small\theissue}}
    \fancyhead[RO]{\small\thepage}            
    \fancyfoot{}
}

% special style for Editors' preface.
\fancypagestyle{preface}{
    \fancyhead[LO]{\small\textit{\theshortcontribution}} 
    \fancyhead[LE]{\small\thepage}            
    \fancyhead[RE]{\scshape{\small\theissue}}
    \fancyhead[RO]{\small\thepage}            
    \fancyfoot{}
}
% special style for first pages of contributions etc.
% DOES include copyright statement
\fancypagestyle{contributiontitlepage}{
    \fancyhead[C]{\scriptsize\centering\copyrightstatement}
    \fancyhead[L,R]{}
    \fancyfoot[CE,CO]{\small\thepage}
}
% special style for first pages of other \chapters.
% DOES NOT include copyright statement
\fancypagestyle{chaptertitlepage}{
    \fancyhead[C,L,R]{}
    \fancyfoot[CE,CO]{\small\thepage}
}
% no page numbers on \part pages 
\renewcommand*{\partpagestyle}{empty}

%%%% FOOTNOTEFORMAT

% footnotes
\renewcommand{\footnoterule}{%
    \kern .5em  % call this kerna
    \hrule height 0.4pt width .2\columnwidth    % the .2 value made the footnote ruler (horizontal line) smaller (was at .4)
    \kern .5em % call this kernb
}
\usepackage{footmisc}               
\renewcommand{\footnotelayout}{
    \hspace{1.5em}    % space between footnote mark and footnote text
}    
\newcommand{\mytodo}[1]{\textcolor{red}{#1}}

%%%% CODESNIPPETS

% colours for code notations
\usepackage{listings}       
	\renewcommand\lstlistingname{Quelltext} 
	\lstset{                    % basic formatting (bash etc.)
	       basicstyle=\ttfamily,
 	       showstringspaces=false,
	       commentstyle=\color{BrickRed},
	       keywordstyle=\color{RoyalBlue}
	}
	\lstdefinelanguage{XML}{     % specific XML formatting overrides
		  basicstyle=\ttfamily,
		  morestring=[s]{"}{"},
		  morecomment=[s]{?}{?},
		  morecomment=[s]{!--}{--},
		  commentstyle=\color{OliveGreen},
		  moredelim=[s][\color{Black}]{>}{<},
		  moredelim=[s][\color{RawSienna}]{\ }{=},
		  stringstyle=\color{RoyalBlue},
 		  identifierstyle=\color{Plum}
	}
    % HOW TO USE? BASH EXAMPLE
    %   \begin{lstlisting}[language=bash]
    %   #some comment
    %   cd Documents
    %   \end{lstlisting}

\author{Dipanjan Maitra}

\title{Ariadne Nunes, Joana Moura and Marta Pacheco Pinto, eds. \emph{Genetic Translation Studies: Conflict and Collaboration in Liminal Spaces}}

\begin{document}
\maketitle
%%%%%%%%%%%%%%%%%%%%%%%%%%%%%%%%%%%%%%
%% DESCRIPTION OF THE REVIEWED BOOK %%
%%%%%%%%%%%%%%%%%%%%%%%%%%%%%%%%%%%%%%

Review of Ariadne Nunes, Joana Moura and Marta Pacheco Pinto, eds. \emph{Genetic Translation Studies: Conflict and Collaboration in Liminal Spaces}. London and New York, Ny.: Bloomsbury Academic, 2021. 256 pp. ISBN: 978–1–350–14681–5.

%%%%%%%%%%%%%%%%%%%%%%%%%%%%%
%% YOUR REVIEW STARTS HERE %%
%%%%%%%%%%%%%%%%%%%%%%%%%%%%%

% remove asterisk (*) if you want to number your sections
% add a title for your section in between the {curly brackets} if you need one
\section*{} 
``Enhancing textual awareness is one of the major roles translation can
play in the nexus between genetic criticism and translation'', noted
Dirk Van Hulle (\citeyear[43]{van_hulle_translation_2015}). It is precisely this ``nexus'' between
genetic criticism and translation studies which forms the basis of
\emph{Genetic Translation Studies: Conflict and Collaboration in Liminal
Spaces}, edited by Ariadne Nunes, Joana Moura and Marta Pacheco Pinto.
Genetic criticism attempts to isolate itself from other modes of textual
criticism by resisting a teleological impulse towards a \emph{final},
published text. Instead, a genetic study of \emph{avant-textes} focuses
on the writerly process, its twists and turns, unraveling, to use a
Borgesian metaphor, a text's multiple ``forking paths'', its myriad
possibilities. The ``independence'' of the genetic process from the
published text sees a parallel in translation studies where the
independence of the translated text from its ``original'' ur-text has
become a talking point. Perhaps if one sees genetic criticism as a
description of the genesis of a text, then translation studies could be
seen as concerned with the post-text and its future reincarnations. In
an oft-quoted essay by Cordingley and Montini, the authors thus
postulate that ``Genetic criticism has given little attention to the
post-text while at the same time much attention in recent translation
studies has been directed towards establishing the independence of the
translated text'' (\citeyear[5]{cordingley_genetic_2015}). The significance of incorporating genetic
approaches to translation studies is revisited powerfully in
\emph{Genetic Translation Studies}.

Conceived initially as an international conference on \emph{Unexpected
Intersections: Translation Studies and Genetic Criticism} under the
aegis of the University of Lisbon in 2017, the book includes fourteen
essays grouped under three broad tenets: (1) genetic approaches to
translation and collaboration, (2) translators' stories and testimonies,
and (3) translators at work. Expectedly, several of the essays focus on
case studies of Portuguese authors but the anthology does a commendable
job in mapping a wide linguistic canvas with essays on translation
projects in several major European languages. These include translated
texts from French to English (Esa Christine Hartmann on Robert
Fitzgerald's translation of Saint-John Perse's poem \emph{Chronique};
\citeyear{perse_chronique_1961}), English to Polish (Ewa Kołodziejczyk on Czesław Miłosz's
translation of ``Negro spirituals''; \cite{Miłosz_negro_1948,Miłosz_wiersze_1948}), from English to French
(Patrick Hersant on Maurice Coindreau's translations of American novels;
e.g. Steinbeck \citeyear{steinbeck_souris_1939} and Capote \citeyear{capote_harpe_1952}), French to German (Joana Moura on
Peter Handke's translation of René Char's poems from \emph{Le nu perdu};
\citedate{char_ruckkehr_1984}) and even from Sanskrit to Portuguese (Marta Pacheco Pinto and
Ariadne Nunes on the Portuguese Orientalist Vasconcelos Abreu's efforts
at translating the \emph{Panchatantra}).

João Dionísio's opening piece lays out the challenges (or, as he calls
it, ``latency'') of genetic translation studies in Portugal. For him,
this is a result of the absence of ``a self-aware interpretive
community'' (28). Despite these obstacles, his two main examples, a
fifteenth-century treatise by King Duarte of Portugal on Latin
translations and a Fernando Pessoa poem which contains transcreations of
a poem by the Victorian poet Alfred Austin, both showcase how genetic
studies of translations challenge any unitary understanding of
authorship. Dionísio ably demonstrates how a genetic analysis of these
authors separated by centuries can still reveal authorship as a complex,
multifarious process, involving external agents or a ``network of
interacting and negotiating subjects'' (39). Laura Ivaska's meticulously
researched article on a Finnish translation of Nikos Kazantzakis's novel
\emph{The Fratricides} (\citeyear{kazantzakis_veljesviha_1967}), which used a French translation for its
``best text'' is a further elaboration on this network of interacting
and negotiating subjects.

If translators are crucial ``intermediaries'' in introducing ``foreign''
texts into a literary field (to echo Pascale \citeauthor{casanova_world_2004}), then perhaps one
of the more intriguing case studies in the volume is of a translation
that is not even complete: Marta Pacheco Pinto and Ariadne Nunes's
account of Guilherme de Vasconcelos Abreu's Orientalist projects
(1842--1907), in what would have been the first Portuguese translation
of the ancient Sanskrit text of fables \emph{Panchatantra} (200 B.C.
ca.). Pacheco Pinto and Nunes analyze Abreu's extant notes for the
translation in two unpublished volumes and demonstrate how even an
unfinished, unpublished project such as Abreu's can shed light on the
debates that informed Orientalist scholarship in its early days in
Portugal. What is only tantalizingly hinted at here and therefore opens
up avenues for future scholarship, are larger questions about print
culture and the communications circuits that connected Portuguese
Orientalists like Abreu with books published in Calcutta via the India
Office in London (221).

This question of continual mediation in translation studies becomes even
more apparent as we introspect the increasing proximity of digital and
computational technologies and genetic criticism. Understandably,
several essays in the collection refer to ongoing and future digital
humanities projects. To give one example, Elsa Pereira examines the
prospects of a digital edition of Pedro Homem de Melo's poetry. Pereira
argues that because of the complexity of de Melo's genetic dossier -- a
mixture of allographic translations (translations of de Melo's poetry by
translators other than the poet), self-translations and what Pereira
terms ``alloglottic rewriting'' involving several stages of rewritings
of his own poems in French -- a multi-layered, para-genetic edition is
the need of the hour. For a ``multi-level'' edition of this kind, a
digital edition becomes a necessity.

What these accounts highlight more acutely, however, is that, despite
new scholarship and groundbreaking theorizations in the field, the
translators's role still lacks visibility and due recognition. Marisa
Mourinha quotes Gregory Rabassa -- easily one of the undisputed
authorities in the field, due to his much-venerated translations of
Gabriel García Márquez (\citeyear{marquez_one_1970}, \citeyear{marquez_autumn_1975} and \citeyear{marquez_chronicle_1983}) -- from an interview
released during the central years of the Latin American Boom, arguing
that ``It is time for at least one definitive statement about
translation: it is impossible to make a handsome living out of it''
(\citedate[37--38]{rabassa_if_1974}; qtd. 149). As Mourinha notes, Rabassa's statement may
still ring true today, in a field that is still dominated by academics.
This is why, perhaps, one of the more memorable sections of this
collection consists of chapters which chart out translators's
testimonies, self-reflections and correspondence, as well as the very
real physical labor of translation, as in Ivančić and Zepter's chapter
(123--34). This is also where the genetic dossier becomes a treasure
trove of personal anecdotes, scholarly journeys and where textuality
touches on the experiential. They highlight for us moments when, as
Emily Apter would note in a slightly different context, translators
decide to go on, ward off challenges -- philological, socio-economic,
philosophic (``the singularity of untranslatable alterity'') -- and
decide to translate \emph{quand même} (\citedate[104]{apter_translation_2006}).


\begin{flushleft}
    % use smallcaps for author names
    \renewcommand*{\mkbibnamefamily}[1]{\textsc{#1}}
    \renewcommand*{\mkbibnamegiven}[1]{\textsc{#1}} 
\printbibliography
\end{flushleft}

\end{document}