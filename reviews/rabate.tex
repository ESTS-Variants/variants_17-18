\documentclass{article}
%%%% CLASS OPTIONS 

\KOMAoptions{
    fontsize=10pt,              % set default font size
    DIV=calc,
    titlepage=false,
    paper=150mm:220mm,
    twoside=true, 
    twocolumn=false,
    toc=chapterentryfill,       % for dots: chapterentrydotfill
    parskip=false,              % space between paragraphs. "full" gives more space; "false" uses indentation instead
    headings=small,
    bibliography=leveldown,     % turns the Bibliography into a \section rather than a \chapter (so it appears on the same page)
}

%%%% PAGE SIZE

\usepackage[
    top=23mm,
    left=20mm,
    height=173mm,
    width=109mm,
    ]{geometry}

\setlength{\marginparwidth}{1.25cm} % sets up acceptable margin for \todonotes package (see preamble/packages.tex).

%%%% PACKAGES

\usepackage[dvipsnames]{xcolor}
\usepackage[unicode]{hyperref}  % hyperlinks
\usepackage{booktabs}           % professional-quality tables
\usepackage{nicefrac}           % compact symbols for 1/2, etc.
%\usepackage{microtype}          % microtypography
\usepackage{lipsum}             % lorem ipsum at the ready
\usepackage{graphicx}           % for figures
\usepackage{footmisc}           % makes symbol footnotes possible
\usepackage{ragged2e}
\usepackage{changepage}         % detect odd/even pages
\usepackage{array}
\usepackage{float}              % get figures etc. to stay where they are with [H]
\usepackage{subfigure}          % \subfigures witin a \begin{figure}
\usepackage{longtable}          % allows for tables that stretch over multiple pages
\setlength{\marginparwidth}{2cm}
\usepackage[textsize=footnotesize]{todonotes} % enables \todo's for editors
\usepackage{etoolbox}           % supplies commands like \AtBeginEnvironment and \atEndEnvironment
\usepackage{ifdraft}            % switches on proofreading options in the draft mode
\usepackage{rotating}           % provides sidewaysfigure environment
\usepackage{media9}             % allows for video in the pdf
\usepackage{xurl}               % allows URLs to (line)break ANYWHERE

%%%% ENCODING

\usepackage[full]{textcomp}                   % allows \textrightarrow etc.

% LANGUAGES

\usepackage{polyglossia}
\setmainlanguage{english} % Continue using english for rest of the document

% If necessary, the following lets you use \texthindi. Note, however, that BibLaTeX does not support it and will report a 'warning'.
 \setotherlanguages{hindi} 
 \newfontfamily\hindifont{Noto Sans Devanagari}[Script=Devanagari]

% biblatex
\usepackage[
    authordate,
    backend=biber,
    natbib=true,
    maxcitenames=2,
    ]{biblatex-chicago}
\usepackage{csquotes}

% special characters  
\usepackage{textalpha}                  % allows for greek characters in text 

%%%% FONTS

% Palatino font options
\usepackage{unicode-math}
\setmainfont{TeX Gyre Pagella}
\let\circ\undefined
\let\diamond\undefined
\let\bullet\undefined
\let\emptyset\undefined
\let\owns\undefined
\setmathfont{TeX Gyre Pagella Math}
\let\ocirc\undefined
\let\widecheck\undefined

\addtokomafont{disposition}{\rmfamily}  % Palatino for titles etc.
\setkomafont{descriptionlabel}{         % font for description lists    
\usekomafont{captionlabel}\bfseries     % Palatino bold
}
\setkomafont{caption}{\footnotesize}    % smaller font size for captions


\usepackage{mathabx}                    % allows for nicer looking \cup, \curvearrowbotright, etc. !!IMPORTANT!! These are math symbols and should be surrounded by $dollar signs$
\usepackage[normalem]{ulem}                       % allows for strikethrough with \sout etc.
\usepackage{anyfontsize}                          % fixes font scaling issue

%%%% ToC

% No (sub)sections in TOC
\setcounter{tocdepth}{0}                

% Redefines chapter title formatting
\makeatletter                               
\def\@makechapterhead#1{
  \vspace*{50\p@}%
  {\parindent \z@ \normalfont
    \interlinepenalty\@M
    \Large\raggedright #1\par\nobreak%
    \vskip 40\p@%
  }}
\makeatother
% a bit more space between titles and page numbers in TOC

\makeatletter   
\renewcommand\@pnumwidth{2.5em} 
\makeatother

%%%% CONTRIBUTOR

% Title and Author of individual contributions
\makeatletter
% paper/review author = contributor
\newcommand\contributor[1]{\renewcommand\@contributor{#1}}
\newcommand\@contributor{}
\newcommand\thecontributor{\@contributor} 
% paper/review title = contribution
\newcommand\contribution[1]{\renewcommand\@contribution{#1}}
\newcommand\@contribution{}
\newcommand\thecontribution{\@contribution}
% short contributor for running header
\newcommand\shortcontributor[1]{\renewcommand\@shortcontributor{#1}}
\newcommand\@shortcontributor{}
\newcommand\theshortcontributor{\@shortcontributor} 
% short title for running header
\newcommand\shortcontribution[1]{\renewcommand\@shortcontribution{#1}}
\newcommand\@shortcontribution{}
\newcommand\theshortcontribution{\@shortcontribution}
\makeatother

%%%% COPYRIGHT

% choose copyright license
\usepackage[               
    type={CC},
    modifier={by},
    version={4.0},
]{doclicense}

% define \copyrightstatement for ease of use
\newcommand{\copyrightstatement}{
         \doclicenseIcon \ \theyear. 
         \doclicenseLongText            % includes a link
}

%%%% ENVIRONMENTS
% Environments
\AtBeginEnvironment{quote}{\footnotesize\vskip 1em}
\AtEndEnvironment{quote}{\vskip 1em}

\setkomafont{caption}{\footnotesize}

% Preface
\newenvironment{preface}{
    \newrefsection
    \phantomsection
    \cleardoublepage
    \addcontentsline{toc}{part}{\thecontribution}
    % enable running title
    \pagestyle{preface}
    % \chapter*{Editors' Preface}    
    % reset the section counter for each paper
    \setcounter{section}{0}  
    % no running title on first page, page number center bottom instead
    \thispagestyle{chaptertitlepage}
}{}
\AtEndEnvironment{preface}{%
    % safeguard section numbering
    \renewcommand{\thesubsection}{\thesection.\arabic{subsection}}  
    %last page running header fix
    \protect\thispagestyle{preface}
}
% Essays
\newenvironment{paper}{
    \newrefsection
    \phantomsection
    % start every new paper on an uneven page 
    \cleardoublepage
    % enable running title
    \pagestyle{fancy}
    % change section numbering FROM [\chapter].[\section].[\subsection] TO [\section].[\subsection] ETC.
    \renewcommand{\thesection}{\arabic{section}}
    % mark chapter % add author + title to the TOC
    \chapter[\normalfont\textbf{\emph{\thecontributor}}: \thecontribution]{\vspace{-4em}\Large\normalfont\thecontribution\linebreak\normalsize\begin{flushright}\emph{\thecontributor}\end{flushright}}    
    % reset the section counter for each paper
    \setcounter{section}{0}  
    % reset the figure counter for each paper
    \renewcommand\thefigure{\arabic{figure}}    
    % reset the table counter for each paper
    \renewcommand\thetable{\arabic{table}} 
    % no running title on first page, page number center bottom instead, include copyright statement
    \thispagestyle{contributiontitlepage}
    % formatting for the bibliography

}{}
\AtBeginEnvironment{paper}{
    % keeps running title from the first page:
    \renewcommand*{\pagemark}{}%                            
}
\AtEndEnvironment{paper}{
    % safeguard section numbering
    \renewcommand{\thesubsection}{\thesection.\arabic{subsection}}  
    % last page running header fix
    \protect\thispagestyle{fancy}%                              
}
% Reviews
\newenvironment{review}{
    \newrefsection
    \phantomsection
    % start every new paper on an uneven page 
    \cleardoublepage
    % enable running title
    \pagestyle{reviews}
    % change section numbering FROM [\chapter].[\section].[\subsection] TO [\section].[\subsection] ETC.
    \renewcommand{\thesection}{\arabic{section}} 
    % mark chapter % add author + title to the TOC
    \chapter[\normalfont\textbf{\emph{\thecontributor}}: \thecontribution]{}    % reset the section counter for each paper
    \setcounter{section}{0}  
    % no running title on first page, page number center bottom instead, include copyright statement
    \thispagestyle{contributiontitlepage}
    % formatting for the bibliography
}{}
\AtBeginEnvironment{review}{
% keeps running title from the first page
    \renewcommand*{\pagemark}{}%                                   
}
\AtEndEnvironment{review}{
    % author name(s)
    \begin{flushright}\emph{\thecontributor}\end{flushright}
    % safeguard section numbering
    \renewcommand{\thesubsection}{\thesection.\arabic{subsection}} 
    % last page running header fix
    \protect\thispagestyle{reviews}                           
}

% Abstract
\newenvironment{abstract}{% 
\setlength{\parindent}{0pt} \begin{adjustwidth}{2em}{2em}\footnotesize\emph{\abstractname}: }{%
\vskip 1em\end{adjustwidth}
}{}

% Keywords
\newenvironment{keywords}{
\setlength{\parindent}{0pt} \begin{adjustwidth}{2em}{2em}\footnotesize\emph{Keywords}: }{%
\vskip 1em\end{adjustwidth}
}{}

% Review Abstract
\newenvironment{reviewed}{% 
\setlength{\parindent}{0pt}
    \begin{adjustwidth}{2em}{2em}\footnotesize}{%
\vskip 1em\end{adjustwidth}
}{}

% Motto
\newenvironment{motto}{% 
\setlength{\parindent}{0pt} \small\raggedleft}{%
\vskip 2em
}{}

% Example
\newcounter{example}[chapter]
\newenvironment{example}[1][]{\refstepcounter{example}\begin{quote} \rmfamily}{\begin{flushright}(Example~\theexample)\end{flushright}\end{quote}}

%%%% SECTIONOPTIONS

% command for centering section headings
\newcommand{\centerheading}[1]{   
    \hspace*{\fill}#1\hspace*{\fill}
}

% Remove "Part #." from \part titles
% KOMA default: \newcommand*{\partformat}{\partname~\thepart\autodot}
\renewcommand*{\partformat}{} 

% No dots after figure or table numbers
\renewcommand*{\figureformat}{\figurename~\thefigure}
\renewcommand*{\tableformat}{\tablename~\thetable}

% paragraph handling
\setparsizes%
    {1em}% indent
    {0pt}% maximum space between paragraphs
    {0pt plus 1fil}% last line not justified
    

% In the "Authors" section, author names are put in the \paragraph{} headings. To reduce the space after these  headings, the default {-1em} has been changed to {-.4em} below.
\makeatletter
\renewcommand\paragraph{\@startsection {paragraph}{4}{\z@ }{3.25ex \@plus 1ex \@minus .2ex}{-.4em}{\normalfont \normalsize \bfseries }
}
\makeatother

% add the following (uncommented) in environments where you want to count paragraph numbers in the margin
%    \renewcommand*{\paragraphformat}{%
%    \makebox[-4pt][r]{\footnotesize\theparagraph\autodot\enskip}
%    }
%    \renewcommand{\theparagraph}{\arabic{paragraph}}
%    \setcounter{paragraph}{0}
%    \setcounter{secnumdepth}{4}
    
%%%% HEADERFOOTER

% running title
\RequirePackage{fancyhdr}
% cuts off running titles that are too long
%\RequirePackage{truncate}
% makes header as wide as geometry (SET SAME AS \TEXTWIDTH!)
\setlength{\headwidth}{109mm} 
% LO = Left Odd
\fancyhead[LO]{\small\emph{\theshortcontributor} \hspace*{.5em} \theshortcontribution} 
% RE = Right Even
\fancyhead[RE]{\scshape{\small\theissue}}
% LE = Left Even
\fancyhead[LE]{\small\thepage}            
% RE = Right Odd
\fancyhead[RO]{\small\thepage}    
\fancyfoot{}
% no line under running title; cannot be \@z but needs to be 0pt
\renewcommand{\headrulewidth}{0 pt} 

% special style for authors pages
\fancypagestyle{authors}{
    \fancyhead[LO]{\small\textit{Authors}} 
    \fancyhead[LE]{\small\thepage}            
    \fancyhead[RE]{\scshape{\small\theissue}}
    \fancyhead[RO]{\small\thepage}            
    \fancyfoot{}
}

% special style for book reviews
\fancypagestyle{reviews}{
    \fancyhead[LO]{\small\textit{Book Reviews}} 
    \fancyhead[LE]{\small\thepage}            
    \fancyhead[RE]{\scshape{\small\theissue}}
    \fancyhead[RO]{\small\thepage}            
    \fancyfoot{}
}

% special style for Editors' preface.
\fancypagestyle{preface}{
    \fancyhead[LO]{\small\textit{\theshortcontribution}} 
    \fancyhead[LE]{\small\thepage}            
    \fancyhead[RE]{\scshape{\small\theissue}}
    \fancyhead[RO]{\small\thepage}            
    \fancyfoot{}
}
% special style for first pages of contributions etc.
% DOES include copyright statement
\fancypagestyle{contributiontitlepage}{
    \fancyhead[C]{\scriptsize\centering\copyrightstatement}
    \fancyhead[L,R]{}
    \fancyfoot[CE,CO]{\small\thepage}
}
% special style for first pages of other \chapters.
% DOES NOT include copyright statement
\fancypagestyle{chaptertitlepage}{
    \fancyhead[C,L,R]{}
    \fancyfoot[CE,CO]{\small\thepage}
}
% no page numbers on \part pages 
\renewcommand*{\partpagestyle}{empty}

%%%% FOOTNOTEFORMAT

% footnotes
\renewcommand{\footnoterule}{%
    \kern .5em  % call this kerna
    \hrule height 0.4pt width .2\columnwidth    % the .2 value made the footnote ruler (horizontal line) smaller (was at .4)
    \kern .5em % call this kernb
}
\usepackage{footmisc}               
\renewcommand{\footnotelayout}{
    \hspace{1.5em}    % space between footnote mark and footnote text
}    
\newcommand{\mytodo}[1]{\textcolor{red}{#1}}

%%%% CODESNIPPETS

% colours for code notations
\usepackage{listings}       
	\renewcommand\lstlistingname{Quelltext} 
	\lstset{                    % basic formatting (bash etc.)
	       basicstyle=\ttfamily,
 	       showstringspaces=false,
	       commentstyle=\color{BrickRed},
	       keywordstyle=\color{RoyalBlue}
	}
	\lstdefinelanguage{XML}{     % specific XML formatting overrides
		  basicstyle=\ttfamily,
		  morestring=[s]{"}{"},
		  morecomment=[s]{?}{?},
		  morecomment=[s]{!--}{--},
		  commentstyle=\color{OliveGreen},
		  moredelim=[s][\color{Black}]{>}{<},
		  moredelim=[s][\color{RawSienna}]{\ }{=},
		  stringstyle=\color{RoyalBlue},
 		  identifierstyle=\color{Plum}
	}
    % HOW TO USE? BASH EXAMPLE
    %   \begin{lstlisting}[language=bash]
    %   #some comment
    %   cd Documents
    %   \end{lstlisting}

\author{Jean-Michel Rabaté}

\title{Alfred Jarry. {[}Messaline\emph{,} Olalla \emph{(Robert Louis
Stevenson),} Le surmâle\emph{,} Pieter de Delft\emph{,} Jef\emph{,} Le
bon roi Dagobert\emph{,} Le manoir enchanté\emph{,} L'amour maladroit
\emph{et derniers poèmes}{]}. Ed. Henri Béhar et al.}

\begin{document}
\maketitle

%%%%%%%%%%%%%%%%%%%%%%%%%%%%%%%%%%%%%%
%% DESCRIPTION OF THE REVIEWED BOOK %%
%%%%%%%%%%%%%%%%%%%%%%%%%%%%%%%%%%%%%%

\begin{reviewed}
Review of [Messaline\emph{,} Olalla \emph{(Robert Louis
Stevenson),} Le surmâle\emph{,} Pieter de Delft\emph{,} Jef\emph{,} Le
bon roi Dagobert\emph{,} Le manoir enchanté\emph{,} L'amour maladroit
\emph{et derniers poèmes}{]}. Ed. Henri Béhar et al. (Vol. 5 of
\emph{Oeuvres complètes}. Gen ed. Henri Béhar). Paris: Classiques
Garnier, 2019. 856 pp. ISBN: 978--2--406--08505--8.
\end{reviewed}

%%%%%%%%%%%%%%%%%%%%%%%%%%%%%
%% YOUR REVIEW STARTS HERE %%
%%%%%%%%%%%%%%%%%%%%%%%%%%%%%

% remove asterisk (*) if you want to number your sections
% add a title for your section in between the {curly brackets} if you need one
\section*{} 
For lovers of Alfred Jarry, the fifth volume of the \emph{Collected
Works} edited by Henri Béhar offers a real treasure-trove. Faithful to
chronological order, we move between 1901 and 1905. By a mathematical
coincidence that would have pleased Jarry, just half of the hefty volume
is taken up by prose: \emph{Messaline}, \emph{The Supermale} and a
translation of Robert Louis Stevenson's \emph{Olalla}, a novella first published in 1885 and later collected in \emph{The Merry Men and Other Tales and Fables} (\citeyear{stevenson_merry_1887}). If we compare
this corpus with the second volume of the \emph{Oeuvres complètes},
published in the Pléiade collection and edited by Michel Arrivé and Henri Bordillon, with the collaboration of Patrick Besnier and Bernard Le Doze (\citedate{jarry_oeuvres_1972}), one verifies that a lot of new ground has
been covered by Béhar and collaborators.

Previous editors did not have access to the notes taken by Jarry when
preparing his historical novel \emph{Messaline}. Here, instead, the
volume reproduces the passages that Jarry condensed from Juvenal, Dion
Cassius, Petronius, Seneca, Suetonius, Tacitus and a few others, along
with the chapters in which they were used, including doodles or drawings
of places and objects like the famous spinning top (149--98). Indeed, as
Julien Schuh writes, this novel was composed differently from the
others, for Jarry took great pains to be accurate. He worked like
Flaubert drafting \emph{Salammbô} (\citeyear{flaubert_salammbo_1862}). Jarry confessed to his friend and
mentor Rachilde that he was working ``come un simple Zola'' [like a simple Zola] (15). However, the complex texture of \emph{Messaline}
shows that the point was not just to recreate ancient Rome in all its
flavor and exotic details; here like elsewhere, Jarry aimed at
appropriating and rewriting previous sources so as to create a network
of layered and overdetermined relationships that work at different
levels by harmonically echoing one another. \emph{Messaline} is composed
like a poem by Mallarmé, a sort of prose \emph{Hérodiade} (begun in 1864 and revised until Mallarmé's death in 1898). A principle
of reversibility is key: ROMA is always the double of AMOR. Most
contemporary readers perceived that the emperor Claudius was a
Mallarméan figure, and he is presented as always working on a \emph{Book
of Dice}. It makes sense to read reviews penned by contemporaries,
beginning with Rachilde's praise in the \emph{Mercure de France} of February 1901, which closes with the following remarks:

\begin{quote}
Moi[,] qui ne sais ni latin ni grec, je garde l’impression d’avoir lu un manuscrit qui aurait été entièrement écrit à l’époque de crimes ingénus et de barbares fantaisies où vivait Claude lui-même, bon traducteur du si curieux \emph{livre des dés}, et durant la course de l’impératrice le long des jardins de Priape, ``durant qu’elle accroche son manteau à griffes d’or sur les vases murrhins'', j’ai vu vraiment, non sans un frisson d’effroi, se dresser Mnester, la pâle et vivante énigme, le sphinx qui dansait quelquefois dans la nuit d’une éclipse.

\vspace{1em}

[I, who know neither Latin nor Greek, keep the impression of having read a
manuscript that would have been entirely written at the time of
ingenuous crimes and barbarous fantasies when Claudius himself was
alive, the good translator of the curious \emph{Book of Dice,} during
the empress's flight along the gardens of Priapus, ``catching her mantle
of gold talons on the murrhine vases'', I have seen really, and not
without a shiver of fear, Mnester rise, the pale and living enigma, the
sphinx who at times would dance in the night of eclipses.]

\begin{flushright}
(51)
\end{flushright}
\end{quote}

\noindent It is no less fascinating to read the draft of Jarry's review of his own
book (198--99). One guesses that the main idea that he wanted to expose
was that he was evoking a pre-Christian world, ``le nom de \emph{Christ} ou \emph{Chrest} n’est [\ldots{}] qu’un surnom banal des esclaves'' [the name of \emph{Christ} or \emph{Chrest} was only [\dots{}] the common name of slaves] (200), also when a more
terrifying divinity possessed the protagonists: Priapus. A subsequently
crossed out sentence defined the kind of writing with bold assurance:
``Ce livre est écrit dans une forme nette et absolue définitive'' [This book is written in a form that is clean and absolute definitive] (199, \emph{sic}). Jarry was aware that he
was inventing a new way of writing prose.

If Jarry was an excellent Latinist who had trained (always in vain,
alas) for the entrance examination of the École Normale Supérieure, his
English was not at the level expected from a translator. Nevertheless,
his misreadings or mistakes when rendering Stevenson's lush prose are
suggestive. Thus, when Olalla and the narrator cross their gazes, Jarry
renders ``embraced'' as ``s'embrasèrent'' {[}caught fire{]} (in ``nos
regards une fois de plus se rencontrèrent et s'embrasèrent''; 261),
which somehow improves the original. Admittedly, such felicitous
metamorphoses are rare; indeed, often Jarry muddles or weakens
Stevenson's text. Why did he decide to translate this novella? He must
have taken to this Gothic tale with a faint suggestion of a vampire
story (when the half-crazy mother of the girl, instead of helping the
narrator who has cut his arm, bites his bleeding arm). Another draw may
have come from the fact that the Scottish soldier who narrates is
repeatedly presented as a ``non-Christian'' lover of a young woman. She,
on the other hand, resists him and convinces him to leave by invoking
the picture and the name of Jesus.

\emph{Olalla} provides a neat transition to \emph{Le Surmâle,} a novel
whose program derives from a few words, culled in Juvenal's sixth satire
which describe Messalina’s sexual habits: ``\emph{Tamen ultima cellam / Clausit,
adhuc ardens RIGIDAE tentigine vulvae}'' (347). Indeed,
\emph{The Supermale} is not only the first ``modern'' but also the first
``modernist'' novel, shot through as it is with speculations about a
futurist intermixing of men and machines. Marcueil is fascinated by the
fact that Messalina's sexuality is so extreme that she suffers less from
nymphomania than from a feminine form of ithyphallicism, priapism or
satyriasis. Marcueil wants to be the male equivalent of a Messalina in a
state of perpetual erection. Jarry conveys the vertigo he feels facing
by numbers related to sexuality and machines. His hero wants to exceed
the limits of human strength and reach a numerical infinity, or
immortality accessible through sexual excess. Marcueil's fantasy of
making love indefinitely explains the idea of a ``supermale'' who
combines the rigidity of a phallus and the productivity of an engine.
Henri Béhar is right to see the plot ultimately predicated upon the
philosophical principle of the identity of the contraries (292), a
medieval conceit that Jarry grafts onto his mixture of Nietzscheism and
Bergsonism.

The plays and opera \emph{Pieter de Delft, Jeff}, \emph{Le bon roi
Dagobert}, \emph{Le manoir enchanté} and \emph{L'amour maladroit} show
that Jarry was groping for a new theatrical vein after the success of
the cycle of \emph{Ubu}, but failed to find it. If here and there a few
gems shine in rare or funny rhymes, in verbal echoes of advertisements
or geographical phrases, most of these texts sound laborious, the jokes
dated and the \emph{boulevardier} humour forced. They were mostly
products of collaborations with friends. Jarry was not successful in
hitting upon a lighter note, as if he did not know how to write comedies
without being sarcastic or dark. In \emph{Jeff,} he alternates different
accents, with marked clichés and obvious stereotypes of British,
Marseillais and Belgian intonations; all the speeches conclude with
exclamation marks. Jarry hoped that some of these attempts would be
musicalized---that would have been one way of saving them.

However, the ``last poems'' gathered in this volume show real promise.
They announce a poetic departure cut short by Jarry's untimely demise.
They are annotated in precise detail by Alain Chevrier. ``Bardes et
cordes'' announces the project of writing under constraint that will
become the trademark of the Oulipo group half a century later. The poem
is generated by the false rhyme of \emph{hallebarde} and
\emph{miséricorde,} a rhyme that had become proverbial. Its echoes
create the scene of the death of Louis XVI, an evocation combining
historical details with an impressive degree of semantic opacity
predicated on inner rhymes and vibrant assonances. I agree with Chevrier
when he rejects the gloss of the Pléiade editors: on the strength of the
poem's topic, they assert that Jarry had become a reactionary royalist
in 1903, mistaking elaborate verbal games for serious ideological
affiliation. The cumulative effect is not far from the classical poems
retranslated into a French devoid of ``-e'' in Georges Perec's \emph{La
Disparition} (\citeyear{perec_disparition_1969}):

\begin{quote}
Aux barrières du Louvre elle dormait, la garde:

Les palais sont de grands ports où la mort aborde;

Corse, kamoulcke, kurde, iroquoise et lombarde,

Le catafalque est ceint de la jobarde horde.

Sa veille n'eût point fait camuse la camarde:

Il faut qu'un rictus torde et qu'une bouche morde.

\begin{flushright}
    (806)   
\end{flushright}
\end{quote}

\noindent ``Madrigal'' remains Jarry's poetic masterpiece, with its imagistic
evocation of a prostitute who calls up Messalina. She allows the speaker
to discover not only a sublime peace but also glimpse the ``absolute''.
Its last lines condense all of Baudelaire along with the gist of
\emph{Messalina} in just eight lines:

\begin{quote}
Comment s'unit la double destinée?

Tant que je n'eus point pris votre trottoir

Vous étiez vierge et vous n'étiez point née,

Comme un passé se noie en un miroir.

La boue à peine a baisé la chaussure

De votre pied infinitésimal,

Et c'est d'avoir mordu dans tout le mal

\emph{Qui vous a fait une bouche si pure.}
\begin{flushright}
    (792)   
\end{flushright}
\end{quote}

\noindent Let me attempt a poor translation of this superb song, which is often
recited by performers:
\begin{quote}
How is double destiny tight-lipped?

As long as I avoided your sidewalk

You were a virgin, you had not been born,

Like a past drowning in a mirror.

Mud hardly kissed the shoe

Of your infinitesimal foot,

It was from having bitten into all evil

That your mouth was made so pure.
\end{quote}

\noindent Jarry's unique genius shines in so many of these necessary pages that
one can only rejoy when discovering that the contents are available in a
cheaper paperback edition.


\begin{flushleft}
    % use smallcaps for author names
    \renewcommand*{\mkbibnamefamily}[1]{\textsc{#1}}
    \renewcommand*{\mkbibnamegiven}[1]{\textsc{#1}} 
\printbibliography
\end{flushleft}

\end{document}