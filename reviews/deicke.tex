\documentclass{article}
%%%% CLASS OPTIONS 

\KOMAoptions{
    fontsize=10pt,              % set default font size
    DIV=calc,
    titlepage=false,
    paper=150mm:220mm,
    twoside=true, 
    twocolumn=false,
    toc=chapterentryfill,       % for dots: chapterentrydotfill
    parskip=false,              % space between paragraphs. "full" gives more space; "false" uses indentation instead
    headings=small,
    bibliography=leveldown,     % turns the Bibliography into a \section rather than a \chapter (so it appears on the same page)
}

%%%% PAGE SIZE

\usepackage[
    top=23mm,
    left=20mm,
    height=173mm,
    width=109mm,
    ]{geometry}

\setlength{\marginparwidth}{1.25cm} % sets up acceptable margin for \todonotes package (see preamble/packages.tex).

%%%% PACKAGES

\usepackage[dvipsnames]{xcolor}
\usepackage[unicode]{hyperref}  % hyperlinks
\usepackage{booktabs}           % professional-quality tables
\usepackage{nicefrac}           % compact symbols for 1/2, etc.
%\usepackage{microtype}          % microtypography
\usepackage{lipsum}             % lorem ipsum at the ready
\usepackage{graphicx}           % for figures
\usepackage{footmisc}           % makes symbol footnotes possible
\usepackage{ragged2e}
\usepackage{changepage}         % detect odd/even pages
\usepackage{array}
\usepackage{float}              % get figures etc. to stay where they are with [H]
\usepackage{subfigure}          % \subfigures witin a \begin{figure}
\usepackage{longtable}          % allows for tables that stretch over multiple pages
\setlength{\marginparwidth}{2cm}
\usepackage[textsize=footnotesize]{todonotes} % enables \todo's for editors
\usepackage{etoolbox}           % supplies commands like \AtBeginEnvironment and \atEndEnvironment
\usepackage{ifdraft}            % switches on proofreading options in the draft mode
\usepackage{rotating}           % provides sidewaysfigure environment
\usepackage{media9}             % allows for video in the pdf
\usepackage{xurl}               % allows URLs to (line)break ANYWHERE

%%%% ENCODING

\usepackage[full]{textcomp}                   % allows \textrightarrow etc.

% LANGUAGES

\usepackage{polyglossia}
\setmainlanguage{english} % Continue using english for rest of the document

% If necessary, the following lets you use \texthindi. Note, however, that BibLaTeX does not support it and will report a 'warning'.
 \setotherlanguages{hindi} 
 \newfontfamily\hindifont{Noto Sans Devanagari}[Script=Devanagari]

% biblatex
\usepackage[
    authordate,
    backend=biber,
    natbib=true,
    maxcitenames=2,
    ]{biblatex-chicago}
\usepackage{csquotes}

% special characters  
\usepackage{textalpha}                  % allows for greek characters in text 

%%%% FONTS

% Palatino font options
\usepackage{unicode-math}
\setmainfont{TeX Gyre Pagella}
\let\circ\undefined
\let\diamond\undefined
\let\bullet\undefined
\let\emptyset\undefined
\let\owns\undefined
\setmathfont{TeX Gyre Pagella Math}
\let\ocirc\undefined
\let\widecheck\undefined

\addtokomafont{disposition}{\rmfamily}  % Palatino for titles etc.
\setkomafont{descriptionlabel}{         % font for description lists    
\usekomafont{captionlabel}\bfseries     % Palatino bold
}
\setkomafont{caption}{\footnotesize}    % smaller font size for captions


\usepackage{mathabx}                    % allows for nicer looking \cup, \curvearrowbotright, etc. !!IMPORTANT!! These are math symbols and should be surrounded by $dollar signs$
\usepackage[normalem]{ulem}                       % allows for strikethrough with \sout etc.
\usepackage{anyfontsize}                          % fixes font scaling issue

%%%% ToC

% No (sub)sections in TOC
\setcounter{tocdepth}{0}                

% Redefines chapter title formatting
\makeatletter                               
\def\@makechapterhead#1{
  \vspace*{50\p@}%
  {\parindent \z@ \normalfont
    \interlinepenalty\@M
    \Large\raggedright #1\par\nobreak%
    \vskip 40\p@%
  }}
\makeatother
% a bit more space between titles and page numbers in TOC

\makeatletter   
\renewcommand\@pnumwidth{2.5em} 
\makeatother

%%%% CONTRIBUTOR

% Title and Author of individual contributions
\makeatletter
% paper/review author = contributor
\newcommand\contributor[1]{\renewcommand\@contributor{#1}}
\newcommand\@contributor{}
\newcommand\thecontributor{\@contributor} 
% paper/review title = contribution
\newcommand\contribution[1]{\renewcommand\@contribution{#1}}
\newcommand\@contribution{}
\newcommand\thecontribution{\@contribution}
% short contributor for running header
\newcommand\shortcontributor[1]{\renewcommand\@shortcontributor{#1}}
\newcommand\@shortcontributor{}
\newcommand\theshortcontributor{\@shortcontributor} 
% short title for running header
\newcommand\shortcontribution[1]{\renewcommand\@shortcontribution{#1}}
\newcommand\@shortcontribution{}
\newcommand\theshortcontribution{\@shortcontribution}
\makeatother

%%%% COPYRIGHT

% choose copyright license
\usepackage[               
    type={CC},
    modifier={by},
    version={4.0},
]{doclicense}

% define \copyrightstatement for ease of use
\newcommand{\copyrightstatement}{
         \doclicenseIcon \ \theyear. 
         \doclicenseLongText            % includes a link
}

%%%% ENVIRONMENTS
% Environments
\AtBeginEnvironment{quote}{\footnotesize\vskip 1em}
\AtEndEnvironment{quote}{\vskip 1em}

\setkomafont{caption}{\footnotesize}

% Preface
\newenvironment{preface}{
    \newrefsection
    \phantomsection
    \cleardoublepage
    \addcontentsline{toc}{part}{\thecontribution}
    % enable running title
    \pagestyle{preface}
    % \chapter*{Editors' Preface}    
    % reset the section counter for each paper
    \setcounter{section}{0}  
    % no running title on first page, page number center bottom instead
    \thispagestyle{chaptertitlepage}
}{}
\AtEndEnvironment{preface}{%
    % safeguard section numbering
    \renewcommand{\thesubsection}{\thesection.\arabic{subsection}}  
    %last page running header fix
    \protect\thispagestyle{preface}
}
% Essays
\newenvironment{paper}{
    \newrefsection
    \phantomsection
    % start every new paper on an uneven page 
    \cleardoublepage
    % enable running title
    \pagestyle{fancy}
    % change section numbering FROM [\chapter].[\section].[\subsection] TO [\section].[\subsection] ETC.
    \renewcommand{\thesection}{\arabic{section}}
    % mark chapter % add author + title to the TOC
    \chapter[\normalfont\textbf{\emph{\thecontributor}}: \thecontribution]{\vspace{-4em}\Large\normalfont\thecontribution\linebreak\normalsize\begin{flushright}\emph{\thecontributor}\end{flushright}}    
    % reset the section counter for each paper
    \setcounter{section}{0}  
    % reset the figure counter for each paper
    \renewcommand\thefigure{\arabic{figure}}    
    % reset the table counter for each paper
    \renewcommand\thetable{\arabic{table}} 
    % no running title on first page, page number center bottom instead, include copyright statement
    \thispagestyle{contributiontitlepage}
    % formatting for the bibliography

}{}
\AtBeginEnvironment{paper}{
    % keeps running title from the first page:
    \renewcommand*{\pagemark}{}%                            
}
\AtEndEnvironment{paper}{
    % safeguard section numbering
    \renewcommand{\thesubsection}{\thesection.\arabic{subsection}}  
    % last page running header fix
    \protect\thispagestyle{fancy}%                              
}
% Reviews
\newenvironment{review}{
    \newrefsection
    \phantomsection
    % start every new paper on an uneven page 
    \cleardoublepage
    % enable running title
    \pagestyle{reviews}
    % change section numbering FROM [\chapter].[\section].[\subsection] TO [\section].[\subsection] ETC.
    \renewcommand{\thesection}{\arabic{section}} 
    % mark chapter % add author + title to the TOC
    \chapter[\normalfont\textbf{\emph{\thecontributor}}: \thecontribution]{}    % reset the section counter for each paper
    \setcounter{section}{0}  
    % no running title on first page, page number center bottom instead, include copyright statement
    \thispagestyle{contributiontitlepage}
    % formatting for the bibliography
}{}
\AtBeginEnvironment{review}{
% keeps running title from the first page
    \renewcommand*{\pagemark}{}%                                   
}
\AtEndEnvironment{review}{
    % author name(s)
    \begin{flushright}\emph{\thecontributor}\end{flushright}
    % safeguard section numbering
    \renewcommand{\thesubsection}{\thesection.\arabic{subsection}} 
    % last page running header fix
    \protect\thispagestyle{reviews}                           
}

% Abstract
\newenvironment{abstract}{% 
\setlength{\parindent}{0pt} \begin{adjustwidth}{2em}{2em}\footnotesize\emph{\abstractname}: }{%
\vskip 1em\end{adjustwidth}
}{}

% Keywords
\newenvironment{keywords}{
\setlength{\parindent}{0pt} \begin{adjustwidth}{2em}{2em}\footnotesize\emph{Keywords}: }{%
\vskip 1em\end{adjustwidth}
}{}

% Review Abstract
\newenvironment{reviewed}{% 
\setlength{\parindent}{0pt}
    \begin{adjustwidth}{2em}{2em}\footnotesize}{%
\vskip 1em\end{adjustwidth}
}{}

% Motto
\newenvironment{motto}{% 
\setlength{\parindent}{0pt} \small\raggedleft}{%
\vskip 2em
}{}

% Example
\newcounter{example}[chapter]
\newenvironment{example}[1][]{\refstepcounter{example}\begin{quote} \rmfamily}{\begin{flushright}(Example~\theexample)\end{flushright}\end{quote}}

%%%% SECTIONOPTIONS

% command for centering section headings
\newcommand{\centerheading}[1]{   
    \hspace*{\fill}#1\hspace*{\fill}
}

% Remove "Part #." from \part titles
% KOMA default: \newcommand*{\partformat}{\partname~\thepart\autodot}
\renewcommand*{\partformat}{} 

% No dots after figure or table numbers
\renewcommand*{\figureformat}{\figurename~\thefigure}
\renewcommand*{\tableformat}{\tablename~\thetable}

% paragraph handling
\setparsizes%
    {1em}% indent
    {0pt}% maximum space between paragraphs
    {0pt plus 1fil}% last line not justified
    

% In the "Authors" section, author names are put in the \paragraph{} headings. To reduce the space after these  headings, the default {-1em} has been changed to {-.4em} below.
\makeatletter
\renewcommand\paragraph{\@startsection {paragraph}{4}{\z@ }{3.25ex \@plus 1ex \@minus .2ex}{-.4em}{\normalfont \normalsize \bfseries }
}
\makeatother

% add the following (uncommented) in environments where you want to count paragraph numbers in the margin
%    \renewcommand*{\paragraphformat}{%
%    \makebox[-4pt][r]{\footnotesize\theparagraph\autodot\enskip}
%    }
%    \renewcommand{\theparagraph}{\arabic{paragraph}}
%    \setcounter{paragraph}{0}
%    \setcounter{secnumdepth}{4}
    
%%%% HEADERFOOTER

% running title
\RequirePackage{fancyhdr}
% cuts off running titles that are too long
%\RequirePackage{truncate}
% makes header as wide as geometry (SET SAME AS \TEXTWIDTH!)
\setlength{\headwidth}{109mm} 
% LO = Left Odd
\fancyhead[LO]{\small\emph{\theshortcontributor} \hspace*{.5em} \theshortcontribution} 
% RE = Right Even
\fancyhead[RE]{\scshape{\small\theissue}}
% LE = Left Even
\fancyhead[LE]{\small\thepage}            
% RE = Right Odd
\fancyhead[RO]{\small\thepage}    
\fancyfoot{}
% no line under running title; cannot be \@z but needs to be 0pt
\renewcommand{\headrulewidth}{0 pt} 

% special style for authors pages
\fancypagestyle{authors}{
    \fancyhead[LO]{\small\textit{Authors}} 
    \fancyhead[LE]{\small\thepage}            
    \fancyhead[RE]{\scshape{\small\theissue}}
    \fancyhead[RO]{\small\thepage}            
    \fancyfoot{}
}

% special style for book reviews
\fancypagestyle{reviews}{
    \fancyhead[LO]{\small\textit{Book Reviews}} 
    \fancyhead[LE]{\small\thepage}            
    \fancyhead[RE]{\scshape{\small\theissue}}
    \fancyhead[RO]{\small\thepage}            
    \fancyfoot{}
}

% special style for Editors' preface.
\fancypagestyle{preface}{
    \fancyhead[LO]{\small\textit{\theshortcontribution}} 
    \fancyhead[LE]{\small\thepage}            
    \fancyhead[RE]{\scshape{\small\theissue}}
    \fancyhead[RO]{\small\thepage}            
    \fancyfoot{}
}
% special style for first pages of contributions etc.
% DOES include copyright statement
\fancypagestyle{contributiontitlepage}{
    \fancyhead[C]{\scriptsize\centering\copyrightstatement}
    \fancyhead[L,R]{}
    \fancyfoot[CE,CO]{\small\thepage}
}
% special style for first pages of other \chapters.
% DOES NOT include copyright statement
\fancypagestyle{chaptertitlepage}{
    \fancyhead[C,L,R]{}
    \fancyfoot[CE,CO]{\small\thepage}
}
% no page numbers on \part pages 
\renewcommand*{\partpagestyle}{empty}

%%%% FOOTNOTEFORMAT

% footnotes
\renewcommand{\footnoterule}{%
    \kern .5em  % call this kerna
    \hrule height 0.4pt width .2\columnwidth    % the .2 value made the footnote ruler (horizontal line) smaller (was at .4)
    \kern .5em % call this kernb
}
\usepackage{footmisc}               
\renewcommand{\footnotelayout}{
    \hspace{1.5em}    % space between footnote mark and footnote text
}    
\newcommand{\mytodo}[1]{\textcolor{red}{#1}}

%%%% CODESNIPPETS

% colours for code notations
\usepackage{listings}       
	\renewcommand\lstlistingname{Quelltext} 
	\lstset{                    % basic formatting (bash etc.)
	       basicstyle=\ttfamily,
 	       showstringspaces=false,
	       commentstyle=\color{BrickRed},
	       keywordstyle=\color{RoyalBlue}
	}
	\lstdefinelanguage{XML}{     % specific XML formatting overrides
		  basicstyle=\ttfamily,
		  morestring=[s]{"}{"},
		  morecomment=[s]{?}{?},
		  morecomment=[s]{!--}{--},
		  commentstyle=\color{OliveGreen},
		  moredelim=[s][\color{Black}]{>}{<},
		  moredelim=[s][\color{RawSienna}]{\ }{=},
		  stringstyle=\color{RoyalBlue},
 		  identifierstyle=\color{Plum}
	}
    % HOW TO USE? BASH EXAMPLE
    %   \begin{lstlisting}[language=bash]
    %   #some comment
    %   cd Documents
    %   \end{lstlisting}

\author{Aline Deicke}

\title{Elena Spadini, Francesca Tomasi and Georg Vogeler, eds., \emph{Graph Data-Models and Semantic Web Technologies in Scholarly Digital Editing}}

\begin{document}
\maketitle

%%%%%%%%%%%%%%%%%%%%%%%%%%%%%%%%%%%%%%
%% DESCRIPTION OF THE REVIEWED BOOK %%
%%%%%%%%%%%%%%%%%%%%%%%%%%%%%%%%%%%%%%


\begin{reviewed}
  Review of Elena Spadini, Francesca Tomasi and Georg Vogeler, eds., \emph{Graph Data-Models and Semantic Web Technologies in Scholarly Digital Editing}. (Band 15 of \textit{Schriften des Instituts für Dokumentologie und Editorik}). Norderstedt: Books on Demand, 2021, <\href{https://kups.ub.uni-koeln.de/54577}{https://kups.ub.uni-koeln.de/54577}>. [Accessed 30 June 2023]. VI–214 pp. ISBN: 978–3–7543–4369–2.
  \end{reviewed}
  
  
%%%%%%%%%%%%%%%%%%%%%%%%%%%%%
%% YOUR REVIEW STARTS HERE %%
%%%%%%%%%%%%%%%%%%%%%%%%%%%%%

% remove asterisk (*) if you want to number your sections
% add a title for your section in between the {curly brackets} if you need one
\section*{} 
As early as 2001, Tim Berners-Lee, together with James Hendler and Ora
Lassila, formulated their idea of a semantic web as an ``extension of the
current {[}internet{]}, in which information is given well-defined
meaning, better enabling computers and people to work in cooperation'' \parencite{berners-lee_semantic_2001}. In their vision, concepts from a variety of
what they termed ``subcultures'' would be linked, enabling relations
between entities which are not only machine \emph{readable}, but
\emph{understandable} and \emph{interpretable}. This prospect of a ``web
of knowledge'' that facilitates and encourages new and innovative ways
of computational analysis appeals particularly to Digital Humanists. Be
it graph databases and the corresponding data models, the Linked Open
Data Cloud of the semantic web, or network analysis as a specific subset
of graph theory, the idea of putting connections --- between texts,
between actors, between objects and concepts --- at the core of
investigation becomes more and more influential in an environment where,
increasingly, ``{[}collective life{]} is mediated by network
technologies'' and logic \parencite[513]{venturini_actor-network_2019}.

As such, graphs have also reached the field of scholarly digital
editions as one of the main research disciplines of the Digital
Humanities, which has been shaping the field for decades. Traditionally,
one crucial influence on this field have been the guidelines developed
by the Text Encoding Initiative (TEI) since its foundation in 1987 \parencite{tei_consortium_tei_2007}. Though the
TEI aims to develop ``hardware- and software-independent methods for
encoding humanities data'' \parencite{tei_consortium_history_2023}, for most of the Digital Humanities
community, its standard has become synonymous with XML and the so-called
\mbox{``X-technologies''} (XSLT, XPath and eXist). While the advantages of
such an established, widely used and continuously developed standard to
the community of digital scholarly editors can hardly be overstated, the
expansion of digital ecosystems and infrastructures as well as interests
in sharing and interlinking data, in extracting semantic relationships
from texts, or in modelling hierarchies, events and processes beyond the
document-centric representations XML encourages have led to the
exploration of new research directions in representing \mbox{textual
information.}

It is in this tension at the intersection of TEI/XML, stand-off
properties and graph technologies that the collection \emph{Graph
Data-Models and Semantic Web Technologies in Scholarly Digital Editing}
is situated. According to the editors, it aims to explore ``possible
interactions between digital texts, the graph data-model, scholarly
editions, and the semantic web'' (1). As such, for the first time, it
provides a comprehensive overview over the state of graph-based
approaches in scholarly digital editing. Emerging from a workshop at the
University of Lausanne in June 2019, it includes ten contributions by
twenty-five authors on a variety of topics, technologies, and
disciplines connected to the wider field of editorial work in the
digital domain, mostly focusing on one of two avenues: text modelled as
a graph, or the representation of textual information through semantic
web technologies; and the extension and enriching of the TEI/XML
standard through the integration of Linked Open Data and RDF.

For context, it has to be noted that the editors --- Elena Spadini,
Francesca Tomasi and Georg Vogeler --- have to be seen as uniquely
qualified to head up such an effort due to their extensive contributions
to the field which are too numerous to list in this limited space. In
addition, the book has been released in the publication series of the
Institute for Documentology and Scholarly Editing (\citeyear{institute_of_documentary_and_scholarly_editing_schriftenreihe_2023}), which has significantly advanced
scholarly editorial research in Germany and beyond for decades.

The collection begins with a concise introductory chapter outlining the
status quo of digital scholarly editing, the gaps and challenges to the
field, and the resulting desiderata addressed here. The editors also
give insights into their own research and motivations, followed by the
usual description of the selected contributions.

Subsequently, the collection is structured into three sections focusing
on ``Infrastructures and Technologies'', ``Formal Models'', and
``Projects and Editions'', though naturally, several chapters could have
appeared in either of these categories.

With five chapters, the first section on ``Infrastructures and
Technologies'' takes up the largest space. Here, it becomes evident how
much groundwork is still necessary for the humanities to be able to
effectively apply graph and semantic web technologies. From Peter Boot
and Marijn Koolen, describing several models of connecting RDF content
to a TEI/XML document, to Hugh Cayless and Matteo Romanello, advocating
for a resolution service for Text URIs, to Iian Neill and Desmond
Schmidt as well as Miller C. Prosser and Sandra R. Schloen, proposing
different editors and their underlying data models as graph-based tool
solutions to working with textual data, the chapters present existing
work but also gaps in the current technologies and services that need to
be addressed to further develop a semantic web of textual data, and to
incorporate these approaches into the realm of TEI/XML. These rather
technical considerations are contextualized by Georg Vogeler's
reflections on the epistemological consequences of the metaphors central
to the technologies discussed, evoking important discussions about
path-dependencies and the influence of widely adopted technology stacks
on the emergence of data and knowledge cultures 
(\cite{mahoney_historical_2006}; see also \cite{acker_data_2019}).

The following section, ``Formal Models'', takes a step back to focus on
the data models and ontologies forming the basis of the semantic web and
graph-based editorial work. Hans Cools and Roberta Padlina give a
comprehensive overview of ``the implementation and development of
Semantic Web technology (SWT) within the ongoing project \emph{National
Infrastructure for Editions}'' that should be a required reading for
every scholar aiming to familiarize themselves with semantic web
technologies and ontology engineering in the Arts and Humanities.
Francesca Giovannetti's chapter leads the reader back to the question of
how to integrate the TEI standard and graph-based technologies,
proposing a data model to transfer a TEI module, in this case the
critical apparatus, and its inherent logic into a serialization as a RDF
knowledge graph to complement data marked up in XML.

In the final section on ``Projects and Editions'', three use cases from
widely different fields demonstrate the potentials and pitfalls of the
approaches and models described so far. Burrows and co-authors, in a
process similar to Giovannetti's proposed model, outline necessary
prerequisites and challenges in transferring a module of the TEI into a
graph structure. Münnich and Ahrend add the perspective of digital
musicology to the collection, discussing graph-based approaches to the
study of philological and music historical processes. Interestingly,
they also include interpretative processes of scholarly argumentation
expressed through the CRMinf model \parencite{stead_crminf_2015} into their
considerations. Last but certainly not least, Sippl, Burghardt and Wolff
demonstrate in detail the challenges of applying graph-based methods to
a ``legacy'' project, integrating different data formats in different
states of digitization and enhancing these digital sources with methods
of natural language processes.

Throughout the collection, it is noticeable that the demand for a more
profound engagement with what Burrows and co-authors call the
``relationship between TEI [\ldots] and the world of Linked Data,
RDF, and ontologies'' (145) is motivated by a number of recurringly
mentioned challenges or dissatisfactions with the current state of
TEI/XML-based editorial practices.\footnote{For a discussion of a
  broader range of what he calls ``myths and misconceptions'' about the
  TEI, some of them also addressed in this collection, see also \cite{cummings_world_2019}.} Among those mentioned most frequently are the difficulties of
dealing with overlapping annotations in XML; the need to represent
textual works or information from multiple perspectives, be it several
variants of a text as a whole, or individual elements adhering to
various, potentially overlapping hierarchies; the potential of
connecting editions with the greater context of the semantic web through
the use of LOD and RDF; and the promise of introducing
machine-interpretable semantics into textual data through ontologies to
enable, for example, the analysis and querying of what Sippl and
co-authors call ``higher-level relationships'' (198). As outlined above,
the chapters indeed provide a plethora of solutions and approaches to
these challenges, in the process identifying further research and
development needs but also potentials of semantic scholarly digital
editions.

In doing so, it has to be noted that, thankfully, neither the editors
nor the authors employ the fallacy of pitting XML and its technology
stack against graph technologies; instead, as Georg Vogeler puts it,
``The debate about the best technology stack {[}moves{]} towards a
debate on the best method for a given combination'' of tasks (88).
Several authors explicitly situate their contributions as complements to
the current use of the TEI guidelines, filling gaps and extending the
potentials of XML-encoded text editions --- for example, Münnich and
Ahrend, who see RDF as ``another complementary perspective {[}that{]}
adds a level of differentiation that goes beyond the expressiveness of
XML'' (173). Still, in many aspects, the collection definitively leaves
the document-centric realm of the TEI behind, opening up exciting new
possibilities for editorial work as a connected, interlinked task
creating and completing a web of knowledge in the best sense of Tim
Berners-Lee's intentions.

Published in 2021, it has been a few years since this collection was
made available. In the meantime, work in the addressed research fields
has certainly continued and advanced --- thanks to the contributing or
cited authors, among others.\footnote{Representative of current
  tendencies in graph technologies in the humanities is the programme of
  GrapHNR 2023: a joint conference of the Historical Network Research
  community and Graphs \& Networks in the Humanities (\citeyear{historical_network_research_and_graphs__networks_in_the_humanities_graphnr_2023}).} Yet, even in the fast-paced discipline
of Digital Humanities, the collection still stands as a landmark in the
exploration of graph-based digital scholarly editions. Many of the
addressed gaps and potentials remain open questions, waiting to be
answered. In the quest to address them, this collection will serve as a
guide and a road map for many more years to come.

\vspace{4em}

\begin{flushleft}
    % use smallcaps for author names
    \renewcommand*{\mkbibnamefamily}[1]{\textsc{#1}}
    \renewcommand*{\mkbibnamegiven}[1]{\textsc{#1}} 
\printbibliography
\end{flushleft}

\end{document}