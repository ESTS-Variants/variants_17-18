%%%%%%%%%%%%%%
%% METADATA %%
%%%%%%%%%%%%%%

\contributor{
% your name(s)
Aline Deicke}

\contribution{Elena Spadini, Francesca Tomasi and Georg Vogeler, eds. \emph{Graph Data-Models and Semantic Web Technologies in Scholarly Digital Editing}}


\begin{review}
\renewcommand*{\pagemark}{}

%%%%%%%%%%%%%%%%%%%%%%%%%%%%%%%%%%%%%%
%% DESCRIPTION OF THE REVIEWED BOOK %%
%%%%%%%%%%%%%%%%%%%%%%%%%%%%%%%%%%%%%%


\begin{reviewed}
Review of \thecontribution. (Band 15 of \textit{Schriften des Instituts für Dokumentologie und Editorik}). Norderstedt: Books on Demand, 2021, <\href{https://kups.ub.uni-koeln.de/54577}{https://kups.ub.uni-koeln.de/54577}>. [Accessed 30 June 2023]. VI–214 pp. ISBN: 978–3–7543–4369–2.
\end{reviewed}


%%%%%%%%%%%%%%%%%%%%%%%%%%%%%
%% YOUR REVIEW STARTS HERE %%
%%%%%%%%%%%%%%%%%%%%%%%%%%%%%

% remove asterisk (*) if you want to number your sections
% add a title for your section in between the {curly brackets} if you need one
\section*{} 
As early as 2001, Tim Berners-Lee, together with James Hendler and Ora
Lassila, formulated their idea of a semantic web as an ``extension of the
current {[}internet{]}, in which information is given well-defined
meaning, better enabling computers and people to work in cooperation'' \parencite{berners-lee_semantic_2001}. In their vision, concepts from a variety of
what they termed ``subcultures'' would be linked, enabling relations
between entities which are not only machine \emph{readable}, but
\emph{understandable} and \emph{interpretable}. This prospect of a ``web
of knowledge'' that facilitates and encourages new and innovative ways
of computational analysis appeals particularly to Digital Humanists. Be
it graph databases and the corresponding data models, the Linked Open
Data Cloud of the semantic web, or network analysis as a specific subset
of graph theory, the idea of putting connections --- between texts,
between actors, between objects and concepts --- at the core of
investigation becomes more and more influential in an environment where,
increasingly, ``{[}collective life{]} is mediated by network
technologies'' and logic \parencite[513]{venturini_actor-network_2019}.

As such, graphs have also reached the field of scholarly digital
editions as one of the main research disciplines of the Digital
Humanities, which has been shaping the field for decades. Traditionally,
one crucial influence on this field have been the guidelines developed
by the Text Encoding Initiative (TEI) since its foundation in 1987 \parencite{tei_consortium_tei_2007}. Though the
TEI aims to develop ``hardware- and software-independent methods for
encoding humanities data'' \parencite{tei_consortium_history_2023}, for most of the Digital Humanities
community, its standard has become synonymous with XML and the so-called
\mbox{``X-technologies''} (XSLT, XPath and eXist). While the advantages of
such an established, widely used and continuously developed standard to
the community of digital scholarly editors can hardly be overstated, the
expansion of digital ecosystems and infrastructures as well as interests
in sharing and interlinking data, in extracting semantic relationships
from texts, or in modelling hierarchies, events and processes beyond the
document-centric representations XML encourages have led to the
exploration of new research directions in representing \mbox{textual
information.}

It is in this tension at the intersection of TEI/XML, stand-off
properties and graph technologies that the collection \emph{Graph
Data-Models and Semantic Web Technologies in Scholarly Digital Editing}
is situated. According to the editors, it aims to explore ``possible
interactions between digital texts, the graph data-model, scholarly
editions, and the semantic web'' (1). As such, for the first time, it
provides a comprehensive overview over the state of graph-based
approaches in scholarly digital editing. Emerging from a workshop at the
University of Lausanne in June 2019, it includes ten contributions by
twenty-five authors on a variety of topics, technologies, and
disciplines connected to the wider field of editorial work in the
digital domain, mostly focusing on one of two avenues: text modelled as
a graph, or the representation of textual information through semantic
web technologies; and the extension and enriching of the TEI/XML
standard through the integration of Linked Open Data and RDF.

For context, it has to be noted that the editors --- Elena Spadini,
Francesca Tomasi and Georg Vogeler --- have to be seen as uniquely
qualified to head up such an effort due to their extensive contributions
to the field which are too numerous to list in this limited space. In
addition, the book has been released in the publication series of the
Institute for Documentology and Scholarly Editing (\citeyear{institute_of_documentary_and_scholarly_editing_schriftenreihe_2023}), which has significantly advanced
scholarly editorial research in Germany and beyond for decades.

The collection begins with a concise introductory chapter outlining the
status quo of digital scholarly editing, the gaps and challenges to the
field, and the resulting desiderata addressed here. The editors also
give insights into their own research and motivations, followed by the
usual description of the selected contributions.

Subsequently, the collection is structured into three sections focusing
on ``Infrastructures and Technologies'', ``Formal Models'', and
``Projects and Editions'', though naturally, several chapters could have
appeared in either of these categories.

With five chapters, the first section on ``Infrastructures and
Technologies'' takes up the largest space. Here, it becomes evident how
much groundwork is still necessary for the humanities to be able to
effectively apply graph and semantic web technologies. From Peter Boot
and Marijn Koolen, describing several models of connecting RDF content
to a TEI/XML document, to Hugh Cayless and Matteo Romanello, advocating
for a resolution service for Text URIs, to Iian Neill and Desmond
Schmidt as well as Miller C. Prosser and Sandra R. Schloen, proposing
different editors and their underlying data models as graph-based tool
solutions to working with textual data, the chapters present existing
work but also gaps in the current technologies and services that need to
be addressed to further develop a semantic web of textual data, and to
incorporate these approaches into the realm of TEI/XML. These rather
technical considerations are contextualized by Georg Vogeler's
reflections on the epistemological consequences of the metaphors central
to the technologies discussed, evoking important discussions about
path-dependencies and the influence of widely adopted technology stacks
on the emergence of data and knowledge cultures 
(\cite{mahoney_historical_2006}; see also \cite{acker_data_2019}).

The following section, ``Formal Models'', takes a step back to focus on
the data models and ontologies forming the basis of the semantic web and
graph-based editorial work. Hans Cools and Roberta Padlina give a
comprehensive overview of ``the implementation and development of
Semantic Web technology (SWT) within the ongoing project \emph{National
Infrastructure for Editions}'' that should be a required reading for
every scholar aiming to familiarize themselves with semantic web
technologies and ontology engineering in the Arts and Humanities.
Francesca Giovannetti's chapter leads the reader back to the question of
how to integrate the TEI standard and graph-based technologies,
proposing a data model to transfer a TEI module, in this case the
critical apparatus, and its inherent logic into a serialization as a RDF
knowledge graph to complement data marked up in XML.

In the final section on ``Projects and Editions'', three use cases from
widely different fields demonstrate the potentials and pitfalls of the
approaches and models described so far. Burrows and co-authors, in a
process similar to Giovannetti's proposed model, outline necessary
prerequisites and challenges in transferring a module of the TEI into a
graph structure. Münnich and Ahrend add the perspective of digital
musicology to the collection, discussing graph-based approaches to the
study of philological and music historical processes. Interestingly,
they also include interpretative processes of scholarly argumentation
expressed through the CRMinf model \parencite{stead_crminf_2015} into their
considerations. Last but certainly not least, Sippl, Burghardt and Wolff
demonstrate in detail the challenges of applying graph-based methods to
a ``legacy'' project, integrating different data formats in different
states of digitization and enhancing these digital sources with methods
of natural language processes.

Throughout the collection, it is noticeable that the demand for a more
profound engagement with what Burrows and co-authors call the
``relationship between TEI [\ldots] and the world of Linked Data,
RDF, and ontologies'' (145) is motivated by a number of recurringly
mentioned challenges or dissatisfactions with the current state of
TEI/XML-based editorial practices.\footnote{For a discussion of a
  broader range of what he calls ``myths and misconceptions'' about the
  TEI, some of them also addressed in this collection, see also \cite{cummings_world_2019}.} Among those mentioned most frequently are the difficulties of
dealing with overlapping annotations in XML; the need to represent
textual works or information from multiple perspectives, be it several
variants of a text as a whole, or individual elements adhering to
various, potentially overlapping hierarchies; the potential of
connecting editions with the greater context of the semantic web through
the use of LOD and RDF; and the promise of introducing
machine-interpretable semantics into textual data through ontologies to
enable, for example, the analysis and querying of what Sippl and
co-authors call ``higher-level relationships'' (198). As outlined above,
the chapters indeed provide a plethora of solutions and approaches to
these challenges, in the process identifying further research and
development needs but also potentials of semantic scholarly digital
editions.

In doing so, it has to be noted that, thankfully, neither the editors
nor the authors employ the fallacy of pitting XML and its technology
stack against graph technologies; instead, as Georg Vogeler puts it,
``The debate about the best technology stack {[}moves{]} towards a
debate on the best method for a given combination'' of tasks (88).
Several authors explicitly situate their contributions as complements to
the current use of the TEI guidelines, filling gaps and extending the
potentials of XML-encoded text editions --- for example, Münnich and
Ahrend, who see RDF as ``another complementary perspective {[}that{]}
adds a level of differentiation that goes beyond the expressiveness of
XML'' (173). Still, in many aspects, the collection definitively leaves
the document-centric realm of the TEI behind, opening up exciting new
possibilities for editorial work as a connected, interlinked task
creating and completing a web of knowledge in the best sense of Tim
Berners-Lee's intentions.

Published in 2021, it has been a few years since this collection was
made available. In the meantime, work in the addressed research fields
has certainly continued and advanced --- thanks to the contributing or
cited authors, among others.\footnote{Representative of current
  tendencies in graph technologies in the humanities is the programme of
  GrapHNR 2023: a joint conference of the Historical Network Research
  community and Graphs \& Networks in the Humanities (\citeyear{historical_network_research_and_graphs__networks_in_the_humanities_graphnr_2023}).} Yet, even in the fast-paced discipline
of Digital Humanities, the collection still stands as a landmark in the
exploration of graph-based digital scholarly editions. Many of the
addressed gaps and potentials remain open questions, waiting to be
answered. In the quest to address them, this collection will serve as a
guide and a road map for many more years to come.

\vspace{4em}

\begin{flushleft}
    % use smallcaps for author names
    \renewcommand*{\mkbibnamefamily}[1]{\textsc{#1}}
    \renewcommand*{\mkbibnamegiven}[1]{\textsc{#1}} 
\printbibliography
\end{flushleft}

\end{review}