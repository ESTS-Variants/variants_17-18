\documentclass{article}
\include{variantex}

\author{Beci Carver}

\title{W. H. Auden, \emph{Poems}. Ed. Edward Mendelson}

\begin{document}
\maketitle

%%%%%%%%%%%%%%%%%%%%%%%%%%%%%%%%%%%%%%
%% DESCRIPTION OF THE REVIEWED BOOK %%
%%%%%%%%%%%%%%%%%%%%%%%%%%%%%%%%%%%%%%

\begin{reviewed}
Review of W. H. Auden, \emph{Poems}. Ed. Edward Mendelson. (\textit{The Complete Works of W. H. Auden. \emph{Ed. Edward Mendelson}}). 2 vols. Princeton, Nj.: Princeton University Press, 2022. 848 pp. (vol. 1) and 1,120 pp. (vol. 2). ISBNs: 978–0–691–21929–5 (vol. 1) and 978–0–691–21930–1 (vol. 2).
\end{reviewed}

%%%%%%%%%%%%%%%%%%%%%%%%%%%%%
%% YOUR REVIEW STARTS HERE %%
%%%%%%%%%%%%%%%%%%%%%%%%%%%%%

% remove asterisk (*) if you want to number your sections
% add a title for your section in between the {curly brackets} if you need one
\section*{} 
Auden's favourite colour? Blue: the colour of ``boys' overalls'' (1:
25). His favourite sight? The ``bosom, backside, crotch'' of a man on a
beach mid-saunter, viewed from a supine position (2: 361). The Auden
that emerges in Edward Mendelson's majestic new \emph{Poems} is
profusely, volubly, magnificently queer. He is even more so if we
consider that, as Mendelson notes, ``when Auden marked up friends'
copies of his early books with dates and places of composition, he often
wrote in the initials of the subjects of his love poems'' \parencite[43]{mendelson_early_2017}. Working closely with Christopher Isherwood and other friends'
editions of the texts, which carry a wealth of unique ``minor
corrections'' (2: front matter), we may imagine Mendelson afloat in a
fully amorous discourse. But even for us, the expanded Auden must appear
remarkably open about who and how he loved.

This queer openness is at once consistent with the ethos we have long
associated with Auden's circle and at odds with the legacy of his
Victorian and Edwardian writer-predecessors. The gay
writers he invokes in his work were uniformly driven to the closet.
Wilfred Owen, or ``Wilfred'' as Auden fondly names the author of the
doctrine ``The poetry is in the pity'' (1: 205), was outed only after
his death. E. M. Forster, whose mantra ``Only connect'' \parencite[195]{forster_howards_1991} Auden championed, did not come out in print (as it were) till his
last novel. A. E. Housman ``kept tears like dirty postcards in a
drawer'' while writing hundreds of elegies for beloved killed
soldiers (1: 326). Edward Lear ``wept to himself in the night'', until,
on being ``pushed'' by libidinal desperation to write, he fled from his
misery into a fantasyland: ``children swarmed to him like settlers. He
became a land'' (1: 326--27). There is of course exaggeration in these
squashed biographies of Auden's, but there is truth as well, and within
that truth a worrying verdict on the prospects of gay self-acceptance.
This verdict was, it is clear, horrific to Auden, his own intermittent
discretion being prompted not by shame but by a felt entitlement to do
as he pleased in private. His deepest wish on behalf of modern
civilisation was that it should register his right to a privacy whose
contents were his own business: ``May this for which we dread to lose /
Our privacy, need no excuse / But to that strength belong'' (1: 186).

In \emph{Poems}, Auden's effort to resist a historical norm of queer
self-rebuke stands out as a manifest mission, revealing his commitment
throughout his career to creating what he describes in \emph{The Double
Man} as an alternative ``civitas'' (2: 10). Where in Lear, it is only
children who may reach the nonsensical-queer New World, in Auden's Long
Island in the sun, where bronzed gay men listen to Dietrich Buxtehude's
organ arrangements on ``the very morning that the war / Took action on
the Polish floor'', a ``civitas of sound'' has the ability to offset
``crystallize{[}d{]} hostilities'':

\begin{quote}
For art had set in order sense

And feeling and intelligence,

And from its ideal order grew

Our local understanding too

\begin{flushright}
(2: 10)
\end{flushright}
\end{quote}

\noindent Auden is riffing here on Hardy's line about the Titanic: ``In shadowy
silent distance grew the Iceberg too'' \parencite[248]{hardy_complete_1976}. But it is the
global war that now, in 1941, mounts in the background a threatening
crystal, while Auden's own beach life is somewhere else, somewhere
safer. In this other-land, made up both of the ``sound'' of Danish
\emph{passacaglias} and of Auden's verse, there is scope to think more freely
about what it means to be sexually oneself.

Literary history has traditionally split Auden in two, categorising his
early poems as cryptic, Marxist and preoccupied with promiscuous
romance, and the poems he wrote after leaving England as more
accessible, more politically conservative and more single-mindedly devoted to
Chester Kallman. This cutting in half may be blamed to an extent on
Auden's contemporaries. Randall Jarrell influentially found in Auden's
first American publication, \emph{Another Time} (\citeyear{auden_another_1940}), a fateful
neatening of his method. Evelyn Waugh in \emph{Put Out More Flags} (\citeyear{waugh_put_1942})
reimagined him and Isherwood as Parsnip and Pimpernel: a camp double act
whose sheer effeminacy leads them to leave England on the eve of war.
And George Orwell claimed to sniff out in Auden's leftism a bourgeois
complacency. Yet Jarrell, Waugh and Orwell were clearly conscious in
launching these attacks of addressing not just the poet himself but all queers. They were opening fire on a populace, albeit one composed
of a constitutively fragile minority. Mendelson tells us that
Auden's strongest incentive for cleaving to the left in the 1930s was a
social one: ``He [\dots] hope{[}d{]} he could find in politics an
escape from his personal isolation'' (1: xiv). His long friendship with
Isherwood began with an entanglement of queer and Marxist sympathies;
and it was Isherwood who became his point of entry into the larger
community of the British left. Moreover, if we read Auden's poems
\emph{en masse} as the soundtrack or ``civitas of sound'' for a growing
queer self-awareness, his dissent from Marx as the grisly costs of
economic justice revealed themselves in Spain, may seem like less of a
U-turn. The essence of ``poetry'' was for him, as ``Wilfred'' had said,
``pity''; and what greater provocation to pity could there be than the spectacle of his friends' sacrificed lives? Read as one lump, Auden's oeuvre is also more
even in quality than we have been led to assume. There are more dud
poems among the early work, more diamonds in the reputedly over-neat
American half. And what finally asserts itself is not discontinuity but
a continuity of investment in telling the truth about love.

Auden's account of love is also substantially thickened by Mendelson's meaty volumes. In the 1920s and 1930s, Auden evolved an
alternative amorous philosophy to that of conventional straight romance:
one that privileged freedom, choice, play, movement. Queer love for him
was a discovery as opposed to a destiny: an empirical matter of
``learning'' to notice when ``eyes'' ``return / My glances every day''
(1: 185) and of knowing when the game was over or up. In a poem of 1936,
Auden ends with a discovery that ends a relationship:

\begin{quote}
Oh but what worm of guilt

Or what malignant doubt

Am I the victim of;

That you then, unabashed,

Did what I never wished

Confessed another love;

And I, submissive, felt

Unwanted and went out?

\begin{flushright}
    (1: 213)
\end{flushright}
\end{quote}

\noindent Love in this stanza lives or dies on the strength of a sensation of
being ``wanted'', while the exit cue ``went out'' is already active in
the rhyme-sound by the time the terminus comes. Mendelson's helpful
notes on this poem also inform us that the question mark after ``went
out'' remained uncertain throughout the poem's life in print, as though
Auden were in two minds as to whether to keep wondering. Yet
significantly, on reading proofs for a new poetry collection in 1973,
the year he would die, Auden deleted the mark and, Mendelson continues,
``said (mistakenly) that it had never been present'' (1: 660). This
final insistence seems to me metonymic of the larger division in his
thinking between the era before and after Kallman, or put
chronologically, before and after the two men's first meeting in 1939.
It followed over three decades of doubt about his American boyfriend,
who neither quite ``wanted'' him nor ever definitively ``went out''.
Auden's later work is obsessed with the irony of a love that was firm on
Auden's side, and on Kallman's, like the queer attachments of Auden's
youth: ``water / Running away in the dark'' (1: 36).

In the early poems, vows are easy for Auden –– he quips: ``as easy as a
vow'' (1: 206) –– but later, they are pleas more than vows, binding the
other to nothing but honouring the integrity of his own freedom. They
make a down payment of Auden's complete devotion in return for
``voluntary love'' (1: 213). But the alternative is meaninglessly transactional:
``it cannot be, / As I think, of such consequence to want, / What anyone
is given, if they want'' (2: 245). The risks of reliance on the other's choice are brought out more clearly in the later poems. In ``Homage to Clio'', Auden's
Prospero reflects in an address to his former slave, Clio, who may leave
any time: ``It is only youth that still believes / It will get away with
anything, while age / Knows only too well that it has got away with
nothing'' (2: 130). But this all-or-nothing gamble is still fair, in its way. Within Auden's phrase ``get away'' here is a recognition
not just that nothing is costless but that nothing may really be
possessed: to ``get away with something'' is a little theft, gleefully
pocketed though it melts. Getting away with anything is how queer love
flourishes in Auden, and getting away with nothing is how it suffers. It
needs to go back and forth to remain in health.


\begin{flushleft}
    % use smallcaps for author names
    \renewcommand*{\mkbibnamefamily}[1]{\textsc{#1}}
    \renewcommand*{\mkbibnamegiven}[1]{\textsc{#1}} 
\printbibliography
\end{flushleft}

\end{document}