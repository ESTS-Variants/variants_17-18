\documentclass{article}
\include{variantex}

\author{Peter M. W. Robinson}

\title{Philipp Roelli, ed., \emph{Handbook of Stemmatology: History, Methodology, Digital Approaches}}

\begin{document}
\maketitle

%%%%%%%%%%%%%%%%%%%%%%%%%%%%%%%%%%%%%%
%% DESCRIPTION OF THE REVIEWED BOOK %%
%%%%%%%%%%%%%%%%%%%%%%%%%%%%%%%%%%%%%%
\begin{reviewed}
Review of Philipp Roelli, ed., \emph{Handbook of Stemmatology: History, Methodology, Digital Approaches}. Berlin and Boston, Ma.: De
Gruyter, 2020, DOI: \href{https://doi.org/10.1515/9783110684384}{10.1515/9783110684384}. 694
pp. ISBN (hardback): 978--3--11--067417--0; e-ISBN (pdf):
978--3--11--068438--4.
\end{reviewed}

%%%%%%%%%%%%%%%%%%%%%%%%%%%%%
%% YOUR REVIEW STARTS HERE %%
%%%%%%%%%%%%%%%%%%%%%%%%%%%%%

% remove asterisk (*) if you want to number your sections
% add a title for your section in between the {curly brackets} if you need one

\section*{} 
This review is an essay (and a long essay) for two reasons.
Firstly, the single book published as a ``handbook'' by De Gruyter,
under the title \emph{Handbook of Stemmatology}, is actually two books,
and this review unpacks the history of how one book, with its own title,
editors and contents, came to be published with a different title and
editor, and explores the implications of these shifts. Secondly, despite
the ``handbook'' title, which promises a complete and up-to-date survey
of the subject, two significant areas of textual scholarship related to
stemmatology are almost completely excluded from the book. This review
essay attempts to remedy that omission.

The first paragraph of the book (1) offers its own account of
how the collection came to be. In the account provided, scholars
associated with the series of \emph{Studia stemmatologica} workshops,
initiated by three scholars from the University of Helsinki and held in
various locations across Europe \parencite{heikkila_studia_2010}, determined to
create an online lexicon of terminology under the title \emph{Parvum
lexicon stemmatologicum} (PLS). This reached a ``first final version''
in November 2015 and is now available at web addresses quoted in the
volume \parencite{roelli_parvum_2015}. In the same year, the contributors to the
PLS decided that they would write a series of essays dealing at greater
length with various aspects of stemmatology. According to the account
given by Phillipp Roelli, editor of the volume, in the same first
paragraph, at that time he ``volunteered to become the
editor-in-chief''. According to Roelli's account, the series of essays
became a ``handbook of stemmatology'' and was published as such by De
Gruyter in 2020. A reader might reasonably infer that Roelli became the
editor of the ``handbook of stemmatology'' in 2015 and that, from then
onwards, authors and editor worked together to create the collection
under review here.

However, this is not what happened. Firstly, Roelli may have
offered to become the ``editor-in-chief'' of this collection of essays
in 2015, but it appears that for several years his offer was not
accepted. One can find web references to the collection as jointly
edited by a varying group of scholars, always including Roelli ––
sometimes called ``chief editor'' and sometimes not. Secondly, all the
way from 2015 and as late as February 2019 \parencite{noauthor_terminology_2019}, the collection had the title \emph{An Introduction to
Stemmatology in the Digital Age}. It is only with the publication of the
collection that it was denominated as a ``handbook'' and not as an
``introduction''. This shift matters for two reasons. The presentation
of events in the first paragraph is strictly correct. But the omission
of parts of the narrative leads the reader to presume that something
happened which actually did not: Roelli did not become
``editor-in-chief'' of a projected ``handbook'' in 2015. This omission
prefigures more consequential omissions in the collection. Secondly, the
essays in the book were written (with a very few possible exceptions)
not for a ``handbook of stemmatology'' but for a collection titled
\emph{An Introduction to Stemmatology in the Digital Age}. There is no
acknowledgement anywhere in the book that almost all the essays it
contains were written for a collection which had a different purpose.

So many of the essays in the book were written for \emph{An
Introduction to Stemmatology in the Digital Age} that one may review
them as part of the collection as originally conceived. I take a
personal pleasure in the range of these essays. In the late 1990s, I
joined with Dick Van Vliet, then director of what was the Constantijn
Huygens Institute, based in Den Haag, to found the European Society for
Textual Scholarship (ESTS), whose journal is \emph{Variants}. Our
primary aim was to provide a venue where textual scholars of different
orientations could engage with each other. We had in mind the series of
debates between Peter Shillingsburg and Bodo Plachta about different
national models of editing, broadly conceived, at Society for Textual
Scholarship conferences and elsewhere through the 1990s, and David
Greetham's \emph{Scholarly Editing: A Guide to Research} (\citeyear{greetham_scholarly_1995}), which
offers a series of accounts of editing across the entire range of
textual scholarship, variously categorized by subject (the Bible, Greek
and Latin classics, Shakespeare) and by national school (Italian,
medieval Spanish, German and more). We thought that a European
counterpart to the North American-based Society for Textual Scholarship
would help textual scholars talk to each other. Hence ESTS was born and
hence the journal in which this review appears was initiated. Hence too
my delight in finding in this \emph{Introduction to Stemmatology in the
Digital Age}, within the covers of a ``handbook'', accounts of textual
scholarship across multiple domains. At the very least: many of the
essays in this ``introduction'' offer a much-needed update of the
multiple surveys contained in Greetham's 1995 \emph{Guide}, itself a
collection of essays by specialists in each area.

The first chapter of the book, following the ``Introduction'',
is titled ``Textual Traditions''. It is edited by Elisabet Göransson, a
distinguished Latinist at Lund University, and contains contributions by
Gerd Haverling, Sinéad O'Sullivan, Outi Merisalo, Iolanda Ventura and
Peter Stokes. The titles of the contributions give a clear sense of the
conception of the chapter as a whole: Haverling's title promises an
overview of ``Literacy and Literature'' all the way from antiquity;
O'Sullivan's ``Transmission of Texts'' a survey of how texts were
transmitted; Merisalo's ``Book Production and Collection'' an overview
of how books were made and preserved; Ventura's ``Textual Traditions and
Early Prints'' a focus on one significant area of textual production;
Stokes' ``Palaeography, Codicology and Stemmatology'' an account of
three fundamental tools used by scholars of all these texts, all in
under fifty pages. Not everything can be covered in so little space. In
fact, four of the five essays (the exception is Stokes) focus almost
exclusively on Latin texts. Of course, Latin texts are important: but to
say almost nothing of the mass of vernacular texts (and indeed Biblical
texts) produced in Europe in the period, and nothing at all of texts
outside Europe (apart from a single mention of Vedic texts by Haverling;
11), seems an omission. Further, and most importantly, the essays have
almost nothing to say of stemmatology. The Stokes essay includes
``stemmatology'' in its title, gives an excellent discussion of the
other two areas referenced in the same title (palaeography and
codicology), but does not address stemmatology at all (which is not
surprising, as his area of specialty, Old English, has few texts which
exist in more than one manuscript). Ventura does discuss the use of
stemmata in some of the editions she describes, while none of Haverling,
Merisalo or O'Sullivan refer at all to stemmatology. One might have
expected, in a book on stemmatology, that in a chapter focussing on
works in Latin, existing in manuscript and early print, there might have
been some mention of Prue Shaw's 2006 digital edition of Dante's
\emph{Monarchia}, later published online in a second edition (\citeyear{alighieri_monarchia_2019}),
which I will discuss more extensively later on.

As is the case for almost the entire collection, the essays in
this chapter are deeply knowledgeable and very readable. Haverling's
essay is an excellent introduction to the history of Latin texts through
the centuries. The O'Sullivan and Merisalo essays offer detailed
instances of the transmission of texts and manuscript book production,
with Merisalo's lucid introduction of digitization initiatives (31--32)
nicely supplemented by Stokes' discussion of the impact of digital
methods (49--50 and 54--56). Ventura reviews the first incunabula print
editions of older texts and their use by later scholars, observing their
consistent preference (which amounts to a prejudice) to base editions on
manuscripts rather than early print editions –– a preference corrected
in recent work in patristic texts. However, one finds oneself wondering:
who is the intended audience for this chapter and indeed for the whole
book? The essays in the chapter are well-suited to a general
introduction to textual scholarship and might be at home in an
``introduction'' to stemmatology. Their place in a ``handbook'' seems
less secure. It is notable too that the Ventura essay is actually the
first half of a broader essay. The second half (512--24) appears in the
chapter ``Philological Practices'', taking up the narrative about
editions from the invention of printing and extending it through to the
modern period. One might question the utility of the chapter divisions,
when a single essay might easily slip between two distinct chapters.
Indeed, the half of the Ventura essay which appears in the first chapter
appears out of place. The other essays in the same chapter deal with
general editorial and textual issues, whereas this essay deals with one
category of editions only, leaving the reader wondering why editions
focussed on early print books get an essay to themselves in this
chapter, especially as stemmatology is irrelevant to most of the
editions it discusses.

The title of the second chapter, ``The Genealogical Method''
(57--138) promises to bring the focus squarely onto stemmatology. A
useful introduction by the chapter editor, Odd Einar Haugen, reminds us
that genealogical thinking about textual relationships did not begin
with Lachmann. Once again, the essay titles within this chapter promise
a comprehensive overview: ``Background and Early Developments''
(Haverling), ``Principles and Practice'' (Chiesa), ``Criticism and
Controversy'' (Palumbo) and ``Neo-Lachmannism: A New Synthesis?''
(Trovato). However, as with the first chapter, the essays are not quite
what one would expect in a collection focussed on stemmatology. Most of
Haverling's essay is, like her contribution in the first chapter, an
account of issues relating to textual editing without a particular focus
on stemmatology, or even on editing texts in many versions, once more
with a concentration on Greek and Latin texts, treading again (very
lucidly, and with sharply pointed examples) ground traversed by L. D.
Reynolds and N. G. Wilson in \emph{Scribes and Scholars}, first
published in 1968 and now in its fourth edition (\citeyear{reynolds_scribes_2013}). Where she writes
about instances of scholars in the period who demonstrate an awareness
of textual relations among multiple witnesses over time, she is
well-informed and succinct. One might have wished for less general
history and more discussion of significant specific instances. The two
paragraphs on Erasmus' editing of the Greek New Testament (71) and the
single paragraph on later editing of the Greek New Testament (73) could
have been usefully elaborated. There is no such difficulty about the
essays by Chiesa and Palumbo, which keep closely to their brief. Chiesa
neatly contrasts the simplicity of the principle of ``descent with
modification'' (to use Darwin's phrase) with the infinite complications
of how it works in practice. This leads him to suggest that it is useful
to think of genealogical hypotheses as metaphors, liable to contesting
interpretations, and not as statements of historical fact. Where Chiesa
is mostly theoretical, Palumbo puts forward a historical perspective,
beginning with Lachmann (repeating a point made many times in this
volume that Lachmann himself was not a Lachmannian editor) and
proceeding by analyses of the contributions of Gaston Paris, Joseph
Bédier, Dom Quentin (with an account of Jacques Froger's elaborations of
Quentin), Paul Maas and Giorgio Pasquali's critique of Maas. Of these,
the discussion of the relationship between Paris and his student Bédier
is the most detailed and illuminating, in revealing a complexity usually
simplified in the common narrative of Bédier's assault on Lachmannism as
he \mbox{understood it}.

Palumbo's essay is followed by an essay by Trovato, entitled
``Neo-Lachmannism: A New Synthesis?''. Here is where odd things start
happening. Firstly, Palumbo declares in his opening paragraph that he is
taking the narrative ``six decades up to 1934'', the date of Pasquali's
masterly \citetitle{pasquali_storia_1934}.
Consequently, one presumes that the ensuing essay by Trovato might begin
at 1934, but it does not: Trovato declares in his first sentence that
his scope is from 1929 to the present. The overlap between the two
essays is much greater than just a few years. Palumbo does not stop at
1934. He includes a pointed critique of the ``New Philology'' movement,
deriving from Bernard Cerquiglini's \emph{Éloge de la variante} (\citeyear{cerquiglini_eloge_1989}),
and the last page invokes ``neo-Lachmannism'', not as an introduction to
Trovato's essay but as a conclusion to his own. Trovato, in turn, does
not start at 1929, but feels obliged to go all the way back to Lachmann
and traverse once more the same ground as Palumbo. Trovato then tries to
establish that North American scholarship has not paid attention to the
``genealogical method'' by an elaborate table (2.4--1), surveying
mentions of European scholarship on Lachmannism in chapters of
Greetham's 1995 \emph{Guide}. It is hard to see that anything is proven
by the failure of (for example) Peter Shillingsburg, Donald H. Reiman
and Joel Myerson, in their chapters on 19th-century print literature, to
refer to genealogical methodologies typically used for ancient
manuscript texts. It is also ironic that Trovato should attempt to
establish that North American scholarship has ignored Lachmannism in a
volume which completely ignores Anglophone scholarship on the
genealogical method, as I will argue later on. But even odder is to
come. The bulk of his essay, fully twenty-one pages (longer, on its own,
than all but two other essays in the whole volume), is taken up with
section 2.4.3: ``a provisional list'' of ``contributions to the
improvement of the method'' (114), starting in 1929. One might
reasonably expect a scholar at this level to connect the pieces into a
thematic narrative, weighing each innovation. One might also expect that
Gianfranco Contini's work and influence over this period would be given
more prominence than a single short entry on his concept of
``diffraction''. There is a good argument to be made that
``neo-Lachmannism'' can be located in Contini's thinking about editing,
with his advocacy of editors as makers of ``working hypotheses'',
pragmatically using whatever tools are available to create and modify
them, as mentioned later in the collection by Frédéric Duval (462; see
also \cite{pugliatti_textual_1998}). Indeed, a list could be made, including key
publications or events not present in what Trovato himself recognizes as
a ``personal and subjective'' list, such as George Kane's landmark
assault on Lachmannian editing (\citeyear{kane_john_1984}). Even odder: the ``list'' in the
collection effectively stops at 2004, with only a brief note about
``multi-text codices'' and a longer report on Federico Marchetti's
startling claim, made in his PhD thesis (\citeyear{marchetti_scribal_2019}), that professional
copyists of Dante's \emph{Commedia} make ``significant errors'' at the
rate of just one every 800 lines. The ferment of thinking about
stemmatics and textual scholarship provoked by the advent of digital
methods finds no place in this essay.

The final section (2.4.4) promises a discussion of
``Neo-Lachmannism in the Third Millennium''. This section begins with a
declaration that the ``promises of amazing advances'' and ``polemical
stances against the method of common errors'', made by certain unnamed
scholars, have, it seems, ``diminished'' (135), being presumably
withdrawn by the same unnamed scholars. One wonders who these unnamed
scholars are. Trovato cites only an email from Odd Einar Haugen, editor
of the chapter, as proof of this declaration, and quotes a blog post of
mine as confirmation of the same (136), but neither Haugen's email, nor
my blog post, nor any formal article known to me, refer to any such
extravagant ``promises'' or take any such ``polemical stances''. The
remainder of the essay does not attempt any definition of
``neo-Lachmannism'', instead listing scholarly domains which practise
some form of genealogical methods (several of which are dealt with fully
in chapter 7, which is not cross-referenced) and ending with a
declaration of victory by Trovato: what he calls ``the method of common
errors'' will achieve, he says, ``a vaster diffusion'' and ``a more
conscious application'' in the new millennium (138).

Perhaps the oddest thing about this chapter on the history of
the genealogical method is what it does not include. In chapter 7,
``Philological Practices'', there is an essay by Ralf Plate entitled
``Medieval German'', which surveys how German courtly epic poetry (c.
1200) has been treated by German editors. The essay begins with the most
complete account of Lachmann as an editor I have read, emphasizing his
eclectic choice of variants, unbounded by strict genealogical
considerations. It continues with surveys of work by Karl Bartsch,
Hermann Paul, Wilhelm Braune, Karl Stackmann and Joachim Bumke. Ironies
and surprises abound here. Bédier's advocacy of ``best-text'' editions
was anticipated by the ``Deutsche Texte des Mittelalters'' series, from
1904 on. If ``neo-Lachmannism'' is to be identified with Contini's
pragmatic approach to editing, then Paul was ``neo-Lachmannian'' some
sixty years before Contini. It might have been useful to have had this
article appear in chapter 2, rather than 7. In chapter 2 it could have
served as a corrective to the tendency of non-German scholars to write
about German stemmatic scholarship as being completely defined by Maas
(thus, Trovato in this volume) or, even worse, by A. E. Housman's acid
attacks (this may be the only substantial publication on stemmatics in
English with no references whatever to Housman).

Two other essays in chapter 7, by Frédéric Duval (``Medieval
Romance Philology'') and Heinz-Günther Nesselrath (``Classical Greek''),
both include detailed discussions of the history of stemmatics, much of
it duplicating (as Plate does not) what is in chapter 2. Palumbo in
chapter 2 deals well with the ``New Philology'' as well as with the
relationship between Gaston Paris and Joseph Bédier (Nesselrath offers
yet a third view of Bédier). Duval, while repeating much of what Palumbo
says about Paris, Bédier and the ``New Philology'', offers a clearer and
more complete account of ``neo-Lachmannism'' than does Trovato, and
updates the discussion of ``New Philology'' to take in the recent surge
in ``documentary digital editions'' (464).

From this survey of the first two chapters, two common themes
appear. Firstly, the individual essays, apart from the Trovato essay,
are of exceptional quality. They are well-written and show deep
knowledge of their subject. Secondly, there is instance after instance
where the chapter editors and the general editor did not edit. There are
many pages in the essays which lack any focus on the declared subject of
``stemmatology''. There are overlaps between essays. There is at least
one essay which seems to be in the wrong chapter. The Trovato essay is
far below the standard of other essays and it is disappointing that the
editors did not call out his fondness for ``straw man'' arguments, also
permitting the lengthy and unfocussed ``provisional list''. There is
ambiguity as to who this collection is for: some essays are appropriate
for scholars who know nothing of textual scholarship (thus the survey
essays by Gerd Haverling and others), other essays for scholars who
already know a great deal (thus the specialist essays on the history of
stemmatics). One does not expect such editorial failures in a
publication with two layers of editing, by the chapter editors and the
general editor.

One finds these two patterns recurring across the rest of the
volume. In chapter 3, ``Towards the Construction of a Stemma'', edited
by Marina Buzzoni, the individual essays are at the same high standard
as most of the essays in the first two chapters. Gabriel Viehhauser's
``Heuristics of Witnesses'' surveys editions of Wolfram von Eschenbach's
\emph{Parzival} (c. 1200), including an account of Lachmann's edition
(\citeyear{von_eschenbach_wolfram_1833}), and with an excellent description of Michael Stolz's digital
edition (\citeyear{von_eschenbach_parzival-projekt_2022}) –– in which Viehhauser himself was involved, though this
is not stated –– which one could wish longer. It contains a useful few
pages on online manuscript catalogues, though one misses here (as
throughout the whole book) a reference to the transformative work
enabled by the IIIF consortium. Caroline Macé's ``Indirect Tradition''
sets out the case for looking at the many witnesses to a text which
appear in ``indirect'' form: in translations, in florilegia, in
quotations (including self-quotation), interpolations, adaptations,
inscriptions and more. Editorial use of such materials is well-known in
the New Testament scholarship (to which one might expect a reference,
given the apparent audience of beginning scholars), and the value of
this essay lies in the detailed description of traditions (e.g., the
Greek \emph{Physiologus}) which are less familiar. I will return to the
two other essays in this chapter, by Tara Andrews and Joris van Zundert.

Chapter 4 (``The Stemma''), edited by Tara Andrews, brings us
to the heart of the matter. It opens with an essay by Roelli,
``Definition of Stemma and Archetype'', which confronts (as does
Chiesa's essay) the simplicity of the fundamental concepts with their
infinitely variable expressions. There are some surprising statements.
Roelli suggests that a ``hyparchetype'' ``is situated directly below the
archetype in the stemma'' (221), though acknowledging in the same
paragraph that it might be situated lower down in the stemma, as the
ancestor of any family group. A ``hyparchetype'' is better defined as
the ancestor of any family group within the tradition, which might be
directly below the archetype, or many layers deep within the tradition.
This might appear a very slight difference, but on the next page (223)
Roelli attempts to identify the fundamental evolutionary biology concept
of the ``most recent common ancestor'' (MRCA) with the archetype. He
then declares: ``Here [\ldots] the similarity to phylogenetics
ends'', because he claims that in biology ``the concept of an original
makes little sense'' (223). If he is right, this demolishes any argument
that phylogenetics might be in any way useful for textual scholars. But
the equivalence he draws is wrong. First, there are cases in biology
where the concept of an original indeed makes sense: witness the many
attempts to locate the ancestor of \emph{Homo sapiens} in time and
space. Further, there are textual scholars for whom the attempt to
reconstruct the ``original'' makes no more sense than it does for some
biologist. Second, the phylogenetic ``most recent common ancestor'' is
not just equivalent to the archetype; it can be equivalent to both a
hyparchetype and the archetype. It is equivalent to the hyparchetype in
cases where the most recent common ancestor is of a genetic group
embedded within a tradition, as is ``ð'' in Roelli's Figure 4.1--8; it is
equivalent to the archetype in cases where all the descendants of the
most recent common ancestor are all the members of the tradition.

Armin Hoenen's ``The Stemma as a Computational Model'' takes us
deep into formal models of stemmata, building on graph theory. Hoenen
acutely raises the possibility that the range of stemmata is so great as
to render any one general model vacuous; yet there is value in
particular models of particular phenomena. This chapter is usefully read
alongside Chiesa's subtle characterization of genealogical hypotheses as
``metaphors'' as well as Dirk Van Hulle's essay, which I will address
later. A question not raised directly by Hoenen, but one which may be
vital to the discipline, is this: is it possible to arrive at a formal
model of textual transmission? Matters such a model would address
include: the likelihoods of contamination, shifts of exemplar,
coincident variation and of different kinds of variation. Aidan Conti's
comprehensively-detailed essay on ``A Typology of Variation and Error''
might be taken as a step towards such a model, with its survey of the
language used to categorize what most authors in this volume call
``errors'' but which are more properly termed ``innovations'' or
``secondary readings''. An essay by Tuomas Heikkilä on ``Dealing with
Open Textual Traditions'' addresses contamination, or ``horizontal
transmission'': the mixing of variants from different originals in a
single copy, famously identified by Maas as disabling any kind of
stemmatic analysis \parencite{maas_textual_1958}. Heikkilä's summary of multiple scholarly
attempts to do what Maas says cannot be done is lucid and accurate,
though one misses any reference to the use I make of relative
frequencies of variants from ``fundamental witness groups'' (building on
Froger's ideas about detecting contamination) to determine contamination
and its sources for Chaucer manuscripts and for the LauSC manuscript of
Dante's \emph{Commedia} (\cite{robinson_stemmatic_1997}; see also \cite{robinson_l0_2021}). The examples given by Macé in her essay ``The Stemma as a
Historical Tool'', again of Greek texts less familiar to most textual
scholars, will convince anyone who might have any doubt of the range of
knowledge, skills and tools needed to cope with highly-complex textual
scenarios. Her use of phylogenetic methods is interesting in its
evocation of its utility in making what she calls a ``tree'', not a
``stemma'': it shows ``only the relative textual proximity of the
witnesses'' (285). 

\newpage

\noindent Chiesa might call this ametaphorical representation,
whereas Van Hulle and others a map, as it will be discussed below.

Somewhat confusingly, in this chapter both Hoenen and Roelli
refer to ``Greg trees'' without explaining that, as I presume, these are
trees derived from W. W. Greg's notation in his \emph{Calculus of
Variants} (\citeyear{greg_calculus_1927}). The term is put in an open-space font in the text, as
if a reference to a definition somewhere. There is no discussion of the
term in the \emph{Parvum lexicon stemmatologicum} \parencite{roelli_parvum_2015}, no illumination to be had from the index (despite the first page
of the ``Introduction'') and no links from the text to any discussion,
anywhere. Indeed, one wonders what is served by this use of the
open-space font, which seems to indicate only that someone in the
editing process at some time thought the terms so marked were somehow
important.

Setting aside for now chapter 5, on ``Computational Methods and
Tools'', chapter 6 on ``Editions'', edited by Aidan Conti, takes us past
what stemmata are and how they are made to how they are used in
editions. Odd Einar Haugen's ``Types of Editions'' highlights the
distinction between editions which seek to ``reconstruct'' the text on
the basis of a genealogical analysis, and ``non-reconstructive'' (or
``monotypic'') editions, which align with Bédierist best-text
principles. A particularly surprising practice among Old Norse editors
is to present a stemma and then ignore it in the making of the text
(378--80). In theory, once one has made a stemma, one should use it to
adjudicate between readings in order to achieve a ``reconstructed
text''. Yet there are many editors (and not just editors of Old Norse)
who make a stemma and then choose a single manuscript and base the text
on that, as if the stemma were not to be trusted. One recalls Macé's
insistence that, perhaps, a tree might not be a stemma. Some examples of
Marina Buzzoni's ``Text-Critical Analysis'' suggest that a tree is
indeed a stemma, at least in the case of the three manuscripts of the
Old English \emph{Heliand}. Yet, even here, editorial judgement is
required to determine the reading which makes sense, and many of her
examples turn on the application (or not) of the \emph{lectio
difficilior} principle. It is striking how in every case she gives,
stemmatics alone is not sufficient to determine matters. Judgement
always matters.

Franz Fischer's ``Representing the Critical Text'' surveys how
editions both in print and in digital form present text and apparatus.
Among digital editions, he distinguishes between those which imitate
print and those whose content, presentation and tools go beyond print.
He includes useful thumbnails of various digital editions, including my
\emph{Wife of Bath's Prologue} \parencite{chaucer_wife_1996}, Michael Stolz's
\emph{Parzival} \parencite{von_eschenbach_parzival-projekt_2022} and Fischer's own \emph{Saint
Patrick's Confessio} (\citeyear{harvey_saint_2011}). One could wish this chapter were longer and
more detailed. Several of the editions which Fischer references contain
significant stemmatic materials, notably Shaw's editions of Dante's
\emph{Commedia} (e.g., \citeyear{alighieri_commedia_2021}), which are discussed nowhere in this
volume, as I will have the chance to repeat later on. This chapter
closes with Tara Andrews on ``Publication of Digitally Prepared
Editions''. The account in this essay on current technical solutions,
accurate and reasonably complete as of 2022, is likely to age quickly.
Already, advances in ``Linked Open Data'', built on ``semantic web''
technologies, are promising a step-change in the infrastructure behind
digital editions: see the Canadian LINCS \parencite{brown_lincs_2020} and ``Good
Things'' \parencite{woods_good_2021} projects.

Chapter 7, edited by Caroline Macé, is a catch-all chapter
holding surveys of editorial practice across multiple textual domains.
It contains essays by Christian-Bernard Amphoux on ``The New
Testament'', Heinz-Günther Nesselrath on ``Classical Greek'', Frédéric
Duval on ``Mediaeval Romance Philology'', Ralf Plate on ``Mediaeval
German'', Alessandro Bausi on ``Ethiopic'', Chaim Milikowsky on
``Hebrew'', Christopher Nugent on ``Chinese'', and Iolanda Ventura on
``Early Modern Printed Texts''. From my point of view, these chapters
contain some of the most useful and interesting material of the whole
volume. In their lucid and compact summaries of diverse textual
situations, they are worthy successors to the survey chapters in
Greetham's 1995 \emph{Guide}. Macé did well to find such knowledgeable
authors, who write so clearly. Of course, such a survey must omit much.
One regrets that a chapter on Arabic texts was withdrawn at the last
moment, and Macé (438) enumerates many areas which are not covered.
Surprisingly, her list does not include medieval English: an omission
which I will address later. Somewhat apart from the other essays in this
chapter is Dirk Van Hulle's ``Genetic Maps in Modern Philology''. On the
face of it, his expertise in ``genetic texts'' (essentially, editions of
modern texts focussing on the authorial revision process) should have
nothing to contribute to a volume on stemmatics, which is a completely
different field of editing. Yet, his essay is perhaps the single most
suggestive contribution to the whole volume. We have noted that Chiesa
saw that stemmata might be regarded as metaphors, that Macé proposed
that a stemma might not be a tree, and Haugen remarked that editors do
not use stemmata to determine readings. Instead, Van Hulle argues that
we might think of stemmata, however generated, as ``maps''. Thus, a
stemma does not represent an unambiguous genealogy. Like a map, it
represents in part proximity, in part routes from one node to another.
Just as in a map, one might suggest that a variant has travelled by the
shortest route from one place to another, that is by direct descent from
an ancestor to its descendant. But a variant might also travel a much
longer route, by roads less-travelled from one distant node to another,
as the result of contamination. Indeed, thinking of these figures we
call stemmata as maps, not genealogical trees, accords well with how
textual scholars (as shown by Buzzoni and Haugen) actually use
``stemmata'': they are used as guides, not as iron rules. One is
reminded of how the Greek New Testament Coherence Based Genealogical
Method uses ``textual flow'' diagrams as road maps which the editor can
travel from variant to variant in the way which makes most sense. It
also happens to accord with the way shrub-like representations
(``variant maps'') are used in several editorial projects with which I
am involved: the \emph{Canterbury Tales Project} \parencite{chaucer_canterbury_1991} as
well as Shaw's editions of Dante's \emph{Monarchia} (\citeyear{alighieri_monarchia_2019}) and
\mbox{\emph{Commedia} (\citeyear{alighieri_commedia_2021}).}

\newpage

\noindent The last chapter of the volume, ``Evolutionary Models in Other
Disciplines'', edited by Armin Hoenen, offers essays focussing on the
use of phylogenetic/stemmatic methods by Christopher Howe and Heather
Windram (``Phylogenetics'') on evolutionary biology, by Dieter Bachmann
(``Linguistics'') on languages, by Jamshid Tehrani (``Anthropology'') on
cultural phenomena and by Cristina Urchueguía (``Musicology'') on
histories and editions of music. Howe and Windram are familiar from
their work over nearly three decades, with myself and others, exploring
the use of phylogenetic tools on textual traditions. Every textual
scholar should note their caveat: ``it is essential for textual scholars
to use their experience in interpreting the results of phylogenetic
analysis'' (547). Bachmann's essay is a model of both concision and
readability. It would be pleasant if scholars engaged in stemmatics
could avoid the conflicts which have riven the historical linguistics
community. Tehrani's essay offers a very clear account of Bayesian
methods, and its outline of how quantitative methods may illuminate our
understanding of cultural change reinforces the claim put forward by
Howe and Windram for ``phylogenetics'' as a discipline applicable to
many domains, from Persian carpets to languages. Finally, Urchueguía
remarks on the late embrace of stemmatic methods, by musicologists, and
on their use in ways beyond the establishment of a text, for example by
Allan Atlas, to trace the origins of chansons as well as how they came
to be formed into a collection (583).

Had these seven chapters of the collection (that is, excluding
chapter 5, on ``Computational Methods and Tools'') been published
together as originally proposed by the chapter editors, as \emph{An
Introduction to Stemmatology in the Digital Age}, the review would have
finished here. My review would have noted the excellence of almost all
the essays. I would have sympathized with the editors in their
difficulties with essays withdrawn at the last minute and with senior
academics who are reluctant to write to their remit. I would have
commented on the structuring of some of the chapters and the
distribution of some of the essays between the chapters. By definition,
an ``introduction'' can claim a certain elasticity. The word
``introduction'' itself points to the existence of alternative
approaches, which might not gain purchase in this first view of the
topic.

But that is not what was published. Instead, the collection is
titled \emph{Handbook of Stemmatology: History, Methodology, Digital
Approaches} (not ``a handbook'', but ``handbook''). Further, the
published volume was not edited by a collective, comprising all the
chapter editors plus Philipp Roelli, but by Roelli alone. On the face
of it, Roelli was not an obvious choice to edit this collection, ranging
so widely over so many areas of scholarship and including scholars from
so many disciplines. He had worked only with Latin texts and had never
edited any comparable volume. Nonetheless, he became the single editor
and the group of essays formerly collected as an ``introduction''
appeared in 2020 as a ``handbook''.

An ``introduction'' is not a ``handbook''. Nor is this just any
``handbook'': it is a ``handbook'' published by De Gruyter. Over many
years, De Gruyter has made a formidable name for itself as a publisher
of handbooks. Indeed, it publishes so many, and so many in multi-volume
series, that it is difficult to determine quite how many De Gruyter
handbooks there are. A search for ``handbook'' on the De Gruyter website
suggests 1,126 books, but some of these might come from other
publishers, some might be double-counted as both titles of series and
volumes within series. By another measure: De Gruyter declares that it
publishes 1,100 books every year; some 73 publications in the last year
seem to be handbooks \parencite{noauthor_publication_2022}. There appears to be a De Gruyter
handbook for every intellectual discipline: for example, on
``Nanoethics'', on ``Empirical Literary Studies'', on ``Zoology'' (nine
volumes), on ``Pragmatics'' (fourteen volumes), on ``Flavoproteins''
(two volumes). While there seems to be no general statement by De
Gruyter available about their handbooks, one can glean from
introductions to some of the handbooks published online what makes a
typical De Gruyter handbook. For example, from the preface to the
thirteen volumes of handbooks on pragmatics (here quoted from its
appearance in the first volume):

\begin{quote}
{[}This series{]} sets out to reflect the field by presenting in-depth
surveys covering the central and multifarious theories and
methodological approaches as well as core concepts and topics
characteristic of pragmatics as the analysis of language use in social
contexts. All articles are written specifically for this handbook
series. They are both state of the art reviews and critical evaluations
of their topic in the light of recent developments.

\begin{flushright}
    \parencite[v]{bublitz_foundations_2011}
\end{flushright}
\end{quote}

\noindent Two characteristics stand out. Firstly, the handbooks aim to be
comprehensive, to offer an up-to-date overview of the whole field.
Secondly, the essays are ``critical evaluations''. This is why De
Gruyter titles each volume simply ``handbook'', not ``a handbook''. Once
such a handbook has been made, why should any other be needed?

Manifestly, these essays, collected for an ``introduction'',
were not written specifically for a ``handbook''. This need not have
mattered, if the editors (with the authors' co-operation) had taken the
opportunity to determine what needed to be done to turn the
``introduction'' into a ``handbook''. Chapter 7, with its overview of
the application of stemmatics across multiple textual domains, could
have been expanded. The failure to deal with Middle English editing
could have been remedied. Any account of stemmatics over the last
century which omits George Kane's assault on Lachmannism (\cite{kane_john_1984}; see also \cite[102--18]{donaldson_psychology_1970}), chiefly in his editions of \emph{Piers Plowman} \parencite{langland_piers_1960,langland_piers_1975}, is radically incomplete. A chapter on Sanskrit
texts, far and away the largest textual tradition of all, could have
been supplied. Room could have been made for an expanded chapter seven
by focussing the volume more sharply on stemmatics, removing areas which
offer general introductions to textual scholarship or to computer
methods (thus, van Zundert's chapter 3.4). Overlaps between essays, the
distribution of essays across the chapters and cross-referencing between
essays could have been addressed. There is no evidence of any such
revision, or even that it was contemplated. At the least, this was an
opportunity missed.

The gulf between ``introduction'' and ``handbook'' is widest in
one key area: the treatment of computing methods. If the volume had
published the seven chapters without chapter 5, that would have been
adequate for an ``introduction'', since there is enough about computing
methods and digital editions in these seven chapters to serve for an
``introduction''. However, the subtitle of the volume published as a
``handbook'' has ``digital approaches'' as one of the three areas it
will cover. In the context of the ferment in digital humanities activity
focussed on scholarly editions in the last decades, a reader would
expect a handbook to offer a comprehensive and balanced account of
computing methods, and the subtitle promises that. This chapter is
fundamental to the ``handbook''. There are multiple problems with this
chapter, though. The first problem is the choice of chapter editor.
Joris van Zundert is one of the leaders of the rising generation of
digital humanists and has done excellent work in many areas, but he has never worked in stemmatics. The
second problem is given by some of the contributions to the chapter. One
expects that the first essay, ``History of Computer-Assisted
Stemmatology'', by Armin Hoenen, will offer a straightforward account of
developments in the application of digital methods to stemmatology in
scholarly editing. It does not. There are startling omissions. Vinton
Dearing's pioneering work on the text of Dryden, dating as far back as
the Fifties, is ignored (\citeyear{dearing_methods_1962}), as are many other fundamental
contributions in the Sixties and Seventies which Hoenen could have
easily gleaned from Susan Hockey's \emph{A Guide to Computer
Applications in the Humanities} (\citeyear{hockey_guide_1980}): a book apparently unknown to
Hoenen. Where is Peter \citeauthor{shillingsburg_scholarly_1986}'s foundational \emph{Scholarly
Editing in the Computer Age} (\citeyear{shillingsburg_scholarly_1986}) and his \emph{CASE} collation
software? Where is Wilhelm and Tobias Ott's \emph{TUSTEP} (\citeyear{ott_tustep_2011}), also
incorporating collation software, and Stefan Hagel's \emph{Classical
Text Editor} (\citeyear{hagel_classical_1997}), later discussed in Hoenen's essay on ``Software
Tools'', both of which were used to make hundreds of editions? It is a
phenomenon worthy of note that many editors have used digital tools to
make editions, but not used them for stemmatic purposes. Hoenen does not
note this. Scant treatment is given to the work done in stemmatics in
the Netherlands in the 1990s, by scholars associated with Evert Wattel
including Pieter van Reenen, Margot van Mulken and Ben Salemans, which
resulted in two \emph{Studies in Stemmatology} volumes (\cite{reenen_studies_1996}; see also \cite{reenen_studies_2004}). Both volumes canvas
quantitative methods which do not come from evolutionary biology, and
the failure to recognize that there are such methods is compounded by
chapter 5.1.2, which describes phylogenetic methods as if they were the
only methods used for stemmatic analysis.

There are also statements which are flat wrong. The contrast
Hoenen draws between \emph{CollateX} \parencite{dekker_collatex_2010} and my
\emph{Collate} program \parencite{robinson_collate_2014}, that \emph{CollateX} ``finally
collates all by itself'' (295), shows that he is ignorant of both
programs. His stern warning that stemmatologists should ``not readily
assume'' that phylogenetic methods are appropriate for textual studies
(297) suggests that there are stemmatologists who actually ``assume''
this. Who are these stemmatologists? He names none, and I know of none
who assume any such thing. But most unsatisfactory of all is his section
5.1.3, ``Recent Developments''. This surveys several experiments with
digital artificial traditions. It completely ignores the work done by
multiple scholars on actual, real, complete textual traditions: thus,
the work by Michael Stolz on \emph{Parzival} \parencite{von_eschenbach_parzival-projekt_2022}, by
Godfried Croenen and Peter Ainsworth on French chronicles (\citeyear{croenen_online_2010}), by
myself and Robert O'Hara on Old Norse (\citeyear{robinson_report_1992}) and, with Barbara
Bordalejo, on the \emph{Canterbury Tales} \parencite{chaucer_canterbury_1991}, and by Prue
Shaw on Dante's \emph{Monarchia} (\citeyear{alighieri_monarchia_2019}) and \emph{Commedia} (\citeyear{alighieri_commedia_2021}).
Hoenen attended several \emph{Studia stemmatologica} workshops, where
these editorial projects were presented. Yet, for this chapter, those
projects do not exist. It is particularly difficult to understand the
omission of Prue Shaw's editions. She was Contini's student. Contini
entrusted her with editing the \emph{Monarchia}, knowing that she
intended to use computers in preparing the edition. She is the only
non-Italian, and one of the few women, to have edited a text in the
\emph{Edizione Nazionale delle Opere di Dante}. Her digital edition of
the \emph{Commedia} (which I presented multiple times at \emph{Studia
stemmatologica} workshops) may claim to be the most advanced and
complete example to date of the fusion of traditional and digital
scholarship. It is mentioned precisely once in the entire volume (not by
Hoenen), and her edition of the \emph{Monarchia} not at all. Her second
edition of the \emph{Commedia} is now available online (\citeyear{alighieri_commedia_2021}), so
readers may make up their own minds. This one edition alone gives the
lie to Hoenen's assertion that ``investigations with a closer link to
traditional methods [\ldots] became rarer after the 1990s'' (297).
From this essay, one might assume that the application of digital
methods to stemmatology had not gone further than a few somewhat
inconclusive experiments. That is not the case.

The titles of the next two essays (``Terminology and Methods'',
by Sara Manafzadeh and Yannick M. Staedler, and ``Computational
Construction of Trees'', by Teemu Roos) suggest a clear differentiation.
Yet, both authors write essentially the same essay: both describe the
difference between distance and discrete (or character-based) methods;
both describe maximum likelihood, parsimony and Bayesian methods, while
Roos strays into a description of software, which is properly the
subject of Hoenen's essay on ``Software Tools'' (itself, a useful
summary). To top it off, the best description of Bayesian methods in the
volume is given not in this chapter, but by Jamshid Tehrani in chapter
8.3, which is not cross-referenced here. The last essay in the chapter
(``Criticisms of Digital Methods'', by Jean-Baptiste Guillaumin) manages
to be even odder. There are just two pages in which Guillaumin does what
the essay title says: discuss various criticisms of digital methods,
which he does lucidly and well. He then does something extraordinary. He
creates a small artificial tradition of ten manuscripts of a ten-word
test and then proceeds to use distance methods to create trees of that
tradition. There are many problems here. There is his choice to use
distance methods alone, apparently in the belief that distance methods
are most appropriate for phylogenetic analysis of manuscript traditions.
This is questionable, to say the least. Inevitably, too, the pages taken
up with setting up and analyzing this ``tradition'' are pages not spent
addressing the supposed topic of the essay.

One may wonder: where were the editors responsible for this
chapter? Did they not see the inadequacy of Hoenen's cursory history, or
the overlaps between the central three essays, or the irrelevance of
most of Guillaumin's essay? The omission of work done on major textual
traditions by Shaw, Stolz, Croenen, Ainsworth, Bordalejo and myself, as
well as by others, is contrary to a fundamental principle of a De
Gruyter handbook: that it should offer a comprehensive account of
scholarly work in the field, even where the handbook's editors disagree
with that work.

Nor is this the only major recent development in the
application of textual scholarship and stemmatics which is left out of
the volume. All the scholars mentioned in the last paragraph, and the
New Testament editing projects based at Münster and Birmingham, follow
the same model for the creation of an apparatus. We may name this model
the \emph{Canterbury Tales Project} model, due to its articulation by
that project \parencite{robinson_canterbury_1993}, though parts of this model had been
anticipated by several collation tools developed as early as the 1960s \parencite{hockey_guide_1980}, implementing varieties of full-text transcription and
collation, as did Gary A. Stringer's variorum edition of John Donne's
poetry \parencite{donne_digitaldonne_1981} and Peter Shillingsburg's \emph{CASE} program
(\citeyear{shillingsburg_scholarly_1986}). Here is a one-sentence account of the model:

\begin{quote}
This Project aims to make available over a ten year period full
transcripts of the text of every manuscript and pre-1500 printed edition
of the \emph{Canterbury Tales}, together with computer images of every
page of every manuscript and early edition, collations of all these
texts, and analyses of the textual tradition based on the transcripts.

\begin{flushright}
    \parencite{robinson_canterbury_1993}
\end{flushright}

\end{quote}

\noindent Setting aside the irrational optimism of the ``ten year period'', these
methods, requiring the creation of full digital transcripts of
manuscripts and collation of these transcripts to make an apparatus, are
central to our work. There are fuller accounts elsewhere of how we
undertake transcription and collation (\cite{bitner_macron_2021}; see also \cite{bordalejo_youre_2021}). There have been previous collations which
sought to record, word-by-word, every variant at every point, notably in
the Twenties and Thirties of the eighty-plus manuscripts of the
\emph{Canterbury Tales} \parencite{manly_text_1940}. However, before the
digital era it was not possible to create digital transcripts of every
word in the manuscripts and then use digital tools to help create a
record of exactly how at every point the manuscripts differ. The
collation tool –– now Catherine Smith's \emph{Collation Editor}
(\citeyear{smith_collation_2018}), used by ourselves and several other projects, including the
New Testament editing projects –– is optimized to permit the editor to
intervene in the collation (hence, ``computer-assisted collation''). It
is the editor who determines exactly how individual manuscript forms
should appear in the collation (``regularization''), which different
forms actually are variants, and exactly how the variant texts should be
aligned against each other (see \cite{houghton_editio_2020}, for the links
between myself and the New Testament editing projects). For just a
single line of verse, in some sixty manuscripts of the \emph{Canterbury
Tales}, this requires hundreds of decisions by the editor.

Further, the collation must serve two purposes: it must present
the variants both so they can be most easily understood by the human
editor and most productively analyzed by other digital tools. I find it
hard to imagine a universe where these choices can be completely
automated. This process is far from what most editors still do: create
or choose some kind of collation base and compare each manuscript, one
at a time, to the base, recording the variants either on paper or in
electronic form using Microsoft Excel spreadsheets or databases (330).
Yet the Andrews chapter on ``Transcription and Collation'' makes no
mention of Smith's \emph{Collation Editor} at all, despite Andrews
having been at several sessions, during \emph{Studia stemmatologica}
workshops, where it was presented. Her account suggests that while my
\emph{Collate} software did indeed permit editorial intervention, ``the
current generation of collation tools'' are fully automated (168),
apparently on the basis that advances in these tools remove the need for
editorial intervention. Nor does her brief discussion of transcription
show any awareness of the particular requirements for transcription when
it is to be used as the source for collation. Hoenen also seems unaware
of the \emph{Collation Editor} and of the approach to collation which it
embodies. He lists just two collation tools, declaring that the only
choice is between ``manual'' and ``automatic'' collation (329--30).

This is not just failure to mention a single piece of software.
It is failure to acknowledge that over the last thirty years, a new
method has been developed for discovering and recording how texts
differ. This new method (the \emph{Canterbury Tales Project} model), of
full-text transcription followed by computer-assisted scholarly
collation, creates complete word-by-word records of every difference
found by the scholar between the texts, after normalization of spelling
forms and alignment of variants by the scholar. Further, the original
forms and their normalizations are recoverable, opening up a new area of
scholarship: research not just into how the manuscripts differ
stemmatically (typically relying only on variants judged as significant
after normalization of spellings), but how they agree and differ in
their spellings. I have created data files recording the original
spelling forms and how they were regularized for six sections of the
\emph{Canterbury Tales}. Preliminary analysis of the patterns of
spellings suggests that, for example, different manuscripts were written
by the same scribe, even though the manuscripts are far apart
stemmatically. In terms of stemmatics, this process creates a mass of
evidence of agreements and disagreements among the manuscripts far
beyond that available from traditional ``manual'' collation, which
typically focuses only on a selection of variants. Hoenen explains how
the availability of massive amounts of data generated by DNA sequencing
has altered biology, but denies that there has been, or ever will be,
any comparable development in textual scholarship: ``no input-data
revolution like the one replacing chosen character data by sequenced DNA
has taken place, and it is unlikely that something similar will ever
happen in stemmatology'' (297). But it has happened. The 100,393
readings in the 14,223 lines of the \emph{Commedia}, and even greater
numbers in the Münster and Birmingham New Testament projects, created by
their application of the \emph{Canterbury Tales Project} model, witness
this change. There is an argument that the development of this new
method for discovering and recording how texts differ is more
significant than developments in how we create hypotheses about the
relationships among texts.

These two omissions are so egregious, requiring that so many
scholars failed to see that the ``handbook'' omits them, that one asks:
how did this happen? To answer this question, one has to look again at
the foundations of this volume in the meetings of the \emph{Studia
stemmatologica} group, from 2010 on. Over the course of these meetings,
it emerged that the sharpest and most persistent schism between
attending scholars concerned the application of scholarly judgement in
the assessment of variants. This difference crystallized around two
words: ``common errors''. For one group, construction of any kind of
genealogy of the relations between manuscripts must have, as its
fundamental first step, the determination by the editor of which variant
readings are ``shared indicative errors'' (Maas' \emph{Leitfehler}; 4).
Only these ``errors'' can be used in the making of any tree. It follows
that the only people able to make genealogical hypotheses of manuscript
relations will be those who can discriminate these ``shared indicative
errors'' from the great mass of variant readings. There is much that is
attractive about this view. It valorizes editorial judgement, which all
agree is crucial to the making of good editions. It offers a powerful
tool, perhaps even a shortcut, towards the understanding of how
manuscripts are related. Instead of having to look at every reading in
every manuscript at every word, the editor can focus on identifying
those readings which are ``shared indicative errors''. This is the basis
of the ``Barbi \emph{loci}'': the choice by Michele Barbi of four
hundred (actually, 396) lines of Dante's \emph{Commedia} in which, Barbi
argued, analysis of the variants would provide a reliable base for
understanding how all the manuscripts in the tradition –– some 600
complete manuscripts and another 200 partial or fragmentary manuscripts
–– are related to one another \parencite{barbi_canone_1891}. Roelli is a forceful
advocate of this view (``the strongest tool for the endeavour are shared
indicative errors''; 4), as is Odd Einar Haugen (a method ``there is no
need to defend''; 135) and Trovato (``the method of common errors which,
not without hesitation, we have proposed calling the neo-Lachmannian
method''; 138).

I am not the first scholar to find difficulties in centring
stemmatics around the identification of ``indicative common errors''.
Famously, Pasquali found Maas' schematization of ``common errors'' so
distant from what actually happens in textual traditions that first he
wrote a review longer than Maas' book (\citeyear{pasquali_paul_1929}) and then his comprehensive
\emph{Storia della tradizione e critica del testo} (\citeyear{pasquali_storia_1934}). Indeed,
Palumbo's essay in this volume gives due weight to Pasquali's criticisms
of Maas (105--07). The medieval English textual scholar George Kane
mounted a full assault on the method after he found it failed to give
useful results in his work on the manuscripts of Langland's \emph{Piers
Plowman} (\cite{kane_john_1984}; see also \cite{langland_piers_1960,langland_piers_1975}). I am on the side of
those who doubt. Consider this statement from Roelli: ``the existence of
an archetype different from the original can be proved by finding at
least one error common to the entire tradition, one the author could not
have written'' (223). This ascribes to the author the god-like ability
never to make a mistake, and to the editor the god-like ability to
identify every mistake ever made (see also Buzzoni's ``Text-Critical
Analysis''; 385). As Kane's collaborator F. Talbot Donaldson observed:
if a scholar is able to discriminate original from error with such
certainty, why not use that ability at every word and have done with the
whole business of stemmatics \parencite[107]{donaldson_psychology_1970}.

The power to discriminate just which ``errors'' are ``indicative'' also
seems god-like, as the editor must choose which readings are truly
``monogenetic'' and therefore likely to have occurred just once in the
manuscript tradition and hence will only be found among descendants of
the copy which introduced the ``error''. Further, the scholar must
assert that the character of the reading is such that scribes will not
restore the original reading, or indeed substitute some other reading
–– or, if they do, the editor will be able to show Contini's
``diffraction'' at work, where a ``hard'' reading generates a range of
readings in the same way that light diffracts into its constituent
colours when it meets an obstacle. Accordingly, the editorial projects
which have adopted the \emph{Canterbury Tales Project} model concentrate
editorial effort on the transcription and collation processes,
attempting to achieve the most transparent and productive representation
of all the variation across the texts, rather than on trying to select
``indicative errors'' and collate only at those points (see \cite{andrews_analysis_2016}, for a similar approach). It is not that the scholars using this
method are attempting to avoid editorial judgement. The transcription
and collation processes require intense expenditure of editorial
judgement; editorial judgement is required to ``orient'' the unrooted
trees typically produced by phylogenetic analysis; editorial judgement
is required to explain what we can learn of each tradition from all the
evidence we have, including phylogenetic and other analysis; editorial
judgement is required to construct a critical text from all this
information.

The stance scholars take on ``common errors'' and the role of
editorial judgement determines how they stand on the \emph{Canterbury
Tales Project} model of full-text transcription and computer-assisted
collation. If you think that scholars can determine stemmatic
relationships from ``indicative shared errors'', then there is no need
to carry out a full word-by-word collation of every word in every
manuscript. Just select those ``indicative shared errors'' and collate
those. Indeed, as every such error by definition must serve to specify
the one stemma which applies across the whole tradition, then it may be
enough to identify just a very few readings as ``indicative'' and build
your argument on those, as it seems Federico Sanguineti did when arguing
that fully 600 manuscripts of the \emph{Commedia} represent a single
group on the basis of just four readings (\cite[8]{robinson_textual_2012}; see also \cite[50--67]{shaw_introduction_2021}).

I do not present these arguments here to establish that Roelli
and others are wrong, and I am right. There are strong arguments on both
sides. To speak just of practicalities, Trovato (\citeyear{trovato_everything_2017}) is right to
question whether the expenditure of so much effort in transcription and
collation is worthwhile. The \emph{Canterbury Tales Project} is in its
fourth decade, and is likely to require another fifteen years to
complete the forthcoming \citetitle{chaucer_critical_nodate}, which I am co-editing with
Barbara Bordalejo. The central importance of these two matters in the
editing of large textual traditions –– the validity of the ``common
errors'' approach, the utility of full-text transcription and
computer-assisted collation –– mandates that they will loom large in
editorial discussions in the next decades. This is a debate which must
be had and this collection would have been an excellent place to have
started it.

At the beginning of this review, I noted that, from Roelli's
presentation of how his edited collection came to be, the reader might
gain the impression that something happened which indeed did not happen.
From the omission of any references to the challenges to the ``common
errors'' method and to traditional selective collation, the reader might
gain the impression that something is not happening which indeed is
happening. The New Testament editions coming from Münster and
Birmingham, Shaw's \emph{Commedia} as well as our forthcoming
\emph{Canterbury Tales} edition all counter the assertion set forward in
this volume, which is to say that the ``common errors'' method is fixed
and accepted orthodoxy, and counter the view that selective collation
justified by that method is entirely adequate. The failure of this
volume to acknowledge the existence of these challenges disqualifies its
claim to be a handbook, and particularly a handbook conformant to the
established De Gruyter model. The immense effort expended by the
editors, including Roelli, and the many excellent essays in this volume
are devalued by their appearance in a volume which is not what it claims
to be. One is left to wonder how De Gruyter came to be persuaded that it
is a handbook and how the Swiss National Science Foundation were
persuaded to pay the substantial sum necessary to make the volume
available free to all. Usually, open access is a great boon, but not
when it gives prominence to a publication which purports to offer a
comprehensive and unbiassed view of a whole field, and does not.
Long-held models of the textual scholarship of texts in many versions
are being challenged, as old models are refreshed and new models
developed. The failure of this ``handbook'' to present these challenges
is a disservice to the discipline of textual scholarship.

There are many lessons from this publication. One lesson is
that scholars should be aware of the context in which their essays
appear. Like most scholars, when I accept an invitation to write an
essay for a collection, I write the essay, submit it, revise it and
think no more of it. I will not be so trusting in the future. For the
readers of this volume: take what is good in the individual essays
(there is much that is very good), as if it were the ``introduction''
which was originally intended and not the ``handbook'' which it fails to
be.


\begin{flushleft}
    % use smallcaps for author names
    \renewcommand*{\mkbibnamefamily}[1]{\textsc{#1}}
    \renewcommand*{\mkbibnamegiven}[1]{\textsc{#1}} 
\printbibliography
\end{flushleft}

\end{document}