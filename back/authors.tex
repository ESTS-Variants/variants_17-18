% add Authors section as a \part in the ToC
\addcontentsline{toc}{part}{Authors}
% use "authors" style for headers and footers
\pagestyle{authors}
% format "Authors" section as a new \chapter
\chapter*{Authors}
% pagenr at the bottom
\protect\thispagestyle{chaptertitlepage}

% Short Bios of authors start here
% Each author gets 1 \paragraph
% Each \paragraph starts with author name
% This author name is formatted as the \paragraph title to make it pop out more (bold and some space)


\paragraph{Olga Beloborodova} is a postdoctoral researcher at the University of Antwerp (Centre for Manuscript Genetics). She has published extensively on Samuel Beckett, cognition and genetic criticism. She is member of the Editorial Board of the \emph{Beckett Digital Manuscript Project}. Her first monograph, \emph{The Making of Samuel Beckett’s ``Play'' / ``Film''} (Bloomsbury) was published in 2019. Her second book, \emph{Postcognitivist Beckett} (CUP) came out in 2020.

\paragraph{Elli Bleeker} is a researcher at the Huygens Institute in Amsterdam, The Netherlands. She specializes in digital scholarly editing and computational philology, with a focus on modern manuscripts, genetic criticism, and automated collation. Elli completed her PhD in 2017 at the University of Antwerp on the role of the scholarly editor in the digital environment. During her Early Stage Research Fellowship in the Marie Skłodowska-Curie funded DiXiT network (2014–2017), she received advanced training in manuscript studies, text modeling, and XML technologies. She has participated in the organization and teaching of workshops on scholarly editing with a focus on knowledge transfer and the application of computational methods in a humanities environment.

\paragraph{Beci Carver} is the author of \emph{Granular Modernism} (Oxford
University Press, 2014) and of numerous articles and essays on modernism
and its penumbrae. Her new book, \emph{Modernism's Whims}, is
forthcoming from Oxford University Press, and she has another in the
works, on \emph{Pleasure}. She is a Senior Lecturer at the University of
Exeter.

\paragraph{Carlotta Defenu} is a Post-Doctoral research fellow within the research project Modernismo.pt at the Institute for the Study of Literature and Tradition of the NOVA University of Lisbon. She holds a Ph.D. in Textual Criticism from the University of Lisbon; her doctoral thesis examined the genesis and revision process of Fernando Pessoa’s orthonymous poems. She has published various articles and book chapters in the fields of textual criticism, genetic criticism, and genetic translation studies.

\paragraph{Aline Deicke} studied Pre- and Protohistory, Classical Archaeology and Anthropology at Johannes Gutenberg-University Mainz, Pécs University and
Eötvös Loránd University. Since 2009, she has worked at the Digital
Academy of the Academy of Sciences and Literature \textbar{} Mainz.
Since 2021, she is teaching as a professor of Digital Humanities at
Philipps-University Marburg. Her research focuses on historical and
archaeological network research; knowledge engineering in the Arts and
Humanities, often in conjunction with graph-based technologies; and the
critical reflection of digital knowledge production and hermeneutics. In
recent years, she has pursued these interests as the co-lead of three
projects: ``Correspondences of Early Romanticism. Scholarly Edition ---
Annotation --- Network Research'', ``disiecta membra. Stone Architecture
and Urbanism in Roman Germany'', and ``HERMES. Humanities Education in
Research, Data, and Methods''.

\paragraph{Wout Dillen} is a Senior Lecturer in Library and Information Science at the University of Borås, Sweden. In 2015, he completed his Ph.D in Literature at the University of Antwerp, where his dissertation focussed on Digital Scholarly Editing in the field of Genetic Criticism. As part of this project, he also developed the Lexicon for Scholarly Editing (\url{https://lexiconse.uantwerpen.be}). In 2016, he held an Experienced Researcher fellowship in the Marie Skłodowska-Curie ITN on Digital Scholarly Editing called DiXiT. Wout has been an active member of the Board of the European Society for Textual Scholarship since 2015, and has functioned as its Secretary from 2017 to 2024. He is currently also the General Editor of the Society’s journal: \textit{Variants}. In addition, Wout has acted as one of the General Editors of the \textit{Journal of DHBenelux} since its inception in 2019, and has recently taken up the role of Deputy Editor of the LIS journal \textit{Information Research}.

\paragraph{Carolina Escobar-Vargas} is the Vice-Dean for Academic Affairs at the Facultad de Ciencias Humanas y Económicas, Universidad Nacional de Colombia (Medellín). She serves as an Associate Professor of Medieval History at the same faculty. Her research focuses on the Central Middle Ages, with publications on topics such as magic, Geoffrey of Monmouth, and the distinctions between fiction and reality in medieval chronicles.

\paragraph{Paola Italia} teaches Italian Philology and Scholarly Editing at the University of Bologna. She has worked on various nineteenth- and twentieth-century authors and topics, with a particular focus on editions of paper and digital texts (Editing Novecento, Salerno, 2013; Editing Duemila, Salerno, 2020), Authorial variants (\textit{What is authorial philology?} OBP, Cambridge, 2021) and Creativity in Manuscripts (\url{https://site.unibo.it/manoscrittidigitali/en}). In the twentieth-century field, she has worked on Savinio, Bassani, Tobino and Gadda (Come lavorava Gadda, Carocci, 2017, French translation: Dans l'Atélier de Gadda, Hermann, Paris, 2022). With Giorgio Pinotti and Claudio Vela she co-directs the new Adelphi edition of Gadda's works, and has edited the critical edition of Gadda’s WWI Notebok (Adelphi, 2023), and a collection of 219 Gadda's words: Il Gaddabolario (Carocci, 2022). In Digital Humanities her domains of competence are in the field of Textual Theory, Text modelling and encoding, Variants Mining, Authorial Knowledge Sites (Wiki Gadda, Wiki Leopardi, Leopardi Ecdosys, Manzonionline). She founded and directs the site \url{www.filologiadautore.it} (more than 1.200.000 hits since 2010), and collaborates with DHARC, where she coordinates scholarly digital editions of Manzoni (Philoeditor and LeggoManzoni) and of VASTO.

\paragraph{Juan Lorente Sánchez} is a lecturer of the Department of English, French and German Philology at the University of Málaga (Spain), where he has taught English for International Tourism and English Language, and currently teaches English Phonetics and Phonology. He has also collaborated in the extension of the Early Modern English component of the so-called \emph{Málaga Corpus of Early English Scientific Prose,} and has worked in the compilation and POS-tagging process of its Late Modern English subset. His research interests include text editing, manuscript studies, historical linguistics and corpus linguistics.

\paragraph{Martin Navrátil} has worked at the Institute of Slovak Literature of the Slovak Academy of Sciences in the Department of Textology and Digital Projects since 2018. He is especially focusing on the Slovak poetry of the 20\textsuperscript{th} century and the question of Slovak textology and co-organizes annual Textological seminar.

\paragraph{Dipanjan Maitra} is Assistant Professor of British Literature at
Louisiana Tech University, USA. He was formerly a Marion L. Brittain
Postdoctoral Fellow at Georgia Institute of Technology. He obtained his
PhD in English from State University of New York at Buffalo. His
dissertation explores the role of press-cutting agencies in advancing
the careers and interests of James Joyce, Gertrude Stein and larger
colonial institutions. His articles on James Joyce, genetic criticism
and modernist print culture have appeared in \emph{Modernism/modernity
Print Plus}, \emph{James Joyce Quarterly}, \emph{Genetic Joyce Studies},
\emph{Joyce Studies in Italy} and other peer-reviewed journals and
edited collections.

\paragraph{Kiyoko Myojo} is a professor in the Faculty of Arts and Literature, Seijo University. She established the Research Center for Textual Scholarship at the university in 2022 and is currently the director. In 1998, she obtained her PhD from the University of Tokyo after studying at the University of Munich for three years. In 2004, she received the ``Japan Society for German Literature Award'' for her book, The New Kafka (in Japanese). She worked at Saitama University for 20 years, from 2000 to 2010 as an associate professor and then from 2010 to 2020 as a professor. She also served at the "Japan Society for the Promotion of Science" as a program officer from 2014 to 2017. From 2024 to 2025 she is a visiting senior research fellow at Jesus College, University of Oxford.

\paragraph{Jean-Michel Rabaté} is Professor of English and Comparative Literature at
the University of Pennsylvania, co-editor of the \emph{Journal of Modern
Literature}, co-founder of Slought: Public Trust Foundation, is a fellow
of the American Academy of Arts and Sciences. He has authored or edited
fifty books on modernism, psychoanalysis, philosophy and literary
theory. Recent titles include \emph{Beckett and Sade} (Cambridge
University Press, 2020), \emph{Rires prodigues. Rire et jouissance chez
Marx, Freud et Kafka} (Éditions Stilus, 2021), \emph{Joyce, hérétique et
prodigue} (Éditions Stilus, 2022), \emph{Lacan l'irritant} (Éditions
Stilus, 2023), and \emph{Lacan and Psychoanalytic Obsolescence: The
Importance of Lacan as Irritant} (Routledge, 2024).

\paragraph{Peter Robinson} is Professor of English at the University of
Saskatchewan. He has developed several computer-based tools for the
creation of scholarly editions, and collaborated in the development of
standards for digital resources. He has published and lectured on
computing and textual editing, on text encoding, digitization, on
electronic publishing, and on Geoffrey Chaucer's \emph{Book of the Tales
of Canterbury}. As well as his own editions of Old Norse and Middle
English texts, he has collaborated with other scholars on the
publication of editions of collections of historical documents, Armenian
texts, the Greek New Testament and Dante's \emph{Monarchia} and
\emph{Commedia}.

\paragraph{Jorge Manuel Gomes da Silva Rocha} (1952) graduated from the Faculty of Fine Arts of Lisbon and PhD in History, Art and Archaeology from the \textit{Université Libre de Bruxelles} where he was a member of the \textit{Groupe de Recherche en Histoire Médiévale (GRHM)}. He taught Art and Art History in Macau, Belgium and Lisbon for more than forty years. He is currently a cooperating professor at \textit{Universidade Lusófona}.

\paragraph{Stefano Rosignoli} received an MA in Modern Literature (2006) and an MPhil in Publishing Studies (2008) from the University of Bologna. He worked in publishing for several years before focusing on his PhD, which he completed at Trinity College Dublin (2024). In 2018, he was a James Joyce visiting fellow and J-1 short-term scholar at the Humanities Institute, State University of New York at Buffalo, and a visiting research scholar at Rare and Manuscript Collections, Cornell University Library, Cornell University. His academic training is grounded in textual studies at large, from philology to genetic criticism, integrated with formalism, structuralism and the semiotics of the text, and his main field of enquiry is the philosophical exogenesis of Irish literature in English. He has recent or forthcoming publications dedicated to Samuel Beckett, T.S. Eliot and James Joyce; he teaches modern literature and cinema at Trinity College and University College, in Dublin; and he serves as review editor for \emph{Variants: The Journal of the European Society for Textual Scholarship}.


\paragraph{Paulius V. Subačius} is Professor of the theory of literature at Vilnius University. He is the author/editor of seventeen books in fields of literature, history, textual scholarship, religion, and academic politics. Among them are the first \emph{Guide on Textual Scholarship} in Lithuanian and several critical editions, printed and digital, of diaries, letters, sermons and poetry of Lithuanian authors. In his approach to textual genesis and variation, he seeks to conciliate textual scholarship and literary interpretation from a historical, sociological and religious standpoint.

\paragraph{Dirk Van Hulle} is Professor of Bibliography and Modern Book History at the University of Oxford, director of the Oxford Centre for Textual Editing and Theory (OCTET) and of the Centre for Manuscript Genetics at the University of Antwerp. He co-directs the \emph{Beckett Digital Manuscript Project} (\url{www.beckettarchive.org}) and is a curator of the Bodleian exhibition Write Cut Rewrite (2024-25). His publications include \emph{Textual Awareness} (2004), \emph{Samuel Beckett’s Library} (2013, with Mark Nixon), \emph{Modern Manuscript} (2014) and \emph{Genetic Criticism} (OUP, 2022).

\paragraph{Jason Wiens} is Professor in the Department of English and Associate Dean, Student Success in the Faculty of Arts at the University of Calgary in Calgary, Canada. He has published widely on Canadian literature, archival studies, and contemporary poetry, and is a member of the SpokenWeb partnership, a Canadian network of researchers exploring critical, pedagogical, and archival approaches to literary audio.

\paragraph{Gabriele Wix} lectured at the University of Bonn, Department of German, Comparative and Cultural Studies until the summer semester 2022, focusing on the interface between art and literature of the 20\textsuperscript{th} and 21\textsuperscript{st} century. She obtained her PhD in 2009. She is a member of the board of ESTS, and member of AGE. She is also the curator of exhibitions on international artists’ books and on writing processes. Her latest book publications include: \textit{Lawrence Weiner. Sail on.} (2024); \textit{Max Ernst, Die Schriften} (ed.; 2022); Thomas Kling, \textit{Werke in vier Bänden} (ed. of vol. 1; 2020).