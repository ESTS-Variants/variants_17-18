
%%%%%%%%%%%%%%
%% METADATA %%
%%%%%%%%%%%%%%

\contributor{
% Add all authors
Carlotta Defenu
}

\contribution{
% Add full title
The Revision Process in ``Hora Absurda''
}

\shortcontributor{
% short version of authors for running header
Carlotta Defenu
}

\shortcontribution{
% short version of title for running header
The Revision Process in ``Hora Absurda''
}

\begin{paper}
\renewcommand*{\pagemark}{}

\begin{abstract}
% write your abstract here
Due to the crisis of conscience which affected the transition between the nineteenth and the twentieth centuries in Western Europe, artistic expression in modernist literature often moved away from the idea of a unitary existence to represent the heterogeneity of the human nature.
In the Portuguese literary context, one of the most emblematic concretizations of self-fragmentation was, doubtlessly, Fernando Pessoa’s heteronomy. 
By creating multiple fictitious personalities, each with its own artistic individuality, Pessoa sought to convey the multiplicity of human experience, following some earlier orthonymous work (i.e., signed under his own name), oriented to the artistic expressions of his heterogeneity. 
Such is the case of ``HORA ABSURDA'', a poem published in the journal \textit{Exílio} in 1916, whose genesis will be the object of reflection in the present paper. 
The relevance of the poem’s genetic process resides in the numerous textual variants noticeable in the manuscript, which show an intense writing and revision work carried out to the end of aptly articulating the annihilation of self-unity. The genetic analysis will be focused on the observation of those variants which modify the expression of the subject, in order to cast light on the process through which the self’s fragmentation is expressed. 
\end{abstract}

\section{Introduction} 
\textsc{The aim of this paper} is to present a study of the genesis of the poem
``Hora Absurda,'' by Fernando Pessoa (1888--1935).

The interest in the poem's compositional process lies in the intrinsic
characteristics of the emendations, which can be observed in the
autograph manuscript: some of the authorial corrections in the
manuscript of ``Hora Absurda,'' in fact, show structural similarities,
modifying a specific category of words through a similar procedure; in
this sense, by representing the concretization of a specific modifying
intention, those interventions seem to testify the authorial will to
introduce a specific poetic expression along the text. Furthermore, the
authorial corrections that will be analyzed in this paper appear to reflect a
particular aspect of the author's poetics, offering the possibility to
establish a concrete interdependence between Fernando Pessoa's writing
practice and the main aspects of his artistic personality.

Considering the distinctive features of the emendations found in the
poem's manuscript, it is possible to infer that the poet had a specific
expressive aim that guided him during the revision of his composition.
In other words, this article will attempt to demonstrate that there is a
correlation between the corrections made by Fernando Pessoa during the
genetic process of his poem ``Hora Absurda'' and the most characteristic
elements of the author's poetics in the period in which the poem was
written and published.

Therefore, this research, which falls within the area of genetic
criticism, will focus on the more structured and thoughtful aspects of
poetic composition. However, it is important to note that this does not
imply a devaluation of the significance of poetic inspiration in Fernando
Pessoa's writing process, which is in fact a fundamental element of
every creative act. Rather, this study seeks to highlight the fact that
it is possible to support the hypothesis that, after a moment of
inspired writing, Pessoa devoted himself to a moment of careful and
calculated revision of his poem. This hypothesis appears to be confirmed
by one of the most discussed aspects of Fernando Pessoa's poetic
production: the strong imbalance between the poems written by the poet
and those that were actually published during his lifetime.

In this context, the analysis of the genetic process of ``Hora Absurda''
would substantiate the image of Fernando Pessoa as a hyper-rational
author, who, in the vast majority of cases, did not publish his works
until they had been subjected to a lengthy and meticulous revision
process.

Before analyzing the genetic process of ``Hora Absurda'', it is deemed
appropriate to provide a brief overview of the most relevant features of
Fernando Pessoa's work and poetics, particularly during the 1910s
decade, when the author wrote and published his poem ``Hora Absurda''.

\section{Main aspects of Fernando Pessoa's poetics}

Today, the Portuguese poet Fernando Pessoa is considered one of the most
representative authors of Portuguese literature and Modernist literature
in general; he actively participated in the modernist literary life in
Portugal during the first decades of the twentieth century, also co-founding
the literary magazine \emph{Orpheu}, which hosted the most influential
voices of Portuguese modernism, and whose name has marked an entire
literary generation in Portugal.\footnote{The first generation of
  modernist artists in Portugal is commonly referred to as the ``orpheu
  generation'' \parencite{martins_orpheu_1994}.}

Scarcely acclaimed while the author was still alive, Fernando Pessoa's
work later became highly celebrated inside and outside Portuguese
national borders, especially for the poet's capacity to embody the
spirit of his own literary era, which reacted to the nostalgic
atmosphere of Romantism by promoting the creation of a new literary
path.

Pessoa's work strongly reflected the crisis of conscience which affected
the artistic scene during the transition between the nineteenth and
twentieth centuries in Western Europe; as a result of the rejection of
the Romantic era's ideals, which were characterized by a strong
individualism and the overevaluation of the author's role in art,
modernist literature often moved away from the idea of a unitary
existence in order to represent the heterogeneity of the human nature.

Fernando Pessoa sought to convey the multiplicity of human experience 
through the literary invention of the heteronymy; through this
literary device, which soon became one of the most emblematic aspects of
his literary activity, the Portuguese poet created multiple fictitious
personalities, each one with its own artistic individuality, whose
literary production appeared alongside the works which Pessoa
signed with his own civil name (and which is referred to as the
``orthonymic production'', as opposed to the ``heteronymic
production'').\footnote{For an overview on Pessoa's heteronyms, see \cite{zenith_pessoa_2021} and \cite{pessoa_eu_2013}}

Heteronyms must not be mistaken with pseudonyms: while a pseudonym is
merely a different name adopted by authors to sign their own work, a
heteronym work is, according to the definition given by Pessoa himself
in his \emph{Tábua Bibliographica}, ``[a obra] do auctor fora da sua
pessoa'' [the work by the author outside his own person]
\parencite{pessoa_tabua_1928}. Pessoa's heteronyms are fictitious characters that have arisen from the
creative mind of the poet; they are invented, made-up, they do not exist
in the tangible world, but they are, in fact, full-fledged artistic
personalities, each one with its own idiosyncrasies and with its own
literary expression.

Pessoa's tendency to create fictitious personalities dates back to his
infancy, when he invented an imaginary pen pal named \emph{Chevalier de
Pas,} who could be considered Pessoa's first heteronym;\footnote{A
  valuable document for exploring the birth of heteronymy is the famous
  letter written by Fernando Pessoa to his friend, the poet Adolfo
  Casais Monteiro, in 1935 \citep[251--262]{pessoa_cartas_1998}. This document serves
  as a fundamental basis for the mythology linked to Pessoa's heteronymy
  and offers an invaluable source for the interpretation of the
  heteronymy phenomenon from Pessoa's own perspective \citep{sousa_nos_2015}.
  Although some of the information provided in the letter is likely the
  result of Pessoa's tendency to romanticise reality, the narration
  regarding Pessoa's inclination to create fictitious personalities
  since his childhood seems reliable: this fact is corroborated by some
  documentary sources preserved in the author's archive and private
  library, such as the volume \emph{The floral birthday book,} in which
  an annotation made by Pessoa himself references to the proto-heteronym
  Chevalier de Pas (the document is available online at
  \url{www.casafernandopessoa.pt}).} however, the public appearance of
Pessoa's heteronymic work did not occur until 1915, when the heteronym
Álvaro de Campos published in the journal \emph{Orpheu} the poem ``Ode
Triunfal''. Since then, Pessoa's heteronyms carry on their own literary
activities alongside with Pessoa himself.

Heteronymy is, doubtlessly, the most emblematic representation of
modernist literature in Pessoa's work; however, even before the first
heteronymic poem was published, Pessoa had already dedicated a great
deal of his literary activity to the theorization of another distinctive
feature of his production, which was also an emblematic expression of
Portuguese's modernism: the ``literary isms''.

Pessoa's \emph{isms} are various literary tendencies which the poet
idealized and theorized with the aim of renewing Portuguese poetry. The
first concrete formulation of these literary \emph{isms} dates back to
1912 when Pessoa published in the journal \emph{A Águia} a series of
three articles, in which he argued for the necessity of rejuvenating
Portuguese literature, providing details on what the main features of
the new Portuguese poetry should have been.

A large number of documents written between 1913 and 1917, currently
kept in Fernando Pessoa's archive at Portugal's National Library in
Lisbon, testify to the theorization of those literary tendencies, which
Pessoa invented with the aim of reviving Portuguese's literary context,
following that urgency of establishing new forms of art which
distinguished the Modernist era. These documents, which were published
posthumously,\footnote{A comprehensive edition of all the documents
  related to Pessoa's literary \emph{isms} was compiled by Jeronimo
  Pizarro and published in 2009 by the Imprensa Nacional-Casa da Moeda,
  in Lisbon \parencite{pessoa_sensacionismo_2009}.} provide detailed insight into the founding features of
Pessoa's \emph{isms}, whose nature was not only literary, but also
social, philosophical, or even religious.

In 1914, Pessoa published the poem ``Pauis'', which marked the public
debut of his literary \emph{isms} project. The poem ``Pauis'' was
conceived and published by Pessoa as the manifesto of his first literary
\emph{ism}, whose name, \emph{Paulismo,} directly derives from the
poem's title. The text of ``Pauis'' is the poetic embodiment of the
three tenets which Pessoa had outlined in his 1912 article as the
fundamental principles of the new Portuguese poetry (``vagueness,
subtlety and complexity''):

\begin{quote}
 Perscrutemos qual a \emph{estética} da nova poesia portuguesa. A
primeira constatação analítica que o raciocínio faz ante a nossa poesia
de hoje é que o seu arcaboiço espiritual é composto de três elementos
-- \emph{vago}, \emph{subtileza} e \emph{complexidade}. São
\emph{vagas}, \emph{subtis} e \emph{complexas} as expressões
características do seu verso, e a sua ideação é, portanto, do mesmo
triplo carácter.

\vspace{1em}

[Let us explore the aesthetic of the new Portuguese poetry. The first
analytical observation that reason makes of our poetry today is that its
spiritual framework is composed of three elements –– vagueness, subtlety
and complexity. The characteristic expressions of its verse are vague,
subtle and complex, and its ideation is therefore of the same threefold
character.]

\begin{flushright}
    \parencite[90]{pessoa_nova_1912}
\end{flushright} 
\end{quote}

\noindent From to Pessoa's perspective, \emph{Paulismo} represented an
evolution of symbolism, inheriting its nature of a ``subjective poetry''
and its poetics of vagueness and subtlety, and implementing a third
element of complexity \parencite[91]{pessoa_nova_1912}. This element of complexity will
be one of the distinctive traits of Pessoa's poems bounded to the
poetics of literary \emph{isms}, and its poetic expression is
characterized by the presence of rhetorical figures, such as metaphor,
synaesthesia, or oxymoron, which violate the utterance's logical
sequence and the syntax's norms, thus creating an effect of oddness \parencite[22]{morna_poesia_1982}.

The three fundamental principles described by Pessoa as the three tenets
of the new Portuguese poetry, and associated with \emph{Paulismo}'s
aesthetic, will be adopted for the theorization of subsequent literary
\emph{isms} and will become a distinctive feature of Pessoa's poetry
throughout his literary activity. As a matter of fact, whereas at the
beginning of his literary project Pessoa referred to his literary
\emph{isms} as three distinct literary movements, over time the
distinction between the various literary \emph{isms} became blurred, and
the fundamental basis of \emph{Paulismo} will be integrated into the
following literary tendencies. Consequently, the second literary
\emph{ism} conceived by Pessoa, \emph{Interseccionismo,} can be seen as
an evolution of \emph{Paulismo,} purged from its strongest symbolist
reminiscence and closer to the modernist European movements.

\emph{Interseccionismo} was a literary aesthetic that drew influence
from European avant-garde movements, such as Cubism and Futurism. As its
name suggests, \emph{Interseccionismo}'s poetics was based on the idea
of intersection between a ``estado de alma'' [state of the soul] and
a ``paisagem'' [landscape], that is, the intersection between an
object and the sensation it evokes. This principle of intersection is
similar to the creative principle of avant-garde experiences, which was
based on overlapping distinct planes of the reality. Various documents
kept in Pessoa's archive testify to the poet's desire to publish a
manifesto of \emph{Interseccionismo}, though it was never completed nor
published. In one of these documents, Pessoa announces the founding
principle of \emph{Interseccionismo} and its connection to the
avant-garde movements:

\begin{quote}
Ora a Arte busca a Sensação em absoluto. Mas a sensação, como vimos,
compõe-se do Objecto sentido e da Sensação propriamente tal.

Intersecção do Objecto comsigo proprio: cubismo. (Isto é, intersecção dos
varios aspectos do mesmo Objecto uns com os outros).

Interseccção do Objecto com as idéas objectivas que suggere: Futurismo.

Intersecção do Objecto com a nossa sensação d'elle: Interseccionismo,
propriamente dito; o nosso.

\vspace{1em}

[Now, Art seeks the Sensation in absolute terms. But sensation, as we
have seen, is composed of the sensed Object and the Sensation itself.

Intersection of the Object with itself: cubism (i.e., intersection of
the various aspects of the same Object with each other).

Intersection of the Object with the objective ideas it suggests:
Futurism.

Intersection of the Object with our sensation of it: Intersectionism,
the actual one; ours.]  

\begin{flushright}
    \parencite[122]{pessoa_sensacionismo_2009}
\end{flushright}
\end{quote}


\noindent Therefore, as Maria Aliete Galhoz pointed out, the theorization of
\emph{Interseccionismo} represented the desire to introduce the
Portuguese literature in the context of European avant-garde \parencite[XLI]{galhoz_o_1980}.

Subsequently, both \emph{Paulismo} and \emph{Interseccionismo} were
incorporated into the last literary \emph{ism} idealized by Pessoa,
\emph{Sensacionismo,} which represented a progression and a synthesis of
all the literary \emph{isms} previously theorised.\footnote{The special
  relevance of \emph{Sensacionismo,} when compared to the other
  \emph{isms} conceived by Pessoa, is widely recognized and it has been
  declared by the author himself in an unpublished fragment, originally
  in English: ``The sensationism movement (represented by the Lisbon
  quarterly `Orpheu') represents the final synthesis. It gathers into
  one organic whole (for a synthesis is not a sum) the several threads
  of modern movements, extracting honey from all the flowers that have
  blossomed in the gardens of European fancy'' \parencite[159]{pessoa_sensacionismo_2009}.} This
is evidenced by the fact that in some documents preserved in Pessoa's
archive, which are related to the theorisation of this last literary
movement, the author refers to the earlier \emph{isms} as ``species'' of
\emph{Sensacionismo}, rather than autonomous full-fledged literary
tendencies \parencite[149]{pessoa_sensacionismo_2009}.

Furthermore, in the documents related to the conception of
\emph{Sensacionismo}, it is possible to discern the emergence of the
first ideas related to self-plurality; thus, from a certain perspective,
some of the precepts of \emph{Sensacionismo} appear to be closely linked
to the poetics of heteronymy \parencite[9]{pessoa_sobre_2015}.

The founding principle of Sensationism is the identity between reality
and sensation, as Pessoa expressed in an unpublished fragment: ``todo o
objecto é uma sensação nossa'' [every object is a sensation of our
own] \parencite[145]{pessoa_sensacionismo_2009}. This means that sensation is not merely a
result of the observation of the external world but is an integral part
of the exterior reality.\footnote{In a famous, yet unpublished, text,
  titled \emph{A arte da quarta dimensão} [\emph{The art of the fourth
  dimension}] Pessoa describes sensations as a dimension of the
  objects: ``Mas se as coisas existem como existem apenas porque nós
  assim as sentimos, segue que a `sensibilidade' (o poder serem
  sentidas) é uma quarta dimensão d'ellas'' [But if things exist as
  they do only because we feel them so, it follows that 'sensibility'
  (the fact that they can be felt) is a fourth dimension of them]
  \parencite[148--149]{pessoa_sensacionismo_2009}.}

From this fundamental principle, Pessoa develops two consequent tenets,
which anticipate the idea of self-multiplicity bounded to Sensationism:

\begin{quote}
 \begin{enumerate}
\def\labelenumi{\arabic{enumi}.}
\item
  A única realidade é a sensação
\item
  A máxima realidade será dada sentindo tudo de todas as maneiras (em
  todos os tempos).
\item
  Para isso era preciso ser \emph{tudo} e \emph{todos}
\end{enumerate}

\begin{center}
\begin{equation*}
 \left[\begin{tabular}{p{.835\textwidth}}
    \text{1.\quad The only reality is sensation}\\
    \text{2.\quad The ultimate reality will be achieved by feeling everything in every way} \\
    \text{\qquad (at all times)}\\
    \text{3.\quad To do that, it was necessary to be \emph{everything} and \emph{everyone}}\\
    \end{tabular}\right]
\end{equation*}
\end{center}

\begin{flushright}
    \parencite[149]{pessoa_sensacionismo_2009}
\end{flushright}
\end{quote}

\noindent Here, the utterance ``ser \emph{tudo} e \emph{todos}'' [to be \emph{everything} and \emph{everyone}] foreshadows the universe
of fictitious personas that will form Fernando Pessoa's heteronymic
project; this demonstrates that Pessoa's \emph{isms} are a fundamental
starting point for the development of the author's poetics. In the
following paragraphs, I will explore how Pessoa's \emph{isms} guide and
influence his poetic production, especially through the analysis of the
genetic process of the poem ``Hora Absurda''.

\section{The genesis of ``Hora Absurda''}

The poem ``Hora Absurda\emph{''}, published in 1916 in the Portuguese
journal \emph{Exílio}, is described by Fernando Pessoa as a ``poema
interseccionista'' \citep[116]{pessoa_sensacionismo_2009}. The manuscript of the poem was
composed in 1913, when the author's literary activity was strictly bound
to the project of literary \emph{isms}.

The poem's structure is composed of a succession of metaphoric images
through which the sensations of the speaking \emph{I} (or those of the
interlocutor) establish a correlation with the exterior reality, as
exemplified by the first two stanzas of the poem:

\begin{quote}
\begin{minipage}{.9\textwidth}
O teu silêncio é uma nau com todas as velas pandas\ldots{}

Brandas, as brisas brincam nas flâmulas, teu sorriso\ldots{}

E o teu sorriso no teu silêncio é as escadas e as andas

Com que me finjo mais alto e ao pé de qualquer paraíso\ldots{}

Meu coração é uma ânfora que cai e que se parte\ldots{}

O teu silêncio recolhe-o e guarda-o, partido, a um canto\ldots{}

Minha ideia de ti é um cadáver que o mar traz à praia\ldots{}, e entanto

Tu és a tela irreal em que erro em cor a minha arte\ldots{}
\end{minipage}

\vspace{1em}

\begin{minipage}{.9\textwidth}    
[Your silence is a ship, each sail taut in the wind\ldots{}

Softly the breezes play in the pennants, your smile\ldots{}

And your smile in your silence is the stairway and stilts

I use to pretend to be up higher, at the foot of some Paradise\ldots{}

My heart's an amphora that falls and breaks\ldots{}

Your silence gathers and guards the shards in a corner\ldots{}

My idea of you is a corpse the sea hauls ashore\ldots{}, which makes you

The unreal canvas where I wander through color my art\ldots{}]

\begin{flushright}
    \parencite[134]{pessoa_poems_1986}
\end{flushright}
\end{minipage}
\end{quote}


\noindent The poem is made up of 25 stanzas, and almost each line follows the same
semantic structure: the first part of the line expresses an
emotion, or a feeling, which may or may not be expressed metaphorically,
while the second part of the line presents an element of the exterior
reality associated with the previously expressed sensation. As Caio
Gagliardi pointed out, this ``intersectionist syntax'' \parencite[97]{gagliardi_fernando_2005} is
used to poetically express the intersection of a ``state of the soul''
and the ``landscape'' that Pessoa had established as the cornerstone of
intersectionist aesthetic.

It is also worth noting that, in most cases, the part of the verse
dedicated to expressing subjective sensations is composed of a
synecdoche; this means that, when expressing the feelings of the
speaking \emph{I} or those of the interlocutor, the poet never refers to
them as whole, entire, unities, but rather only to a part of them (v. 3
``your smile''; v. 5 ``my heart''). Thus, the frequent use of
synecdoches serves as a means of expression through which Fernando
Pessoa conveys the idea of the fragmentation of the self, of a ``soul
broken into pieces'' \parencite[152]{pessoa_sensacionismo_2009}, which was, in fact, part of the
intersectionist program.\footnote{As Clara Rocha has noted, the
  intersectionist aesthetic is characterized by the presence of an
  ``interseccionismo pessoal'', which conveys the idea of the
  fragmentation of the self and the intersection of its various voices,
  and which will later give rise to Fernando Pessoa's heteronymic
  project \parencite[259]{rocha_revistas_1985}.}

\begin{figure}
    \centering
    \frame{\includegraphics[width=.75\textwidth]{media/defenu1.png}}
    \caption{First page of Fernando Pessoa's ``Hora Absurda'' manuscript. Fernando Pessoa's Archive. National Library of Portugal. BNP/E3~117--1\textsuperscript{r}.}
    \label{fig:defenu1}
\end{figure}

In many cases, this unique way of expressing the idea of a self divided
into its own sensations appears in the manuscript as a first attempt at
writing. Upon examining the first page of the manuscript, it is possible
to notice that the number of corrections introduced by the author is
scarce (see Fig. \ref{fig:defenu1}).

\begin{figure}[H]
    \frame{\includegraphics[width=\textwidth]{media/defenu2.png}}
    \begin{center}
    \begin{tabular}{p{.9\textwidth}}
        \hspace{4.5em} \textsubscript{alma} \\
        E \sout{ha} minha é aquella luz que não mais haverá \\
        \hspace{13em} nos candelabros\ldots{} \\
    \end{tabular}
    \end{center}
    
    \begin{center}
    \begin{equation*}
	\left[\begin{tabular}{p{.9\textwidth}}
   		\text{\hspace{4em}\textsubscript{my soul}} \\
        \text{And \sout{mine} is the light that will never be again} \\
        \text{\hspace{12em}in the candelabrums\ldots{}} \\
	\end{tabular}\right]
    \end{equation*}
    \end{center}

\subfigure[Line 39]{
    \centering\small
    \hspace{\textwidth}
    \label{fig:defenu2a}
}
    \frame{\includegraphics[width=\textwidth]{media/defenu3.png}}
    \begin{center}
    \begin{tabular}{p{.9\textwidth}}
        \hspace{.2em}\textsuperscript{O meu} amar-te é uma catedral de silêncios eleitos\ldots.\\
        \textsuperscript{\sout{A minha}}\\
    \end{tabular}
    \end{center}
    
    \begin{center}
    \begin{equation*}
	\left[\begin{tabular}{p{.9\textwidth}}
   		\text{\hspace{.2em}\textsuperscript{My} love for you is the cathedral of the elected sciences\ldots.}\\
        \text{\textsuperscript{\sout{Mine}} \hspace{5.5em}} \\
	\end{tabular}\right]
    \end{equation*}
    \end{center}
    
\subfigure[Line 75]{
    \centering\small
    \hspace{\textwidth}
    \label{fig:defenu2b}
}
    \caption{Fernando Pessoa's ``Hora Absurda'' manuscript; lines 39 and 75.
    \protect\footnotemark} 
    \label{fig:defenu2}
\end{figure}\footnotetext{In the published version, these lines respectively read: ``E a minha alma é aquella luz que não mais haverá nos candelabros\ldots{}'' [And my soul is the light that will never be again in the candelabrums\ldots) \mancite\parencite[42; line 39]{pessoa_mensagem_2018}, and ``O meu amar-te é uma cathedral de silencios eleitos\ldots'' [My love for you is a cathedral of elected silences\ldots] \parencite[43; line 75]{pessoa_mensagem_2018}. Unless otherwise stated, all the genetic
  transcriptions and quotes from the published versions of the poems are
  retrieved from the critical-genetic edition of \emph{Mensagem e Poemas
  Publicados em Vida} \parencite{pessoa_mensagem_2018}.}


The number of authorial interventions, however, increases in the
following sheets. The ink's typology and the calligraphy's graphic
traits remain unaltered throughout the manuscript, suggesting that the
text was composed in a limited period of time. The emendations' position
within the manuscript indicates that most of the corrections were
introduced by the author after the first redaction of the
text,\footnote{Almuth Grésillon designed this typology of variants
  ``Variantes de lecture'' \parencite[112]{gresillon_les_1988}.} during a phase of \emph{labor
limae}, in which the Portuguese poet corrected portions of the text
which he found unsatisfying from an expressive point of view, without
distorting the poem's textual content. The corrections made by the
author appear to have been intended to adapt the poetic expression of
some portions of the text to the overall tone of the composition.

Furthermore, there is a certain consistency in terms of the textual
content and the modality of alteration with regards to a particular
group of variants, which seem to be closely related to the poetics of
sensationism and self-fragmentation. These alterations display analogous
traits resulting from the same correction modality: the reference to an
entity as an entire unity is erased and replaced by a synecdoche, which
introduces a reference to the same entity by referring to just a part of
it.

For instance, in lines 39 and 75 of the manuscript, Pessoa replaced the
possessive pronoun ``mine'' with two different synecdoches. In line 39,
``And mine is the light that will never be again in the candelabrums''
is changed to ``my soul is the light that will never be again in the
candelabrums''; in line 75, ``mine is the cathedral of elected silence''
becomes ``my love for you is the cathedral of elected
silence'' (see Fig. \ref{fig:defenu2}).

In these examples, Pessoa adopted a process of revision at the beginning of the line, in order to
maintain the same kind of structure we
saw in the first two stanzas of the poem: that is achieved by replacing
a word which specifically represents a whole speaking \emph{I} (such as
the possessive pronoun ``mine'') with a phrasing referring just to a
specific part of the same speaking \emph{I}. It is also worth mentioning
that the words that Pessoa introduced to build these synecdoches (``my
soul'', ``My love for you'') belong to the field of sensitiveness,
referring to a feeling (such as in the case of ``love'') or to the site
where feelings reside (such as in the case of ``soul''). Sometimes, the fragmentation process implemented during the reviewing
phase is made by introducing  a reference to a physical part
of the speaking \emph{I}: for example, in line 50, Pessoa replaces the
word ``me'' with ``my feet'' (see Fig. \ref{fig:defenu3}).

This process of fragmentation modifies not only the expression of the
speaking \emph{I}, but also of other entities, as in line 45
and 79 (see Fig. \ref{fig:defenu4}). There, references to whole entities are replaced by synecdoches: thus the
``peacocks'' in line 45 become ``tales of peacocks'', and the pronoun
``you'', in line 79, is replaced by a phrasing which, once again, is
bounded to the expression of a feeling (``your boredom'').

In every correction that we could observe so far, the emendation process
is the same: the author erases a word that refers to a living being as a
whole and replaces it with a reference to the same entity by means of a
synecdoche, by referring just to a part of it, hence contributing to
convey the idea of fragmentation. This type of variants can be included in a category proposed by the
Italian critic Gianfranco Contini as ``correzioni che rinviano a luoghi
esorbitanti dell'opera dell'autore'' \parencite[72]{contini_esercizi_1974}, i.e., corrections that
refer to a distinctive feature of an author's work.

It is also worth mentioning that the analysis of the genetic process of
other poems written by Fernando Pessoa in the same period as ``Hora
Absurda'', reveals a similar \emph{modus operandi}; a case in point is
represented by the poem ``Ella canta, pobre ceifeira'', written in 1914.
In the poem's manuscript (document no. 117--42a), line 18, the expression
``ouvi-te quase a chorar'' [I heard you almost crying] is replaced
by ``o que em mim ouve está chorando'' [in me what hears is crying];
in line 27, the pronoun ``me'' in the utterance ``make me your
weightless shadow'' is replaced by the synecdoche ``my soul'': ``make my
soul your weightless shadow'' (see Fig. \ref{fig:defenu5}).

\begin{figure}[H]
    \frame{\includegraphics[width=\textwidth]{media/defenu4.png}}
    \begin{center}
    \begin{tabular}{p{.9\textwidth}}
        As relvas de todos os prados foram frescas \sout{sob mim} \\
        \hspace{20em}\textsubscript{pés frios} \\
        \hspace{18em} sob \hspace{.5em} meus \\
    \end{tabular}
    \end{center}

    \begin{center}
    \begin{equation*}
    	\left[\begin{tabular}{p{.9\textwidth}}
       		\text{All the meadows grasses were fresh beneath \sout{me}} \\
        	\text{\hspace{18.5em}\textsubscript{cold feet}} \\
            \text{\hspace{15em} beneath \hspace{.5em} my}
    	\end{tabular}\right]
    \end{equation*}
    \end{center}

    \caption{Fernando Pessoa's ``Hora Absurda'' manuscript; line 50.\protect\footnotemark} 
    \label{fig:defenu3}
\end{figure}\footnotetext{In the published version, this line  reads: ``As relvas de todos os prados foram frescas sob meus pés frios\ldots{}'' [All the meadows grasses were fresh beneath my cold feet\ldots] \parencite[42; line 50]{pessoa_mensagem_2018}}
\null
\vfill

\newpage
\null
\vfill

\begin{figure}[H]
    \frame{\includegraphics[width=\textwidth]{media/defenu5.png}}

    \begin{center}
    \begin{tabular}{p{.9\textwidth}}
        \hspace{3.5em}\textsubscript{caudas de pavões todas olhos nos jardins de outr'ora}  \\
        Já não há pavões nos jardins de outr'ora\ldots{}\\
    \end{tabular}
    \end{center}
    
    \begin{center}
    \begin{equation*}
	\left[\begin{tabular}{p{.9\textwidth}}
   		\text{\hspace{5em}\textsubscript{tails of peacocks all eyes left in the gardens of yore}} \\
        \text{There are no peacocks  left in the gardens of yore} \\
	\end{tabular}\right]
    \end{equation*}
    \end{center}

\subfigure[Line 45]{
    \centering\small
    \hspace{\textwidth}
    \label{fig:defenu4a}
}
    \frame{\includegraphics[width=\textwidth]{media/defenu6.png}}
    \begin{center}
    \begin{tabular}{p{.9\textwidth}}
        \hspace{1em}\textsubscript{t\sout{u}}\scriptsize eu \small\textsubscript{tedio é}\\
        Oh, \sout{és} statua de uma mulher que ha de vir,\\
    \end{tabular}
    \end{center}
    
    \begin{center}
    \begin{equation*}
	\left[\begin{tabular}{p{.9\textwidth}}
   		\text{\hspace{1em} \textsubscript{you}\scriptsize{r} \small \textsubscript{boredom is}}\\
        \text{Oh, \sout{are} a statue of a woman still to come,}\\
	\end{tabular}\right]
    \end{equation*}
    \end{center}

\subfigure[Line 79]{
    \centering\small
    \hspace{\textwidth}
    \label{fig:defenu4b}
}
    \caption{Fernando Pessoa's ``Hora Absurda'' manuscript; lines 45 and 79.\protect\footnotemark} 
    \label{fig:defenu4}
\end{figure}
\footnotetext{In the published version, these lines respectively read: ``Já não ha caudas de pavões todas olhos nos jardins de outr'ora\ldots{}'' [There are no tales of peacocks left in the gardens of yore\ldots] \mancite\parencite[42; line 45]{pessoa_mensagem_2018}, and ``Ah, o teu tedio é uma estatua de uma mulher que ha de vir,'' [Ah, your boredom is a statue of a woman still to come] \parencite[43; line 79]{pessoa_mensagem_2018}.}
\null
\vfill

\newpage
\null
\vfill

\begin{figure}[H]
    \frame{\includegraphics[width=\textwidth]{media/defenu7.png}}
    \begin{center}
    \begin{tabular}{p{.9\textwidth}}
        \hspace{1em}\textsuperscript{O que em mim ouve} \hspace{.5em}\textsubscript{está \hspace{1em}chorando} \\
        Estão \hspace{1em} \textsuperscript{\sout{Ouvi-te}} quase \sout{a chorar}\ldots.\\
    \end{tabular}
    \end{center}
    
    \begin{center}
    \begin{equation*}
	\left[\begin{tabular}{p{.9\textwidth}}
        \text{\hspace{1em}\textsuperscript{In me what hears} \hspace{3.5em}\textsubscript{is \hspace{1em}crying}} \\
        \text{They are \textsuperscript{\sout{I heard you}} almost \sout{crying}\ldots.}\\
	\end{tabular}\right]
    \end{equation*}
    \end{center}
\subfigure[Line 18]{
    \centering\small
    \hspace{\textwidth}
    \label{fig:defenu5a}
}

    \frame{\includegraphics[width=\textwidth]{media/defenu8.png}}
    \begin{center}
    \begin{tabular}{p{.9\textwidth}}
        \hspace{1em} \textsuperscript{Minh'alma}\\
        Que eu seja a vossa sombra leve\\
    \end{tabular}
    \end{center}
    
    \begin{center}
    \begin{equation*}
	\left[\begin{tabular}{p{.9\textwidth}}
        \text{\hspace{2em} \textsuperscript{My soul}}\\
        \text{Make me your weightless shadow}\\
	\end{tabular}\right]
    \end{equation*}
    \end{center}


\subfigure[Line 27]{
    \centering\small
    \hspace{\textwidth}
    \label{fig:defenu5b}
}

    \caption{Fernando Pessoa's ``Ella canta, pobre ceifeira'' manuscript; lines 18 and 27.\protect\footnotemark} 
    \label{fig:defenu5}
\end{figure}
\footnotetext{This translation is
  retrieved from an edition of Pessoa's selected poems compiled and
  translated by Richard Zenith \citep{pessoa_little_2006}. In the published version, these lines respectively read ``O que em mim ouve está chorando'' [In me what hears is crying] \parencite[45; line 18]{pessoa_mensagem_2018}, and ``Minh'alma a vossa sobra leve!...'' [My soul your weightless shadow!...] \parencite[45; line 27]{pessoa_mensagem_2018}.}

Despite the fact that similar corrections as those observed in the
manuscript of ``Hora Absurda'' can be found in other manuscripts
composed by Pessoa in the same period, the frequency and modality of
this correction phenomenon is not the same as in the reviewing process
of the 1913's poem.

The particular care taken during the composition of ``Hora Absurda'' is
also revealed by the comparison between the manuscript and the published
version of the poem: it is possible to notice that, before the
publication, Pessoa corrected certain formal aspects of the text and
introduced expressive elements closely linked to the aestetics of
literary \emph{isms} (such as the addition of the initial capital letter
in words charged with symbolic meaning, the exaggerated and unusual use
of suspension dots, and punctuation in general).

The reason behind the special attention given by Pessoa to the reviewing
process of ``Hora Absurda'' could be related to the fact that this poem
was considered by the author himself as a representative text of his
poetic ideals in the period around 1914; this is corroborated by the
references to the poem made by Pessoa in the documents related to the
theorizations of the literary \emph{isms}.

Furthermore, it is important to note that between the publication of the
article regarding the new Portuguese poetry \citep{pessoa_nova_1912} and 1915, when the
first heteronymic work appears publicly and Pessoa begins to move away
from the literary \emph{isms}' project,\footnote{In a letter dated 19th
  January 1915, Pessoa declared that he was experiencing a profound
  crisis and that he intended to cease any publication related to the
  literary isms, as he no longer considered that project an honest
  representation of his poetic ideals \citep[356]{pessoa_sensacionismo_2009}. This letter
  has been subject to various interpretations, often conflicting (see,
  among others, \cite{gagliardi_uma_2004}, \cite{amado_orpheu_2015}, \cite{miraglia_dia_2020})} the number
of poems published by the Portuguese poet was extremely scarce, and a
large part of the poems composed in this period were published by
Fernando Pessoa after 1916.

The lack of correspondence between the number of written and published
works in Pessoa's lifetime, is a distinctive feature of his creative
approach in general; the scarcity of published works and the meticulous
\emph{labor limae} process that we were able to observe by analysing the
manuscript of ``Hora Absurda'' suggest that the Portuguese poet was an
extremely perfectionist writer, who felt the need to carefully and
repeatedly revise and refine his work before publishing it.

The meticulous revision process conducted by Pessoa during the genesis
of ``Hora Absurda'' resulted in a coherent and well-structured text,
which effectively expresses Pessoa's poetic ideals. The poem's
expressive homogeneity is maintained throughout its 25 stanzas, which
led George Rufold Lind to see the text of ``Hora Absurda'' as the result
of a ``pre-established compositional program'' \citep[23]{lind_teoria_1970}. However, the
inspirational factor necessary for any creative act prevents a poetic
text from being considered the mere fulfillment of a pre-established
program; the observation of the manuscript of ``Hora Absurda'' reveals
how the poem's composition is a product of combination between
inspiration and revision.

In summary, the textualization of the poetic expressions related to the
self-fragmentation and Sensationism in ``Hora Absurda'' occurs in two
stages: in an initial phase, largely guided by inspiration, where this
type of expression appears instinctively in the manuscript as a first
attempt at writing; and in a second phase, in which Pessoa revises his
text in order to refine those lines that, as they were initially
written, inadequately expressed the idea of a self fragmented
into its sensations.

The interpretation of the genetic data provided by the manuscript of
``Hora Absurda'' does not negate the importance of the inspirational
process in poetry writing, but also reveals how a careful revision
process contributes to the final form of a poem, and how a critical
interpretation of a work from a genetic perspective can shed light on
aspects of the creative process that would otherwise remain uncharted.

\section*{Acknowledgements}
This research was supported by the Fundação para a Ciência e a Tecnologia (SFRH/BD/143789/2019).

\begin{flushleft}
    % use smallcaps for author names
    \renewcommand*{\mkbibnamefamily}[1]{\textsc{#1}}
    \renewcommand*{\mkbibnamegiven}[1]{\textsc{#1}} 
\printbibliography
\end{flushleft}

\end{paper}
