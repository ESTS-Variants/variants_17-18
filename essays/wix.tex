\documentclass{article}
%%%% CLASS OPTIONS 

\KOMAoptions{
    fontsize=10pt,              % set default font size
    DIV=calc,
    titlepage=false,
    paper=150mm:220mm,
    twoside=true, 
    twocolumn=false,
    toc=chapterentryfill,       % for dots: chapterentrydotfill
    parskip=false,              % space between paragraphs. "full" gives more space; "false" uses indentation instead
    headings=small,
    bibliography=leveldown,     % turns the Bibliography into a \section rather than a \chapter (so it appears on the same page)
}

%%%% PAGE SIZE

\usepackage[
    top=23mm,
    left=20mm,
    height=173mm,
    width=109mm,
    ]{geometry}

\setlength{\marginparwidth}{1.25cm} % sets up acceptable margin for \todonotes package (see preamble/packages.tex).

%%%% PACKAGES

\usepackage[dvipsnames]{xcolor}
\usepackage[unicode]{hyperref}  % hyperlinks
\usepackage{booktabs}           % professional-quality tables
\usepackage{nicefrac}           % compact symbols for 1/2, etc.
%\usepackage{microtype}          % microtypography
\usepackage{lipsum}             % lorem ipsum at the ready
\usepackage{graphicx}           % for figures
\usepackage{footmisc}           % makes symbol footnotes possible
\usepackage{ragged2e}
\usepackage{changepage}         % detect odd/even pages
\usepackage{array}
\usepackage{float}              % get figures etc. to stay where they are with [H]
\usepackage{subfigure}          % \subfigures witin a \begin{figure}
\usepackage{longtable}          % allows for tables that stretch over multiple pages
\setlength{\marginparwidth}{2cm}
\usepackage[textsize=footnotesize]{todonotes} % enables \todo's for editors
\usepackage{etoolbox}           % supplies commands like \AtBeginEnvironment and \atEndEnvironment
\usepackage{ifdraft}            % switches on proofreading options in the draft mode
\usepackage{rotating}           % provides sidewaysfigure environment
\usepackage{media9}             % allows for video in the pdf
\usepackage{xurl}               % allows URLs to (line)break ANYWHERE

%%%% ENCODING

\usepackage[full]{textcomp}                   % allows \textrightarrow etc.

% LANGUAGES

\usepackage{polyglossia}
\setmainlanguage{english} % Continue using english for rest of the document

% If necessary, the following lets you use \texthindi. Note, however, that BibLaTeX does not support it and will report a 'warning'.
 \setotherlanguages{hindi} 
 \newfontfamily\hindifont{Noto Sans Devanagari}[Script=Devanagari]

% biblatex
\usepackage[
    authordate,
    backend=biber,
    natbib=true,
    maxcitenames=2,
    ]{biblatex-chicago}
\usepackage{csquotes}

% special characters  
\usepackage{textalpha}                  % allows for greek characters in text 

%%%% FONTS

% Palatino font options
\usepackage{unicode-math}
\setmainfont{TeX Gyre Pagella}
\let\circ\undefined
\let\diamond\undefined
\let\bullet\undefined
\let\emptyset\undefined
\let\owns\undefined
\setmathfont{TeX Gyre Pagella Math}
\let\ocirc\undefined
\let\widecheck\undefined

\addtokomafont{disposition}{\rmfamily}  % Palatino for titles etc.
\setkomafont{descriptionlabel}{         % font for description lists    
\usekomafont{captionlabel}\bfseries     % Palatino bold
}
\setkomafont{caption}{\footnotesize}    % smaller font size for captions


\usepackage{mathabx}                    % allows for nicer looking \cup, \curvearrowbotright, etc. !!IMPORTANT!! These are math symbols and should be surrounded by $dollar signs$
\usepackage[normalem]{ulem}                       % allows for strikethrough with \sout etc.
\usepackage{anyfontsize}                          % fixes font scaling issue

%%%% ToC

% No (sub)sections in TOC
\setcounter{tocdepth}{0}                

% Redefines chapter title formatting
\makeatletter                               
\def\@makechapterhead#1{
  \vspace*{50\p@}%
  {\parindent \z@ \normalfont
    \interlinepenalty\@M
    \Large\raggedright #1\par\nobreak%
    \vskip 40\p@%
  }}
\makeatother
% a bit more space between titles and page numbers in TOC

\makeatletter   
\renewcommand\@pnumwidth{2.5em} 
\makeatother

%%%% CONTRIBUTOR

% Title and Author of individual contributions
\makeatletter
% paper/review author = contributor
\newcommand\contributor[1]{\renewcommand\@contributor{#1}}
\newcommand\@contributor{}
\newcommand\thecontributor{\@contributor} 
% paper/review title = contribution
\newcommand\contribution[1]{\renewcommand\@contribution{#1}}
\newcommand\@contribution{}
\newcommand\thecontribution{\@contribution}
% short contributor for running header
\newcommand\shortcontributor[1]{\renewcommand\@shortcontributor{#1}}
\newcommand\@shortcontributor{}
\newcommand\theshortcontributor{\@shortcontributor} 
% short title for running header
\newcommand\shortcontribution[1]{\renewcommand\@shortcontribution{#1}}
\newcommand\@shortcontribution{}
\newcommand\theshortcontribution{\@shortcontribution}
\makeatother

%%%% COPYRIGHT

% choose copyright license
\usepackage[               
    type={CC},
    modifier={by},
    version={4.0},
]{doclicense}

% define \copyrightstatement for ease of use
\newcommand{\copyrightstatement}{
         \doclicenseIcon \ \theyear. 
         \doclicenseLongText            % includes a link
}

%%%% ENVIRONMENTS
% Environments
\AtBeginEnvironment{quote}{\footnotesize\vskip 1em}
\AtEndEnvironment{quote}{\vskip 1em}

\setkomafont{caption}{\footnotesize}

% Preface
\newenvironment{preface}{
    \newrefsection
    \phantomsection
    \cleardoublepage
    \addcontentsline{toc}{part}{\thecontribution}
    % enable running title
    \pagestyle{preface}
    % \chapter*{Editors' Preface}    
    % reset the section counter for each paper
    \setcounter{section}{0}  
    % no running title on first page, page number center bottom instead
    \thispagestyle{chaptertitlepage}
}{}
\AtEndEnvironment{preface}{%
    % safeguard section numbering
    \renewcommand{\thesubsection}{\thesection.\arabic{subsection}}  
    %last page running header fix
    \protect\thispagestyle{preface}
}
% Essays
\newenvironment{paper}{
    \newrefsection
    \phantomsection
    % start every new paper on an uneven page 
    \cleardoublepage
    % enable running title
    \pagestyle{fancy}
    % change section numbering FROM [\chapter].[\section].[\subsection] TO [\section].[\subsection] ETC.
    \renewcommand{\thesection}{\arabic{section}}
    % mark chapter % add author + title to the TOC
    \chapter[\normalfont\textbf{\emph{\thecontributor}}: \thecontribution]{\vspace{-4em}\Large\normalfont\thecontribution\linebreak\normalsize\begin{flushright}\emph{\thecontributor}\end{flushright}}    
    % reset the section counter for each paper
    \setcounter{section}{0}  
    % reset the figure counter for each paper
    \renewcommand\thefigure{\arabic{figure}}    
    % reset the table counter for each paper
    \renewcommand\thetable{\arabic{table}} 
    % no running title on first page, page number center bottom instead, include copyright statement
    \thispagestyle{contributiontitlepage}
    % formatting for the bibliography

}{}
\AtBeginEnvironment{paper}{
    % keeps running title from the first page:
    \renewcommand*{\pagemark}{}%                            
}
\AtEndEnvironment{paper}{
    % safeguard section numbering
    \renewcommand{\thesubsection}{\thesection.\arabic{subsection}}  
    % last page running header fix
    \protect\thispagestyle{fancy}%                              
}
% Reviews
\newenvironment{review}{
    \newrefsection
    \phantomsection
    % start every new paper on an uneven page 
    \cleardoublepage
    % enable running title
    \pagestyle{reviews}
    % change section numbering FROM [\chapter].[\section].[\subsection] TO [\section].[\subsection] ETC.
    \renewcommand{\thesection}{\arabic{section}} 
    % mark chapter % add author + title to the TOC
    \chapter[\normalfont\textbf{\emph{\thecontributor}}: \thecontribution]{}    % reset the section counter for each paper
    \setcounter{section}{0}  
    % no running title on first page, page number center bottom instead, include copyright statement
    \thispagestyle{contributiontitlepage}
    % formatting for the bibliography
}{}
\AtBeginEnvironment{review}{
% keeps running title from the first page
    \renewcommand*{\pagemark}{}%                                   
}
\AtEndEnvironment{review}{
    % author name(s)
    \begin{flushright}\emph{\thecontributor}\end{flushright}
    % safeguard section numbering
    \renewcommand{\thesubsection}{\thesection.\arabic{subsection}} 
    % last page running header fix
    \protect\thispagestyle{reviews}                           
}

% Abstract
\newenvironment{abstract}{% 
\setlength{\parindent}{0pt} \begin{adjustwidth}{2em}{2em}\footnotesize\emph{\abstractname}: }{%
\vskip 1em\end{adjustwidth}
}{}

% Keywords
\newenvironment{keywords}{
\setlength{\parindent}{0pt} \begin{adjustwidth}{2em}{2em}\footnotesize\emph{Keywords}: }{%
\vskip 1em\end{adjustwidth}
}{}

% Review Abstract
\newenvironment{reviewed}{% 
\setlength{\parindent}{0pt}
    \begin{adjustwidth}{2em}{2em}\footnotesize}{%
\vskip 1em\end{adjustwidth}
}{}

% Motto
\newenvironment{motto}{% 
\setlength{\parindent}{0pt} \small\raggedleft}{%
\vskip 2em
}{}

% Example
\newcounter{example}[chapter]
\newenvironment{example}[1][]{\refstepcounter{example}\begin{quote} \rmfamily}{\begin{flushright}(Example~\theexample)\end{flushright}\end{quote}}

%%%% SECTIONOPTIONS

% command for centering section headings
\newcommand{\centerheading}[1]{   
    \hspace*{\fill}#1\hspace*{\fill}
}

% Remove "Part #." from \part titles
% KOMA default: \newcommand*{\partformat}{\partname~\thepart\autodot}
\renewcommand*{\partformat}{} 

% No dots after figure or table numbers
\renewcommand*{\figureformat}{\figurename~\thefigure}
\renewcommand*{\tableformat}{\tablename~\thetable}

% paragraph handling
\setparsizes%
    {1em}% indent
    {0pt}% maximum space between paragraphs
    {0pt plus 1fil}% last line not justified
    

% In the "Authors" section, author names are put in the \paragraph{} headings. To reduce the space after these  headings, the default {-1em} has been changed to {-.4em} below.
\makeatletter
\renewcommand\paragraph{\@startsection {paragraph}{4}{\z@ }{3.25ex \@plus 1ex \@minus .2ex}{-.4em}{\normalfont \normalsize \bfseries }
}
\makeatother

% add the following (uncommented) in environments where you want to count paragraph numbers in the margin
%    \renewcommand*{\paragraphformat}{%
%    \makebox[-4pt][r]{\footnotesize\theparagraph\autodot\enskip}
%    }
%    \renewcommand{\theparagraph}{\arabic{paragraph}}
%    \setcounter{paragraph}{0}
%    \setcounter{secnumdepth}{4}
    
%%%% HEADERFOOTER

% running title
\RequirePackage{fancyhdr}
% cuts off running titles that are too long
%\RequirePackage{truncate}
% makes header as wide as geometry (SET SAME AS \TEXTWIDTH!)
\setlength{\headwidth}{109mm} 
% LO = Left Odd
\fancyhead[LO]{\small\emph{\theshortcontributor} \hspace*{.5em} \theshortcontribution} 
% RE = Right Even
\fancyhead[RE]{\scshape{\small\theissue}}
% LE = Left Even
\fancyhead[LE]{\small\thepage}            
% RE = Right Odd
\fancyhead[RO]{\small\thepage}    
\fancyfoot{}
% no line under running title; cannot be \@z but needs to be 0pt
\renewcommand{\headrulewidth}{0 pt} 

% special style for authors pages
\fancypagestyle{authors}{
    \fancyhead[LO]{\small\textit{Authors}} 
    \fancyhead[LE]{\small\thepage}            
    \fancyhead[RE]{\scshape{\small\theissue}}
    \fancyhead[RO]{\small\thepage}            
    \fancyfoot{}
}

% special style for book reviews
\fancypagestyle{reviews}{
    \fancyhead[LO]{\small\textit{Book Reviews}} 
    \fancyhead[LE]{\small\thepage}            
    \fancyhead[RE]{\scshape{\small\theissue}}
    \fancyhead[RO]{\small\thepage}            
    \fancyfoot{}
}

% special style for Editors' preface.
\fancypagestyle{preface}{
    \fancyhead[LO]{\small\textit{\theshortcontribution}} 
    \fancyhead[LE]{\small\thepage}            
    \fancyhead[RE]{\scshape{\small\theissue}}
    \fancyhead[RO]{\small\thepage}            
    \fancyfoot{}
}
% special style for first pages of contributions etc.
% DOES include copyright statement
\fancypagestyle{contributiontitlepage}{
    \fancyhead[C]{\scriptsize\centering\copyrightstatement}
    \fancyhead[L,R]{}
    \fancyfoot[CE,CO]{\small\thepage}
}
% special style for first pages of other \chapters.
% DOES NOT include copyright statement
\fancypagestyle{chaptertitlepage}{
    \fancyhead[C,L,R]{}
    \fancyfoot[CE,CO]{\small\thepage}
}
% no page numbers on \part pages 
\renewcommand*{\partpagestyle}{empty}

%%%% FOOTNOTEFORMAT

% footnotes
\renewcommand{\footnoterule}{%
    \kern .5em  % call this kerna
    \hrule height 0.4pt width .2\columnwidth    % the .2 value made the footnote ruler (horizontal line) smaller (was at .4)
    \kern .5em % call this kernb
}
\usepackage{footmisc}               
\renewcommand{\footnotelayout}{
    \hspace{1.5em}    % space between footnote mark and footnote text
}    
\newcommand{\mytodo}[1]{\textcolor{red}{#1}}

%%%% CODESNIPPETS

% colours for code notations
\usepackage{listings}       
	\renewcommand\lstlistingname{Quelltext} 
	\lstset{                    % basic formatting (bash etc.)
	       basicstyle=\ttfamily,
 	       showstringspaces=false,
	       commentstyle=\color{BrickRed},
	       keywordstyle=\color{RoyalBlue}
	}
	\lstdefinelanguage{XML}{     % specific XML formatting overrides
		  basicstyle=\ttfamily,
		  morestring=[s]{"}{"},
		  morecomment=[s]{?}{?},
		  morecomment=[s]{!--}{--},
		  commentstyle=\color{OliveGreen},
		  moredelim=[s][\color{Black}]{>}{<},
		  moredelim=[s][\color{RawSienna}]{\ }{=},
		  stringstyle=\color{RoyalBlue},
 		  identifierstyle=\color{Plum}
	}
    % HOW TO USE? BASH EXAMPLE
    %   \begin{lstlisting}[language=bash]
    %   #some comment
    %   cd Documents
    %   \end{lstlisting}
\author{Gabriele Wix}
\title{Born-Digital Manuscripts and the Contemporary Author. ``Carport'' from Marcel Beyer's Ezra Pound Poem Cycle, for Example.}

\begin{document}
\maketitle

\begin{abstract}
% write your abstract here
Today, the basic material evidence of authorial activity has migrated to
the electronic realm. Considering the theme of the 2022 ESTS conference
\emph{Histories of the Holograph.} \emph{From Ancient to Modern
Manuscripts and Beyond,} we find that there is no clear cut between
``modern'', e.g. private documents, be they handwritten or typed, and
``beyond'', e.g. digitally created. Rather, there is a transitory stage
between modern and digital writing processes, which is the focus of the
essay. This in-between does not only concern contemporary authors who
were anchored in typewriting and then switched to creating their
manuscripts on computers. The so-called digital natives born from the
1980s onwards have also grown up in a culture of paper, and publishing
in books and other print media continues to coexist with digital forms
of publication.

From a philological and \emph{critique génétique} perspective, the first
part of the paper addresses the question to what extent the transitional
situation is reflected in present born-digital writing processes. The
genesis of ``Carport'', a four-line poem by the 2016 Büchner Prize
winner Marcel Beyer will serve as a case study for this. As part of a
cycle on Ezra Pound, Beyer's text may be of special interest to
English-language scholars. Beyond its brevity, the poem lends itself to
study: The born-digital writing process spans eight years and is
documented in twelve bundles of drafts, saved in private storage media.
In addition, the storages are carefully paginated, dated and annotated,
which blends the roles of author and editor. We are thus faced with the
rather uncommon situation, even in born-digital writing processes, that
the author himself provides a well-ordered \emph{dossier génétique.} A
provocation for the editor? A fortunate circumstance for the textual
scholar and the genetic critic?

Engaging with digital writing processes usually means engaging with a
living author. Moreover, in the present case, the drafts are embedded in
a kind of logbook, generating an emphatic presence of the author beyond
the \emph{writer} in the sense of Roland Barthes. This is why the
second part of this paper reflects on which questions the
contemporaneity of author and scholar raises for research, especially
with regard to the current boom in academic studies of contemporary
literature.
\end{abstract}

%%%%%%%%%%%%%%%%%%%%%%%%%%%%
%% YOUR ESSAY STARTS HERE %%
%%%%%%%%%%%%%%%%%%%%%%%%%%%%

% remove asterisk (*) if you want to number your sections
% add a title for your section in between the {curly brackets} if you need one
\section*{Modern and Beyond} 

Sifting through piles of manuscripts, deciphering and dating the
documents (mostly handwritten notes, drafts, excerpts and typescripts),
sorting out so-called hot and cold stacks of paper, reconstructing the
chronology, creating diplomatic or linear transcriptions of manuscripts
and revised typescripts, but also the joy in the aesthetics of the
manuscripts, rightly considered as art objects by the \emph{critique
génétique} \citep[157]{gresillon_literarische_1999}, the pleasure of having read into a
handwriting so that one succeeds in deciphering initially unreadable
passages, the delight of finding the pieces that complete the puzzle --
that was the editor's work creating a \emph{dossier génétique} or a
facsimile edition in the modern era.

Since the 1990s, the basic material evidence of contemporary authorial
activity has migrated to the electronic realm. Handwritten manuscripts
of contemporary poets appear to us almost as historical and as a
reminder of a pre-digital, `modern' age. Considering the theme of the
2022 ESTS conference: \emph{Histories of the Holograph. From Ancient to
Modern Manuscripts and Beyond,} the preposition ``beyond'' seems chosen
for good reason: The ``born-digital manuscript'' or the ``born-digital
holograph'' spans a wide field. On the one hand, new literary practices
are emerging in the context of media and technical upheavals. According
to Annette Gilbert, the possibilities of automated processing of texts
are being used aesthetically far beyond the mere capturing, copying and
typesetting of texts; the spectrum ranges from filtering and searching
through a text, to counting, sorting, comparing, summarising and
translating (\cite[511]{gilbert_zukunfte_2019}; see also: \cite[34-50]{gilbert_literatures_2022}). Rainald Goetz, for example, wrote an
internet diary as early as 1998 -- the term \emph{blog} did not yet
exist at that time -- into which any user could plug in. But
eventually, it appeared as an 860-page book in 1999 and was then removed
from the net \citep{goetz_abfall_1999}. Jason Huff and Elizabeth Tonnard, to give
another example, used the possibilities of automated processing of texts
aesthetically. In their case, they used Word's auto-summarise function to generate the
shortest poems from famous works of world literature \citep[511]{huff_autosummarize_2010, tonnard_speak_2010, gilbert_zukunfte_2019}. Authors, such as Thomas Kling
(1957--2005), are in turn deeply attached to writing by hand \citep{stussel_thomas_2013,wix_stratigraphic_2016}, and his digital writing processes \citep{ries_rationale_2018},
which tend to replace the typewriter rather than initiate medially
induced experimental writing processes, account for a comparatively
short, albeit significant period of his literary output.

The phrase ``modern and beyond'' already suggests it: There is no clear
cut between ``modern'' e.g. private documents, be they handwritten or
typed, and ``beyond'', e.g. digitally created. Rather, there is a
transitional phase between modern and digital writing processes, and
this transitory stage is the focus of the essay. The in-between does not
only concern authors who were anchored in typewriting and then switched
to creating their manuscripts on computers. The so-called digital
natives born from the 1980s onwards have also grown up in a culture of
books and paper, and the publication of books and other print media continues
to coexist with digital forms of publication.

\section*{Considering the In-Between: A Born-Digital Holograph by Marcel Beyer}

\subsection*{Author and Poetology}

From a philological and \emph{critique génétique} perspective, this paper
addresses the question to what extent the transitional situation is
reflected in present born-digital writing processes. The genesis of
``Carport'', a four-line poem by the 2016 Büchner Prize winner Marcel
Beyer, will serve as a case study for this. It is part of a 203-page PDF
documenting the material traces of a writing process that spanned eight
years.

Within the discourse about the death and the return of the author, to
which I will come back later, it is an almost subversive pleasure to first
introduce the author with an iconic photo (see Fig. \ref{fig:wix1}).

\begin{figure}[H]
    \centering
    \includegraphics[width=.75\textwidth]{media/wix1.png}
    \caption{Figure \ref{fig:wix1}: Photograph of Marcel Beyer, by Jens Gyarmaty. Reproduced with permission of the copyright holder.}
    \label{fig:wix1}
\end{figure}

Image reproduced with permission from the copyright holder.

Beyer is a highly acclaimed poet, novelist, librettist, and essayist.
Two examples may outline his poetological approach. First, a recent
work, which has met with great public response, the libretto for an
opera by the Japanese composer Toshio Hosokawa, premiered in July 2018
\citep{beyer_libretto_2018}. It is based on the novella \emph{Das Erdbeben in Chili} (The
Earthquake in Chile) by Heinrich Kleist. The plot and the central theme
of the novella were transformed into an opera collage that transmits the
suspension of any social norms in this historical disaster scenario to
the contemporary events surrounding the nuclear accident due to the
tsunami in Fukushima, as well as to the current right-wing populist
attacks \parencite{wix_programmheft_2018}. Second, the poetry collection in question, \emph{Graphit}
(Graphite), published in 2014 \citep{beyer_graphit_2014}. The first association of
the title might be a pencil, a common writing tool. But according to an
author's statement in a talk in Cologne in 2015, the title refers to the
graphite dust which his wife, artist Jacqueline Merz, works with in her
studio -- Beyer does not use a pencil, but a Lamy pen, lead F for
handwritten notes. The title of the poetry volume returns in the title
of the first poem. In this text, Beyer blends his memories of a visit to
his friend, poet Thomas Kling at Neuss in the 1990s, into a report on
the shooting of Eisenstein's film \emph{Alexander Newski} in Moscow in
1938. The key link between the two places is a view on the production of
artificial snow: machine made snow using water at the Ski Hall in
Neuss, naphthalene and chalk in Moscow. The poem ends with the lines,

\begin{quote}
{[}\ldots{]} Einmal quer durchs

Jahrhundert führt, am Pistenrand

hier, eine Schattenspur: Graphit. 

\vspace{1em}
 
{[}\ldots~Once leading straight through the

century, at the edge of the slope

here, a shadow trace: graphite.{]}

\begin{flushright}
    \parencite[17]{beyer_graphit_2014}
\end{flushright} 
\end{quote}

This image of ``a shadow trace'' mirrors a crucial point of Beyer's
poetological program: Tracing historical events from the present back to
the past, from the past into the present, blending motifs,
combining and installing dissembled elements, unmasking the layers of
historical tradition -- especially these layers which are tending to
conceal themselves. Accordingly, the writing process is based on a kind of collage or
montage technique, which is a key point in Beyer's writings \citep{wix_editionsphilologie_2017, wix_genese_2019}.

\subsection*{``Carport''}

The poetry collection \emph{Graphit} is structured in nine chapters. A
cycle of five poems including ``Carport'' forms the eighth chapter
\citep[171 -- 192]{beyer_graphit_2014}. Even if notes at the end of the book only give
Ezra Pound as a reference for ``Mein Blauhäher'' (My Blue Jay), which is
the longest poem with sixteen parts, centrally placed in third place,
the five poems gathered in this chapter all make reference to Ezra
Pound's life and work \citep[67]{mengeringhaus_was_nodate}. Regarding the complexity
of the genesis and the German language, with which not all readers will
be familiar, I focus on the first poem of only four short lines from
chapter VIII, titled ``Carport'', to present specific issues of the
digital writing process. Figures of the first two pages may give an
impression of this born-digital manuscript (Fig. \ref{fig:wix2} and Fig. \ref{fig:wix3}). They also
provide the frame of reference.

\begin{quote}
Carport

Ich saß da, in meiner Heimat, und

die Sprache hielt mich fest, in

meinem Carport, wo der Tod läuft

und das Leben aus Pappe ist.

\vspace{1em}
 
{[}Carport

I sat there, in my homeland, and

the language held me tight, in

my carport, where the death runs

and life is made of cardboard.{]}

\begin{flushright}
    \parencite[173]{beyer_graphit_2014}
\end{flushright} 
\end{quote}

These verses might remind the reader of another text, and this is right.
Beyer refers to the last lines of Ezra Pound's Canto CXV, the one that
immediately came to Allen Ginsberg's mind upon hearing of Ezra Pounds
death during a radio interview. Ginsberg spontaneously quoted the text
and commented: ``Great despairing end ...'' \citep[185]{ginsberg_allen_1974}.

\begin{quote}
IN MEINER HEIMAT

WHERE THE DEAD

WALKED AND THE

LIVING WERE MADE

\vspace{1em}

OF CARDBOARD
\end{quote}

This is the way to read Pound's verses at the very beginning of Beyer's
drafts, bundle 1, p. 1 (Fig. \ref{fig:wix2}). They are written in capital letters, and
additional line breaks divide the three lines of Pound's original verses into five
\citep[1172]{pound_cantos_2012}. ''OF CARDBOARD'' is separated by a blank
line.


\begin{figure}[ph]
    \centering
    \includegraphics[width=.9\textwidth]{media/wix2.png}
    \caption{Figure \ref{fig:wix2}: Marcel Beyer: Mein Blauhäher, Konvolut 1, page 1.}
    \label{fig:wix2}
\end{figure}

Image reproduced with permission from the copyright holder.

\begin{figure}[ph]
    \centering
    \includegraphics[width=.9\textwidth]{media/wix3.png}
    \caption{Figure \ref{fig:wix3}: Marcel Beyer: Mein Blauhäher, Konvolut 1, page 2.}
    \label{fig:wix3}
\end{figure}

Image reproduced with permission from the copyright holder.

Alluding to Pound's idiosyncratic and recreative way of translating
\citep{hesse_tertium_2001} -- his transmissions from the source language to the target
language are primarily inspired by visual, sonic and rhythmic
perceptions, irrespective of the semantic content -- Beyer transforms
the word ``cardboard'', German ``Pappe'', into the phonetically related
word ``Carport'', an Anglicism frequently used in German. Likewise, the
literal translation ``Pappe'' comes into play. Michael Kindellan's words
describing Pound could be transferred one-to-one to Beyer, ``pre-eminent
poet-philologist, literally in \emph{love} with \emph{logos}''
\citep[1]{Kindellan_late_2017}. Pound's German phrase ``In meiner Heimat'' (``in my homeland'') is
evocative. It moves into Beyer's poem almost casually, inserted in
commas. While Pound's phrase may also resonate as a rejoinder to the
Wagnerian, ever-repeated formula of longing to "unserer Heimat" (``our
homeland'') (Wagner 2013, 167 et al.), the focus now shifts to the
contemporary connotations of the term "Heimat" in the field of tension
between trivialising folklore, affection, existential need and right-wing
populist agitation.

The initial title of the poem, ``Heim'', later removed in favour of
``Carport'', also underlines the significance of Pound's German
expression. Non-German-speaking readers are reminded of the difference
of ``Heim'' and ``Heimat'': ``Heimat'' can be translated as ``home'' as
well as ``homeland''. ``Heim'', however, covers ``home'' in the sense of
house or apartment in which one lives and feels comfortable, and it does
not cover ``homeland''.



\subsection*{The digital-born \emph{dossier génétique}}

The digital-born writing process of this four-line poem ``Carport'' is
documented within twelve bundles of drafts for chapter VIII from
\emph{Graphit,} the part which relates to Ezra Pound's poetry. The
digital documents are saved on Beyer's storage media; they were
submitted as PDFs via an email attachment. The analysis focuses on these
digital objects only. It does not consider any other (meta-)data,
including any further temporary files or the ``fragmented traces of the
writing process scattered across multiple system locations'' \citep[391]{ries_rationale_2018}; data of the automatic process running in the background that would
be relevant from a digital forensic perspective. The manuscript
comprises a total of 203 pages and is dated January 24, 2006, to
February 18, 2014. The organization into twelve individual bundles
reflects the work steps, and the time they cover varies greatly, from a
few minutes to over several years.

The texts are not written consecutively, since the author usually starts
a new page after each saving, a working routine resulting from writing
on the typewriter:

\begin{quote}
-- das kommt noch vom alten Arbeiten mit der Schreibmaschine her: Wenn's
nicht weitergeht, wird das Blatt aus der Maschine gezogen, ein neues
Blatt eingespannt, die neben der Maschine liegende Textstufe abgetippt
bzw. korrigierend weiterentwickelt.

\vspace{1em}

[-- this still comes from the old way of working with the typewriter: if
it doesn't go any further, the sheet is pulled out of the machine, a new
sheet is clamped in place, the text stage lying next to the machine is
typed or corrected further.] 
\begin{flushright}
    \parencite{beyer_e-mail_2015}
\end{flushright} 
\end{quote}

This reference to the writing practice acquired with the typewriter and
its transfer to the work on the computer leads to a fundamental question
regarding born-digital manuscripts. At the same time, it confirms the
thesis of an intermediate stage set up at the beginning: The transfer
from writing practices with the typewriter to the computer outlined by
Beyer is not only an individual experience. Rather, the orientation to
paper (and book) as a text carrier and to the format of the sheet is
preserved in the word processor, and the idea of the physical document
is still to be found in the terminology of textual scholarship, as two
quotes from a temporal distance of around 20 years will show.

\begin{quote}
The figure~of~the text ``processed'' on a computer is like a phantom to
the extent that it is less bodily, more ``spiritual,'' more ethereal.
There~is~something like a disincarnation~of~the text in this. But its
spectral silhouette remains, and what's more, for most intellectuals and
writers, the program, the ``software'' of machines, still conforms to
the spectral model of the book. Everything that appears on the screen is
arranged~\emph{with a view~}to books: writing, lines, numbered pages,
coded indications~of forms (italics, bold, etc.), the differences of the
traditional shapes and characters. There are some tele-writing machines
that don't do this, but ``ours'' still respect the figure of the~book --
they~serve it and mimic it, they are wedded to it in a way that is
quasispiritual, ``pneumatic,'' close to breathing: as if you had only to
say the word and it would be printed.~
\begin{flushright}
    \parencite[30]{derrida_paper_2005}
\end{flushright} 

From the perspective of today's everyday usage of the term `digital
document', it might be considered odd to still tie the term `document'
to the physicality of a text carrier, although obviously the term and
concept is historically derived from physical documents and graphical
user interfaces are still mimicking the physical document on the screen.
\begin{flushright}
    \parencite[146]{ries_philology_2017}
\end{flushright} 
\end{quote}

Obviously, the pictorial idea of physical pages that can be touched,
counted, and ordered, seems to facilitate dealing with the fluid data and
processes hidden beneath the user interface. It remains to be seen
whether this is a temporary transitional phenomenon or perhaps the
expression of a fundamental need for spatial orientation after all.

Beyer's \emph{dossier génétique} has a stringent structure. In the
footer line, you find the pagination, the author's name, the title of
the work he is writing, and the number of the bundle. Usually the
end of the respective draft includes a date and exact time, and
sometimes some remarks on the working situation, the weather, or a
comment on the draft itself. This is why the manuscript may be
characterized as logbook (``Arbeitsjournal''), a format that causes a
noticeable presence of the author's person and is usually classified
under ``ego-material''. What is even more, Beyer not only used to storage his drafts, but also his excerpts, references and so on. From a
\emph{critique génétique} perspective, it is first worth taking a closer
look at the two manuscript pages. On page one, various writing
approaches can be traced below the Pound citation, whereby it is
striking that a certain appropriation process can already be observed in
the altered line break of the quotation described at the beginning. On
the second page, there is a series of disjointed notes below the date of
filing: material that the author did not use initially, but kept in
stock.

Considering this type of material, we encounter a discourse on the
continued existence of textual criticism that already emerged with the
first advent of digital literary writing processes. As Thorsten Ries points out,
Wolf Kittler already suggested in 1991 that instant corrections no longer left material traces and thus there was no longer
any material basis for the editor. And in the case that the author
himself saved deleted passages, the editor was out of a job 
(\cite[235]{kittler_literatur_1991};  \cite[129]{ries_philology_2017}). Regardless of fundamental differences
between writing with a word processor, by hand, or with a typewriter,\footnote{Jacques Derrida already reflected with great clarity in 1996 (\cite{derrida_paper_2005}, 24), and treating this issue further would go beyond the scope of
this article.} we can argue that the first two pages of Beyer's born-digital
holograph already prove that in the digital realm, too, it is quite
possible to follow a complex writing process in its inner dynamics, its
virtual ramifications, its dead ends. The material provided by Beyer is
not a document of digital self-archiving in favour of a fetishised
preservation of the ultimate first draft, a behaviour that Kittler
described with critical distance \citep[130]{ries_philology_2017}; it is a document of
the author's working material to keep the writing process going. This
becomes all the more clear when tracing the genesis of the short poem
over the entire material.

Bundle 1 consists of ten pages, which all refer to ``Carport'', titled
``Heim'' (home) at this stage. The writing process lasted one day,
January 24 in 2006. There are six different storages within a very short
time span, from 4:05 p.m. to 5:20 p.m., and a last one without time
specification.

To give an idea of the paratextual comments: the six
storages in bundle 1 read:

\begin{quote}
Notizen, fünf nach vier, 24.1.06

\vspace{1em}

[Notes, five past four, 24/1/2006]
\end{quote}

\begin{quote}
Notizen, fünf vor halb fünf, 24.1.06

\vspace{1em}

[Notes, five before half past five, 24/1/2006]
\end{quote}

\begin{quote}
bis zehn nach fünf, 24.1.06

\vspace{1em}

[till ten past five, 24/1/2006]
\end{quote}

\begin{quote}
viertel nach fünf, 24.1.06 (fließt optisch nicht)

\vspace{1em}

[quarter past five, 24/1/2006 (does not flow optically)]
\end{quote}

\begin{quote}
zwanzig nach fünf, 24.1.06 so vielleicht? alles sehr rasch

\vspace{1em}

[twenty past five, 24/1/2006 maybe? all very quickly]
\end{quote}

\begin{quote}
(25.1.06: als ich es morgens, gleich nach dem Aufstehen, lese, habe ich
plötzlich ‚Und der Haifisch, der hat Zähne` im Ohr, das gefällt mir
nicht.)

\vspace{1em}

[(25/01/2006: when I read it in the morning, just after getting up,
I suddenly have ``And the shark, it has teeth'' ringing in my ears, I do
not like it.)]
\end{quote}

Bundle 2 and 3 last till February 22, 2006. Beyer reviews the ``sudden''
writing process and considers the notion of ``Heim'', writing down some
further associations and research on Pound. While working on other poems
of the Pound cycle, titled ``Alba'' (``Daybreak'') and ``Hähersprache'' (``Jay
Tongue''), Beyer returns to the quotes of Pound's canto CXV as well as to
the draft for ``Home'', later titled ``Carport'', and inserts the lines
into different parts of his poem ``Hähersprache''. In 2007, a poem
titled ``Hähersprache'' was published in a magazine \parencite{beyer_hahersprache_2007}.
Single lines of the drafts for ``Heim'' were scattered in the poem, and
a text of four lines -- identical with the one of the poem later titled
``Carport'' -- appeared as part IX. An essay by Beyer on E. E. Cummings'
early poem "pound pound pound" was published in the same journal \citep{beyer_zu_2007}. Here Beyer also refers to Cumming's later poem to Pound,
``crazy blue jay'', from which a relation to Beyer's poem
``Hähersprache'' (``Jay Tongue'') can be drawn.

More than 5 years later, in August 2011, Beyer returns to the Pound cycle. At
the beginning of bundle 4, he notes on August 29:

\begin{quote}
seit vorgestern ungefähr geistert mir im Kopf herum: Hähersprache neu
schreiben. Als Eröffnung einen Wochendurchgang, dann jeweils einen Tag
(aus dem bestehenden, noch nicht wieder angeschauten Material der alten
Fassung)

\vspace{1em}

[since the day before yesterday, I've been thinking about rewriting Jay
Tongue. As an opening a weekly run, then a day at a time (from the
existing material of the old version that has not yet been looked at
again)]
\end{quote}

He copies and pastes the version from 2006, and many approaches deal
with ``Heimat'' and ``Heim''. With longer breaks, he works on the Pound
cycle again in September 2011, February 2012 and February 2014. Either
he does not refer to the lines of the poem later titled ``Carport'' at
all, or the procedure is similar to the prior ones. As far as bundle 1
to 12 is concerned, the logbook does not give any information about when
Beyer decided to finally title the poem ``Carport'' and to place it at
the beginning of his Pound cycle in \emph{Graphit}. But the exact
wording of the poem had already been realized on January 22, 2006, eight
years before. In the end, the Pound cycle consists of five poems:
``Carport'', ``Alba'', ``Mein Blauhäher'' (``My Blue Jay''), a new version
of ``Hähersprache'', ``Alba'' -- a poem of identical title, but different
text -- and ``Argot''.

To finally return to Derrida's notion of digital writing machines being
``wedded to the figure of the book'': It can be traced in detail that
Beyer's writing processes, as documented in the born-digital
\emph{dossier génétique,} are arranged with a view to the book, the
division into chapters, the importance of the white space, the axes of
symmetry in the arrangement of the poems, etc.

\subsection*{Ships at Sea}

In another \emph{dossier génétique}, Beyer used to finish a part of the
planned cycle without ever returning to it or changing anything
anymore. In this case, the writing process is different. Over a long
period of time, the writing process returns to the beginning again and
again. In an e-mail-correspondence, the author answered my question
regarding a certain stage in the writing process:

\begin{quote}
Es gibt ja für mich im Schreibprozeß keine Position des neutralen
Beobachters, wie wir sie jetzt beide einnehmen, da wir die
Arbeitsmaterialien vor uns haben. Aha, hier wurde diese Entscheidung
getroffen, und aha, dort wurde jener Ansatz weiterverfolgt. Im in eine
unbestimmte Zukunft gerichteten Schreibprozeß (wird aus diesen Ansätzen
etwas oder nicht?) ist ständig die Hauptfrage: Wie dynamisiere ich
dieses Material -- ist dies ein Wort, das andere Wörter nach sich zieht
-- wie kann ich ein Stop \& Go entstehen lassen, dessen jeweiligen
Endpunkte / Löcher / Pausen eben nicht endgültige Löcher / Enden
ergeben, sondern selbst wieder eine Spannung aufbauen, die in weiteren
Textverlauf mündet?

Man will ja beim Schreiben nichts anderes als: Schreiben.

Insofern trifft man natürlich Entscheidungen -- aber doch Entscheidungen
weniger der Art von: ``mit diesem Ansatz komme ich nicht weiter, also
wähle ich einen anderen``, es läuft also gar nicht souverän ab, sondern
eher wie: ``aha, hier könnte ich vielleicht rasch etwas probieren.``

Ich habe also vielleicht drei Zeilen geschrieben -- mir kommt in den
Sinn: vielleicht könnte die ganze Sache gleich viel mehr Speed
aufnehmen, wenn ich das zuletzt geschriebene Wort (oder den Halbsatz
oder den Vers) an den Anfang stellen würde -- gut, probiere ich das
rasch einmal aus, speichere aber zur Sicherheit den bisherigen Ansatz
ab, falls sich das Ausprobieren als Sackgasse erweist, denn ich möchte
ja womöglich zum ersten Ansatz zurück. Klar unterscheiden sich solche
Ansätze dann mitunter nur in einem Wort, oder gar nur darin, daß der
Zeilen- / Versumbruch an einer anderen Stelle vorgenommen wird, aber auf
diesen Minimalentscheidungen liegt im Moment des Schreibens die GANZE
Hoffnung (nämlich in diesem Sinne: Ich glaube, wenn ich dieses letzte
Wort in der Zeile an den Anfang der nächsten Zeile setzen würde, ergäbe
sich eine Dynamik für alles, was noch kommen mag (von dem ich aber zu
diesem Zeitpunkt noch gar nicht weiß)). 

\vspace{1em}

[There is, after all, no position of neutral observer for me in the
writing process, as we both occupy now that we have the working
materials at hand. Aha, this decision was made here, and aha, that
approach was pursued there. In the writing process, which is directed
into an indeterminate future (will these approaches become something or
not?), the main question is always: How do I dynamise this material - is
this a word that draws other words after it - how can I create a stop \&
go whose respective end points / holes / pauses do not result in final
holes / endings, but rather build up a tension again that leads to
further text development?

When you write, you want nothing more than: To write. 

In this respect, of course, you make decisions -- but decisions less
like: ``I can't get any further with this approach, so I'll choose
another one'', so it doesn't happen confidently at all, but more like:
``aha, maybe I could quickly try something here''.

So I've written maybe three lines -- it occurs to me: maybe the whole
thing could pick up a lot more speed if I put the last word I wrote (or
the half-sentence or the verse) at the beginning -- well, I'll try that
out quickly, but save the previous approach for safety, in case trying
it out turns out to be a dead end, because I might want to go back to
the first approach. Of course, such approaches sometimes differ only in
one word, or even only in the fact that the line / verse break is made
at a different place, but at the moment of writing, the WHOLE hope lies
in these minimal decisions (namely in this sense: I believe that if I
were to place this last word in the line at the beginning of the next
line, there would be a dynamic for everything that may still come (but
of which I am not yet aware at this point).]

\begin{flushright}
    \parencite{beyer_e-mail_2016}
\end{flushright} 

\end{quote}

In his answer, Beyer points out the difference between the position of
an observer who oversees the complete material at hand, and the author
who just wants to keep the writing process going on and who does not
know where that leads to at the end. Beyer's digital \emph{dossier
génétique} is, in my view, an excellent example for writing into an
openness for anything that comes, and simultaneously, a proof of the
instability and indeterminacy of writing processes, the risk and the
adventure involved in them. To put it in an aphorism by Lawrence Weiner:
``We are ships at sea, not ducks on a pond.''

\subsection*{Contemporaneity}

What was considered ``an impossibility'' 20 years ago \citep{steinfeld_literaturwissenschaft_1997,spoerhase_literaturwissenschaft_2014} is now a programme. ``Academic research on contemporary
literature is experiencing a boom'' \parencite{noauthor_dfg-graduiertenkolleg_nodate}. This coeval turn leads to a fundamental change in
the academic culture. Julika Griem, who also refers to Spoerhase,
describes the emergence of an institutional ``Gemengelage'', as we say in German \citep[101]{griem_standards_2015}. It is ``an institutional situation producing role
conflicts and increasingly fuzzy relations of proximity and distance
{[}\ldots{]} when authors lecture on poetics and participate in academic
conferences, when academics chair these authors' readings and write
reviews of their newest works'' \parencite{noauthor_info_nodate}.
Referencing of author interviews (and e-mails) has become a habitualised
research practice. As a result, the contemporaneity of author and editor
redefines their roles as well as their relationship, particularly so
regarding the postmodern decentring of the author.

Beyer himself is very present in the academic discourse and open to
answer any questions as he did, for example, concerning the semantic
content of the title of his poetry volume. But this is just the moment,
when the notion of authorship gets back close to its etymological
meaning of \emph{auctoritas}, authority, and here we are reminded of the
academic discourse of the 1970s initiated by Roland Barthes' 1968 essay
``The death of the author'' and Michel Foucault's 1969 essay ``Qu'est-ce
qu'un auteur?'', a theory which was fundamentally questioned by Roger
Chartier's proposition about the return of the author in 1992. In
Germany, the discourse was fuelled in 1999 due to an anthology edited by
Fotis Jannidis et al., titled ``Die Rückkehr des Autors'' (The return of
the author) \citep{jannidis_ruckkehr_1999}. Authors like Beyer are up to date with the
academic discussion. They know the discourse around the role of the
``author'' and reflect on their doing as ``writers'', which is a vital
prerequisite to collaborate with an author in academic research.
Secondly, Beyer is well aware of the fact that a text once published is
beyond the author's authority.

The turn to contemporary literature offers new research opportunities
for textual scholarship. At one point, a practice of ``Vorlass'' (both on the part of the archives
and on the part of the authors) has increasingly emerged alongside the practice of ``Nachlass'' \citep{sina_nachlassbewusstsein_2017}. The term ``Vorlass'' does not exist in English. While
``Nachlass'' refers to an author's posthumous papers, the term
``Vorlass'' denotes the papers an author has donated to an archive while
he or she is still alive. Secondly, instead of traditional research in
the literary archive or in a poet's estate, there is the possibility of
directly accessing  living authors' manuscripts, as is the
case here in the context of a book project on Marcel Beyer's writing
processes. But that opens us up to sensitive areas in many regards.

At an institutional level, giving insight into the working process can
be honoured as an expression of Beyer's personal concern for an author's
educational mandate, and at a personal level, as an act of confidence
between author and scholar. And yet conflicts can arise. Possibly, there
are expectations on the part of the author. Even if only assumed by
the scholar, this might lead to self-censorship and concerns about living
up to the claim, or about corresponding to the generosity in an appropriate
way. There is no general answer. In the end, this is a question of
maintaining academic probity and a question of personality of both
author and scholar. From the institutional point of view, Griem is
right: Changed academic contact zones facilitate data collection and
research, coincidently creating these ``fuzzy relations of proximity and
distance''. Above all, the access to the genesis of contemporary texts
is a far-reaching and stimulating opportunity for textual scholarship.
For various reasons, be it the growing interest of literary studies in
contemporary literature, or the change in posthumous (``Nachlass'') and
anthumous (``Vorlass'') politics on the part of authors and archives,
there is an increasing wealth of contemporary holographs, analogue and
ever more digital -- both ready for philology and genetic criticism.


\begin{flushleft}
    % use smallcaps for author names
    \renewcommand*{\mkbibnamefamily}[1]{\textsc{#1}}
    \renewcommand*{\mkbibnamegiven}[1]{\textsc{#1}} 
\printbibliography
\end{flushleft}

\end{document}