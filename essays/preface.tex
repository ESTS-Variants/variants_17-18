\title{Editors' Preface}
\contribution{Editors' Preface}
\shortcontribution{\thecontribution}
\author{}
\begin{preface}
\maketitle
\thispagestyle{contributiontitlepage}
\section*{}
\label{preface}
\vskip 3em

\textsc{Life is, to a large extent, unpredictable} and known to put a spoke in the wheel of a careful planning. It might be unusual to start an editorial preface with such philosophical truisms, but they serve to explain the long hiatus between the last \emph{Variants} issue and the present one. Over the past two years, we have both had to take some time off for personal reasons which significantly delayed the publication process. We are very thankful to the authors for their kind patience and understanding. Special thanks also to Manon Beentjes, our first editorial assistant, who worked hard alongside us to get the issue published before the end of the year. It is therefore with great pleasure that we announce the publication of this double issue, which was well worth the long wait.

Issue 17--18 contains essays from the consecutive GENESIS and ESTS 2022 conferences as well as essays and reviews that have been submitted outside this context but fit beautifully within the joint themes of creative revision \mbox{(GENESIS)} and the history and study of ancient and modern holographs (ESTS). The issue starts off with a comprehensive introduction to the theme by the conference organizers Dirk Van Hulle and Olga Beloborodova, who make a compelling argument for expanding literary historiography with a comparative study of literary drafts. Foregrounding the creative process rather than the end product can yield new and unexpected results in reception–, theatre–, and translation studies, among others. It can also shed new light on the (collaborative) nature of authorship, and on the influence of writing tools or the use of different media on the genesis of a work.

The essays in this issue offer, each in their own way, a fitting illustration of the value of comparative genetic criticism and holograph studies. In her analysis of the holograph manuscript of Pessoa's ``Hora Absurda'', Carlotta Defenu demonstrates how the author's use of the synecdoche figure of speech is gradually introduced into his drafts, as he continues to revise the poem. Next, a theoretical discussion of the status of variants in holograph manuscripts is offered by Paola Italia, who presents a typology of the different kinds of ``alternative variants''. And, as Italia argues, although this type of variant is more readily associated with the modern manuscripts of the twentieth century, the Italian tradition is in a unique position to include examples in this typology that date all the way back to Petrarch. Martin Navrátil, on the other hand, has to start from a less luxurious position. For his reconstruction of \textit{Ruža}, a collection of Slovak poems by Vojtech Mihálik's that came second in a literary competition in 1946, only a tentative list of poem titles recorded in the author's manuscript notebooks survives. This leads to an interesting case study that explores methods to both minimize the editor's conjecture, and give the reader more agency in the process.

``Creativity is not an exclusively textual matter'' write Van Hulle and Beloborodova (\pageref{qtVanhulle1}), a statement clearly substantiated in the essay of Jorge Silva Rocha. He undertook a comparative study into the variation of the images adjacent to the text of one particular copy of the Beatus \emph{Commentary}. By applying stemmatic methods to explore the pictoral tradition of the Beatus, Silva's contribution indeed broadens the study of variance, showing how the images evolved independent of the text but dependent on each other. Paulius Subačius, then, introduces us to some ``curious incidents'' in the history of holographs, taking various examples from the works of twentieth-century Lithuanian authors. Following ``ancient-like writing habits'' (\pageref{qtSubacius}), these authors added notes to their texts of which the meaning is not always evident. Through a classification of these modern authorial glosses, Subačius provides some guidance to scholarly editors who may wonder how to handle them. Finally, our essay section is concluded by a contribution from Gabriele Wix, who brings the discussion back to what Van Hulle and Beloborodova call ``the born-digital holograph''(\pageref{qtVanhulle3}), by using a case study of a Marcel Beyer poem cycle to examine interactions between analogue and born-digital writing processes. And what are the implications for the genetic editor, Wix asks, when an author presents us with a well-organized and prearranged \textit{genetic dossier} of born-digital draft materials?

The `Holograph Description' section is new in \emph{Variants} and although we do not usually publish descriptive contributions of manuscripts, we found that they were well fitted within the ``Histories of the Holograph''--theme of the present issue. The section contains two contributions about a twelfth century and a seventeenth century manuscript, by Carolina Maria Escobar Vargas and Juan Lorente Sanchez respectively. Both authors demonstrate a remarkably detailed level of subject-specific knowledge and their reports provide ample incentives for future research. 

This issue also includes another `Work in Progress' section, with contributions that contain shorter and less formal research narratives that are more practice oriented and allow researchers to report on more preliminary research findings. Here, Kyoko Myojo's contribution demonstrates the importance of examining what Van Hulle and Beloborodova called the ``no-man's-land between the author's desk and the publishing house''(\pageref{qtVanhulle2}), in this case the editorial interventions in Franz Kafka's \emph{Der Process}. Her genetic analysis of the manuscripts made her reconsider the conventional chapter order as well as the entire text of the current Japanese edition. Jason Wiens, then, closes this section with a report on some of the activities and results from a project that aims to visualize data, links, and writing processes found in the University of Calgary's archive of the Alice Munro papers. As such, the project means to analyse the genetic development and interrelation of the stories contained in the archive of this Canadian author who sadly passed away recently, and to open up avenues for future research on her oeuvre.

After the `Work in Progress' section, the issue again ends with an interesting review dossier, compiled by our Review Editor Stefano Rosignoli. It starts off with an elaborate review essay, where Peter Robinson casts his critical eye on the \emph{Handbook of Stemmatology}, edited by Philipp Roelli. In his review essay, Robinson praises the quality of many of the volume's individual contributions, but challenges its claim to be a true De Gruyter ``Handbook'', representative of the field, by exposing a series of lacunae in the volume as a whole. We hope that this essay may spark a lively debate in the field; perhaps even through future contributions in our Journal. Afterwards, the dossier continues with four more conventionally sized reviews that cover a wide range of topics. This includes two reviews of recent volumes published in the ``Complete'' or ``Collected'' Works editions of W. H. Auden (reviewed by Beci Carver) and Alfred Jarry (reviewed by Jean-Michel Rabaté) respectively, as well as two reviews of edited volumes, with one on graph data-models and semantic web technologies in Digital Scholarly Editing (reviewed by Aline Deicke), and another on genetic translation studies (reviewed by Dipanjan Maitra). We want to thank our reviewers for their thoughtful considerations regarding these relevant publications in our field, and Stefano for the great effort and care he put into commissioning and  editing them.

Finally, we end our preface by introducing some changes in our editorial board, that will take effect upon the publication of this volume. Elli, who became an Associate Editor of \textit{Variants} in the lead-up to our 14\textsuperscript{th} volume (2019), will be stepping down from this role once the current issue is published. Elli has been an invaluable member of our team for a long time, and we will be sad to see her go. Still, we have already found a worthy successor to take her place: ESTS board member and its newly appointed Secretary, Elsa Pereira, who has already been a great help in the preparations of our next volume. In addition to Elsa joining our team, we are also in the process of recruiting two new Assistant Editors to help us with the publication of our issues, to minimize the risk of significant delays in our publication schedule in the future. 

We hope that you enjoy the research presented in this volume. If you do, you may be happy to learn that the work on our next volume is already well underway. We have a number of essays accepted and under review that are expanded full-paper versions of research presented at last year's GENESIS conference in Taipei, the ESTS conference in Leicester, as well as some conference independent submissions. As such, \textit{Variants} 19, which is scheduled for publication in 2025, is already shaping up to be another sizeable volume of relevant research that, like our current issue, we hope will provide plenty food for thought and help inspire spirited discussions in the field.






\begin{flushright}
\emph{
Wout Dillen, General Editor\\
Elli Bleeker, Associate Editor
}
\end{flushright}

\end{preface}