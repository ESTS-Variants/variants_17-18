```
\documentclass{article}
%%%% CLASS OPTIONS 

\KOMAoptions{
    fontsize=10pt,              % set default font size
    DIV=calc,
    titlepage=false,
    paper=150mm:220mm,
    twoside=true, 
    twocolumn=false,
    toc=chapterentryfill,       % for dots: chapterentrydotfill
    parskip=false,              % space between paragraphs. "full" gives more space; "false" uses indentation instead
    headings=small,
    bibliography=leveldown,     % turns the Bibliography into a \section rather than a \chapter (so it appears on the same page)
}

%%%% PAGE SIZE

\usepackage[
    top=23mm,
    left=20mm,
    height=173mm,
    width=109mm,
    ]{geometry}

\setlength{\marginparwidth}{1.25cm} % sets up acceptable margin for \todonotes package (see preamble/packages.tex).

%%%% PACKAGES

\usepackage[dvipsnames]{xcolor}
\usepackage[unicode]{hyperref}  % hyperlinks
\usepackage{booktabs}           % professional-quality tables
\usepackage{nicefrac}           % compact symbols for 1/2, etc.
%\usepackage{microtype}          % microtypography
\usepackage{lipsum}             % lorem ipsum at the ready
\usepackage{graphicx}           % for figures
\usepackage{footmisc}           % makes symbol footnotes possible
\usepackage{ragged2e}
\usepackage{changepage}         % detect odd/even pages
\usepackage{array}
\usepackage{float}              % get figures etc. to stay where they are with [H]
\usepackage{subfigure}          % \subfigures witin a \begin{figure}
\usepackage{longtable}          % allows for tables that stretch over multiple pages
\setlength{\marginparwidth}{2cm}
\usepackage[textsize=footnotesize]{todonotes} % enables \todo's for editors
\usepackage{etoolbox}           % supplies commands like \AtBeginEnvironment and \atEndEnvironment
\usepackage{ifdraft}            % switches on proofreading options in the draft mode
\usepackage{rotating}           % provides sidewaysfigure environment
\usepackage{media9}             % allows for video in the pdf
\usepackage{xurl}               % allows URLs to (line)break ANYWHERE

%%%% ENCODING

\usepackage[full]{textcomp}                   % allows \textrightarrow etc.

% LANGUAGES

\usepackage{polyglossia}
\setmainlanguage{english} % Continue using english for rest of the document

% If necessary, the following lets you use \texthindi. Note, however, that BibLaTeX does not support it and will report a 'warning'.
 \setotherlanguages{hindi} 
 \newfontfamily\hindifont{Noto Sans Devanagari}[Script=Devanagari]

% biblatex
\usepackage[
    authordate,
    backend=biber,
    natbib=true,
    maxcitenames=2,
    ]{biblatex-chicago}
\usepackage{csquotes}

% special characters  
\usepackage{textalpha}                  % allows for greek characters in text 

%%%% FONTS

% Palatino font options
\usepackage{unicode-math}
\setmainfont{TeX Gyre Pagella}
\let\circ\undefined
\let\diamond\undefined
\let\bullet\undefined
\let\emptyset\undefined
\let\owns\undefined
\setmathfont{TeX Gyre Pagella Math}
\let\ocirc\undefined
\let\widecheck\undefined

\addtokomafont{disposition}{\rmfamily}  % Palatino for titles etc.
\setkomafont{descriptionlabel}{         % font for description lists    
\usekomafont{captionlabel}\bfseries     % Palatino bold
}
\setkomafont{caption}{\footnotesize}    % smaller font size for captions


\usepackage{mathabx}                    % allows for nicer looking \cup, \curvearrowbotright, etc. !!IMPORTANT!! These are math symbols and should be surrounded by $dollar signs$
\usepackage[normalem]{ulem}                       % allows for strikethrough with \sout etc.
\usepackage{anyfontsize}                          % fixes font scaling issue

%%%% ToC

% No (sub)sections in TOC
\setcounter{tocdepth}{0}                

% Redefines chapter title formatting
\makeatletter                               
\def\@makechapterhead#1{
  \vspace*{50\p@}%
  {\parindent \z@ \normalfont
    \interlinepenalty\@M
    \Large\raggedright #1\par\nobreak%
    \vskip 40\p@%
  }}
\makeatother
% a bit more space between titles and page numbers in TOC

\makeatletter   
\renewcommand\@pnumwidth{2.5em} 
\makeatother

%%%% CONTRIBUTOR

% Title and Author of individual contributions
\makeatletter
% paper/review author = contributor
\newcommand\contributor[1]{\renewcommand\@contributor{#1}}
\newcommand\@contributor{}
\newcommand\thecontributor{\@contributor} 
% paper/review title = contribution
\newcommand\contribution[1]{\renewcommand\@contribution{#1}}
\newcommand\@contribution{}
\newcommand\thecontribution{\@contribution}
% short contributor for running header
\newcommand\shortcontributor[1]{\renewcommand\@shortcontributor{#1}}
\newcommand\@shortcontributor{}
\newcommand\theshortcontributor{\@shortcontributor} 
% short title for running header
\newcommand\shortcontribution[1]{\renewcommand\@shortcontribution{#1}}
\newcommand\@shortcontribution{}
\newcommand\theshortcontribution{\@shortcontribution}
\makeatother

%%%% COPYRIGHT

% choose copyright license
\usepackage[               
    type={CC},
    modifier={by},
    version={4.0},
]{doclicense}

% define \copyrightstatement for ease of use
\newcommand{\copyrightstatement}{
         \doclicenseIcon \ \theyear. 
         \doclicenseLongText            % includes a link
}

%%%% ENVIRONMENTS
% Environments
\AtBeginEnvironment{quote}{\footnotesize\vskip 1em}
\AtEndEnvironment{quote}{\vskip 1em}

\setkomafont{caption}{\footnotesize}

% Preface
\newenvironment{preface}{
    \newrefsection
    \phantomsection
    \cleardoublepage
    \addcontentsline{toc}{part}{\thecontribution}
    % enable running title
    \pagestyle{preface}
    % \chapter*{Editors' Preface}    
    % reset the section counter for each paper
    \setcounter{section}{0}  
    % no running title on first page, page number center bottom instead
    \thispagestyle{chaptertitlepage}
}{}
\AtEndEnvironment{preface}{%
    % safeguard section numbering
    \renewcommand{\thesubsection}{\thesection.\arabic{subsection}}  
    %last page running header fix
    \protect\thispagestyle{preface}
}
% Essays
\newenvironment{paper}{
    \newrefsection
    \phantomsection
    % start every new paper on an uneven page 
    \cleardoublepage
    % enable running title
    \pagestyle{fancy}
    % change section numbering FROM [\chapter].[\section].[\subsection] TO [\section].[\subsection] ETC.
    \renewcommand{\thesection}{\arabic{section}}
    % mark chapter % add author + title to the TOC
    \chapter[\normalfont\textbf{\emph{\thecontributor}}: \thecontribution]{\vspace{-4em}\Large\normalfont\thecontribution\linebreak\normalsize\begin{flushright}\emph{\thecontributor}\end{flushright}}    
    % reset the section counter for each paper
    \setcounter{section}{0}  
    % reset the figure counter for each paper
    \renewcommand\thefigure{\arabic{figure}}    
    % reset the table counter for each paper
    \renewcommand\thetable{\arabic{table}} 
    % no running title on first page, page number center bottom instead, include copyright statement
    \thispagestyle{contributiontitlepage}
    % formatting for the bibliography

}{}
\AtBeginEnvironment{paper}{
    % keeps running title from the first page:
    \renewcommand*{\pagemark}{}%                            
}
\AtEndEnvironment{paper}{
    % safeguard section numbering
    \renewcommand{\thesubsection}{\thesection.\arabic{subsection}}  
    % last page running header fix
    \protect\thispagestyle{fancy}%                              
}
% Reviews
\newenvironment{review}{
    \newrefsection
    \phantomsection
    % start every new paper on an uneven page 
    \cleardoublepage
    % enable running title
    \pagestyle{reviews}
    % change section numbering FROM [\chapter].[\section].[\subsection] TO [\section].[\subsection] ETC.
    \renewcommand{\thesection}{\arabic{section}} 
    % mark chapter % add author + title to the TOC
    \chapter[\normalfont\textbf{\emph{\thecontributor}}: \thecontribution]{}    % reset the section counter for each paper
    \setcounter{section}{0}  
    % no running title on first page, page number center bottom instead, include copyright statement
    \thispagestyle{contributiontitlepage}
    % formatting for the bibliography
}{}
\AtBeginEnvironment{review}{
% keeps running title from the first page
    \renewcommand*{\pagemark}{}%                                   
}
\AtEndEnvironment{review}{
    % author name(s)
    \begin{flushright}\emph{\thecontributor}\end{flushright}
    % safeguard section numbering
    \renewcommand{\thesubsection}{\thesection.\arabic{subsection}} 
    % last page running header fix
    \protect\thispagestyle{reviews}                           
}

% Abstract
\newenvironment{abstract}{% 
\setlength{\parindent}{0pt} \begin{adjustwidth}{2em}{2em}\footnotesize\emph{\abstractname}: }{%
\vskip 1em\end{adjustwidth}
}{}

% Keywords
\newenvironment{keywords}{
\setlength{\parindent}{0pt} \begin{adjustwidth}{2em}{2em}\footnotesize\emph{Keywords}: }{%
\vskip 1em\end{adjustwidth}
}{}

% Review Abstract
\newenvironment{reviewed}{% 
\setlength{\parindent}{0pt}
    \begin{adjustwidth}{2em}{2em}\footnotesize}{%
\vskip 1em\end{adjustwidth}
}{}

% Motto
\newenvironment{motto}{% 
\setlength{\parindent}{0pt} \small\raggedleft}{%
\vskip 2em
}{}

% Example
\newcounter{example}[chapter]
\newenvironment{example}[1][]{\refstepcounter{example}\begin{quote} \rmfamily}{\begin{flushright}(Example~\theexample)\end{flushright}\end{quote}}

%%%% SECTIONOPTIONS

% command for centering section headings
\newcommand{\centerheading}[1]{   
    \hspace*{\fill}#1\hspace*{\fill}
}

% Remove "Part #." from \part titles
% KOMA default: \newcommand*{\partformat}{\partname~\thepart\autodot}
\renewcommand*{\partformat}{} 

% No dots after figure or table numbers
\renewcommand*{\figureformat}{\figurename~\thefigure}
\renewcommand*{\tableformat}{\tablename~\thetable}

% paragraph handling
\setparsizes%
    {1em}% indent
    {0pt}% maximum space between paragraphs
    {0pt plus 1fil}% last line not justified
    

% In the "Authors" section, author names are put in the \paragraph{} headings. To reduce the space after these  headings, the default {-1em} has been changed to {-.4em} below.
\makeatletter
\renewcommand\paragraph{\@startsection {paragraph}{4}{\z@ }{3.25ex \@plus 1ex \@minus .2ex}{-.4em}{\normalfont \normalsize \bfseries }
}
\makeatother

% add the following (uncommented) in environments where you want to count paragraph numbers in the margin
%    \renewcommand*{\paragraphformat}{%
%    \makebox[-4pt][r]{\footnotesize\theparagraph\autodot\enskip}
%    }
%    \renewcommand{\theparagraph}{\arabic{paragraph}}
%    \setcounter{paragraph}{0}
%    \setcounter{secnumdepth}{4}
    
%%%% HEADERFOOTER

% running title
\RequirePackage{fancyhdr}
% cuts off running titles that are too long
%\RequirePackage{truncate}
% makes header as wide as geometry (SET SAME AS \TEXTWIDTH!)
\setlength{\headwidth}{109mm} 
% LO = Left Odd
\fancyhead[LO]{\small\emph{\theshortcontributor} \hspace*{.5em} \theshortcontribution} 
% RE = Right Even
\fancyhead[RE]{\scshape{\small\theissue}}
% LE = Left Even
\fancyhead[LE]{\small\thepage}            
% RE = Right Odd
\fancyhead[RO]{\small\thepage}    
\fancyfoot{}
% no line under running title; cannot be \@z but needs to be 0pt
\renewcommand{\headrulewidth}{0 pt} 

% special style for authors pages
\fancypagestyle{authors}{
    \fancyhead[LO]{\small\textit{Authors}} 
    \fancyhead[LE]{\small\thepage}            
    \fancyhead[RE]{\scshape{\small\theissue}}
    \fancyhead[RO]{\small\thepage}            
    \fancyfoot{}
}

% special style for book reviews
\fancypagestyle{reviews}{
    \fancyhead[LO]{\small\textit{Book Reviews}} 
    \fancyhead[LE]{\small\thepage}            
    \fancyhead[RE]{\scshape{\small\theissue}}
    \fancyhead[RO]{\small\thepage}            
    \fancyfoot{}
}

% special style for Editors' preface.
\fancypagestyle{preface}{
    \fancyhead[LO]{\small\textit{\theshortcontribution}} 
    \fancyhead[LE]{\small\thepage}            
    \fancyhead[RE]{\scshape{\small\theissue}}
    \fancyhead[RO]{\small\thepage}            
    \fancyfoot{}
}
% special style for first pages of contributions etc.
% DOES include copyright statement
\fancypagestyle{contributiontitlepage}{
    \fancyhead[C]{\scriptsize\centering\copyrightstatement}
    \fancyhead[L,R]{}
    \fancyfoot[CE,CO]{\small\thepage}
}
% special style for first pages of other \chapters.
% DOES NOT include copyright statement
\fancypagestyle{chaptertitlepage}{
    \fancyhead[C,L,R]{}
    \fancyfoot[CE,CO]{\small\thepage}
}
% no page numbers on \part pages 
\renewcommand*{\partpagestyle}{empty}

%%%% FOOTNOTEFORMAT

% footnotes
\renewcommand{\footnoterule}{%
    \kern .5em  % call this kerna
    \hrule height 0.4pt width .2\columnwidth    % the .2 value made the footnote ruler (horizontal line) smaller (was at .4)
    \kern .5em % call this kernb
}
\usepackage{footmisc}               
\renewcommand{\footnotelayout}{
    \hspace{1.5em}    % space between footnote mark and footnote text
}    
\newcommand{\mytodo}[1]{\textcolor{red}{#1}}

%%%% CODESNIPPETS

% colours for code notations
\usepackage{listings}       
	\renewcommand\lstlistingname{Quelltext} 
	\lstset{                    % basic formatting (bash etc.)
	       basicstyle=\ttfamily,
 	       showstringspaces=false,
	       commentstyle=\color{BrickRed},
	       keywordstyle=\color{RoyalBlue}
	}
	\lstdefinelanguage{XML}{     % specific XML formatting overrides
		  basicstyle=\ttfamily,
		  morestring=[s]{"}{"},
		  morecomment=[s]{?}{?},
		  morecomment=[s]{!--}{--},
		  commentstyle=\color{OliveGreen},
		  moredelim=[s][\color{Black}]{>}{<},
		  moredelim=[s][\color{RawSienna}]{\ }{=},
		  stringstyle=\color{RoyalBlue},
 		  identifierstyle=\color{Plum}
	}
    % HOW TO USE? BASH EXAMPLE
    %   \begin{lstlisting}[language=bash]
    %   #some comment
    %   cd Documents
    %   \end{lstlisting}
\author{Jorge Silva Rocha}
\title{Stemmatics and Image. Remarks on the Twelfth-Century Beatus Commentary on the Apocalypse of Lorvão}


\begin{document}
\renewcommand*{\pagemark}{}

\begin{abstract}
% write your abstract here
Most images produced in the Middle Ages are closely linked to written
texts, as is quite evident in the field of illustrated manuscripts. In
the same way that texts were copied and circulated in space and time,
images also tended to accompany them with relative formal stability
constituting, like the texts, cycles of works. For this reason and in
some cases, textual criticism (and more specifically, stemmatics) has
extended its domain to pictorial production, comparing images for a
phylogenetic relationship and elements for the reconstruction of
hypothetical originals. This has been the case for the illuminated
manuscripts containing the \emph{Commentary} \emph{on the Apocalypse} by
Beatus de Liébana, where stemmatics has had a significant influence in
the approach to art history. Some notes on a twelfth-century
\emph{Commentary} \emph{on the Apocalypse}, the codex of Lorvão, reflect
on the functionalities and limits of the transposition of these textual
analysis methodologies for the appreciation of pictorial creation.
\end{abstract}

%%%%%%%%%%%%%%%%%%%%%%%%%%%%
%% YOUR ESSAY STARTS HERE %%
%%%%%%%%%%%%%%%%%%%%%%%%%%%%

% remove asterisk (*) if you want to number your sections
% add a title for your section in between the {curly brackets} if you need one
\section*{Introduction} 
Stemmatics belongs to the field of philology and appears to have nothing
to do with the realm of the creation and transmission of images. In the
history of art ––  especially in the history of medieval art ––  we
find, however, the same methodological basis to develop pictorial
\emph{stemmata} of artistic cycles, in particular when the origin of the
iconographic programmes are unknown. This is the case with the cycle of
illuminated manuscripts containing the \emph{Commentary} \emph{on the
Apocalypse} of John, attributed to the eighth-century Asturian monk
Beatus of Liébana, which is here the subject of remarks based on one
copy of the work, the manuscript of Lorvão.\footnote{\emph{Beatus of Lorvão} –– 
  \emph{Commentary} \emph{on the Apocalypse} by Beatus de Liébana
  attributed to the \emph{scriptorium} of the Monastery of São Mamede de
  Lorvão, 1189. Arquivo Municipal da Torre do Tombo, Lisboa. Call
  number: Ordem de Cister, Mosteiro de Lorvão, códice 44.}

The legitimacy of textual criticism and, by extension, ``stemmatics'' in
the historical analysis of images adjacent to a text is questionable.
Indeed, the production and reproduction of images are not submitted to
the same procedures and constraints as those of texts. However two
orders of reasons can justify a continuation, even a deepening, of image
\emph{stemmata}. The first reason, synthesized by M. Alison Stones, is
based on an epoch when the copying and transmission of text and images
followed specific rules and conventions. Stones notes that, in the
analysis of medieval manuscript cycles, we verify that the position of
the illustrations in the text and the content of these are not original
creations in each work and ``they tend to remain fairly standard, and it
is possible to work out recensions for the iconographical cycles in much
the same way as it may be done with texts'' \citep[96]{stones_secular_1976}. The
second order of reasons is broader and based on an understanding of the
history of art that deepens and develops through the study of series and
families of works: ``No image is ever completely present in isolation.
It often forms part of a series [\ldots]. The only meaningful work is
perhaps that of the entire series: isolating one image will always be
arbitrary and misguided'' \citep[31]{schmitt_images_2003}. If a series of works of
art is built phylogenetically ––  that is, in an evolutionary
relationship ––  we are, in fact, faced with a \emph{stemma codicum}. It
will then make sense to speak of a stemmatics of the image to designate
this process. In this paper, we present considerations of the way in
which the \emph{Lorvão} codex was integrated within the series of
illuminated manuscripts designated by \emph{Beatus} and the genealogies
proposed by historians such as Konrad Miller, Wilhelm Neuss, Peter K.
Klein, or John Williams. It is not within the scope of this paper to
discuss the general criticism that stemmatological research has received
in the course of the last decades, but rather to consider the
observations in relation to art-historical traditional discourses and to
discuss some of the implications for the field's methodology.

\section*{The Beatus family}
In the report of his trip to the kingdoms of Leon and Galicia and to the
principality of Asturias in 1572, under the orders of Filipe II,
Ambrosio de Morales\footnote{\emph{Viage de Ambrosio de Morales por
  Orden del Rey D. Phelipe II in Los Reynos de Leon y Galicia y
  Principado de Asturias para Reconocer las Relíquias de Santos,
  Sepulcros Reales y Libros Manuscritos de las Catedrales y
  Monasterios}. Republication of the report of Ambrosio de Morales's
  expedition between June 1572 and 20 November 20 1573 by Henrique
  Flórez. 1765. Madrid: Antonio Marín, pp. 51--52.} refers to an
``excellent manuscript'' that he has just discovered in the cathedral of
San Isidoro de León, a work that he ``hopes to give to the light, to be
very worthy of it''. Morales suggested the monk Beatus of the monastery
of Santo Toribio de Liébana, Asturias, as the likely author of the
original exegetical text, which was written in the eighth century. The
term ``Beatus'' is then the designation by which all \emph{Commentaries}
of this apocalyptic cycle are known. Twenty-nine illustrated Beatus have
now been identified, including complete works, fragments, and facsimiles.
For five centuries, this work of exegesis was copied and illustrated
with rare frequency in the \emph{scriptoria} of the north of the Iberian
Peninsula, south of France, and in Italy. The text of the
\emph{Revelation} written in Patmos has always given rise to abundant
exegesis, but the cycle that Beatus began in the eighth century is
unique not only for its textual content but also, or above all, for the
profusion of images with which the works are illustrated.

The codices identified and collected since the discovery of Morales have
been the object of reflection and analysis since the eighteenth century,
but several questions raised by the images remain open. Among the
interpellations made to the Beatus manuscripts, one continues
particularly to occupy art historians: the origin and figurative
evolution of the illustrations in the Beatus \emph{Commentary}. Indeed,
there are no witnesses for the first century and a half of its
existence.\footnote{Between the probable date of the first writing of
  the commentary (776) and those of the first fairly complete known
  Beatus (dated from the mid-tenth century), only the Fragment of Silos
  (Biblioteca del Monastero de Santo Domingos, frag. 4) remains ––  a
  single folio with an illumination dated to the last decades of the
  ninth century.} The original and the evolution of subsequent copies
remain unknown. The apparently isolated nature of this apocalyptic cycle
also makes it difficult to understand the genesis of this artistic
phenomenon.

\section*{Textual Criticism and Stemmatics of the Pictorial Tradition}

Stemmatics marked the textual and pictorial analysis of the Beatus cycle
from the beginning, thus merging textual criticism with art history.
Since the first attempts to identify the manuscripts made by Abbot
Ambrosio de Morales in 1573, this exegetical work received renewed
attention in the eighteenth century. In 1770, theologian and historian
Henrique Flórez produced the first unifying text for the
\emph{Commentaries}. In 1895, Konrad Miller drew the first pictorial stemma of the Beatus. By considering only the first image of the
\emph{Prologus libri secundi}, a world map that illustrates the
apostolic mission (\emph{Apostoli graece língua latine missi
interpretantur}) included in ten cited manuscripts, Miller was able to
organize a genealogical tree with two main branches (see Fig.  \ref{fig:rocha:miller1895}).

\begin{figure}[H]
\centering
\includegraphics[width=.45\textwidth]{media/rocha1.png}
\caption{Figure \ref{fig:rocha:miller1895}: Stemma of Konrad \citet{miller_mappaemundi_1895}.}
\label{fig:rocha:miller1895}
\end{figure}


 Image reproduced with permission from the University of Glasgow Library 


 
Miller used two main criteria based on the following presumption:
Beatus' original \emph{Commentary} was very detailed and would obviously
iconographically describe the apostles' missions. He thus divided the
maps into two families, the first consisting of the more detailed maps,
and the second of those that reveal many faults and absences. If the
visual reference to the apostles' missions is a plausible criterion for
the branching of the stemma, the assumption of great detail in
the original illustration is a simple personal conviction. Later, in
1954, Gonzálo Menéndez Pidal reformulated this stemma and
introduced two new world maps (Fig. \ref{fig:rocha:pidal}). But Miller and Menéndez Pidal
did not include the map of the Lorvão codex in their trees because they
were unfamiliar with it. Henry A. Sanders (in 1930) and Wilhelm Neuss
(in 1931) developed the first stemmata which encompass all
illuminated manuscripts known at that time.


\begin{figure}[H]
    \centering
        \includegraphics[width=\textwidth]{media/rocha2.jpg}
        \caption{Figure \ref{fig:rocha:pidal}: Stemma of Gonzálo \citet{menendez_mozarabes_1954}.}
        \label{fig:rocha:pidal}
    \end{figure}


 Image reproduced with permission from the University of Glasgow Library 


 
    \begin{figure}
        \includegraphics[width=\textwidth]{media/rocha3.jpg}
        \caption{Figure \ref{fig:rocha:neuss}Stemma of Wilhelm \citet{neuss_apokalypse_1931}.}
        \label{fig:rocha:neuss}
\end{figure}


 Image reproduced with permission from the University of Glasgow Library 


 
Neuss was the one to formulate, in the first place, the major questions
on all of the manuscripts, the exegetical tradition, the lost archetype,
and the first versions of the \emph{Commentary}. The pictorial programme
and iconographic tradition were also the privileged object of visual
analysis by Neuss in close interaction with the textual analysis; that
is, Neuss analysed the text and images together as a closely related
whole in a single genetic development. He either justifies the nature of
an image by the place it occupied in a codex during the transmission of
the text or, inversely, he relied on the character of the illustration
to consolidate the position of a manuscript in his Beatus stemma
(Fig. \ref{fig:rocha:neuss}). For Neuss, the text and image paths are parallel and
identical. This methodology finds significant expression in two
fundamental conclusions of the historian concerning the archetype: after
determining that the text of the Saint-Sever manuscript was closest to
the archetype, Neuss deduced that the illustration of this codex would
therefore be the most faithful to the original pictorial model. In
accordance with this deduction, he concluded that the style of the model
would be classicist and Italianate like that of Saint-Sever. Henry A.
Sanders proposed a new Beatus stemma based on the theory of the
existence of three initial versions (Fig. \ref{fig:rocha:sanders}). From the calculation of
the \emph{praesens era} included in Book IV in the calculation of the
world ages of Saint Augustine (\emph{septem aetates mundi}), Sanders
concluded that the different dates observed in different codices for the
current era would correspond to different editions \citep[1:165--167]{noauthor_actas_1978}.

\begin{figure}[H]
    \centering
    \includegraphics[width=.45\textwidth]{media/rocha4.jpg}
    \caption{Stemma of Henry \citet{sanders_beati_1930}.}
    \label{fig:rocha:sanders}
\end{figure}


 Image reproduced with permission from the University of Glasgow Library 


 
This idea had direct repercussions on the organization of the pictorial
development of the Beatus later elaborated by Peter K. Klein. However,
Klein would be the first to demonstrate the possibility of a process of
creating illustrations independent of the process of transmitting text.
To seek a more precise and objective method than the conventional
comparative method used by Neuss in the analysis of Beatus
illustrations, Klein used a statistical method called the correlation
test –– already used in quantitative linguistics –– because it is not
limited to analysis of coincidences but allows for differences between
the compared works. Applying this procedure to the works of the family I
from the stemma of Neuss, the results obtained indicate a
divergence of the pictorial tradition from the textual tradition of
Neuss and Sanders. According to Klein, the results are conclusive: the
text and images in these codices do not share a genealogy. Some doubts
about the rigour of this method were immediately raised when these
results were presented at the Madrid colloquium in 1976 \citep[2:107--115]{noauthor_actas_1978}.

Klein's method had the merit, however, of emphasizing the need for an
analysis of the transmission and creation of the images that is not
captive to the textual tradition. Apart from this virtue, the
statistical method used contributed very little to solving the problem
of the archetype or to the development of pictorial creations for the
Beatus during the first century and a half of the cycle of these
codices. As Klein himself noted, while the correlation test makes it
possible to establish differences and approximations between works, it
does not make it possible to determine which of the compared works are
the oldest. This is why Klein, in drawing the stemma of the
Beatus pictorial tradition –– despite the introduction of some changes
to the existing strains ––  essentially preserved the structure
inherited from Neuss and Sanders based on the Lachmannian method (Fig. \ref{fig:rocha:klein}). More recently, in an extensive compilation work John Williams
proposed a correction of the Klein family tree \citep[1:26]{williams_illustrated_1994}. Williams saw a difference between what Klein thought of the IIb
family and its representation in the stemma he proposed. In fact,
Klein agreed with Sanders when he distinguished family IIb from family
IIa, not as two branches from a hypothetical common archetype, but as
one (IIb) being a new revision of the other (IIa) ––  an idea that the
Klein diagram does not reflect. Williams thus proposed the corresponding
rectification (Fig. \ref{fig:rocha:williams}), but essentially everything remains in his
stemma for this cycle.

\begin{figure}[H]
    \centering
    \includegraphics[width=.45\textwidth]{media/rocha5.jpg}
    \caption{Figure \ref{fig:rocha:lein}: Stemma of Peter \citet{klein_altere_1976}.}
    \label{fig:rocha:klein}
\end{figure}


 Image reproduced with permission from the University of Glasgow Library 


 
\begin{figure}[H]
    \centering
    \includegraphics[width=.45\textwidth]{media/rocha6.jpg}
    \caption{Figure \ref{fig:rocha:williams}: Stemma of John \citet{williams_illustrated_1994}.}
    \label{fig:rocha:williams}
\end{figure}


 Image reproduced with permission from the University of Glasgow Library 


 
Finally, Roger Gryson published a new stemma of the
\emph{Commentary}, which differed little from that of Klein and opted
for ramifications not entirely binary (Fig. \ref{fig:rocha:gryson}). For the purposes of our inquiries about the genealogy of the image, however, it is important to
highlight Gryson's critical edition of the Beatus text of 776, published
in 2012. First, the enormous erudition and wisdom involved in textual
criticism and in all subsequent stemmata by all the authors
mentioned is not at issue here; rather, how illustrations were
considered in these manuscripts is the issue. This is where Gryson's
critical edition is a good example of the ambiguity and casualness of
his approach. The illustrations chosen to give an approximate idea of
the Beatus autograph archetype dated to 776 are paintings that belong to
manuscripts containing the text of a later edition (784). We must deduce
that the images were chosen for decorative purposes, which is in
contrast to the rigour required of the texts.

\begin{figure}[H]
    \centering
    \includegraphics[width=.45\textwidth]{media/rocha7.jpg}
    \caption{Figure \ref{fig:rocha:gryson}: Stemma of Roger \citet{gryson_beati_2012}.}
    \label{fig:rocha:gryson}
\end{figure}


 Image reproduced with permission from the University of Glasgow Library 


 
These genealogical constructions are not conclusive. The problem of the
configuration of the original Beatus pictorial work remains open. The
development of exclusively visual variants that each work in this cycle
presents has not yet been organized in a phylogenetic scheme. The
stemmata dedicated to the image persist, in general, dependent on
the textual tradition. On what basis, then, can we build a stemmatics of
the image? Let us look for possible answers in the analysis of some
approaches to images –– influenced by the methodology of textual
criticism –– from a specific Beatus, the manuscript of Lorvão.

\section*{The Beatus of Lorvão}

One of the twenty-nine known illustrated Beatus, this manuscript is
attributed to the \emph{scriptorium} of the monastery of São Mamede de
Lorvão, near the city of Coimbra.\footnote{The set of codices identified
  to date consists of complete and partial works, fragments and a
  facsimile from the seventeenth century, as well as more than 1,500
  illustrations.} Signed and dated 1189, this is a late copy of this
cycle of \emph{Commentaries}. Klein devoted a monograph to this codex \citep{klein_beato_2004}, but much of the iconographic differences identified by the
author in this \emph{Commentary} are attributed to ``errors'' of
interpretation by the artist of Lorvão. However, Klein noted his
impression that the Beatus of Lorvão is, in general, a faithful
testimony of the original tradition of the first archetype and a useful
resource in its reconstruction, although he did not fail to add that
this ``impression is obscured by curious misunderstandings'' \citep[42]{klein_beato_2004} by the illustrator. These supposed misunderstandings would be
justified by the ``style of a very old model, by the lack of theological
knowledge and, perhaps, by the little education of the miniaturist who
could hardly read the biblical text and that of the \emph{commentary}''
\citep[42]{klein_beato_2004}.

We now present two complementary proposals which may help to overcome
this divergence. The first is to pose the hypothesis, already formulated
by Henri Quentin \citeyearpar{quentin_essais_1926}, that an alleged variance can already be part of the archetype. The second, concerning visual creation, is that one
should not consider the existence of ``errors'' in this field, but only
options, because the idea of reproducing a work does not coincide
exactly with the idea of copying. The following briefly reviews some
illustrations from the more than seventy images in the Lorvão
\emph{Commentary} that may shed light on the differences between the
copyist's attitude and the illustrator's method, even when both are
identified as the same person.

\section*{Aspects of Image Creation in the Manuscript of Lorvão}

\subsection*{Iconography}

Elements added to the regular matrix of a cycle of illustrations disrupt
the concept of transmission in stemmatics. The codex of Lorvão presents
a very important number of iconographic elements which are not observed
in other Beatus. These elements have often been attributed to
misinterpretations by the artist, but are neither devoid of meaning nor
intention, as a look at a few examples will reveal.

\subsubsection{1. Super bestiam muliere (Fig. \ref{fig:rocha:super})} Despite its
inclusion in the Prologue of Book II, this painting illustrates the text
of Apoc. 17.3--14. Compared to the fourteen other known illustrations,
the image of Lorvão is the only one to present a set of six signs
inserted in small rectangles in the centre of the composition; five
heads at the back of the beast; the Christogram in the right hand of the
woman, who holds a crown, which we have found parallels to only in the
codex of San Andrés de Arroyos (Bibliothèque Nationale de France, Paris,
Lat. 2290). All of these elements find textual support both in the
biblical and in the exegetical text, which indicates that the author of
Lorvão had a deep knowledge of the substance of the work. The exegetical
text on the vision of Apoc. 17.3--14 can be found in two different
places: in the prologue to Book II, where the image now discussed is
inserted, and in Book IX, where the theme is repeated in the biblical
text with a new illustration. We must resort to two extracts of the
\emph{Commentary} to understand the decisions of the artist of Lorvão.

\begin{figure}[H]
    \centering
    \includegraphics[width=.45\textwidth]{media/rocha8.jpg}
    \caption{Figure \ref{fig:rocha:super}: \emph{Super bestiam muliere}. Folio 43r. Image credit for Lorvão codex: Arquivo Nacional da Torre do Tombo, Lisboa.}
    \label{fig:rocha:super}
\end{figure}


 Image reproduced with permission from the University of Glasgow Library 


 
The six signs are associated through the rectangles, where both the cup
and the symbols are inserted. The golden cup that the woman holds and
the abominations of John's vision (\emph{Et habebat calicem aureum in
manu sua, plenum abominationum}; \citealt[566]{gonzalez_echegaray_obras_1995})
are partially but perfectly identified in the exegetical text of the
Prologue to Book II.\footnote{The text of Beatus de Liébana followed in
  this article is the text unified by Henrique Flórez in 1770, and
  published in a bilingual edition prepared by \citet{gonzalez_echegaray_obras_1995}.} It is said that the woman on the beast represents, among
other evils, ``those who practise augury and enchantments and carry
signs, which the ignorant call Solomon's, or other similar signs that
they decide to make and wear around the neck, and collect herbs while
praying, or the women who observe spider webs or footprints'' (\emph{qui
auguria et incantationes et caracteres, quod signum Salomonis rustici
dicunt, vel alios huiusmodi caracteres, quos solent scribere, et collo
suspendere, et herbas symbolo vel oratione dominica vel cum incantatione
colligere, et mulierculas ad telas araneas vel pedes observare}; \citealt[162]{gonzalez_echegaray_obras_1995}). We can identify, in the small
rectangles, esoteric double spirals, a plant, a possible spider's web
and two stars, which perhaps allude to Solomon despite having five and
not six points (in one of the stars there appears to be a sketchy
attempt to trace a sixth branch). The five heads on the back of the
Beast are easily identifiable in the biblical and exegetical text. In
John's vision, the seven chiefs of the Beast are seven kings, of whom
five have perished, one is alive and the other is yet to come. These
seven kings are, in the \emph{Commentary}, one body of vices and
represent all of the kings of the earth. The five heads of those who
have already died are expressively pasted to the back of the Beast and
identified in the exegesis: Júlio César, Octávio, Tibério, Cláudio and
Galba –– Nero is still alive \citep[574]{gonzalez_echegaray_obras_1995}. The
crown above the woman's head is also explained by exegesis. In the
exegetical text of the Prologue to Book II, the commentator sets out the
different names by which the Church is called. These are paradoxical
designations based on ambiguous explanations, which mark the entire text
of the Beatus. Thus, the Church has the name of fornicator and
prostitute, as well as virgin, sister, bride, wife, mother or queen
(\emph{aliquando dicitur fornicaria, et meretrix; aliquando dicitur
virgo}; \citealt[164]{gonzalez_echegaray_obras_1995}). Regarding the Christogram
in the right hand of the figure, unique in the fourteen known
representations, we find its textual support in the Beatus commentary of
Apoc. 17.3--14 in Book IX. Here the woman is identified with the false
Church (\emph{simulatae veritatis ornata ostenditur, quia de foris
christianitas videtur. [\ldots] Id est, simulatio sanctitati}; \citealt[566]{gonzalez_echegaray_obras_1995}). The commentator's explanations
have complex developments, but these additional iconographic elements
represent a deep link with the texts, especially exegesis, and represent
an effort at a figurative translation and visual synthesis of the
plurality of meanings existing in the Beatus text. Throughout the Lorvão
Codex, we find this difficulty –– which cannot be confused with the
inability to illustrate or ignorance of the texts.

\subsubsection{2. Unclean spirits issuing from the mouths of the dragon, the
beast, and the false prophet (Fig. \ref{fig:rocha:spirits})} The illustration corresponding
to Apoc. 16.13--16 is an image without equivalent illustrations in the
other Beatus codices, and it is apparently unjustified: first, because
the illustration represents two dragons instead of one, and second,
because the amphibians come out not only from the dragon's mouth but
also from the mouth of the three human figures represented, which is not
in accordance with the apocalyptic vision.

\begin{figure}[H]
    \centering
    \includegraphics[width=.45\textwidth]{media/rocha9.png}
    \caption{Figure \ref{fig:rocha:spirits}: Unclean spirits. Folio 182r.}
    \label{fig:rocha:spirits}
\end{figure}


 Image reproduced with permission from the University of Glasgow Library 


 
Now in the text of the
\emph{Commentary} there is an expansion, a doubling of the Devil who is
one of the three prefigurations of evil, as well as the sum of these
three entities (\emph{Unum spiritum vidit, sed pro numero partium unius
corporis draconis, id est, diaboli. Nam et bestia corpus diaboli, et
pseudopropheta, id est, praepositi, corporis diaboliunus spiritus est}; \citealt[552]{gonzalez_echegaray_obras_1995}). The presence of two dragons is
legitimized by the meta-allegory, where the devil is himself and also a
member of his own body. The three human figures represented are
simultaneously the false prophet of John's vision and what this false
prophet represents in exegesis and in the plural: the false priests
(\emph{Pseudo tamen propheta, ipsi pseudosacerdotes sunt};
\citealt[552]{gonzalez_echegaray_obras_1995}). The use of three elements, both for the figures of the
Dragon and the Beast and for the false prophets, reinforces the
recapitulated character of John's vision, which the discourse of the
exegesis does not escape, and which makes the illustrator's work
difficult. Umberto Eco talks about John's text as an ``allegory to the
second power'' –– that is, an allegory which cites another allegory,
which in turn recalls other allegories. But if an allegory is the key to
interpretation, Eco notes, ``Apocalypse, on the other hand, is rhetoric
in action which ambiguously postulates and eludes a metalanguage capable
of defining it'' \citep[26--27]{eco_beato_1982}. The same idea is emphasized by Ives
Christe with regard to the visions of Jean, who ``in different forms and
figures recapitulate, \emph{eadem aliter}, the same reality'' \citep[1:56]{noauthor_actas_1978}. It is therefore natural that, in the \emph{Commentary},
these characteristics are projected and, in turn, the illustrations
devoted to the textual substance reflect the complex narrative
structure. This illustration is one example, among many others from the
Lorvão codex, of the visual translation of recapitulated allusions.

\subsubsection{3. The last attack of Satan (Fig. \ref{fig:rocha:attack})} More questions are
raised by iconography which is not exactly supplemental but rather
divergent from the standard established by the regularity of
corresponding illustrations known to date. 

\begin{figure}[H]
    \centering
    \includegraphics[width=.45\textwidth]{media/rocha10.png}
    \caption{Figure \ref{fig:rocha:attack}: The last attack of Satan. Folio 203v.}
    \label{fig:rocha:attack}
\end{figure}


 Image reproduced with permission from the University of Glasgow Library 


 
In the illustration of the
biblical and exegetical text corresponding to the Last Attack of Satan
(Apoc. 20.7--9), we see in the centre of the composition a reclining
woman who points to a circle with nine inscribed busts; the circle is
flanked by two nimbed figures. This iconography is unparalleled in any
other Beatus. In almost all the codices, instead of this representation
there are figures hidden in a schematic drawing which represent
mountains. It is the visual expression of the exegetical text which says
that, facing the attack of Antichrist \emph{qui in Iudaea sunt, fugiant
montibus [\ldots] tempore Antichristi per loca inaccessibilia multi
esse salvandi} \citep[622]{gonzalez_echegaray_obras_1995}. Only the codex of
Osma does not contain this scene in its composition; that of Geneva does
not represent the hills, but only a circle with the identification in
the legend \emph{ubi absconduntur a facie anti / christus}. \citet[126]{klein_beato_2004} alleges that the representation of the woman in the illustration of
Lorvão ``does not make sense in this context [\ldots] because Satan
does not drive nations to lust but to war.'' Yet it is precisely the
context that seems to give all of its meaning to the iconographic choice
of the Lorvão illustrator. We must emphasize that this illustration is
exclusive to the Beatus, and we do not find it in any other apocalyptic
cycle. It is therefore a creation that emerged in this cycle and
probably has not had a definitive iconography from the first moment the
image was imagined. The decision of the illustrator of Lorvão seems to
precede the graphic representation of the hills where ``those who are in
Judea'' take refuge (\emph{qui in Iudaea sunt, fugiant in montibus}).
This iconography becomes a stable standard in branch II of the Neuss
stemma from the Morgan codex 664, attributed to the painter
Magius. The model followed by the author of the Lorvão codex would
therefore not contain this graphic solution. In the \emph{Commentary} of
Lorvão, the artist privileged the representation of this crucial weapon
of Satan –– seduction. As the exegesis explains, seduction means
``losing oneself and leading to perdition'' (\emph{seducere hic dixit,
quod est dissipare et secum in perditionem adtrahere}; \citealt[620]{gonzalez_echegaray_obras_1995}); in the explanatory text, this often appears
metaphorically associated with the figure of a woman. The Message to the
Church of Thyatira (Apoc. 2.18--19) alludes to Jezebel, who seduces the
innocent (\emph{Ipsa est Iezabel, quae seducit homines simplices}: \citealt[224]{gonzalez_echegaray_obras_1995}); this results in the exclusive
illustration from the Lorvão Beatus with a female figure lying on a bed:
the iconography is very similar to the illustration in Apoc. 20.7--9.
This woman deceives the people and ``gets drunk on the blood of the
saints'' (\emph{Et vidi mulierem ebriam de sanguine sanctorum, et
sanguine martyrum Iesu}; \citealt[568]{gonzalez_echegaray_obras_1995}); she
always receives expressive representation in Apoc. 17.3--14 and in the
above referenced ``Woman on the Beast'' in the Prologue to Book II. It
is true that the exegesis recalls Matthew 24.16 through the evocation of
the call to those of Judea to flee towards the mountains, but the interpretive substance
throughout the text focuses on the doctrinal falsification and false
holiness that the figure of the woman has symbolized since the first
representation of Jezebel. The illustration thus appears to fit
perfectly into the exegetical context. Gog and Magog, who appear as
giants participating in the attack on the Holy City in the illustrations
of all known compositions –– with the exception of the Beatus of Geneva
which does not represent them –– are, we believe, represented
ambiguously beside the bed of the woman, simultaneously representing the
nations seduced and the vanity which animates them (\emph{Gog enim
interpretatur tectum: Dogmate sive de tecto Magog. Omnes quos extulit,
in superbiae suae lapsum adduxit vel elationis tecto sublevavit}; \citealt[620]{gonzalez_echegaray_obras_1995}). In the illustration of Lorvão, Gog
and Magog do not participate in the destructive action, because the
fantastic and dynamic traces of the vision of Jean are clearly relegated
to a secondary level in favour of a spiritual reading; this results in
an iconography that is more emblematic than scenic. The circle
represents the Holy City (\emph{castra sanctorum}) and gives us some
indications of the model used, as well as a possible origin for this
illustration. The circle only appears in two other Beatus; as mentioned,
this circle with busts inside of it appears in the Geneva codex with the
legend \emph{ubi absconduntur a facie anti christus}, while in the
Beatus of Escorial (Biblioteca del Monasterio, II.5), the circle
represents the Holy City, as in the Lorvão Beatus. It seems probable that
this iconography belongs to the model used by the artist from Lorvão and
probably to the original Beatus de Liébana, but it is not clear what
original meaning was attributed to it.

\subsubsection{4. Christ between the clouds (Fig. \ref{fig:rocha:christ})} Going through all of
the Lorvão illuminations, one always finds a deep commitment to the
texts and an attempt to translate the emerging content into images. The
artist's attitude towards the model followed is dynamic and sometimes
even audacious. In the Appearance of Christ Between the Clouds (Apoc.
1.7--8), we see the creative and unique way Alpha and Omega are drawn.
The stems of the letter \textbf{Α} form a gable on the nimbus of Christ. \citet[56]{klein_beato_2004} has identified this motif as unique and ``strange''
without, however, associating it with the text of the vision. The
\textbf{ω} is represented by the curved modelling of Christ's arms, a
posture we do not see in any other depiction throughout the manuscript
and which supports our interpretation.

\begin{figure}[H]
    \centering
    \includegraphics[width=.45\textwidth]{media/rocha11.jpg}
    \caption{Figure \ref{fig:rocha:christ}: Christ between the clouds. Folio 14v.}
    \label{fig:rocha:christ}
\end{figure}


 Image reproduced with permission from the University of Glasgow Library 


 
\subsection*{Formal elements}

The transformative initiative of the monk of Lorvão is also expressed,
beyond the additional iconography, in the realm of form by the
introduction of original graphic elements. In the messages to Smyrna and
Thyatira (Figs. \ref{fig:rocha:smyrna} and \ref{fig:rocha:thyatira}), the chromatic alternation is not only
ornamental, but also an element of the visual narrative. The orange
colour establishes the spatial connection between the message delivered
to John and its descriptive content. Also, not all of the formatted and
irregular backgrounds are the result of hesitation or lack of talent in
the preparation of the compositions. In the image of the fifth Angel who
Throws the Cup on the Throne of the Beast (Apoc. 16.10--11), the
excessive irregularity of the background seems to emphasize the
disharmony induced by the Beast (Figs. \ref{fig:rocha:angel} and \ref{fig:rocha:scarlet}). These irregularities
in the frame and background often appear in compositions where the Beast
is represented; we should not consider this fortuitous, but rather as an
explicit or unconscious intention of the artist of Lorvão.
\vfill
\begin{figure}[H]

    \centering
    \includegraphics[width=\textwidth]{media/rocha12.png}
    \caption{Figure \ref{fig:rocha:smyrna}: Message to Smyrna. Folio 54r.}
    \label{fig:rocha:smyrna}
    \end{figure}


 Image reproduced with permission from the University of Glasgow Library 


 
    \begin{figure}
    \centering
    \includegraphics[width=\textwidth]{media/rocha13.png}
    \caption{Figure \ref{fig:rocha:thyatira}: Message to Thyatira. Folio 64r.}
    \label{fig:rocha:thyatira}
\end{figure}


 Image reproduced with permission from the University of Glasgow Library 


 

\begin{figure}[H]
    \centering
    \includegraphics[width=\textwidth]{media/rocha14.png}
    \caption{Figure \ref{fig:rocha:angel}:The fifth angel poured out his bowl. Folio 181v.}
    \label{fig:rocha:angel}
\end{figure}


 Image reproduced with permission from the University of Glasgow Library 


 
  \begin{figure}
    \centering
    \includegraphics[width=\textwidth]{media/rocha15.png}
    \caption{Figure \ref{fig:rocha:scarlet}: Woman sitting on a scarlet beast. Folio 186v.}
    \label{fig:rocha:scarlet}
\end{figure}


 Image reproduced with permission from the University of Glasgow Library 


 
\subsection*{Composition}

The iconographic and formal elements mentioned have an experimental
character; they appear to be vacillating and in a clear phase of
spontaneous elaboration. Some compositions are clear attempts to
organize spatially the separate graphic elements previously disconnected
in the folio space, such as the Vision of the Son of Man (Apoc.
1.10--20). But there are also compositions which seem to result from a
stable model where the painter does not feel the need to intervene and
reproduces the model without major changes. The Vision of the Lamb and
of the four living creatures (Apoc. 4.6--14) presents an organized and
spatially balanced composition, as well as iconographic stability in all
known codices. We here make special reference to the representation of
the Celestial Jerusalem (Apoc. 21.1--17; Fig. \ref{fig:rocha:celestial-lorvao}), because until
recently, it was apparently unique and contained the ``usual
simplifications and errors'' \citep[136]{klein_beato_2004}. But another illustration
emerged in the new Beatus of Geneva from the Italian province of
Benevento, in which everything was identical to the image from Lorvão
(Fig. \ref{fig:rocha:celestial-geneva}). In two regions so different, the creation of two paintings so
similar to each other can only indicate an archetype containing a stable
composition that has been regularly transmitted.

\begin{figure}[H]
    \centering
    \includegraphics[width=\textwidth]{media/rocha16.png}
    \caption{Figure \ref{fig:rocha:celestial-lorvao}: Celestial Jerusalem. Lorvão. Folio 209v.}
    \label{fig:rocha:celestial-lorvao}
\end{figure}


 Image reproduced with permission from the University of Glasgow Library 


 
  \begin{figure}
    \centering
    \includegraphics[width=\textwidth]{media/rocha17.png}
    \caption{Figure \ref{fig:rocha:celestial-geneva}: Celestial Jerusalem. Geneva. Folio 241r.\footnote{Genève, Bibliothèque de Genève, Ms. lat. 357: The Institutions of Priscian and the Commentary on the Apocalypse by Beatus of Liébana (\url{https://www.e-codices.unifr.ch/en/list/one/bge/lat0357}). Image licensed under \href{https://creativecommons.org/licenses/by-nc/4.0/deed.en}{CC-BY-NC} by the Bibliothèque de Genève.}}
    \label{fig:rocha:celestial-geneva}
\end{figure}


 Image reproduced with permission from the University of Glasgow Library 


 
Finally, we have another order of compositions which consist of
illustrations created entirely by the illustrator, perhaps impelled by
an empty space between the biblical text and the text of the exegesis,
as in the Opening of the Seventh Seal and the representation of the
Silence in Heaven (Apoc. 8.1). Only eight codices visually mark this
vision, and each has a different iconographic solution. Regarding the
Beatus of Lorvão, we do not have uniformity in the genesis of the
paintings. The type of connection between the copyist and the model
seems to change for each illustration. Stemmatics cannot be applied
without taking into account the specificity of the production involved
in each image.

\subsection*{Integral archetypes and promorphic cycles}

We mentioned at the beginning that, for some philologists, the stability
and standardization of the production of images in medieval manuscripts
allows the creation of a recension of iconographic cycles in the same
manner as for the texts. The modes of transmission reported by Stones
are limited to only two or three procedures: the direct transfer of the
model to the space intended for illustration in the work in preparation,
and the schematic sketch of the composition in the margin of the
manuscript, later erased, as well as indications written in the margins
of the folios \citep[96]{stones_secular_1976}. Many scenes have also been taken from
general models (battles, banquets, etc.) and applied to the context of
the stories \citep{scheller_exemplum_1995}. But the production of the images in the
\emph{Commentary} of Lorvão, with a majority containing significant
interventions by the artist's initiative, does not appear to fit into
these copying procedures. Lorvão iconographic diversity cannot be
attributed solely to the difficulties and problems caused by a primitive
model. In fact, in the approach to medieval illustration, copying
processes were seen primarily as a means of pictorial reconstruction of
the sources used –– but they are more than that. There are many other
elements involved in the process of copying illuminated manuscripts. \citet{muller_use_2014} highlights some of these important factors in the
introduction to her work on the use of models in the Middle Ages. The
organization of the work and of the \emph{scriptorium}, the transfer
processes used, the technical-artistic constraints, the cultural
background, the availability of models and motifs, the intention and
objectives of the patrons, the messages that are sought in the selection
of certain models –– all of these are part of the intervening in the
transmission process of a work.

There is, however, also a transformative impulse in response to copying
that is motivated by what we can call an ontological condition of visual
representations. As ``re-presentation'', an image being transmitted is
always embedded in a \emph{hic et nunc}: it is always updated. In view
of this reality, the use of terms such as ``copy'' and ``quote'' should
be made with caution. \citet{muller_use_2014} has emphasized the fact that art
history does not yet have specific terms to describe this phenomenon,
and to cope with its ``structural and semantic consequences'' proposes
the use of Julia Kristeva's concept of intertextuality, where the
hypotext would correspond to the new elements of the copied work of art,
and the hypertext to the basic copied text. Hypotext is thus related to
hypertext and ``Perhaps using this terminology will allow us to describe
in more detail the different levels of the copying process which
translate into different levels of meaning'' \citep[xxvi]{muller_use_2014}. Often,
diversity prevails over unity in a series of works without, however,
losing certain cohesive elements sufficient to give an identity to the
cycle. Is a group of medieval works defined and structured by a single
source (archetype), with which it maintains a commitment to continuity,
or should we first consider the existence of cycles of mutant works?

Regarding the cycle of \emph{Commentaries on the Apocalypse} of Beatus
of Liébana, Neuss was confronted with this problem early on in his
attempts to link and merge the textual \emph{stemma} with the pictorial
\emph{stemma}. Indeed, Neuss could not explain the pictorial divergences
found in the oldest family of the tree, and thus divided his scheme into
two main branches. While the second branch, in particular sub-branch
IIa, exhibited relative consistency between textual genealogy and
illustrations, the older branch –– supposedly closer to the archetype
and, therefore, supposedly more homogeneous –– exhibited significant
incongruities. Neuss justified this reality by a loss of artistic skill
among the copyists of this period. Nevertheless, branch II has a formal
and iconographic stability: the variations are mainly stylistic. Indeed,
although a good number of these manuscripts are signed and include
identification of the scriptoria, which are located in the same region
near the models, the stemmatics with their traditional methodologies can
still be relevant and useful in this case.

We come back to the initial question of legitimacy and utility of a
stemmatics of the image imported from textual criticism for the history
of art. Regarding the intensive and extensive application of this method
in the analysis of the Beatus cycle, we have found that it has been
shown to have some usefulness and methodological adequacy in groups or
branches of works with some peculiarities: works produced in a
circumscribed region near scriptoria and accessible models, or works
with substantial formal stability. Textual criticism and stemmatics used
this hypothesis as a methodological basis, but these cycles and works of
art were rarely born from original works. As Kubler recalls, using the
biological metaphor, the mutant gene at the start of a series of works
can be infinitely small, but this mutant fraction imposes consequences
on the progeny of the art object. Kubler even wonders if the original
objects have a real existence; in exchange, he designates the initial
solutions for a series as ``promorphic'' and the final solutions as
``neomorphic'' \citep[40]{kubler_shape_1970}.

\section*{Retrograde Analysis}

Without more documentary elements of the genesis of an image other than
the image itself –– and perhaps a date, a location and an
identification of the authorship –– and having only a few similar works
to compare, how can one trace its stemma? How can one establish
the connections with previous works? When we approached some aspects of
image creation in the manuscript of Lorvão, we used a procedure very
often used in art criticism, the analysis of the art-making process. We
tried to understand the artist's options and visual solutions and, in
this way, tried to get closer to the meaning of the images. In doing so,
we were able to reconstruct, almost naturally, the starting point of the
illustrator –– that is, we managed to take a reverse path along the
artist's creative process. We can call this process a retrograde
analysis of the genesis of an image.

This is why we are convinced that the first \emph{Commentary} attributed
to the monk Beatus of Liébana was not, in terms of the body of the
illustrations, a complete work as we conceive it today. A few
illustrations recovered from a previous tradition, interspersed between
the apocalyptic message divided into extracts (\emph{historia}) and
respective exegesis (\emph{explanatio}), perhaps generated a desire for
successive completion. A stemmatics of the image should be considered
more in the cartography of this genetic development rather than in the
search for an archetype. But a stemma codicum structured in
bipartite branches is not suited to a process where variants and
transformations eliminate, replace or hide –– often by erosion –– the
sources of transmission. New visual solutions can invert or deviate from
a model; they can use different and anachronistic models, single
patterns or other cycles, building a network that is not compatible with
linear affiliation. Closer to the idea of a rhizome than a tree –– 
evoking the concepts formulated by \citet{deleuze_rhizome_1976} --
the genealogical structure of a cycle of works can be expressed in a
non-arborescent stemma, which develops in multiple,
multidirectional branches. At the same time, these connections tend to a
limit, as in an open function, a limit which may even be an integral
archetype, but more likely just an original trait in a mutant fraction
or a set of promorphic works.

Confronted with existing approaches to the pictorial work of Beatus'
\emph{Commentaries},\footnote{For an overview see \citealp[Vol. I, chap. 3, 31--102]{williams_illustrated_1994}.} the analysis of the creative process of the
illustrator from Lorvão shows us succinct but revealing evidence that
the evolutionary paths of textual transmission and pictorial
transmission are not of the same nature. While the textual tradition has
its main instrument in copying (with mutations), in the pictorial
tradition, in most cases, the identity of a series is preserved through
a significant and non-occasional sequence of deviations. The
predominance of these deviations over the preserved elements partly
dilutes their own filiation. In these cases, it is not appropriate to
speak of ``descent'', ``witnesses'', ``errors'' or ``contamination'' in
the observed variants. Does this compromise the relevancy or even the
workability of a stemmatics of the image? Otto Pächt, writing about the
genealogical tree of a work by Rembrandt, points out this difference
while acknowledging the existence of genetic evolution:

\begin{quote}
Il est peut-être bon ici de bien prendre conscience de la différence
d'essence entre la filiation biologique et la filiation artistique et
spirituelle. [\ldots] L'artiste peut, jusqu'à un certain degré, dans une
mesure qui varie selon la situation historique, choisir ses modèles ;
autrement dit, il détermine en réalité les ancêtres qu'il veut avoir et,
à la différence de ce qui se passe dans la relation parents-enfant, il
peut choisir d'avoir plus de deux géniteurs. [\ldots] Il s'agit bien de
liberté [\ldots] mais, en même temps, il n'est pas question de
``non-dérivabilité'' [\ldots] Cette liberté, il faut se la représenter
comme une liberté de choix, un droit de cogestion, de participation,
accordé au créateur, celui-ci ne pouvant toutefois choisir que ce qui,
en quelque manière, a des affinités avec lui.

\vspace{2em}

{[}It is perhaps good to become aware of the difference in essence
between biological filiation and artistic and spiritual filiation. The artist can, up to a certain degree, to an extent which
varies according to the historical situation, choose his models; in
other words, he actually determines the ancestors he wants to have and,
unlike what happens in the parent-child relationship, he can choose more than progenitors. It is indeed freedom but, at the same
time, it is not a question of ``non-derivability'' This freedom is a
freedom of choice, a right of co-management, of participation granted to the creator, a creator that cannot, however, choose what, in some way, has affinities with him.{]}
\begin{flushright}
    \parencite[135]{pacht_questions_1994}
\end{flushright}
\end{quote}

Since Heinrich Wölfflin and particularly since Alois Riegel the work of
art has been understood as a link in a historical chain. Despite this,
there is no area of art history devoted specifically to the study of
these genetic relationships. Hans Belting emphasizes in his ``Thèses sur
les tâches actuelles de la recherche en art'' {[}Theses on the current
tasks of art history{]} that history of art ``envisage ainsi des series
où une oeuvre est mise en rapport non seulement avec son publique mais
aussi avec un art antérieur, ou avec des artistes ou des oeuvres à
venir'' {[}envisages series where a work is put in relation not only
with its public but also with an earlier art, or with artists or works
to come{]}\citep[44]{belting_histoire_1989}. Under the
very broad and imprecise concept of ``influence'', the history of art
has developed its research without a clarification of what an influence
consists of. An area of study more dedicated to the genesis of images
could make a significant contribution to the making of art history. With
a more accurate methodology it could rescue the history of art from the
reductive simplicity of the iconographic and formal analogies on which
the history of art has been based, try to free itself, as far as
possible, from the ``tyrannie du visible'' {[}tyranny of the visible{]},
to use a Didi-Huberman expression \citep[64]{didi_devant_1990}.

The structuring role of variants (synchronous aspect of a single
witness) in the transmission of images (diachronic aspect of the
tradition) is not exclusive to the pictorial tradition. In philology,
the awareness of the contrast between the verticality of diachrony and
the horizontality of synchrony opened new paths in textual criticism
that can be shared with an eventual stemmatics of the image. The
so-called New Philology started with Cerquiglini's ``Éloge de la
variante''\citep{cerquiglini_eloge_1989} where
the new concept of ``variance'', in contrast to that of variant, opens
new paths for stemmatics,\footnote{The concept of ``mouvance'' had
  already been formulated by \citet{zumthor_essai_1972}.} and where the original itself is
understood as a potential variance in itself\footnote{See also \citealt{canfora_copista_2002}.} or the
emergence of a ``genetic criticism'' that, in contrast to stemmatics,
emphasizes the creative process in the textual transmission rather than
the search for the purity of the original,\footnote{See \citealt{ferrer_production_2002}. See also
  \citealt{biasi_oeuvre_2017}.} providing some
practical and theoretical foundations for a similar approach to
pictorial traditions.


\begin{flushleft}
    % use smallcaps for author names
    \renewcommand*{\mkbibnamefamily}[1]{\textsc{#1}}
    \renewcommand*{\mkbibnamegiven}[1]{\textsc{#1}} 
\printbibliography
\end{flushleft}

\end{document}