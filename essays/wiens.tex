%%%%%%%%%%%%%%
%% METADATA %%
%%%%%%%%%%%%%%

\contributor{
% Add all authors
Jason Wiens
}

\contribution{
% Add full title
The Alice Munro Papers: A Collective Genetic Approach
}

\shortcontributor{
% short version of authors for running header
Jason Wiens
}

\shortcontribution{
% short version of title for running header
The Alice Munro Papers: A Collective Genetic Approach
}

\begin{paper}
\renewcommand*{\pagemark}{}

\begin{abstract}
This essay discusses the ``Visualizing a Canadian Author Archive: Alice Munro'' project, a collective effort to gather data on a segment of the Alice Munro papers held at the University of Calgary in Calgary, Canada, and visualize the interrelations of the archival materials. Researchers gathered data on drafts of stories published in two of Munro’s collections: \emph{Who Do You Think You Are?} (1977) and \emph{The Moons of Jupiter} (1983), as well as correspondence relating to each story with magazines and publishing houses. The goal of the project was to visualize the genetic evolution of the stories, including in relation to each other. The paper includes examples of preliminary visualizations and offers considerations for further research.
\end{abstract}

%%%%%%%%%%%%%%%%%%%%%%%%%%%%
%% YOUR ESSAY STARTS HERE %%
%%%%%%%%%%%%%%%%%%%%%%%%%%%%

% remove asterisk (*) if you want to number your sections
% add a title for your section in between the {curly brackets} if you need one
\section*{} 
\begin{quote}
What I am suggesting by this too-brief discussion of what is to be found
in the Munro archives is that Munro's methods of composition, often readily
discernible there, lend themselves to archival analysis. Those methods
need to be probed more thoroughly than they have been. This is because
Munro's methods are fundamentally organic –– that is, as she has written
in the introduction to her \emph{Selected Stories}, she often imagines a
scene, seeing it as an image, and then tries to write its contexts, to
discover what it means. As she both imagines and works through the
story, focused on such contexts, she leaves clues throughout her
holograph drafts and typescript. {[}\ldots{]} Throughout these drafts,
such clues abound. And because of this, and especially because Munro's
work is in the short story form, hers is an art that demands mapping in
ways not yet done. To my mind, all such maps begin in Calgary.

\begin{flushright}
    \parencite[196]{thacker_reading_2016}
\end{flushright}
    
\end{quote}

\noindent \textsc{This essay discusses} the ``Visualizing a Canadian Author Archive: Alice
Munro'' project,\footnote{\url{https://libguides.ucalgary.ca/munroarchiveproject}
  This site provides a description of the project, a list of the
  researchers, metadata gathered, and visualizations.} a collective
effort to gather data on a segment of the Alice Munro papers held at the
University of Calgary and visualize the interrelations of the archival
materials, offering a preliminary response to leading Munro scholar and
biographer Robert Thacker's call for mappings of her archive in ways not
yet done, ``the need for better maps of Munro's work'' \citep[194]{thacker_reading_2016}. If
genetic critical investigations, like most literary archival work, are
generally undertaken individually by a solitary scholar carefully
examining and comparing drafts of a text in order to trace the creative
process, I want to explore the possibilities –– and limitations –– of
collective genetic critical projects here. In 2008--19, a team of researchers,\footnote{This team included faculty members
  from the Faculty of Arts at the University of Calgary: Dr. Noreen
  Humble, Dr. Murray McGillivray, Dr. Michael Ullyot, and Dr. Jason
  Wiens; collaborators from Libraries and Cultural Resources at the
  University of Calgary: John Brosz, Annie Murray, Ingrid Reiche, and
  Regina Landwehr; a postdoctoral researcher from Computer Science at
  the University of Calgary, Dr. Jagoda Walny; and graduate and
  undergraduate student researchers: Hannah Anderson, Min Lei, Florie
  Moran, Alexandra Mossman, James Robertson, Jon Rozhon, Paige
  Stoffregan, and Leah Van Dyk. The COVID-19 pandemic prevented us from
  extending the project due to the impossibility of working in the
  archives, and several members of the research team have since retired
  or moved to other positions, while most of the students have
  graduated. As I explain at the conclusion of this essay, I intend to
  apply for funding from the Social Sciences and Humanities Research
  Council of Canada to extend this project, which may possibly involve
  the digitization and transcription of the papers as well as the data
  gathering and visualizations I describe in this paper.} in collaboration with library
staff,
read through the drafts of 22 Alice Munro stories. They were all composed in the
late 1970s, and would eventually comprise two of her collections:
\emph{Who Do You Think You Are?} (\cite{munro_who_1977}; published in the United States
as \emph{The Beggar Maid}) and \emph{The Moons of Jupiter} \citep{munro_moons_1983}.
Researchers included faculty, graduate students, and undergraduate
students; each researcher put data from the materials into a template
with the goal of mapping the evolution of these stories in relation to
each other, as well as to the editorial interventions of Munro's editors
at the \emph{New Yorker}, Macmillan, Knopf, and other magazines and
publishing firms. As the sole Canadian literature scholar amongst the
faculty members of the team, as a teacher whose students have worked
with primary sources in the Alice Munro papers for years leading up to
this project, and as a researcher with particular interest in genetic
critical approaches, I took a lead role in the project and have written
this paper discussing the collaborative project from these teaching and
research perspectives. As a collective, ``distant reading'' genetic
investigation across a substantial subsection of an author's oeuvre, the
``Visualizing a Canadian Author Archive: Alice Munro'' project offers a
response to Dirk Van Hulle's question as to ``whether genetic criticism
and textual scholarship can connect to the more ambitious, going
`around' of distant reading and still go `on' doing what they have
always been good at, the careful transcription of drafts and direct,
microscopic analysis of textual relationships'' \citep[189]{van_hulle_genetic_2022}.

Due to ongoing copyright restrictions, this project focuses on
visualizing metadata gathered from a segment of the Munro archive,
rather than digitizing drafts and/or transcribing them so that
they may become searchable. In addition to gathering metadata to answer some
initial research questions about Munro's practice at this particular
stage of her career, as well as to prompt further questions both about
her writing practice and the nature of the archive itself, we sought to
create a public-facing resource which could alert researchers to
possible avenues of further investigation into the Alice Munro Papers.
Due to the time constraints of the project –– less than a year from the
conception and grant application stage to the conclusion of
funding –– securing permissions from Alice Munro and/or her legal
representatives to digitize and/or transcribe the materials was not
possible. Moreover, there are ongoing challenges to securing copyright
permissions in relation to the Alice Munro Papers in general. We feel
that this mapping of metadata, however, is in itself an innovation and
significant contribution: the visualizations we produced supplement the
existing finding aids which describe the collections.

\subsection*{The Alice Munro Papers, Pedagogy, and Genetic Critical
Approaches}

As genetic studies of Munro's fiction by scholars such as Christine
Lorre-Johnston and Nadine Fladd suggest, her work invites and rewards
genetic critical analysis \citep{lorre-johnston_pictures_2015,fladd_alice_2015}. The numerous drafts in the Alice Munro
collection evince a writer who revises heavily and retains the various
iterations of each story. We can trace the genetic sequence through most
stories from holograph drafts in notebooks to numerous typescripts and
galleys with holograph revisions; even after she began using a word
processor in the early 1990s, Munro continued to print drafts and revise
by hand. This process of revision continues even after first
publication, with many stories undergoing substantial revision between
first publication in magazines and their subsequent republication in
book collections, or even their republication in anthologies or
collections of selected stories by Munro. This process of ongoing
revision throughout the life of a story suggests that Munro's fiction
exemplifies John Bryant's observations about ``the fluid text'':

\begin{quote}
that writing is a process; that a literary `work' is not just a final
print version published but the full range of revisions which occur before and even after initial publication; that patterns of revision reveal a writer's ``revision strategies'', which give us a clearer sense of the writer's shifting intentions; and that an
analytic ``narrative of revision'' can be constructed as a critical means of assessing not simply a writer's development but the dynamics of an individual writer's acceptance of or resistance to the ideology of a culture.

\begin{flushright}
    \parencite[4]{bryant_melville_2008}
\end{flushright}
\end{quote}

Grzegorz Koneczniak argues that Konrad Górski's notion of ``textual
awareness'' ``seems to go hand in hand with Munro through her creative
development'' \citep[106]{koneczniak_works_2016} and that ``{[}e{]}ven a cursory look at the manner
in which the Canadian Nobel-Prize winner frequently modified versions of
her short stories {[}\ldots{]} can be treated as an invitation to a more
methodologically-oriented editorial discussion of Munro's oeuvre and the
way she shaped it'' \citep[105]{koneczniak_works_2016}. Because for Munro revision is a process which continues throughout the life of a story – including after publication –– and because interactions with editors of both magazine and
book publishers have significantly shaped her corpus, her work
especially invites textual and editorial scholarship, and, I would add,
genetic critical approaches. Moreover, the instability of Munro's texts
aligns with an ongoing concern with indeterminacy in her work; as Coral
Ann Howells observes, ``Munro's storytelling methods have always
encouraged a plurality of meanings as alternative worlds are positioned
`alongside' each other in the same geographical and physical space''
\citep[70]{howells_alice_1998}. The reader of Munro's stories is placed in a position analogous to
that of the genetic critic, considering and assessing numerous versions
of a story and always arriving at a place of provisional meaning and
completion.

In 1974, Alice Munro was first approached by the University of Calgary library with an offer to acquire her papers, though the arrangement was not
finalized until 1980 \citep[xxix]{tener_archival_1986}. Acquiring her papers was
part of an aggressive effort by then-head of Archives and Special
Collections, Kenneth Glazier, to build the national profile and research
capacity of a university which was only established as an autonomous
institution in 1966. Not unlike decisions made in the middle of the
twentieth century by other North American institutions, such as the
Lockwood Memorial Library at the State University of New York, Buffalo,
or the Harry Ransom Center at the University of Texas, the University of
Calgary saw an opportunity to quickly establish this profile and
capacity through acquiring the papers of contemporary writers, chiefly
Canadian. This strategy was supported by the comparably extensive
resources available to libraries for such acquisitions in the 1970s, not
to mention funding available to support the work of describing the
collections and producing detailed finding aids.\footnote{In his
  ``Series Introduction'' to the Canadian Archival Inventories Series
  produced by the University of Calgary Press in the 1980s, Charles R.
  Steele notes that ``{[}t{]}he establishment of the Canadian Studies --
  Research Tools Program by the Social Sciences and Humanities Research
  Council of Canada has formally acknowledged the frequently lamented
  absence from the field of Canadian Studies of such basic documents of
  research as indexes, inventories and bibliographies'' and that is is
  the aim of the series ``to contribute to the amelioration of this
  situation by providing basic research documentation'' \citep[v]{steele_series_1986}.} In
addition, during this period, contemporary writers were increasingly
willing to either sell their papers to university libraries, or donate
them in exchange for a tax receipt, not only for the financial benefit,
but to relieve them of the burden of holding them and ensure their
preservation in an organized archive. As Kathryn Sutherland puts it:

\begin{quote}
To a living author the mass of papers, sorted and unsorted, that accumulate in attics, garages, and garden sheds or lie encrypted in obsolete electronic
storage devices –– drafts of works published and unpublished, diaries,
and letters –– represents a burden
and, just possibly, money in the bank, a pension; to descendants of the
fashionable, the admired, or the best, a potential windfall. 

\begin{flushright}
    \parencite[31]{sutherland_why_2022}
\end{flushright}
\end{quote}

While Alice Munro was already a prominent and award-winning author in
Canada at the time of the arrangement reached with the University of
Calgary, and –– as I discuss further below –– her then-newly-established
relationship with \emph{The New Yorker} would propel her to prominence
in the United States, at this point the financial benefits of the
arrangement would have been especially welcome. And as Munro's national
and international profile has grown since the agreement was reached, her
papers have become the most internationally significant of the
University of Calgary collections, particularly after her reception of
the Nobel Prize for literature in 2013.\footnote{While this essay was at the proofing stage, Alice Munro passed away in May 2024. On July 7, 2024, her daughter, Andrea Robin Skinner, published a brief memoir in the \emph{Toronto Star} which revealed she was sexually abused by Munro’s second husband, Gerald Fremlin, in 1976, when she was 9. Skinner revealed that she told Alice Munro about the abuse in the early 1990s, and Munro chose to remain with Fremlin, who later pled guilty to indecent assault in 2005 and was given a suspended sentence. Fremlin died in 2013. These revelations have caused consternation among scholars of Munro, including myself, and while they do not impact my overall argument here, they no doubt will reframe how we read Munro’s fiction as well as her archive.}

Although, as Stephen Brier has argued, the digital humanities, broadly
speaking, have tended to focus too narrowly on research and scholarship
while ``also minimizing and often obscuring the larger implications of
DH for how we teach in universities and colleges'' \citep[391]{brier_wheres_2012}, this
project has its roots in an archival-based pedagogy. As well as
supporting scholarship, the Munro papers provide excellent opportunities
for undergraduate and graduate student research projects. I have had
students work with the drafts of Munro stories we are studying as part
of archival projects that have replaced the conventional group project,
but still encourage collaborative peer learning. Students examine drafts
of a story, or correspondence relating to it, collect images of the
materials with their phones, which they share with the class through our
learning management system, and then write short essays in genetic
criticism in which they identify substantive variants between drafts
and the published story, explaining the significance of these revisions
to the story as a whole. In addition to the opportunity for students to
conduct primary archival research, their close work with these
materials, I will admit, has provided a crowdsourced, quantitative
investigation into the Munro archive which has alerted me to curious
compositional histories, enabled a greater sense of her practice's
evolution, and spurred my interest in the possibilities of quantitative
archival and genetic analysis.

Although Munro had initially claimed in correspondence with Calgary that
she had not retained ``that many manuscripts'', this fortunately proved
not to be the case, ``for when a trunk and a suitcase were unpacked in
1980, there were 2.5 metres of papers going back to the early 1950s''
\citep[xxix]{tener_archival_1986}. The materials from the mid-1970s,
when Munro acquired the representation of Virginia Barber in New York
and began to publish in commercial U.S. magazines, most prominently the
\emph{New Yorker}, provide an especially rich resource which gestures
towards the ways in which Munro's practice at this point enters into
dialogue with editorial influences at \emph{The} \emph{New Yorker}. For example, a letter dated
November 18, 1976, from \emph{The New Yorker} editor Charles
``Chip'' McGrath, suggests Munro make the location of her
stories more explicit, because ``American readers think everything
happens in America unless you tell them otherwise'' \citep{mcgrath_letter_1976}. A 1981 fan letter
from a Richard Carr of Solon, Maine, would appear to support McGrath's
suggestion, telling Munro ``{[}i{]}t's refreshing to see a slick
magazine publish something realistic about New England'' \citep{carr_letter_1981}. Carol L. Beran notes \emph{The New Yorker}'s ``editorial
policies regarding language and descriptions that might offend
conservative readers created small conflicts between the magazine and
Munro'' \citep[207]{beran_luxury_1998}, which occasionally led Munro to attenuate the
forcefulness of diction and imagery, particularly as relates to sex and
the body. And so, for example, in ``Royal Beatings'', the first story
Munro would publish in the \emph{New Yorker} in 1976, horses that
wouldn't ``pull your cock out of a pail of lard'' become horses that
wouldn't ``pull your tail out of a pail of lard'' \citep[207--208]{beran_luxury_1998} (``cock''
would be restored to its rightful place in the story's publication two
years later in \emph{Who Do You Think You Are?}). As a result of such
editorial interventions, Beran contends, ``Munro's depiction of
lower-class people in rural Ontario becomes slightly less authentic when
the genteel image of \emph{The} \emph{New Yorker} excises their
characteristic impoliteness and deliberate lack of gentility'' \citep[209]{beran_luxury_1998}.

In her book-length study of Munro's archives, JoAnn McCaig situates the
editorial influence of Munro's editors, at \emph{The New Yorker} and
especially at Norton during the composition of the stories which would
comprise \emph{Who Do You Think You Are}?, within a framework of
``American cultural hegemony'' \citep[121]{mccaig_reading_2002}. Similar to our project,
McCaig focuses on the 1970s as ``a particularly intense and fascinating
period in her authorial history'':

\begin{quote}
It is during this time that she enters the American literary marketplace, marries for
the second time, and struggles with genre privilege while preparing \emph{Who Do You Think You Are?} for publication. In general, the tensions between nationality, gender, and genre all clash during this period, and some very powerful negotiations are needed to bring Munro into the full force of her own authority. 

\begin{flushright}
    \parencite[13--14]{mccaig_reading_2002}
\end{flushright}
\end{quote}

Noting that publishers generally regard a novel as an easier genre to
market than a collection of short stories, McCaig discerns in the
archive an insistence by Munro's editor at Norton, Sherry Huber, that
the stories in that collection be linked closely, an insistence not
shared by Munro's Canadian publisher, Macmillan. Both McCaig's book and
Helen Hoy's essay ``Rose and Janet: Alice Munro's Metafiction'' discuss
the ``tortuous'' \citep[69]{hoy_rose_1989} editorial and publication history of \emph{Who
Do You Think You Are? / The Beggar Maid}; McCaig suggests that Munro's
resistance to linking the stories more closely was both feminist and
postcolonial, as ``the genre issue has links to gender and nationality
or colonial status'' and ``{[}i{]}f the short story ranks below the
novel in the hierarchy of genres, the categories of autobiography or
memoir fall even lower on the scale'' \citep[122]{mccaig_reading_2002}.

\emph{The New Yorker}'s editorial interventions in Munro's writings
would appear to reflect this genre hierarchy which McCaig identifies,
even if its reasons were perhaps more legal than generic. Even more
significantly than \emph{The New Yorker}'s discomfort with the rural
working class language of some of Munro's writing, an ``editorial
dislike of autobiographical stories'' evidently arising from an earlier
``lawsuit about a supposedly fictional work by another author'' \citep[207]{beran_luxury_1998},
would lead to some of Munro's more apparently ``autobiographical''
stories to be rejected or substantially edited. 

Genetic research in the drafts of her story "The Progress of Love," for example, suggests that relatively late in the composition process, Munro changed the narrative perspective of the story from first to third person prior to its publication in \emph{The New Yorker} (October 7, 1985), and then changed it back when it was republished as the title story of her 1986 collection \emph{The Progress of Love}. Here are Munro's edits on one draft of the story's first paragraph, perhaps made at the recommendation of her editor at the magazine:
\vspace{1em}
\begin{figure}[H]
    \centering
    \tiny
    \scalefont{1.2}
\begin{tabular}{p{\textwidth}}
        \text{\textsuperscript{Phemie's} \hspace{5em}\textsuperscript{her} \hspace{7em}\textsuperscript{her that her} \hspace{19em}\textsuperscript{Phemie's} \hspace{5em}\textsuperscript{she}}\\
   		\text{\sout{My} father phoned \sout{me} at work to tell \sout{me that my} mother had died. This was not long after \sout{my} divorce, when \sout{I}}\\\vspace{1em}
        \text{\hspace{8em}\textsuperscript{her}}\\
        \text{finally had both of \sout{my} sons in school, and had started working in the real estate offic. \sout{[QA: something like this?]} It} \\\vspace{1em}
        \text{was a hot enough day in September.}
	\end{tabular}
    \caption{Transcript from: \cite{munro_typescript_nodate}, \citetitle{munro_typescript_nodate}.}
    \label{fig:wiens-ts}
\end{figure}
\vspace{1em}

The shift to third person for the magazine publication, for instance, may have been prompted by a suggestion to further distance the narrator from the author. Although Munro's employment of free indirect discourse in the
third-person narration of \emph{The New Yorker} version allows the
reader access to the narrator Phemie's thoughts, recollections, and
perspectives, this narrative perspective creates a further detachment
between story and narration, the told and the telling. This is all the
more consequential in a story which establishes an unreliable narrative
perspective in Phemie's recollections, for example, in her uncertainties
over whether or not her father witnessed her mother burning her
inheritance in the stove \citep[28--29]{munro_progress_1986}. The return to first person narration in subsequent publications, then, suggests Munro regarded the
shift to third person for publication in \emph{The New Yorker} as a
compromise, and that Phemie's self-questioning at the story's
end –– ``How hard it is for me to believe that I made that up. It seems
so much the truth it is the truth; it's what I believe about them''
\citep[30]{munro_progress_1986} –– more effectively embodies the story's concerns with
the imperfections of memory when narrated in first person.

In a 1982 interview with Geoff Hancock, Munro describes her
compositional practice as sequential, with the composition of each story
a singular process: ``I see everything separate. Right now, all that
matters to me is making a new story. It's as if I had never made any in
the past. When that story's finished, it's only the new one that will
matter'' \citep[77]{hancock_interview_1982}. The archive, however, suggests a different
practice, revealing Munro's stories as composed less as discrete works
but rather generated from a compositional matrix in which names,
episodes, imagery, and other elements are continually mutating and
overlapping intertextually as each story takes shape. 

In my own readings through the archive, for example, I discovered an episode in a holograph notebook from the mid-1970s, in a draft of what would become her story "Chaddeleys and Flemings" in which neighbours are alerted to a woman threatening to hang herself in a barn:

\begin{quote}
Franklin's mother was in the barn, just as he had said. She was standing on a kitchen chair in the middle of the space her husband usually parked the car, and she had a noose around her neck. Phyllis and Marjory and the dog were hanging around the barn door. 

"Oh Mrs. Cole!" said Winona's mother in a tired voice. "Come down out of that."
\begin{flushright}
    (\cite[197]{munro_alice_nodate}, \citetitle{munro_alice_nodate})    
\end{flushright}    
\end{quote}

\noindent This episode eventually disappears from that story only to resurface as a key scene in "The Progress of Love":

\begin{quote}
At the end of the yard is a small barn, where they keep firewood, and some tools and old furniture. A chair, a straight-backed wooden chair, can be seens through the open doorway. On the chair, Marietta sees her mother's feet, her mother's black laced shoes. Then the long, printed cotton summer work dress, the apron, the rolled-up sleeves. Her mother's shiny-looking white arms, and neck, and face.

Her mother stood on the chair and didn't answer. She didn't look at Marietta, but smiled and tapped her foot, as if to say, "Here I am, then. What are you going to do about it?" Something looked wrong about her, beyond the fact that she was standing on a chair and smiling in this queer, tight way. Standing on an old chair with back rungs missing, which she had pulled out to the middle of the floor, where it teetered on the bumpy earth. There was a shadow on her neck.

The shadow was a rope, a noose on the end of a rope that hung down from a beam overhead.

"Mama!" says Marietta, in a fainter voice. "Mama. Come down, please."
\begin{flushright}
    \parencite[10]{munro_progress_1986}
\end{flushright}
\end{quote}

Recognizing that in order to trace
the composition of a single story it would be ``advisable to zoom out
and consider the genesis as part of an entire oeuvre in motion'' \citep[60]{van_hulle_genetic_2022}, I began to think about how an individual student or faculty
member's qualitative, individual archival investigations might be
extrapolated into a quantitative, collaborative practice, providing a
``distant reading'' of the archive through the accumulations of hundreds
of close genetic readings. I was also reminded of the urgent need to
digitize these documents, build upon the excellent foundational work in
the finding aids produced in the 1980s, and develop new resources for
researchers interested in these papers.


\subsection*{Visualizing an Author's Archive: Alice Munro}

As timing would have it, an internal grant opportunity would provide the
resources for such a project. In the Fall of 2018, I and several colleagues
from the departments of English, Classics and Religious Studies, and
Computer Science, as well as our digital library, received a subgrant to
develop the project ``Visualizing a Canadian Author Archive: Alice
Munro''. Funded by a grant awarded to the library from the Mellon
foundation, these subgrants are intended to cultivate collaborations
between the library and faculty working in different disciplines. The
amount of the subgrant was modest –– a little over \$20 000 CDN –– and the
timeline short (about eight months), so our scope was limited to a pilot
project. Our initial proposal stated that the project would ``focus on
the Alice Munro collection in Archives and Special Collections,
specifically her correspondence with other writers, her readers, and her
editors and agents as they negotiate revisions and publications of her
work''. We further narrowed our focus to stories relating to two of
Munro's collections –– \emph{Who Do You Think You Are?} \citep{munro_who_1977} and
\emph{The Moons of Jupiter} \citep{munro_moons_1983} –– all of which were composed during
the same timeframe of the mid to late 1970s, that crucial period in
Munro's career when she acquired New York representation and began to
expand her international reputation with publications in U.S. magazines.
The data gathered from the correspondence affiliated with these stories
trace the networks that emerged between Munro, her publishers and
editors at various magazines and publishing houses in New York and
Toronto and, eventually, her readers. One of the goals of our project
was to determine if there was any consistent causal relationship between
the revisions to Munro's drafts and the correspondence with the various
publication venues of her stories. Excellent work has been done on the
editorial influence during this period on stories such as ``Royal
Beatings'' \citep{beran_luxury_1998} and ``The Turkey Season'' \citep{fladd_alice_2015}, and we had hoped
our project may reveal other interesting intersections of editorial
intervention and Munro's writing practice. This ultimately proved impossible to visualize accurately, however, because while the correspondence is entirely dated, the drafts never are.

Nevertheless, we wanted to see if a genetic critical method, a method
typically performed individually and qualitatively, could be scaled up
and performed collectively and quantitatively. And we wanted to see if
these findings could be visualized in interesting and revealing ways. We
established clearly defined data units, hoping to avoid ambiguities and
not rely too much on interpretive decisions of researchers. Indeed, our
work with the archive could be read as a scaled down, analogous version
of Franco Moretti's ``distant reading'' approach. His ``evolutionary
model'', in which the fittest texts not only survive the slaughterhouse
of literature, but establish the literary values that ensure their
continued survival, parallels a genetic critical method. In such an
analogy, the various textual witnesses that are discarded in the
revision process would approximate his notion of literary ``rivals'' to
the canonical works: ``contemporaries who write more or less like
canonical authors\ldots but not quite'' \citep[66--67]{moretti_distant_2013}. The scenario
Moretti identifies as ``Darwinian'' in the evolution of the novel
parallels a genetic understanding of the composition process: ``the
field opens up to a multiplicity of possibilities, it narrows down as an
external pressure selects one and discards the other, then it opens up
again to a new multiplicity'' \citep[264]{moretti_signs_1983}. If Moretti's practice
is a ``\emph{formalism without close reading}'' (Arac cited in \cite[65]{moretti_distant_2013}; italics in original), then ours is a formalism of multiple and
various close readings, with the shared ground of ``identifying
{[}multiple{]} discrete formal trait{[}s{]}, and then following
{[}their{]} metamorphosis through a whole series of texts'' (\cite[65]{moretti_distant_2013}; italics
removed).

Aided by graduate and undergraduate student researchers, we cast a wide
net in gathering metadata, which we hoped would enable us to visualize
these relationships in significant ways. The student researchers and I
spent hundreds of hours reading through all of the drafts and
correspondence relating to the 22 stories which eventually comprised
these two collections, compiling spreadsheets that charted the various
iterations of titles; narrative points of view; character names,
professions, and relationships of each character to the protagonist;
settings, both primary and secondary; cultural references, both ``high''
and ``mass''; historical and geographical references; authorial and
editorial revisions; Munro's typed deletions; even the colour and
dimensions of the paper as well as the colour of pens and typewriter
ribbons. For the correspondence, we recorded the dates; locations and
professional positions of the correspondents; stories discussed;
rejections and acceptances; other individuals mentioned; and references
to other cultural institutions and events. We collectively harvested an
enormous amount of data from the drafts: one spreadsheet I compiled of
metadata from the drafts of ``Chaddeleys and Flemings'' eventually grew
to 239 rows across 20 columns, and extrapolating this across the
comparable number of drafts of 22 stories produced a sizable data set.

Our determinations about which categories to gather data were shaped by
our own understanding of the Munro papers and in which categories many
of the substantive revisions were typically made. But these
determinations were also influenced by existing scholarship on Munro.
The decision to map the changes to character names across the drafts,
for example, arose from a recognition that, as James Carscallen has
noted:

\begin{quote}
{[}i{]}n studying drafts and earlier published forms of the stories, we
find that characters we know from a single name may have had others in the course of their evolution –– Rose {[}the protagonist of \emph{Who Do You Think You
Are?}{]} has had a good dozen. Munro herself has remarked that, while
the names may change, the characters do not, but true as this is (the contrary, by the way, would be equally true), it hardly explains her need to fuss with her names as much as she does, nor could any such
explanation dispose of similar variations within her work in its final
form. 

\begin{flushright}
    \parencite[85--86]{carscallen_other_1992}
\end{flushright}
\end{quote}

Asked in her interview with Hancock if names are a clue to her
characters, Munro responds, ``Well, they might be but I don't choose
them that way. I usually choose names that sound right. But with
difficulty. I change names a lot. And then part ways into a story I
realize that the name I've been calling a character isn't his or her
name. I know what the name is'' \citep[101]{hancock_interview_1982}. Similarly, mapping
changes to both settings and geographical references would seem
especially important in the work of a writer for whom place is such a
powerful determinant of character and narrative; Robert McGill, for
instance, has called Munro's work ``geographic metafiction'', a fiction
which, ``even as it configures space and place, examines its own ability
to do so'' \citep[103]{mcgill_where_2002}.

In addition to comparing different drafts of a text to the completed
version and focusing on isolated variants and their significance, our
approach demanded that we record the data in such a way that
consistencies across drafts were also noted, establishing how and when
in the composition process, for example, certain character names,
settings, or other details solidify, and when others are discarded. I
should note that, because the drafts are undated, we followed the
sequence produced by archivists when the accession of Munro papers were
described; therefore, any inaccuracies in ``original order'' produced in
the finding aid 35 years ago would be reproduced in the course of our
project. We were unable to find any way around this problem, although we
did hope that our meticulous examination of the drafts might identify
any such errors in sequencing, and indeed some of the data gathered were
intended to clarify the temporality of each story's composition. While
we initially transcribed each individual revision, this proved too time
(and therefore resource) consuming, and we instead moved to a
quantitative measure of revisions and deletions. This produced a kind of
distant reading of a particular section of the Munro archive, made all
the more distant and quantitative as it was employed across the
materials relating to 22 stories, rather than just one.

John Brosz, our library's coordinator of data visualizations, and Jagoda
Walney, a postdoctoral fellow working on visualization, then produced
visual models of the data the research team had collected.\footnote{See: \url{https://library.ucalgary.ca/munroarchiveproject}} 
The CSV files of original raw data from our investigations are also
open-source and available for download at this site. We hope our
attempts to this point might raise general questions as to how one might
visualize a literary archive and indeed a literary corpus considered in
the quantitative terms of a distant reading approach. The Exploratory
Visual Analysis (EVA) tool we chose to employ was Tableau, due to its
flexibility in handling both large and small data sets, its
user-friendly functionality, and the clarity of the visualizations it
produces. Reviewing the literature on EVA projects, Leilani Battle and
Jeffrey Heer note that EVA projects often do not have clear goals, and
that some researchers ``describe EVA as a situation where analysts may
not know what they are looking for, but will know something interesting
when they see it'', while others assert that in fact ``EVA is
\emph{more} effective when the goals are \emph{less} clear'' (\cite[147]{battle_characterizing_2019};
emphasis in original). They further note that ``{[}o{]}thers take a
different view, saying that analytical goals evolve throughout the
course of an EVA session: the analyst starts with a vague goal, and
refines and sharpens this goal as they explore'' \citep[147]{battle_characterizing_2019}. Our method
combined both approaches: starting with a particular goal –– attempting
to determine if there is a recurring pattern between the editorial
correspondence during this period and Munro's revision process –– and
then seeing if a visualization of this cross section of the archive
would provoke further research questions.

\subsection*{Preliminary Visualizations and Future Possibilities}

Genetic critical approaches generally seek to reconstruct or trace the
creative process through determining the sequence of the process and
decisions made as evidenced in the drafts of the \emph{avant-texte.}
Genetic criticism may employ a visual ``map'' of the creative process,
such as the ``genetic map'' of Samuel Beckett's \emph{Fin de partie}
(\emph{Endgame}) in the Beckett Digital Manuscript Project \citep[11]{van_hulle_genetic_2022}. According to Van Hulle, ``{[}w{]}hat genetic maps show above all is
that authors' intentions can be highly mutable, to the extent that
several genetic critics no longer see the writer as a monolithic `self'
but as a succession of selves'' \citep[11]{van_hulle_genetic_2022}. Because our approach looked at a
cross-section of an author's archive in relation to not one, but 22
texts, our visualizations would take a different form than such genetic
maps. Van Hulle observes that,

\begin{quote}
{[}w{]}ithin Digital Humanities, both genetic criticism and textual
scholarship are exceptional in that they are forms of microanalysis, whereas the general trend is towards `macroanalysis' (Jockers 2013). But this macroanalysis
generally makes use of only one version of the text in their corpus. What is lacking,
therefore, is macroanalysis across versions. 

\begin{flushright}
    \parencite[176--77]{van_hulle_genetic_2022}
\end{flushright}
\end{quote}

The ``Visualizing a Canadian Author Archive: Alice Munro'' project thus
employs the tools of genetic criticism –– microanalysis of all the
documents in the life of a text, including multiple
publications –– without engaging in interpretation at the ``micro''
level. Rather, we sought to collect data across the lives of multiple
texts written contiguously with each other, and then visualize the data
gathered in what we hoped would be revealing ways.

Because of the temporal and resource constraints of our project, we have
not yet created visualizations in response to all the research questions
we have asked. We first worked to visualize the quantity of
revisions –– revisions and deletions made by Munro, as well as
``external'', i.e., editorial revisions –– over the sequence of drafts of
a particular story through vertical bar graphs. We then juxtaposed these
different graphs, suggesting which stories were more heavily revised and
in which drafts the most revisions occur; these visualizations also
provide a quick comparison of the number of drafts produced for each
story (see Fig. \ref{fig:wiens3}). These visualizations may be of most interest to
genetic scholars, suggesting which stories were more heavily revised and
therefore potentially offering more fruitful directions for research.
Stories such as ``The Turkey Season'' or ``Royal Beatings'', which
scholarship has already established were shaped by significant editorial
intervention, are represented as such. Other stories, such as ``The
Moons of Jupiter'' or ``Providence'', are shown as having been revised
significantly through external editorial interventions, but scholarship
to date has to our knowledge not yet explored the significance of these
interventions.
\begin{figure}[t]
    \centering
    \frame{\includegraphics[width=1\textwidth]{media/wiens3.png}}
    \caption{Occurrences of revisions in Munro's stories. Data visualization created by John Brosz and Jagoda Walny, and reproduced with their permission.}
    \label{fig:wiens3}
\end{figure}

We also mapped both settings and geographical references in the set of
stories (Fig. \ref{fig:wiens4}). Such visualizations perhaps reinforce what we
already know about Munro's work: that she sets most of her stories in
Ontario, southern Ontario in particular, with substantial reference as
well to British Columbia. If we expand the project to eventually
incorporate all of Munro's works, these maps may be of comparative
interest in terms of visualizing the expansions (or contractions) of the
geographic range of Munro's work over her career (or perhaps reinforce a
critical perception that associates her so closely with one particular
region). Alongside the mapping of settings, the visualization also
provides a global map of references to various locations beyond setting
in the 22 stories –– references to places characters visited, for
example, or origins of secondary characters, or geographic points of
comparison. While almost all continents are represented, locations in
North America and Europe appear to dominate, perhaps unsurprisingly.
Although the researchers who gathered the data did not distinguish
between actual locations, such as Sudbury or Lake Huron, and fictional
ones, such as Jubilee or Hanratty, it would not be difficult to modify
the visualizations to distinguish between such locations and thus
establish a ratio between the real and the invented in Munro's fictional
worlds.
\begin{figure}[t]
    \centering
    \frame{\includegraphics[width=1\textwidth]{media/wiens4.png}}
    \caption{Settings and geographical references in Munro's stories. Data visualization created by John Brosz and Jagoda Walny, and reproduced with their permission.}
    \label{fig:wiens4}
\end{figure}

As noted above, we had hoped to visualize a determinative relationship
between the editorial correspondence in the Munro papers and the drafts
of her stories discussed in this correspondence, but because the drafts
are undated, establishing this relationship conclusively proved
impossible. However, we were still able to create visualizations that
could help us roughly establish the compositional time of the stories.
Tracking the paper colours and size of the sheets that comprised the
various drafts, as well as details about the writing apparatus used, we
hypothesized that grouping drafts written, say, on yellow paper, or with
a fading typewriter ribbon placing text in red and black, might indicate
the drafts of these stories were composed contemporaneously with each
other. This logic might lead us to conclude that ``The Moons of
Jupiter'' and ``Royal Beatings'', both typed on blue paper, were written
around the same time, as perhaps were ``Chaddeleys and Flemings'',
``Wild Swans'', and ``Moons of Jupiter'', as drafts of all were produced
on a fading black / red typewriter ribbon. Moreover, data gathered from
the correspondence still proved useful by establishing which stories are
mentioned in letters alongside a particular story. In Figure \ref{fig:wiens5}, the
frequency a story is mentioned alongside Munro's story ``Accident'' is
represented by the relative size of the circle, and the stories
referenced are colour coded by the collection they eventually joined (or
neither in some cases). This tells us that ``Accident'' was actually
referenced in the correspondence more frequently alongside the stories
that would comprise the earlier collection, \emph{Who Do You Think You
Are?}, rather than within the orbit, pardon the pun, of the stories that
would make up \emph{The Moons of Jupiter}, where it was eventually
published. At some point I envision an interactive visualization, in
which by clicking on the circle of one particular story in the orbit, it
centres that story and then re-constellates the other stories around it.

\begin{figure}[t]
    \centering
    \frame{\includegraphics[width=1\textwidth]{media/wiens5.png}}
    \caption{The frequency in which a story is mentioned alongside Munro's ''Accident''. Data visualization created by John Brosz and Jagoda Walny, and reproduced with their permission.}
    \label{fig:wiens5}
\end{figure}

As noted above, researchers were initially recording the text of
revisions in the spreadsheets, which we soon realized would have been
too time- and resource-consuming across the drafts of 22 stories, and
therefore unfeasible given our limited time and resources. However, we
returned to this initial focus on particular revisions through another
possibility we explored in a preliminary way: employing XML to create
interactive digital genetic editions of each story. A graduate research
assistant (Min Lei) used OCR to create digital files of each draft of
the story ``Wild Swans''. Then other graduate assistants conversant in
XML (Hannah Anderson and James Robertson), encoded these drafts,
allowing us to track the revisions across them. Figure \ref{fig:wiens6} presents the
code of a brief section of the story; the LEM represents the published
text, and the RDG tags variants across the revisions. In the products of
the encoding work (see Fig. \ref{fig:wiens7}), the blue shades in the passage represent
text that has been revised; the darker the blue, the more revisions. The
coding will allow a user to hover over any revised text to have a pop-up
reveal the different versions. However, because this element of the
project works with transcriptions of the drafts themselves rather than
metadata about the drafts, to pursue this encoding further will require
copyright permissions from Alice Munro’s identified representatives
such as her family or legal representation. Van Hulle notes another
possible application of XML encoding: the \lstinline[language=XML]!<del>!
and \lstinline[language=XML]!<add>! tags could be used ``to calculate the
percentage of words that remained `stable' across versions, how many
were cut and how many added'' \citep[174]{van_hulle_genetic_2022}. Van Hulle further notes
that ``{[}t{]}o make this kind of research fully operational, however,
it is necessary to present readers with the entirety of the author's
oeuvre and \emph{sous-ouevre}'' \citep[174]{van_hulle_genetic_2022}, something our project moves
towards and which could be part of its expansion in the future.


\begin{figure}[t]
    \tiny
    \begin{lstlisting}[language=xml]    
<p n="32">
    <app loc="p32">
        <lem>The train was filling up.</lem>
        <rdg wit="#MsC_37_12_14">Frances had a window seat.</rdg>
    </app>
    <app loc="p32">
        <lem>At Brantford a man asked if she would mind if he sat down beside her.</lem>
        <rdg wit="#MsC_37_12_14">At Brantford, a man sat down beside her.</rdg>
        <rdg wit="#MsC_37_12_15 #MsC_37_12_15">At Brantford, a man sat down beside me.</rdg>
    </app>
</p>
<p n="33">
        <lem>"It's cooler than you'd think," he said.</lem>
</p>
    \end{lstlisting}
    \caption{Encoding for tracking revisions across drafts. Encoding by Hannah Anderson and James Robertson.}
    \label{fig:wiens6}

\vspace{2em}

    \centering
    \frame{\includegraphics[width=0.92\textwidth]{media/wiens7.png}}
    \caption{A passage that has been revised. Screenshot of work by Hannah Anderson and James Robertson.}
    \label{fig:wiens7}
\end{figure}



Further work with the existing metadata set collected from drafts of the
22 stories could draw upon metadata not yet quantified and visualized
meaningfully. For example, we tracked the different character names
Munro used in the different drafts of her stories; a visualization of
where the same names are used in drafts of different stories could lead
to different hypotheses about the proximity of a story's composition in
relation to others, or perhaps gesture towards thematic similarities
between stories in which a character's name in one story migrated to
another. The metadata we collected on character professions, on ``high''
and ``mass'' cultural references and historical references, may be
especially interesting to consider in relation to the depictions of
class in Munro's work, or on the work's relationship to a postcolonial
context: both areas that have been written about extensively in relation
to Munro's fiction.\footnote{For studies of Munro's fiction in relation
  to class, see Katrin Berndt, ``Trapped in Class? Material
  Manifestations of Poverty and Prosperity in Alice Munro's ``Royal
  Beatings'' and ``The Beggar Maid'' \citep{berndt_trapped_2020} and Isla Duncan, ``Social Class in Alice Munro's `Sunday
  Afternoon' and `Hired Girl'' \citep{duncan_social_2009}. For recent postcolonial
  approaches, see Corinne Bigot, ``Marooning on Islands of Her Own
  Choosing: Inscribing Place and Instability in Alice Munro's
  `Deep-Holes''' \citep{bigot_marooning_2020}, and Pilar Somacarrera, ``Looking at America from
  Edinburgh Castle: Postcolonial Dislocations in Alice Munro's and
  Ann-Marie MacDonald's Scottish Fictions'' \citep{somacarrera_looking_2016}.} Do the professions of
characters across her oeuvre, or the ratio of high cultural references
to mass cultural references, support a reading of Alice Munro as a
writer who represents working class characters and milieu, for instance?
If not, how might we account for differences between our quantitative
data and qualitative readings of the work? Do the locational, cultural,
and historical references across her work place it squarely in a Western
/ Eurocentric frame of reference, and if so, how might that relate to
its postcolonial elements?

\subsection*{Conclusions}

The pedagogical impetus for this project has to some extent come full
circle: just as the project was at least in part prompted by student
work, students continue to build upon the research foundation
established by the Visualizing a Canadian Author's Archive: Alice Munro
project. Although our group's research has been paused due to the
COVID-19 pandemic, a doctoral student in a course I recently taught,
Shazia Hafiz Ramji, was able to access the Munro papers and complete a
project on fan mail collected from the late 1970s and early 1980s, the
same period our project covered. Working once again with John Brosz in
our library, Ramji produced visualizations of these collective letters,
mapping them in terms of gender, geographic location, and positive,
negative, or neutral ``sentiments''. She was struck to find that her
``affective experience with this set of correspondence did not align
with the data'' \citep[3]{ramji_dear_2022}, and indeed I experienced a comparable
reaction when I saw the visualizations of the Munro revisions across the
drafts of the 22 stories. Quantitative visualizations are limited in
their ability to convey the qualitative \emph{degree} and \emph{impact}
of revisionary work across the various extant versions of a story or
stories.

Despite such limitations, I am intrigued by the possibilities once
metadata drawn from the archival materials around all the stories can be
quantified and visualized in this way, leading to a comparative approach
to the materials. Researchers will be able to quickly see which stories
were more heavily revised, for instance, which stories were likely being
composed around the same time, and which characters and episodes are
moved from one story to another. Such a distant reading of the archive
will not replace close reading, of course, but it can alert researchers
as to which archival avenues might prove most productive. Prior to the
COVID-19 pandemic, our team had planned to apply for a large grant from
Canada's federal humanities research agency –– the Social Sciences and
Humanities Research Council of Canada –– for multi-year funding that
would allow us to expand the project to cover all of Munro's drafts,
including the period when she begins to produce born digital materials
using a word processor, which will obviously impact the residual textual
archive in significant ways. In addition to extending the visualizations
of metadata of the sort I have introduced here to the entire archival
corpus, the Taylor Family Digital Library at the University of Calgary
is currently investigating the possibility of digitizing the entire
collection of Munro drafts. A future research project, in addition to
the data collection and visualizations I have described above, could
transcribe the drafts so as to make them searchable (and in the case of
handwritten drafts, more legible), and encode text versions of the
drafts to permit the creation of genetic critical editions of the
stories as well as allow for quantitative analysis of the sort Van Hulle
describes. Now that the pandemic has receded and the idea of
large groups of researchers working in the archives is feasible again, I
plan to apply for this funding to continue, and expand, the project. All
of this, of course, will be contingent not only on success with the
grant application, but on the permission of Alice Munro’s estate.

\section*{Acknowledgements}
I would
  like to acknowledge that I am a settler scholar working and living on
  the traditional territories of the people of the Treaty 7 region of
  southern Alberta, and that the University of Calgary, where the
  research and writing of this paper took place, is also located on
  those traditional territories. I also thank the external reviewers
  whose suggestions for revision strengthened and clarified this paper.

\begin{flushleft}
    % use smallcaps for author names
    \renewcommand*{\mkbibnamefamily}[1]{\textsc{#1}}
    \renewcommand*{\mkbibnamegiven}[1]{\textsc{#1}} 
\printbibliography
\end{flushleft}

\end{paper}