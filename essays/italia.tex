\documentclass{article}
%%%% CLASS OPTIONS 

\KOMAoptions{
    fontsize=10pt,              % set default font size
    DIV=calc,
    titlepage=false,
    paper=150mm:220mm,
    twoside=true, 
    twocolumn=false,
    toc=chapterentryfill,       % for dots: chapterentrydotfill
    parskip=false,              % space between paragraphs. "full" gives more space; "false" uses indentation instead
    headings=small,
    bibliography=leveldown,     % turns the Bibliography into a \section rather than a \chapter (so it appears on the same page)
}

%%%% PAGE SIZE

\usepackage[
    top=23mm,
    left=20mm,
    height=173mm,
    width=109mm,
    ]{geometry}

\setlength{\marginparwidth}{1.25cm} % sets up acceptable margin for \todonotes package (see preamble/packages.tex).

%%%% PACKAGES

\usepackage[dvipsnames]{xcolor}
\usepackage[unicode]{hyperref}  % hyperlinks
\usepackage{booktabs}           % professional-quality tables
\usepackage{nicefrac}           % compact symbols for 1/2, etc.
%\usepackage{microtype}          % microtypography
\usepackage{lipsum}             % lorem ipsum at the ready
\usepackage{graphicx}           % for figures
\usepackage{footmisc}           % makes symbol footnotes possible
\usepackage{ragged2e}
\usepackage{changepage}         % detect odd/even pages
\usepackage{array}
\usepackage{float}              % get figures etc. to stay where they are with [H]
\usepackage{subfigure}          % \subfigures witin a \begin{figure}
\usepackage{longtable}          % allows for tables that stretch over multiple pages
\setlength{\marginparwidth}{2cm}
\usepackage[textsize=footnotesize]{todonotes} % enables \todo's for editors
\usepackage{etoolbox}           % supplies commands like \AtBeginEnvironment and \atEndEnvironment
\usepackage{ifdraft}            % switches on proofreading options in the draft mode
\usepackage{rotating}           % provides sidewaysfigure environment
\usepackage{media9}             % allows for video in the pdf
\usepackage{xurl}               % allows URLs to (line)break ANYWHERE

%%%% ENCODING

\usepackage[full]{textcomp}                   % allows \textrightarrow etc.

% LANGUAGES

\usepackage{polyglossia}
\setmainlanguage{english} % Continue using english for rest of the document

% If necessary, the following lets you use \texthindi. Note, however, that BibLaTeX does not support it and will report a 'warning'.
 \setotherlanguages{hindi} 
 \newfontfamily\hindifont{Noto Sans Devanagari}[Script=Devanagari]

% biblatex
\usepackage[
    authordate,
    backend=biber,
    natbib=true,
    maxcitenames=2,
    ]{biblatex-chicago}
\usepackage{csquotes}

% special characters  
\usepackage{textalpha}                  % allows for greek characters in text 

%%%% FONTS

% Palatino font options
\usepackage{unicode-math}
\setmainfont{TeX Gyre Pagella}
\let\circ\undefined
\let\diamond\undefined
\let\bullet\undefined
\let\emptyset\undefined
\let\owns\undefined
\setmathfont{TeX Gyre Pagella Math}
\let\ocirc\undefined
\let\widecheck\undefined

\addtokomafont{disposition}{\rmfamily}  % Palatino for titles etc.
\setkomafont{descriptionlabel}{         % font for description lists    
\usekomafont{captionlabel}\bfseries     % Palatino bold
}
\setkomafont{caption}{\footnotesize}    % smaller font size for captions


\usepackage{mathabx}                    % allows for nicer looking \cup, \curvearrowbotright, etc. !!IMPORTANT!! These are math symbols and should be surrounded by $dollar signs$
\usepackage[normalem]{ulem}                       % allows for strikethrough with \sout etc.
\usepackage{anyfontsize}                          % fixes font scaling issue

%%%% ToC

% No (sub)sections in TOC
\setcounter{tocdepth}{0}                

% Redefines chapter title formatting
\makeatletter                               
\def\@makechapterhead#1{
  \vspace*{50\p@}%
  {\parindent \z@ \normalfont
    \interlinepenalty\@M
    \Large\raggedright #1\par\nobreak%
    \vskip 40\p@%
  }}
\makeatother
% a bit more space between titles and page numbers in TOC

\makeatletter   
\renewcommand\@pnumwidth{2.5em} 
\makeatother

%%%% CONTRIBUTOR

% Title and Author of individual contributions
\makeatletter
% paper/review author = contributor
\newcommand\contributor[1]{\renewcommand\@contributor{#1}}
\newcommand\@contributor{}
\newcommand\thecontributor{\@contributor} 
% paper/review title = contribution
\newcommand\contribution[1]{\renewcommand\@contribution{#1}}
\newcommand\@contribution{}
\newcommand\thecontribution{\@contribution}
% short contributor for running header
\newcommand\shortcontributor[1]{\renewcommand\@shortcontributor{#1}}
\newcommand\@shortcontributor{}
\newcommand\theshortcontributor{\@shortcontributor} 
% short title for running header
\newcommand\shortcontribution[1]{\renewcommand\@shortcontribution{#1}}
\newcommand\@shortcontribution{}
\newcommand\theshortcontribution{\@shortcontribution}
\makeatother

%%%% COPYRIGHT

% choose copyright license
\usepackage[               
    type={CC},
    modifier={by},
    version={4.0},
]{doclicense}

% define \copyrightstatement for ease of use
\newcommand{\copyrightstatement}{
         \doclicenseIcon \ \theyear. 
         \doclicenseLongText            % includes a link
}

%%%% ENVIRONMENTS
% Environments
\AtBeginEnvironment{quote}{\footnotesize\vskip 1em}
\AtEndEnvironment{quote}{\vskip 1em}

\setkomafont{caption}{\footnotesize}

% Preface
\newenvironment{preface}{
    \newrefsection
    \phantomsection
    \cleardoublepage
    \addcontentsline{toc}{part}{\thecontribution}
    % enable running title
    \pagestyle{preface}
    % \chapter*{Editors' Preface}    
    % reset the section counter for each paper
    \setcounter{section}{0}  
    % no running title on first page, page number center bottom instead
    \thispagestyle{chaptertitlepage}
}{}
\AtEndEnvironment{preface}{%
    % safeguard section numbering
    \renewcommand{\thesubsection}{\thesection.\arabic{subsection}}  
    %last page running header fix
    \protect\thispagestyle{preface}
}
% Essays
\newenvironment{paper}{
    \newrefsection
    \phantomsection
    % start every new paper on an uneven page 
    \cleardoublepage
    % enable running title
    \pagestyle{fancy}
    % change section numbering FROM [\chapter].[\section].[\subsection] TO [\section].[\subsection] ETC.
    \renewcommand{\thesection}{\arabic{section}}
    % mark chapter % add author + title to the TOC
    \chapter[\normalfont\textbf{\emph{\thecontributor}}: \thecontribution]{\vspace{-4em}\Large\normalfont\thecontribution\linebreak\normalsize\begin{flushright}\emph{\thecontributor}\end{flushright}}    
    % reset the section counter for each paper
    \setcounter{section}{0}  
    % reset the figure counter for each paper
    \renewcommand\thefigure{\arabic{figure}}    
    % reset the table counter for each paper
    \renewcommand\thetable{\arabic{table}} 
    % no running title on first page, page number center bottom instead, include copyright statement
    \thispagestyle{contributiontitlepage}
    % formatting for the bibliography

}{}
\AtBeginEnvironment{paper}{
    % keeps running title from the first page:
    \renewcommand*{\pagemark}{}%                            
}
\AtEndEnvironment{paper}{
    % safeguard section numbering
    \renewcommand{\thesubsection}{\thesection.\arabic{subsection}}  
    % last page running header fix
    \protect\thispagestyle{fancy}%                              
}
% Reviews
\newenvironment{review}{
    \newrefsection
    \phantomsection
    % start every new paper on an uneven page 
    \cleardoublepage
    % enable running title
    \pagestyle{reviews}
    % change section numbering FROM [\chapter].[\section].[\subsection] TO [\section].[\subsection] ETC.
    \renewcommand{\thesection}{\arabic{section}} 
    % mark chapter % add author + title to the TOC
    \chapter[\normalfont\textbf{\emph{\thecontributor}}: \thecontribution]{}    % reset the section counter for each paper
    \setcounter{section}{0}  
    % no running title on first page, page number center bottom instead, include copyright statement
    \thispagestyle{contributiontitlepage}
    % formatting for the bibliography
}{}
\AtBeginEnvironment{review}{
% keeps running title from the first page
    \renewcommand*{\pagemark}{}%                                   
}
\AtEndEnvironment{review}{
    % author name(s)
    \begin{flushright}\emph{\thecontributor}\end{flushright}
    % safeguard section numbering
    \renewcommand{\thesubsection}{\thesection.\arabic{subsection}} 
    % last page running header fix
    \protect\thispagestyle{reviews}                           
}

% Abstract
\newenvironment{abstract}{% 
\setlength{\parindent}{0pt} \begin{adjustwidth}{2em}{2em}\footnotesize\emph{\abstractname}: }{%
\vskip 1em\end{adjustwidth}
}{}

% Keywords
\newenvironment{keywords}{
\setlength{\parindent}{0pt} \begin{adjustwidth}{2em}{2em}\footnotesize\emph{Keywords}: }{%
\vskip 1em\end{adjustwidth}
}{}

% Review Abstract
\newenvironment{reviewed}{% 
\setlength{\parindent}{0pt}
    \begin{adjustwidth}{2em}{2em}\footnotesize}{%
\vskip 1em\end{adjustwidth}
}{}

% Motto
\newenvironment{motto}{% 
\setlength{\parindent}{0pt} \small\raggedleft}{%
\vskip 2em
}{}

% Example
\newcounter{example}[chapter]
\newenvironment{example}[1][]{\refstepcounter{example}\begin{quote} \rmfamily}{\begin{flushright}(Example~\theexample)\end{flushright}\end{quote}}

%%%% SECTIONOPTIONS

% command for centering section headings
\newcommand{\centerheading}[1]{   
    \hspace*{\fill}#1\hspace*{\fill}
}

% Remove "Part #." from \part titles
% KOMA default: \newcommand*{\partformat}{\partname~\thepart\autodot}
\renewcommand*{\partformat}{} 

% No dots after figure or table numbers
\renewcommand*{\figureformat}{\figurename~\thefigure}
\renewcommand*{\tableformat}{\tablename~\thetable}

% paragraph handling
\setparsizes%
    {1em}% indent
    {0pt}% maximum space between paragraphs
    {0pt plus 1fil}% last line not justified
    

% In the "Authors" section, author names are put in the \paragraph{} headings. To reduce the space after these  headings, the default {-1em} has been changed to {-.4em} below.
\makeatletter
\renewcommand\paragraph{\@startsection {paragraph}{4}{\z@ }{3.25ex \@plus 1ex \@minus .2ex}{-.4em}{\normalfont \normalsize \bfseries }
}
\makeatother

% add the following (uncommented) in environments where you want to count paragraph numbers in the margin
%    \renewcommand*{\paragraphformat}{%
%    \makebox[-4pt][r]{\footnotesize\theparagraph\autodot\enskip}
%    }
%    \renewcommand{\theparagraph}{\arabic{paragraph}}
%    \setcounter{paragraph}{0}
%    \setcounter{secnumdepth}{4}
    
%%%% HEADERFOOTER

% running title
\RequirePackage{fancyhdr}
% cuts off running titles that are too long
%\RequirePackage{truncate}
% makes header as wide as geometry (SET SAME AS \TEXTWIDTH!)
\setlength{\headwidth}{109mm} 
% LO = Left Odd
\fancyhead[LO]{\small\emph{\theshortcontributor} \hspace*{.5em} \theshortcontribution} 
% RE = Right Even
\fancyhead[RE]{\scshape{\small\theissue}}
% LE = Left Even
\fancyhead[LE]{\small\thepage}            
% RE = Right Odd
\fancyhead[RO]{\small\thepage}    
\fancyfoot{}
% no line under running title; cannot be \@z but needs to be 0pt
\renewcommand{\headrulewidth}{0 pt} 

% special style for authors pages
\fancypagestyle{authors}{
    \fancyhead[LO]{\small\textit{Authors}} 
    \fancyhead[LE]{\small\thepage}            
    \fancyhead[RE]{\scshape{\small\theissue}}
    \fancyhead[RO]{\small\thepage}            
    \fancyfoot{}
}

% special style for book reviews
\fancypagestyle{reviews}{
    \fancyhead[LO]{\small\textit{Book Reviews}} 
    \fancyhead[LE]{\small\thepage}            
    \fancyhead[RE]{\scshape{\small\theissue}}
    \fancyhead[RO]{\small\thepage}            
    \fancyfoot{}
}

% special style for Editors' preface.
\fancypagestyle{preface}{
    \fancyhead[LO]{\small\textit{\theshortcontribution}} 
    \fancyhead[LE]{\small\thepage}            
    \fancyhead[RE]{\scshape{\small\theissue}}
    \fancyhead[RO]{\small\thepage}            
    \fancyfoot{}
}
% special style for first pages of contributions etc.
% DOES include copyright statement
\fancypagestyle{contributiontitlepage}{
    \fancyhead[C]{\scriptsize\centering\copyrightstatement}
    \fancyhead[L,R]{}
    \fancyfoot[CE,CO]{\small\thepage}
}
% special style for first pages of other \chapters.
% DOES NOT include copyright statement
\fancypagestyle{chaptertitlepage}{
    \fancyhead[C,L,R]{}
    \fancyfoot[CE,CO]{\small\thepage}
}
% no page numbers on \part pages 
\renewcommand*{\partpagestyle}{empty}

%%%% FOOTNOTEFORMAT

% footnotes
\renewcommand{\footnoterule}{%
    \kern .5em  % call this kerna
    \hrule height 0.4pt width .2\columnwidth    % the .2 value made the footnote ruler (horizontal line) smaller (was at .4)
    \kern .5em % call this kernb
}
\usepackage{footmisc}               
\renewcommand{\footnotelayout}{
    \hspace{1.5em}    % space between footnote mark and footnote text
}    
\newcommand{\mytodo}[1]{\textcolor{red}{#1}}

%%%% CODESNIPPETS

% colours for code notations
\usepackage{listings}       
	\renewcommand\lstlistingname{Quelltext} 
	\lstset{                    % basic formatting (bash etc.)
	       basicstyle=\ttfamily,
 	       showstringspaces=false,
	       commentstyle=\color{BrickRed},
	       keywordstyle=\color{RoyalBlue}
	}
	\lstdefinelanguage{XML}{     % specific XML formatting overrides
		  basicstyle=\ttfamily,
		  morestring=[s]{"}{"},
		  morecomment=[s]{?}{?},
		  morecomment=[s]{!--}{--},
		  commentstyle=\color{OliveGreen},
		  moredelim=[s][\color{Black}]{>}{<},
		  moredelim=[s][\color{RawSienna}]{\ }{=},
		  stringstyle=\color{RoyalBlue},
 		  identifierstyle=\color{Plum}
	}
    % HOW TO USE? BASH EXAMPLE
    %   \begin{lstlisting}[language=bash]
    %   #some comment
    %   cd Documents
    %   \end{lstlisting}
\author{Paola Italia}
\title{The Double Text. ``Alternative Variants'' in the Italian Tradition.}


\begin{document}
\maketitle


\begin{abstract}
This article deals with the case of \emph{alternative variants}, i.e.
those variants which are written by the author after the first draft of
the text, but on which the author does not give us precise indications:
``competing readings between which the author does not know how to
decide, or in any case does not give certain signs of knowing how to
decide'' \parencite[xxxv]{isella_nota_1983}. The \emph{alternative variants} are a ``double text'': the
first step of rewriting the text, where the author's will manifests
itself in a dual form. Thanks to the extraordinary richness of case
studies offered by the Italian tradition – from the fourteenth to twentieth century (from
Petrarch to Gadda) –– \emph{alternative variants} will be analyzed in
their different typologies, over time to verify whether the ways of
representing the ``double text'', the ``double will'' of the author,
have been similar or different in time, and whether it can be compared
with \emph{alternative variants} of other literary traditions.
\end{abstract}


\section*{}
\textsc{Within the ``history of holographs'',} the \emph{Italian
tradition} is an exception, since, starting with the first and most
famous \emph{scartafaccio} [draft] of the European tradition: the Codex of
Petrarch's draft (1348), and continuing with the manuscripts of
Boccaccio, Ariosto, Bembo, Della Casa, Francesco Maria Molza,
Machiavelli, Guicciardini, Varchi, etc., it preserves the oldest and
most extensive series of autographs or apographs with autograph
corrections, both in the pre- and post-Gutenberg era \parencite{italia_italian_nodate}. This wealth of
quantity and quality makes the Italian case unique in Europe.

Since the the seventeenth century, these author's manuscripts have been the object of
philological edition and critical study, and in particular Petrarch's
manuscripts testifying to the genesis of the \emph{Canzoniere} – generally
considered as a precious testimony of how the poet worked – were published
in a prestigious edition, edited in 1642 by Francesco Ubaldini, a
scholar of the circle of Francesco Barberini, nephew of Urban VIII. This
edition can be considered not only the first ``genetic edition'', or
rather the incunabulum of authorial philology,\footnote{For general
  overview of authorial variants and authorial philology, see \cite{italia_what_2021}.} which would later be
developed with more scientific ecdotic criteria by Francesco Moroncini,
with the first critical edition of Giacomo Leopardi's \emph{Canti}
in 1927; but also an example of proto-``criticism of variants'', borne ten
years after Moroncini's edition, in 1937, with Gianfranco Contini's
Essay: \emph{As Ariosto worked.}\footnote{As discussed, for example, in \cite{brugnolo_incunabula_2021} and \cite{italia_alle_2018}.}

The rich dossier of autographs possessed by the Italian
tradition\footnote{A curated collection of this tradition can be consulted at \url{www.autografi.net}, a project
  coordinated by Matteo Motolese and Emilio Russo, for which see \cite{motolese_autografi_2014}. This project collects a
  very wide range of autographs of Italian literati, from the Thirteenth to the Seventeenth centuries,
  although it does not specifically select cases of author's sketches.}
provides a vast picture of the typology of the authorial variants: from
the \emph{genetic} ones (i.e., immediate corrections, made at the moment
of writing, which lead to a reading considered provisionally valid by
the author) to the \emph{evolutionary corrections} (which are made after
a first draft and replace valid readings with others equally valid, but
preferred by the author).\footnote{For the distinction between immediate
  and late variants, which Gianfranco Contini also defined:
  \emph{instaurative} (immediate) and \emph{substitutive} (late), see \cite{italia_alle_nodate} (with reproduction of Contini's manuscript, in the part
  where he speaks of such variants).} A particular case of variants is
the subject of this essay: the \emph{alternative} \emph{variants}, which
represent a particular manifestation of the author's relationship with
his or her text, of what, with an effective expression, Armando Petrucci
has called: the ``writing relationship'' \parencite[63--64]{petrucci_letteratura_2017}.

The first definition of \emph{alternative variants} was given by
Dante Isella, in the \emph{Note to the text} of Gadda's \emph{Racconto
italiano di Ignoto del Novecento}: an unpublished novel that the great
engineer-writer had begun in 1924, at the beginning of his career, but
which remained in his legendary ``trunks'' until 1983, when it was
published posthumously (ten years after the author's death) by Gadda's
greatest scholar, the founder of authorial philology: Dante Isella.

It is no coincidence that Carlo Emilio Gadda was the author on whom this
discipline was borne and developed: his discontent in the creative use
of language, the ``unfinished'' status of his masterpieces:
\emph{Acquainted with grief} and \emph{That awful mess of via Merulana},
and his collecting mania, which led him to keep everything to the point
of saying of himself, ``I am an archivist-maniac'', make Gadda the best
case study of ``how an author worked'', in its various
phenomenologies.\footnote{For more information, see \cite{italia_come_2017}, now also in French translation: \cite{italia_dans_2022}. For Gadda's passion for ``compulsive'' conservation, see the
  exhibition catalogue \parencite{italia_io_2003}, and \cite{falkoff_possessed_2021} –– in particular
  chapter 2, pp. 95--107.}

Isella defines alternative variants as follows:

\begin{quote}
Competing readings among which the author cannot decide, or at any rate
does indicate by any precise markings that he knows how to
decide.
\begin{flushright}
\parencite[xxxiv]{isella_nota_1983}
\end{flushright}
\end{quote}

\noindent Since the aim of authorial philology is not to reconstruct ``in slow
motion'' what took place in the author's mind and on his worktable, which
is impossible, but to decipher, from the documents that the author has
left us, traces and signs of his work, to analyze them and publish them
in the form most suitable to the text (and subsequently, with the
criticism of the variants, to interpret them in such a way as to
identify guidelines, directives, corrective ``isohypses''), the necessary
condition for an alternative variant to be such is that the \emph{base
text is not deleted}.

The lack of deletion is the sure sign by which an author declares to
himself, and to the future reader of his text (which exists ``virtually'',
since he has decided not to destroy his ``scribblings'', thus showing a
specific ``archival will''),\footnote{On the ``archival will'' cf. \cite{albonico_autore_2015} and \cite{italia_italian_nodate}.} that
his will with respect to a reading written subsequently to the basic
reading is not decided; it is a will still suspended between the
original reading and the alternative reading, which remains, as the word
itself says, in ``alternative'' to the basic reading.

This means that, in the \emph{absence of deletion}, the completion of
the author's will and the philologist's decision cannot be inferred from
other elements and must conform to this ``double will''. We are
faced with a text that is, so to speak, ``double'', a text that does not
show by clear signs what the manuscripts always declare: \emph{the
author's last will}.

Alternative variants are in a sort of ``limbo'': a little below the level
of the text, and a little above the level of the apparatus. Therefore,
the method of representation used by Isella is particular. The
alternative variants are in fact represented in a subordinate position
to the text, in the footnote, referring to the text of which they are,
in fact, the ``alternative'' correction, but in a typographical body
identical to that of the text, and not, as is usually the case for
footnotes, in a smaller body.

Why do we use the same typeface as the text? We do so, to declare to the
reader, in an immediate way and with a clarity of visualization not
secondary in the ecdotic choices, that the text he/she reads is
subordinate to the text, has a lower value, because the author wrote it
at a later time and did not decide whether to prefer that text to the
lecture in line, but we cannot know whether he would have accepted this
variant or rejected it. As for the edition of the manuscript, it is
necessary for the philologist to consider this text as a variant, and
not the critical text, which, according to the ecdotic practice
established by Isella, follows the base reading. This is not, however, a
genetic or evolutionary variant, but an ``alternative'' reading of the
text.

This is why, in the definition of textual levels and the relative
ecdotic practices established by Isella for the representation of the
author's text, according to a series of parameters that reflect the
status of the text and the relationship between text and apparatus, the
\emph{alternative variants} are represented with a particular
formalization (see Table \ref{tab:italia:ecdotic}).

\textbf{Table 1: Formalization of \textit{alternative variants}.}

\begin{figure}
    \centering
    \includegraphics[width=\textwidth]{media/italia-tab1.png}
    \label{tab:italia:ecdotic}
\end{figure}

It is important to remember that, after 1983, the ecdotic praxis applied
by Isella set the standard in the Italian philological tradition, and
his representation of alternative variants was adopted not only in other
editions of Gadda's works, but for all the most important editions of
author's philology, both in prose and poetry, from Manzoni to Leopardi,
from Pasolini to Alberto Savinio \parencite[57--59]{italia_what_2021}. This was
particularly significant because, in the case of re-editions of
scholarly/critical editions in paperback of previously unpublished
texts, the alternative variants were not sacrificed, as is usually
happens for apparatuses, but were considered an integral part of the
text, and therefore published together with it, in the manner indicated
above. This is an ecdotic practice that we believe can also be adopted for other
editions of texts in European literature.

The first page of the original version of the anti-fascist
pamphlet \emph{Eros and Priapus} (see Figure \ref{fig:italia1}), dating from 1944--46, where Gadda presents a
brilliant interpretation of fascism in a psychoanalytical key in a
violent anti-Mussolini treatise written \emph{à la Machiavelli}, offers a
very clear example of the peculiarity of the alternative variants.

\begin{figure}[H]
    \centering
    \includegraphics[width=\textwidth]{media/italia1.png}    
    \caption{Figure 1: First page of the manuscript of \emph{Eros e Priapo} (\emph{Eros
and Priapus}), original version (Archivio Liberati in Villafranca di
Verona), c. 65.}
    \label{fig:italia1}
\end{figure}


The page is a ``clean copy'', with various corrections, dating back to
different moments in the history of the text that was rejected by all the publishers as ``unpublishable'' because it was considered ``intolerably obscene'' \parencite[46, letter from 28 October 1946]{gadda_lettere_1988}. It was only published in 1967, with the author's forced variants, after a period of twenty years in which Gadda had returned to the manuscript on several occasions. Note the different pens, and above all the difference
between the basic text, the corrective series, which presents various
implicit variants, and the alternative variant:

\begin{quote}
\textbf{L'associazione a delinquere,} cui per più di un ventennio è
venuto fatto di poter taglieggiare a \textbf{sua} posta, e coprir d'onte
e stuprare la Italia, e precipitarla finalmente in quella ruina e in
quell'abisso dove Dio medesimo ha paura \textbf{\uline{guardare}},
\textbf{è pervenuta} a dipingere come attività politica, la distruzione
e la cancellazione della vita, la obliterazione totale dei segni della
vita.

\vspace{1em}

[\textbf{The association of criminals}, who, for more than twenty years,
have been able to prey on Italy \textbf{at will}, to cover it with shame
and rape it, and finally plunge it into that ruin and abyss where God
himself is afraid to \textbf{\uline{look}}, \textbf{has come} to portray as a
political action, the destruction and erasure of life, the total
obliteration of the signs of life.]
\begin{flushright}
\parencite[11]{gadda_eros_2016}
\end{flushright}
\end{quote}

The corrections, made later but in the same handwriting and with the
same pen used for the basic text (which we have marked here in
\textbf{bold}) introduce a dissimulating variant, replacing the
locution ``L'associazione a delinquere''  [The association of
criminals], which is more transparent and more immediate in the Italian language\footnote{This form could be defined a ``polyrhematic'' locution. For a  definition of 'polyrhematic' see the comprehensive entry in the
  Treccani Dictionary:
  \url{https://www.treccani.it/enciclopedia/parole-polirematiche_\%28Enciclopedia-dell\%27Italiano\%29/}}
with its archaic variant: ``Li associati a delinquere''. Initially Gadda changes it to ``I'' [The], but then, significantly, he corrects this to: ``Li'', which is
an archaic variant of the determinative article: ``Gli'' [The].

On the one hand, this correction immediately declares to the reader the
literary genre of the text: a modern treatise, pretending to be ancient,
as if it had been written by a new Machiavelli. On the other hand,
however, it conceals a crucial historical reference, inserted by Gadda
right at the beginning of his \emph{pamphlet}: the speech given by
Benito Mussolini in Italian Parliament on 3 January 1925, after the
Matteotti murder, where the Duce, taking moral responsibility for the
crime, set Fascism in motion to become a real regime with the ``leggi fascistissime''
`the ultra-Fascist laws):

\begin{quote}
    
Se il fascismo è stato \textbf{un'associazione a delinquere}, io sono il
capo di questa \emph{associazione a delinquere}!

\vspace{1em}

[If fascism was a \textbf{criminal association}, I am the leader of this \emph{criminal association}!]
\begin{flushright}
\parencite[13--14]{mussolini_discorsi_1926}
\end{flushright}
\end{quote}

\noindent This quotation from Mussolini is thus hidden by the stylistic elevation, by
its ``archaization'', with a process of dissimulation that will be used
very often by Gadda, to mitigate the tremendous violence of the text,
both in the correction of the manuscript and in the revision, in 1967,
on the typescript for printing.\footnote{On the Wiki Gadda platform
  (\url{http://www.filologiadautore.it/wiki/index.php?title=Edizione\_critica\_EP}), which is only accessible with permission of the heirs, it is possible to consult the complete digital critical Wiki edition. See also \cite{giuffrida_edizione_2016}.}

Let us read the final version of the manuscript, which was published by Giorgio Pinotti and myself, in the 2016 Adelphi critical edition:

\begin{quote}
\textbf{L'associazione a delinquere \textgreater{} Li associati a
delinquere,} cui per più di un ventennio è venuto fatto di poter
taglieggiare a \textbf{lor} posta, e coprir d'onte e stuprare la Italia,
e precipitarla finalmente in quella ruina e in quell'abisso dove Dio
medesimo ha paura \textbf{\uline{guardare}}, \textbf{pervennero} a
dipingere come attività politica, la distruzione e la cancellazione
della vita, la obliterazione totale dei segni della vita.

\vspace{1em}

[\textbf{The association of criminals \textgreater{} The criminal
associates}, who, for more than twenty years, have been able to prey on
Italy at their \textbf{own} will, to cover it with shame and rape it,
and finally plunge it into that ruin and abyss where God himself is
afraid to \textbf{\uline{look}}, have come to portray as a political action the
destruction and erasure of life, the total obliteration of the signs of
life.]

\begin{flushright}
\parencite[11]{gadda_eros_2016}
\end{flushright}

\end{quote}


Because of his revision, Gadda is forced to correct the number from singular to plural, but forgets to complete the
correction of the verb due to the speed and the emotional tension. Although he does not correct the ``è'' [is] –– a quite common case
of ``unfinished variant'', he does make his will explicit with a series of ``implicit corrections''. What is even more interesting, however, is that the author's last will, the
one published in the 1967 Garzanti edition, is different still, and even more camouflaged by the variants of the text:

\begin{quote}
\textbf{Li associati,} cui per più d'un ventennio è venuto fatto di
poter taglieggiare a lor posta e coprir d'onta la Italia, e precipitarla
finalmente a quella ruina e in quell'abisso ove Dio medesimo ha paura
guatare, pervennero a dipingere come attività politica la distruzione e
la cancellazione della vita, la obliterazione totale dei segni della
vita.

\vspace{1em}

[\textbf{The associates,} who, for more than twenty years, have been able
to cut down on their own account and cover Italy with shame, and finally
plunge it into that ruin and abyss which God himself is afraid to heal,
have come to portray as a political activity the destruction and
cancellation of life, the total obliteration of the signs of
life.]
\begin{flushright}
\parencite[21]{gadda_eros_2004}
\end{flushright}
\end{quote}

\begin{quote}


\textbf{L'associazione a delinquere \textgreater{} Li associati a
delinquere \textgreater{} Li associati}    

\vspace{1em}

[\textbf{The association of criminals \textgreater{} The criminal associates \textgreater{} The associates}]    
\end{quote}

\noindent A dissimulative variant, which prevented all the readers and scholars
from recognising the
direct attack on Mussolini in the first sentence of the violent pamphlet through the words of the martyr of fascism,
Giacomo Matteotti.

The case of the term ``guardare'' [look] –– marked in bold and underlining –– is
different. This term is not deleted, but marked as an alternative with
its ``archaic'' counterpart: ``guatare''. This variant
 follows the line of correction, illustrated above, of raising the
stylistic register for the purpose of concealment, and which immediately
highlights  the ``false antique'' style of the text to the reader, but which,
as we can clearly see from the reproduction, was applied after the
previous corrections, with \emph{different handwriting and pen}.

In this case, therefore, it is not possible to accept the reading ``in
text'', but it will be represented at the bottom of the page, in
a footnote, linked to the text (``guardare'') with an alphabetical
exponent, and, most importantly, in the same typographical body of the
text:

\begin{figure}[H]
    \centering
    \includegraphics[width=\textwidth]{media/italia2.png}
    \caption{Figure 2: Alternative reading represented at the bottom of the page.}
    \label{fig:italia2}
\end{figure}

However, there are particular cases that must be considered, and which
reveal \emph{the hybrid nature} of these variants, precisely because
they reflect an undecided will, with no clear indication. In these
cases, the precise analysis of the context and situation can help to
identify the correct ecdotic practice.

The position of the variant is very important, because it defines its
relationship with the base text, but other accessory elements, such as
underlining, are equally useful. We will analyze some cases
distinguished by the topography and the spellings of the alternative
variant: those in which the text is written in (1) line spacing, (2)
alignment, (3) within the text; in which (4) the alternative variants
are underlined and, (5) the most particular and tricky case, the one in
which the line spelling is wrong and the alternative variant intervenes
to correct it, even though the author has not deleted the base text.

\section{Alternative Variants in Interlinear}

The cases in which alternative variants occur in interlinear position
reflect a hierarchical relationship with the base text, since this is a
privileged position for the late variant, for a correction of the text,
and the variant is thus placed, topographically, in a position of
strength with respect to the base text. In these cases, it is possible
that the \emph{alternative variant} also reveals, due to its content or
style –– as we have seen in the example from Gadda –– a greater
expressive force.

What happens, however, when the alternative variant is accepted by the
subsequent redaction of the manuscript? Does it not acquire, in this
case, a greater value, given that the author has decided, subsequently,
to ``promote'' it as part of the text? And does this ``promotion'' not
entitle one to choose the alternative variant as a \emph{critical text}?
The answer is negative. The critical edition of a text must be
justified in itself, it is an interpretation that the philologist
establishes (and discusses in the \emph{Note to the text}), that
concerns the text itself, it cannot be modified according to subsequent
witnesses (unless the subsequent witness, in the reconstruction of the
general critical edition, is chosen as the witness to be ``put to text'',
and the first witness is represented, consequently, only in the
\emph{apparatus criticus}). In this case too, therefore, the
\emph{alternative variant} should be represented at the foot of the
page, perhaps indicating with a typographical marker that it has been
``promoted to text'' in the next witness, i.e., that it constitutes a sort
of ``bridging variant''.

This is the case of Alfieri's manuscripts (Fig. \ref{fig:italia3}), which present
numerous cases of \emph{alternative variants} that, in later witnesses
become ``evolutionary'', also because the author used to dictate to
secretary Polidori the revision of his texts by reading the manuscript
directly, and it is likely that he left evidence on the first manuscript
(to preserve it for his memory), of his ``double'' will, when the
secretary copied, in a new witness, his ``final will''.

\begin{figure}[H]
    \centering
    \includegraphics[width=\textwidth]{media/italia3.png}
    \caption{Figure 3: Vittorio Alfieri, Manuscript of the first draft of
\emph{L'America libera} (\emph{The free America}), Canzone Terza, c. 86v
(Ms. Alfieri 13, Laurentian Library, Florence).}
    \label{fig:italia3}
\end{figure}

\section{Alternative Aligned Variants}

The cases in which the alternative variants are \emph{aligned} may
reflect, due to their topography, a lower ``specific weight'' of the
variant than the \emph{interlinear} ones. In fact, the aligning position
is traditionally used for the author's recording of \emph{alternative
readings} from different codices on examples of other authors, and the
alternative variants may be affected by this habit. For Giacomo
Leopardi, Italy's most famous nineteenth century poet, for example, the habit of
recording these variants during his philological activity, constitutes a
model of annotation, so much so that, when he copied texts from an
original dossier that has not been preserved (the autographs by Leopardi
that we have are all ``clean copies''), on the ``good'' copies, together
with the text, the alternative variants were also copied, overwritten
and aligned. Their frequency (which led the field of Leopardian
philology to coin a specific expression for these variants: \emph{varia
lectio}, i.e., ``series of readings different from the basic reading''),
shows how Leopardi considered his papers as ``living papers'', whose
variants continue to stimulate poetic creation even after the text has
been copied, and even after it has been published. And, in fact, most of
Leopardi's papers are kept in the National Library in Naples, where the
poet spent his last years before his death, taking his papers (full of
variants) with him \parencite{italia_lo_2019}.

We see here an example taken from the manuscript of one of Leopardi's
most beautiful songs, \emph{Sappho's Last Song} (Fig. \ref{fig:italia4}):

\begin{figure}[H]
    \centering
    \includegraphics[width=\textwidth]{media/italia4.png}
    \caption{Figure 4: G. Leopardi, Manuscript of the \emph{L'ultimo canto di Saffo}
[\emph{Last Sappho's Song}] (1822), C.L. X.5.2.gamma (National Library
in Naples).}
    \label{fig:italia4}
\end{figure}

\noindent The text is written (i.e., copied) leaving a third of the paper free,
precisely to accommodate the alternative variants (and other metatextual
observations, self-assessment notes or compositional notes). Leopardi,
as can clearly be seen, writes them while he is copying the text
'cleanly', and –– like a pianist rehearsing the various chords on the
keyboard –– transcribes them in succession, even closing the series of
variants with a full stop:

\textbf{Transcription of part of Figure \ref{fig:italia4}}

\begin{figure}
    \centering
    \includegraphics[width=\textwidth]{media/italia4-trans.png}
\end{figure}

Sometimes, the alternative \emph{aligned} variants belong to a later
``corrective series'', and are written transversally to the text, as in
the case of the \emph{La sera del giorno festivo} [\emph{Evening of the
Festive Day}], where the sorrowful invocation: ``Oh giorni orrendi!'' [Oh
horrendous days], is emphasised and intensified in the variant (affixed
with a later pen): ``Oh vita o giorni orrendi`'' [Oh life, oh horrendous
days].\footnote{Note that Leopardi also put the alternative variant in
  the round bracket, to mark alternations to the text even more
  clearly. See all the corrective series and pens used by Leopardi \parencite{giuffrida_print_2021}, with the critical edition of the
  corrective stratigraphy of the ``Quaderni degli idilli'', in Leopardi
  Ecdosys: \url{https://leopardi.ecdosys.org/it/Home/}.}

\begin{figure}[H]
    \centering
    \includegraphics[width=\textwidth]{media/italia5.png}
    \caption{Figure 5: G. Leopardi, \emph{La sera del giorno festivo} [\emph{The
evening of the feast day}], C.L. XIII.22, p. 6 (National Library in
Naples).}
    \label{fig:italia5}
\end{figure}

\section{Alternative Variants in the Text}

Leopardi again provides us with an example of cases –– rarer in other
authors –– in which, following the clean copy of the text, the recording
of the alternative variants is not \emph{interlinear} or \emph{aligned}
with the text, but \emph{in-line}, distinguishing the alternative
variant from the text with a round or square bracket.

The exemplary case is constituted by the autographs of the so-called
``Pisano-Recanatesi'' Songs, that is, written between Pisa and Recanati
from 1828 to 1830, after a period –– dating from 1825 –– of ``poetic
silence''. Returning to writing in verse, as Giacomo told his sister on 2
May 1828, ``truly in the old style and with that old-fashioned heart of
mine'', his compositional method had completely changed: he did not
record the sources of his lexical choices in the manuscript, and tended
increasingly to record alternative variants \emph{within the text}.

This is clearly seen in the manuscript of the \emph{Ricordanze}
[\emph{Remembrances}] (Fig. \ref{fig:italia6}) from 1829, where the alternative
variants, which are also written both aligned and transversally, as
Leopardi did with the manuscripts of the \emph{Idylls} and the
\emph{Songs}, are also transcribed (we must remember that it is always a
``clean copy'') in the text, with a personal system of brackets: round,
which seem to signal a lesser uncertainty in the choice of the term, and
square (visibly corrected on the round), which indicate a greater
uncertainty, a suspension of the variant in the ``limbo'' of the
alternative variants.

\begin{figure}[H]
    \centering
    \includegraphics[width=\textwidth]{media/italia6.png}
    \caption{Figure 6: Leopardi, manuscript of \emph{Le ricordanze}
[\emph{Remembrances}] (1829), C.L. XIII.21 (National Library of Naples)
[detail].}
    \label{fig:italia6}
\end{figure}

\textbf{Transcription of part of Figure \ref{fig:italia6}}

\begin{figure}
    \centering
    \includegraphics[width=\textwidth]{media/italia6-trans.png}
\end{figure}


It may seem like a graphic detail, but it is actually a revolution, because
it encapsulates, a hundred years ahead of its time, a concept that
Mallarmé, and in particular Valéry, would only develop in the 1930s,
from which \emph{genetic criticism} would derive, and which inspired
Gianfranco Contini in founding, in 1937, the ``criticism of variants'':
that is, the concept that the \emph{poetic value} of a text is not given
by its final form, but also by the previous corrective stations, and
that poetry is not a value but an ``approximation to value'' \parencite[233]{contini_esercizi_1974}.\footnote{For more on the critique of variants and the French Symbolists, see \cite{italia_aux_2019}.}

\section{Alternative Variants and Underlining}

In the Leopardian example in Figure \ref{fig:italia4} we have seen that an alternative variant was underlined (italicized in the transcription):

\textbf{Transcription of part of Figure \ref{fig:italia4}}

\begin{figure}
    \centering
    \includegraphics[width=\textwidth]{media/italia4-trans2.png}
\end{figure}


\noindent This is an infrequent but insidious case, which needs to be considered analytically. The underlining of the text, in fact, is always an indication of
importance, but it is not as unambiguous as the deletion. It is only
from the 16th century onwards, in fact, with the spread of printed
correction practices, that underlining is used to indicate to the
printer the need to \emph{italicize} the text when converting it to
print, but in previous centuries, and in fact long afterwards,
underlining is an indication that the author gives to himself, declaring
that the underlined text is ``worthy of attention''.

This means that, in the practices of self-writing, the underlining of
the alternative variant may indicate a ``prestige'' of its own, a dignity
of having the same value as the basic text, and allow for it to be
considered, in the eyes of the author or an external reader, worthy of
being promoted to text. Since the basic reading is not discarded, in the
critical edition it \emph{must be represented} as an \emph{alternative
variant} (footnote, with alphabetical exponent and same typographical
body as the text), indicating the underlining with italics, or by
underlining of the text.

It is not uncommon for the underlined alternative variant to be chosen
by the author in a subsequent redaction and promoted to text, as if the
underlining, which is a sign of importance, also indicates his
``preference'' over the other alternative variants. It should not be
thought, however, that the presence of such a case expresses a univocal
signal.

In the case of Alfieri's manuscripts, for example, if all
\emph{alternative variants} were underlined and accepted in the
following apograph, it would state that, in the face of a
still-undecided will, the apograph would dissolve this indecision by
adopting the alternative variant in the text and that the author had
wanted to indicate his ``preference'' by underlining. This would be even
more probative if there were cases of \emph{alternative variants} that
were not underlined and were not adopted in the text in the same
apograph. Their presence would be a necessary and sufficient condition
to be able to say that the \emph{alternative variants} identified by the
underlining are considered by the author to be worthy of replacing the
base text. Even in this case, however, the reasons for putting the
underlined alternative variant in the text would not be valid, since the
author's intention was revealed only at the copying stage, and not
before, and was not completed by the formatting of the reading in the
line. The correct ecdotic practice to follow in this second case,
therefore, is to leave the base reading in the text, and to publish the
underlined alternative variant (marked with \emph{italics}) in the
footnotes. But one can further refine this system of representation.

Let us return to the Leopardian case we saw earlier. The underlined
reading: ``splendi'' [shine], although underlined, is not then
incorporated into text: from the first print-run of 1824 until the last
of 1837 the text will read: ``E tu che spunti'' [And you who appears].
On the other hand, a few lines below, an \emph{alternative variant},
also underlined, is ``promoted'' in the text and is followed by an
interlinear correction, emerging from the ``forest'' of other variants,
and imposing itself on them as an evolutionary variant:

\begin{figure}[H]
    \centering
    \includegraphics[width=\textwidth]{media/italia7.png}
    \caption{Figure 7: G. Leopardi, Manuscript of \emph{L'ultimo canto di Saffo}
[\emph{The Last Canto of Sappho}] (1822), C.L. X.5.2.gamma (National
Library of Naples) {[}detail{]}.}
    \label{fig:italia7}
\end{figure}

\textbf{Transcription of part of Figure \ref{fig:italia7}}

\begin{figure}
    \centering
    \includegraphics[width=\textwidth]{media/italia7-trans.png}
\end{figure}

These ``special'' alternative variants,\footnote{However, there are two
  special cases, which can be identified by comparing inks and
  spellings: (1) underlining is done before or at the same time as
  writing the alternative variant; in this case the author signals, by
  underlining, a part of the text worthy of attention, and on which is
  undecided, evidenced by the alternative reading; (2) underlining is
  done after writing the alternative variant: in this case, by
  underlining, the author can signal a preference for the stave reading
  over the alternative. In both cases, correct ecdotic practice requires
  leaving the stave reading in the text and presenting the alternative
  in a footnote.} which are ``chosen'' by the author by means of
underlining and become \emph{evolutionary variants}, in the critical
edition edited by Franco Gavazzeni, have been marked by
\uline{underlining} and also with \textbf{bold type}, in order to signal
their ``prestige'' to the reader in a clear and evident way \parencite{leopardi_canti_2009}.


\begin{figure}[H]
    \centering
    \includegraphics[width=\textwidth]{media/italia8.png}
    \caption{Figure 8: Giacomo Leopardi, \textit{Canti e Poesie disperse} [\textit{Songs and scattered poems}], critical edition edited by Franco Gavazzeni, t. I \textit{Canti}, Florence, Presso l'Accademia della Crusca, 2009, p. 225.}
    \label{fig:italia8}
\end{figure}

Let us go one step further in these cases of double text. What happens
when it is the line text that is underlined? In this case too, the
underlining (when not indicating a foreign word) is a mark of
importance. Of course, it must be an underlining and not a deletion.
It is difficult to confuse an underlining mark with a deletion mark in
the case of a unique witness, because deletion has its own unambiguous
marks (erasion, horizontal frieze, spiral, serration, etc.), even though
one may sometimes find oblique and indecisive deletions. Faced with a
manuscript that presents an underlined (but not redacted) stave reading,
and an alternative variant, the underlining should be considered a mark
of importance of the first reading, a sort of reinforcement of the
author's preference for the stave text, leaving the alternative
intention in a subordinate position. In this case, the correct edition
would put the first reading in the text, marking it in \emph{italics}
(and reproducing, according to the usual method, the alternative
variant).

\section{Correct Alternative Variant of an Incorrect Basic Reading}

Let us now look at one last interesting and insidious case. The case of
an \emph{alternative variant} referring to an incorrect base reading, as
in the example taken from Benedetto Varchi's\footnote{Benedetto Varchi,
  a 16th-century polygraph in the court of Cosimo I, for whom he became
  the historiographer, is an author of whom, due to his sudden death and
  Cosimo's need to finish the work he had entrusted to him, we possess
  an enormous quantity of manuscripts. The most interesting case is the
  \emph{Florentine History}, left unfinished. After his death, the work
  was revised by Cosimo's doctor and collaborator, Baccio Baldini, with
  corrections made directly on Varchi's manuscript. See the VASTO
  critical edition, edited by Dario Brancato, on the Dh.Arc website:
  \url{https://dharc-org.github.io/progetto-vasto/index.html\#EVT}; see
  also \cite{brancato_vasto_2021}.}
unfinished commentary on Aristotle's \emph{Meteorologica}, of which only
a rough draft survives. The typology of alternative variants is as
follows:

\subsection{Interlinear Alternative Variants.} In these two examples,
variants are \emph{adiaphora}, but one undoubtedly improves the sense of
the text.

The example in Figure \ref{fig:italia9a}, c. 223r, (``insegnamento'' [teaching] / ``trattamento'' [treatment]), is an example of an \emph{adiaphora} variant.

\begin{figure}[H]
    \centering
    \includegraphics[width=\textwidth]{media/italia9a.jpg}
    \caption{Figure 9: c. 223r}
    \label{fig:italia9a}
\end{figure}

\textbf{Transcription of part of Figure \ref{fig:italia9a}}

\begin{figure}
    \centering
    \includegraphics[width=\textwidth]{media/italia9-trans.png}
\end{figure}

The example in Figure \ref{fig:italia9b}, c. 238r, is similar to that in Figure \ref{fig:italia9a}:
the legitimacy of the reading is derived from the meaning of the
sentence; in it Varchi explains the nature of the vapour that rises from
the water due to the effect of the sun and therefore it is hot and
humid, as he writes a few lines later:

\begin{figure}[H]
    \centering
    \includegraphics[width=\textwidth]{media/italia9b.jpg}
    \caption{Figure 10: c. 238r}
    \label{fig:italia9b}
\end{figure}

\textbf{Transcription of part of Figure \ref{fig:italia9b}}

\begin{figure}
    \centering
    \includegraphics[width=\textwidth]{media/italia10-trans.png}
\end{figure}

\subsection{Basic Readings Underlined with Interlinear Alternative Variants}

It is not clear what Varchi means by \uline{underlining} the basic
reading: on the one hand, one could say that the author wants to keep
the original reading, that is, that the underlining indicates his
``preference'' for the basic text, but in one case it is the opposite,
since the alternative reading is the correct one. Let us look at two
examples: one dubious (1), the other in which the alternative variant is
correct, despite the fact that the text is underlined (2).

In the example of Figure \ref{fig:italia9c}, c. 231v, the basic reading is: ``se la natura mi dà e \emph{la consuetudine mi mantiene}'' (the underlined part in italics). If I have interpreted correctly, Varchi first adds:
``l'uso non mi toglie sopra la consuetudine mi mantiene'', and then writes ``conserva'' above ``mantiene''. Could it be that the underlining indicates that he wants to return to the original reading?

\begin{figure}[H]
    \centering
    \includegraphics[width=\textwidth]{media/italia9c.jpg}
    \caption{Figure 11: c. 231v}
    \label{fig:italia9c}
\end{figure}


The example in Figure \ref{fig:italia9d}, c. 235v is much simpler: the
reading of ``cold'' is undoubtedly the correct one from the point of view
of coherence. Indeed, Varchi explains the phenomenon of
``counter-resistance'' through the example of the middle region of the
atmosphere, which is cooled because the ``cold'', moist vapours rising
from the water towards the extreme region of the air are pushed down by
the latter due to the effect of the heat. 
In this case, it is extremely important to assess the nature of the
error, because the fact that the basic, incorrect reading is not
corrected may be a simple oversight (a case similar to the one we saw
earlier of ``inconclusive'' variant), but also reveal the author's
uncertainty about recognizing the error and correcting the text (and, by
maintaining the basic reading, he signals this indecision to himself).
In the critical edition of the text, if the philologist recognizes the
error as genuine, he is \emph{obliged to amend it}, promoting
the alternative variant (which becomes an evolutionary variant) to text
and indicating in the note to the text that the correction was made on
the basis of the author's intention provided by the alternative variant.
The alternative variant, of course, will not be carried over to the
footnote, because it would be identical to the text (and because it is
no longer an alternative variant).


\begin{figure}[H]
    \centering
    \includegraphics[width=\textwidth]{media/italia9d.jpg}
    \caption{Figure 12: c. 235v}
    \label{fig:italia9d}
\end{figure}

\textbf{Transcription of part of Figure \ref{fig:italia9d}}

\begin{figure}
    \centering
    \includegraphics[width=\textwidth]{media/italia12-trans.png}
\end{figure}

\noindent If, on the other hand, the correction is not recognized as certain, i.e.
if the base reading is not obviously incorrect, the philologist is
\emph{not obliged to correct} the text, rather, it is better to
treat the alternative variant as an ``inconclusive'' reading, perhaps
pointing out to the reader, in the \emph{Note to the text}, the
possibility that the alternative variant has been introduced to amend an
incorrect base text.

These case studies, which represent the most widespread phenomenologies
of \emph{alternative variants}, show how this particular typology of
variants, apparently simple, is in reality very complex, and in some
cases really insidious, but, unlike genetic and evolutionary variants,
which have a unique identity and can be easily represented once the
manuscript has been deciphered and the appropriate formalization
followed, in analyzing the alternative variants the philologist is
obliged to examine the entire tradition of the text, both preceding and
subsequent, in order to understand their ``double'' status and to arrive,
as justifiably as possible, at a critical edition that communicates to
the reader the intermediate textual level that they represent: a sort of
a ``limbo'', but also a place of textual ``possibilities'', of
potentialities yet to come to fruition, which allow us to understand the
most hidden folds of the text, and often its fascinating motivations.

\begin{flushleft}
    % use smallcaps for author names
    \renewcommand*{\mkbibnamefamily}[1]{\textsc{#1}}
    \renewcommand*{\mkbibnamegiven}[1]{\textsc{#1}} 
\printbibliography
\end{flushleft}
\end{document}