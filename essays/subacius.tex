%%%%%%%%%%%%%%
%% METADATA %%
%%%%%%%%%%%%%%

\contributor{
% Add all authors
Paulius V. Subačius
}

\contribution{
% Add full title
Textual Offshoots: Glosses of the Modern Drafts
}

\shortcontributor{
% short version of authors for running header
Paulius V. Subačius
}

\shortcontribution{
% short version of title for running header
Textual Offshoots: Glosses of the Modern Drafts
}

\begin{paper}
\renewcommand*{\pagemark}{}

\begin{abstract}
% write your abstract here
Interlinear or marginalia authorial notes as peculiar textual offshoots
are particularly characteristic of self-revised holographs. It is not
always easy to distinguish this kind of inscriptions from textual
additions, expansions, or alternative variants, i.e., ``competitive
revisions'', when the author did not indicate which of the variants should have priority. The discussion of manuscripts by several
twentieth-century Lithuanian poets reveals illuminative editorial
misapprehensions and brings us to consider the functional variety of
authorial notes. When discussing such cases, we face a more general
theoretical problem: how do we modernize the arrangement of textual
elements in a reading text so that we may stick to the original as much as possible, without sending false signals to the reader about the nature of textual elements?
\end{abstract}

%%%%%%%%%%%%%%%%%%%%%%%%%%%%
%% YOUR ESSAY STARTS HERE %%
%%%%%%%%%%%%%%%%%%%%%%%%%%%%

% remove asterisk (*) if you want to number your sections
% add a title for your section in between the {curly brackets} if you need one
\section*{} 
\textsc{In classical and biblical textual studies}, instances where a scribe failed to recognize a gloss and its insertion in a body text, resulting in several variants of a work transmitted in copies, have been discussed at length (cf. \citealt[17, 31--32]{renehan_greek_1969}; \citealt[16--17]{walker_interpolations_2001}). It seems that this problem does not pertain to modern
manuscripts. It is true that in the modern period, authors, translators,
and editors also needed to add an explanation of a rare word, an
alternative translation to the body text. However, the above-mentioned confusion is easily avoided by the contemporary
practices of the arrangement of text, mainly the use of footnotes and endnotes, the
possibilities of a commentary block offered by various word processors, and,
in particular, a shorter and more direct way of transmitting works. Yet, curious incidents
did occur in the history of holographs, when publishers and scholarly editors erroneously interpreted a modern author who
followed \label{qtSubacius}ancient-like writing habits. I will give some examples of
twentieth-century Lithuanian writers, which are relevant in this
respect. These examples will serve to suggest a certain classification
of the modern glosses. They will reveal another challenging decision that editors have to
make, namely how to transform the positions of
handwritten authorial insertions into the different layout of notes typical for print and digital media. I begin with presenting the
case of poetic translation that has been the cause of almost a
century-long editorial misunderstanding over authorial notes. Next, I
try to show that in some published works the graphic form alone allows
the textual segments to be considered as paratexts rather than parts of
the main text. Since a clear graphic structure of the insertions in the
manuscripts may be absent, the last section of the paper demonstrates
several ways for the gloss-like inscriptions to be interpreted.

In the early 1920s, the famous Lithuanian poet Maironis (1862--1932)
translated five hymns (\emph{suktas}) of \emph{The Rigveda}, a collection of
ancient Vedic Sanskrit hymns, into Lithuanian \citep{maironis_vertimai_1923}. One of
the evident aspirations of this translation was a textbook on world
literature that Maironis was writing at the time and published some
years later \citep{maironis_maironio_1926}. Unlike the first Lithuanian textbooks on
foreign literature for university and high school students written by
several other authors over the same five-year period, Maironis did not
limit himself to Western literature. He was guided by the German
tradition of \emph{Geschichte der Weltliteratur} (\citealt{baumgartner_geschichte_1897}; \citealt{wiegler_geschichte_1914};
\citealt{henschke_geschichte_1922}) and started his \emph{history} with a survey
of early Oriental literature –– Chinese, Japanese, Indian, Persian, and
Arab. Maironis did not know Sanskrit and translated the excerpts of the
\emph{Rigveda} texts from Polish \citep{michalski_czterdziesci_1912} –– the source of translation
was reliably established on the basis of textual matches and Polish
words found in Maironis' rough copies \citep[9v]{maironis_daiktu_1922}.

In contemporary editions of Maironis' poetry,\footnote{The editions
  of Maironis' poetry that appeared in the years of Soviet occupation did not include
  texts with religious motifs, among which translations of \emph{The Rigveda}, because ``they contained references to an
  Unfamiliar God'' \citep[339]{brazaitis_maironis_1957}.} which also include his scarce
translations, an attentive reader will spot two irregularities in
\emph{The Rigveda}'s texts: in the 115\textsuperscript{th}
sukta of the first \emph{mandala} (series of hymns into which \emph{The
Rigveda} is organized), dedicated to the \emph{Sūrya} {[}the Sun{]} (\citealp[232]{maironis_rastai_1987}; \citealp[235]{maironis_pavasario_2012}; \citealp[308]{maironis_pavasario_2020-1}; see Fig. \ref{fig:subacius1}), and
in the 129\textsuperscript{th} sukta of the tenth mandala, titled
``Creation'' in English editions \citep[1607]{jamison_rigveda_2014} and
``Daiktų pradžia'' {[}The Beginning of Things{]} in Maironis's
translation (\citealp[226]{maironis_rastai_1987}; \citealp[230]{maironis_pavasario_2012}; \citealp[300]{maironis_pavasario_2020-1}; see Fig. \ref{fig:subacius2}). Each of these texts contains a line with a word in
brackets, and because of that word, these lines are longer than the
others and do not correspond to the metric model. Moreover, in the
second example, the word in brackets is added at the end of the line,
preceded by a semicolon and a space. It disrupts not only the metre but
also the rhyme, which connects the penultimate rather than the last word
of the first line to the second line, based on the scheme \emph{aabb}.

\begin{figure}[H]
    \centering
    \frame{\includegraphics[width=.9\textwidth]{media/subacius1.jpg}}
    \caption{The 1\textsuperscript{st} stanza of ``Surjai Saulei'' [Sun] in \citealt[308]{maironis_pavasario_2020-1}.}
    \label{fig:subacius1}

\end{figure}

\begin{figure}[H]
    \centering
    \frame{\includegraphics[width=.9\textwidth]{media/subacius2.jpg}}
    \caption{The 4\textsuperscript{th} stanza of ``Daiktų pradžia'' [The Beginning of Things] in \citealt[300]{maironis_pavasario_2020-1}.}
    \label{fig:subacius2}
\end{figure}

What are these two words that stand out from the textual structure? In
the first case, the word in brackets \emph{mirties} {[}death{]} is a
simplified explanation of the quite rare, maybe even authorial
derivational form \emph{stingio} {[}\textasciitilde stagnation{]} used
in the verse. In the English prose translation by Stephanie W. Jamison,
the line is conveyed thus: ``The Sun is the life-breath of both the
moving and the still'' \parencite[267]{jamison_rigveda_2014}. In the second
case, next to the Lithuanian translation \emph{geidimas} {[}desire{]},
the transliterated Hindu concept \texthindi{काम}
%\begin{otherlanguage}{hindi}
%{\dn k{A}m}
%\end{otherlanguage}
-- \emph{kama}, which appears in
\emph{The Rigveda} and has the same meaning of \emph{desire},
\emph{longing}, is given in brackets. In the English prose translation
by Joel P. Brereton, the line reads: ``Then, in the beginning, from
thought there evolved desire'' \parencite[1607]{jamison_rigveda_2014}. Since
the words in brackets perform the function of glosses –– they explain a
rare form and present an equivalent of the original –– by no means
they are to be held the integral elements of the body text.

The question arises, why have these glosses been incorporated directly in a line in the
reading texts of contemporary editions? Is it a
result of the uncritical adoption, or of factual copying of the edition of
\emph{Oeuvre} by Maironis, which came out in the poet's lifetime and has
been considered the most authoritative source of his works ever since \citep[182 and 187]{maironis_maironio_1927}? The typesetter of
\emph{Oeuvre}, in turn, mechanically rendered what he saw in the holograph, which was the
fair copy that was handed to the printing house \citep[227r and 233r]{maironis_vertimai_1927}. A comprehensive comparative analysis of the entire text of
\emph{Oeuvre} and the holograph, which cannot be presented here in full,
shows that the author did not proofread the galleys. He prepared the
fair copy for publishing from heavily self-revised rough copies rather
automatically, inattentively, weakened by advanced-age fatigue. In the
fair copy, the elements that we interpret as glosses are incorporated
into the lines and distinguished only by brackets (see Figs. \ref{fig:subacius3} and \ref{fig:subacius4}).
We do not have any material that would allow us to reliably guess how
the writer himself imagined the final presentation of these two notes in
the printed publication. Glosses separated by brackets are nowhere to be
found in his earlier poetry collections, which were published under his
close supervision. On the contrary, these books include several
authorial annotations in the form of footnotes, e.g., equivalents of a rare
word in other languages and explanations of culture-specific items. Each
of these notes consists of several to a dozen words \citep[30,
45--46, 56--58 and 109]{maironis_pavasario_1920}. Similarly, the translations of \emph{The
Rigveda}'s hymns in the fair copy include as many as four longer
author's notes in the form of footnotes \citep[228r, 231r
twice and 233r]{maironis_vertimai_1927}. Both in \emph{Oeuvre} and in later editions, those
authorial notes are graphically conveyed as footnotes.

\begin{figure}[H]
    \centering
    \frame{\includegraphics[width=\textwidth]{media/subacius3.jpg}    }
    \caption{The 4\textsuperscript{th} line of ``Surjai Saulei'' [Sun] in \citealt[233r]{maironis_vertimai_1927}}
    \label{fig:subacius3}
\end{figure}

\begin{figure}[H]
    \centering
    \frame{\includegraphics[width=\textwidth]{media/subacius4.jpg}    }
    \caption{The 13\textsuperscript{th} line of ``Daiktų Pradžia'' [The Beginning of Things] in \citealt[227r]{maironis_vertimai_1927}}
    \label{fig:subacius4}
\end{figure}

Quite possibly it was the presence of glosses and scholia, to use
the classical terms,\footnote{The concept of \emph{gloss} is used in
  reference to authorial notes in the copy of the published poem
  inscribed next to the body text (as distinguished from authorial
  paratexts arranged differently and performing other functions) by
  Thomas Dilworth in his discussion of \emph{The Anathemata} by David
  Jones \citep{dilworth_david_1980}, and by Maria Dimitrova writing about William
  Empson's poetry \citep{dimitrova_decorus_2012}.} or shorter and longer author's
notes, in the same manuscript that caused confusion and misunderstanding
about the glosses by the first publisher and, later, by the (scholarly) editors
as well. In other words, there might have been some hesitation to
transform the glosses into a more modern form of footnotes, as then they
would not have been discernible from other notes that the author himself formatted as footnotes. In his turn, Maironis most likely followed the old
distinction between glosses and scholia, by which a short note is
written right next to the commented place, and a longer commentary is
moved farther, to the margins or below a text segment. I assume that the
difference between a one-word note and a two or more-word note is
significant in this case precisely for graphical reasons.

The fact that the poet actually regarded the words in brackets as notes
is testified by rough copies \citep[6v and 9v]{maironis_daiktu_1922}. They reveal
that while translating the hymn dedicated to the Sūrya {[}Sun{]},
Maironis grappled with the antonymic definition of deity in the fourth
line (conveyed in English as ``the still'' / ``the moving''). Thus, the
poet left a large empty gap between the first and the last words in the
line for the equivalent of ``the still''. In this gap, he wrote the
variant of the Polish translation \emph{bezwład} {[}inertia{]} faintly
(most likely holding a pen upright to avoid spilling ink), and when he
came up with a (rather dubious) Lithuanian equivalent, he wrote
\emph{tingio}~{[}laziness{]} over it in bold letters (see Fig. \ref{fig:subacius5}).
Later, using another pen, he wrote \emph{mirties}~{[}death{]} above,
between the lines. Even though the latter word was first discussed as an
alternative variant that could replace \emph{tingio} {[}laziness{]}, it
remained in the manuscript with the status of an explanatory gloss. The
word \emph{tingio} was corrected already in the fair copy into a
neologism \emph{stingio} {[}\textasciitilde stagnation{]}, different by
only one letter. At the end of the same line, three variants were tried
in search of a proper antonym (``the moving''): \emph{gyvumo}
{[}vitality{]} → \emph{judrumo} {[}mobility{]} → \emph{gaivumo}
{[}dew{]} (in the fair copy and published version, the variant
\emph{judrumo} {[}mobility{]} was chosen).

\begin{figure}[H]
    \centering
    \frame{\includegraphics[width=\textwidth]{media/subacius5.jpg}    }
    \caption{Authorial revisions in the 4\textsuperscript{th} line of ``Surjai Saulei'' {[}Sun{]} in \citealt[9v]{maironis_daiktu_1922}.}
    \label{fig:subacius5}
\end{figure}

Another rough copy shows even more clearly that in the hymn
``Creation'', the authentic transliterated Sanskrit concept \emph{kama}
found next to \emph{geidimas} {[}desire{]} is a gloss –– the word was
added later (a different ink hue), above the line, in brackets, and
on the side, in letters smaller by one third, though there was more than
enough empty space (see Fig. \ref{fig:subacius6}). These are typical features of the
graphic distinction of glosses. Contemporary editors might be clinging
to the latest authorial sources (the fair copy and the \emph{Oeuvre}) as
they are mystically faithful to the principle of the final authorial
intention (whose authority has been considerably discredited in the
community of textual scholars). However, in their entirety, the
discussed autographs reveal the misunderstanding about the glosses. In
my opinion, the graphic structure of Maironis' fair copy is not a valid
motive for a publication that integrates the notes into the body text, as it has been done so far.
Therefore, in the last –– digital –– edition they are untacked and
transposed into a separate field like the other authorial notes and
commentaries \citep{maironis_pavasario_2020}.

\begin{figure}[H]
    \centering
    \frame{\includegraphics[width=\textwidth]{media/subacius6.jpg}    }
    \caption{Draft version of the 13\textsuperscript{th} line of ``Daiktų Pradžia'' {[}The Beginning of Things{]} in \citealt[6v]{maironis_daiktu_1922}.}
    \label{fig:subacius6}
\end{figure}

Having in mind this curious case study, let us move on to the bigger
picture of gloss-induced complications. Despite the efforts not to
interrupt the purity of aesthetic reception in a contemporary reading
text edition, it is difficult to avoid explanatory notes in translations
of complicated poetic texts. And, truth be told, not only in the case of
ancient literature.\footnote{Recently, a student of mine asked me for
  advice regarding the publication of poems by the Estonian poet Ene
  Mihkelson (1944--2017), which she translated into Lithuanian. In
  Mihkelson's poetry, one frequently comes across personal names and
  other culture-specific items that are completely unknown outside
  Estonia, but without recognizing and understanding them the text
  remains incomprehensible. Thus, the translator was concerned about the
  arrangement of a factual commentary, which, in her opinion, absolutely
  needed to be placed next to the poems. I proposed to situate the notes
  in the very bottom-right corner of the pages and to choose a twice
  smaller font than that of the poems, which were graphically oriented
  to the upper-left side of the pages, so that the interference with the
  body text would be as slight as possible \citep{mihkelson_bokstas_2022}.} It is not extremely rare for poets to supplement the texts of their own works with
commentaries. Such cases have been known since the times of Dante and
Petrarch and the copious endnotes in T.S. Eliot's \emph{The Waste Land} \citep[28]{grafton_footnote_1997}. However, it is one thing when an author's commentary is both
graphically and functionally dissociated from the body text, when it is read
separately and tells the genesis of the work, or allows to understand, analyse, and rethink the details. In that case,
a commentary functions as an important, but still partly autonomous
supplement. Another thing is when a note is seen next to the body text;
moreover, in the author's opinion, the textual segments of the work are
not clear enough to the reader without such a note. In the
unsophisticated press (in the case of Lithuania, 19\textsuperscript{th} and early 20\textsuperscript{th}-century
literature), the absolute majority of the author's notes can be
qualified as a double crutch for a poetically immature text and an
uneducated reader. These are explanations of either archaisms and
newly-coined words that caused doubts to the writers themselves \citep[25 and 60]{maironis_pavasario_1905}, or culture-specific items mentioned in
rhymed historical narratives \citep[30--31]{maironis_pavasario_1913}.

However, in more modern literature, quite different kinds of authorial
annotations emerged. For example, in the works by the Lithuanian poet
Sigitas Geda (1943--2008). Especially in his poetry collection with the
memorable title \emph{Babilono atstatymas} {[}The Rebuilding of
Babylon{]}, the author's scholia are paratexts whose mutual dialogue
with the ``body'' of the poem is not less intense than that of the
titles or epigraphs. Presented at the bottom of the page, in the place
of footnotes, in a smaller font, sometimes clearly linked to the body
text by an asterisk, and sometimes without a reference sign, are phrases
rich in poetic expression, imbued with paradoxes (\citealt[5 and 8]{geda_babilono_1994};
see Fig. \ref{fig:subacius7}). The visual rhetoric, the blank space functioning
as ``overt or implicit invitations to the reader to fill in gaps'' \citep[18]{maguire_rhetoric_2020}
transforms the semantically homogeneous, yet separated, lines into a
metatextual commentary when returning to the already read poem. In their description of this particular structure of the
poetic text (incidentally, Geda's poems are on historical themes),
Claudia Claridge and Sebastian Wagner's insight on Late Modern English
historiography is befitting:

\begin{quote}
Looking at authorial notes through the lenses of paratext and
metadiscourse, one reaches a somewhat paradoxical conclusion. Notes are
positionally and visually clearly metadiscoursal and paratextual,
whereas linguistically and content-wise they exhibit close similarities
to functions and realisations found in the main text. The decision to
separate them from the main text is clearly significant, reflecting some
kind of authorial prioritising.
\begin{flushright}
\citep[69]{claridge_footnote_2020}
\end{flushright}
\end{quote}

\begin{figure}[!p]
    \centering
    \frame{\includegraphics[width=0.75\textwidth]{media/subacius7.jpg}    }
    \caption{Poem ``Antano Baranausko diagnozė'' {[}Diagnosis of Antanas Baranauskas's{]} in \citealt[8]{geda_babilono_1994}.}
    \label{fig:subacius7}
\end{figure}

It would be extremely difficult –– in fact, impossible –– to separate
phrases at the bottom from the lines of Geda's poem if such a textual
segment were not in isolating graphic interplay with the main textual
block. Which is exactly the case with manuscripts, in particular,
self-revised rough copies: unlike
strophic, metric, and rhymed poetry, the \emph{vers libres} text
structure does not allow to make a substantial assumption about
heterogeneous elements on a formal linguistic basis.

To decide if an inscription belongs to the text or the paratext –– and
to which type of paratext –– is close to impossible if we look at some
segments of an unpublished poem by the Lithuanian poet Judita
Vaičiūnaitė (1937--2001) on the motifs of the Spanish Civil War. I have
in mind the two last lines at the bottom of the \emph{recto} page
(\citealt[1r]{vaiciunaite_untitled_1961}; see Fig. \ref{fig:subacius8}). The phrases: ``Lithuanian
internationalists / Compatriots in the international brigades'' can
be either two variants of explanation of a poetic motif, or two versions
of the poem's title, or –– bearing in mind the poem's rhetoric style
and broken strophic system –– alternative variants of continuing the
  support the assumption that the inscription that it separates does not
body text.\footnote{There runs a horizontal line above these specific lines in Vaičiūnaitė's manuscripts, which doesn't necessarily support the assumption that these lines do not belong to the body text: elsewhere, identical graphical signs serve
  to separate strophe from strophe, a zone of corrections from another
  zone of corrections, etc.} Similarly, even the customary
entitling practice cannot be applied to the poem: on the one hand, the
status of the quoted phrases is unclear, and on the other, if we discard
the supposition that these are versions of the title, it remains unclear
which line comes first (quoted in contents and bibliographies as a
conventional reference to the work). As a matter of fact, the three top
lines (``\ldots was killed in the Aragon Offensive\ldots{} [\ldots]'')
can be considered both a pseudo-epigraph and an asymmetrical strophe, as
they are repeated with some variations at the end of the manuscript \citep[1v]{vaiciunaite_untitled_1961}. However, let us return to the main subject of our
discussion: the recognition of authorial notes and their status in an
edition.

\begin{figure}[H]
    \centering
    \frame{\includegraphics[width=0.5\textwidth]{media/subacius8.jpg}    }
    \caption{The first page of the draft version of the unpublished untitled {[}?{]} poem in \citealt[1r]{vaiciunaite_untitled_1961}.}
    \label{fig:subacius8}
\end{figure}

As we have seen, even a phrase- or sentence-long interlinear and
marginalia addition, or simply a line below the textual element, may
confuse an editor about its function: is it a part of the
body text, an author's commentary, or some different paratext? After
all, while reconsidering Gérard Genette's efforts to define and classify
contextualizing elements, his followers emphasize ``the fluidity of the
notions of paratext and (main) text'' \citep*[21]{bos_framing_2020}. When
discussing ``entextualisation processes by which text becomes paratext,
and [\ldots] the reverse process of paratext becoming part of text
[\ldots], for example, when marginal glosses or comments are visually
and linguistically amalgamated with the body text'' \citep[21]{bos_framing_2020}, ancient or early modern examples are more frequently brought
up, but it can also be found in modern holographs, albeit in a subtler
form. A complication to qualify insertions is even more valid in the
case of a word-long text element that looks like a gloss in the
manuscript. After all, supposedly, it is more convenient for the author
to put a note in the manuscript in the same place where variants are
usually written than to anchor a note with an asterisk or other mark and
to move the annotation to the bottom of the page or the end of the draft
notebook. Paola Italia and Giulia Raboni following Dante Isella offer
the definition of \emph{authorial philology}, whose task, among others,
is identifying metatextual notes in holographs and differentiating them
from alternative variants (cf. \citealt[57--58]{italia_what_2021}).
Having in mind the distinction made between authorial and traditional
philology –– the latter of which is the philology of the copy –– we
can make a pun of the former term and turn it around, for the discipline
refers to the cases when the author handles his manuscript like a
philologist himself and arranges glosses. Functionally, at least four
types of authorial inscriptions that look similar to glosses could be
distinguished. In what follows, I will present an example of each type from Lithuanian
holographs, although a lingering shadow of doubt remains around their
interpretation.

First, an alternative variant resulting from the failure to decisively choose between two or more revisional possibilities. 
The presumably latest of three rough copies of Vaičiūnaitė's unpublished poem
``Šerkšnas'' {[}Hoarfrost{]} contains seven pairs of variants (one pair
is phrase-long, and the other six are word-long) (\citealt[11r]{vaiciunaite_serksnas_1967}; see Fig. \ref{fig:subacius9}). Alternatives are written above or below a line, and the
length of the variant is marked by a broken line. In only one of these
cases the author made her choice: she discarded the inscription and
stayed with the original form. Writers have left numerous similar
documents in which the choice between textual alternatives in the form
of glosses is not expressed graphically, i.e., when the discarded
variant is crossed out, and so on. There are few places in such
holographs, particularly in the case of unpublished works, where an
editor may be puzzled if it is an addition or an annotational insert
rather than an alternative variant.
\vfill
\begin{figure}[H]
    \centering
    \frame{\includegraphics[width=0.5\textwidth]{media/subacius9.jpg}    }
    \caption{The 3\textsuperscript{rd} draft version of the unpublished poem ``Šerkšnas'' {[}Hoarfrost{]} in \citealt[11r]{vaiciunaite_serksnas_1967}.}
    \label{fig:subacius9}
\end{figure}
\newpage

The second type of authorial inscription is an expansion of the text. In an early draft of the much-revised poem ``Sniego balandžiai'' {[}Snow Doves{]} by Henrikas Nagys (1920--1996), in the second line above the phrase ``gėralo žalio'' {[}green booze{]},
another epithet ``keisto'' {[}strange{]} is written as a gloss (\citealt[33v]{nagys_sniego_1951}; see Fig. \ref{fig:subacius10}). Since in almost all drafts the poet would
either graphically discard one of the alternative variants or would
write them in a different pen or pencil (as can be seen in the third
line of the facsimile), the second epithet might be considered as an
expansion of the text rather than a replacement of words with one
another. The free rhythmical structure of the poem does not exclude such
a supposition.

\begin{figure}[H]
    \centering
    \frame{\includegraphics[width=.9\textwidth]{media/subacius10.jpg}    }
    \caption{Fragment of the 5\textsuperscript{th} draft version of the poem ``Sniego balandžiai'' {[}Snow Dowes{]} in \citealt[33v]{nagys_sniego_1951}.}
    \label{fig:subacius10}
\end{figure}

Thirdly, a note, i.e., a gloss in the traditional sense. I already presented obvious cases from Maironis' manuscripts, so now I will discuss a rather
peculiar case: the rough copy of Nagys' poem ``Memento (Mariui)''
{[}Memento (to Marius){]}, dedicated to the deceased husband of his
sister, a fellow writer Marius Katiliškis (1914--1980) (\citealt[20v]{nagys_memento_1980};
see Fig. \ref{fig:subacius11}). In the last line, the poet adds a vocative addressed to
the deceased, ``Mariau'', written in a red pen, with which the text was
revised. In a dedicational text, this kind of addition should not look
surprising, were it not for the fact that it disrupts the poem's
rhythmical structure and rhyme (the rhyme was:
{[}\emph{\sout{at}}{]}\emph{neš} \textbar{} \emph{tavęs}, i.e., the word
of the last line before the addition). Thus, we could consider this
inscription as the author's note to himself, as he might have thought
about a cadence at a public reading during a memorial event or a church
service.

\begin{figure}[!p]
    \centering
    \frame{\includegraphics[width=\textwidth]{media/subacius11.jpg}    }
    \caption{The end of draft version of the poem ``Memento (Mariui)'' {[}Memento (to Marius){]} in \citealt[20v]{nagys_memento_1980}.}
    \label{fig:subacius11}
\end{figure}

The fourth and final type of authorial inscription is a note that is totally unrelated to the work and accidentally
written on the same page. This is easy in the case of a telephone number
or an address, but sometimes highly mysterious inscriptions can be
found. In the rough copy of the poem ``Prieš lietų'' {[}Before the
Rain{]} by Marcelijus Martinaitis (1936--2013) (two versions on the same
page), we find segments written in five different pens (\citealt{martinaitis_pries_1967}; see Fig. \ref{fig:subacius12}). Some of the marginalia can be considered additions
and alternative variants, and some of them can be seen as rudimental
lines of two other poems. Yet, the function of the inserted phrase
written in a blue pen ``Kas paims mane ar seserį?'' {[}Who will pick me
or my sister up?{]} is unclear. It is likely a self-reflection or a
matter-of-fact daily reminder, but solid arguments to support the latter
interpretation are lacking.

\begin{figure}[H]
    \centering
    \frame{\includegraphics[width=0.75\textwidth]{media/subacius12.jpg}    }
    \caption{Draft versions of the poem ``Prieš lietų'' {[}Before the
Rain{]} in \cite[17r]{martinaitis_pries_1967}.}
    \label{fig:subacius12}
\end{figure}

From the perspective of the variorum edition, parallel edition,
or other editing models seeking to convey textual fluidity, the
presentation of a gloss both as an alternative variant and as an
addition will be implemented similarly. However, the possibility of a
holograph containing segments that function as authorial notes implies a
threefold challenge for an editor. Firstly, it is difficult to
distinguish a gloss-commentary from the above-mentioned inserts and
additions of another function. Second, a theoretical alternative exists:
is it just the author's note to himself, to the typist, to the
publisher, or is it addressed to the readers? Here we may get involved
in what looks like an eternal discussion about the intention and the
possibilities of its reconstruction, thus I will not elaborate on it
further. I will only note that a proposal to reconceptualize authorial
and sociological orientations to the text as \emph{causal} orientation
allows, in the case of modern holographs, to reformulate the not
necessarily complicated question ``who did the gloss'' (cf. \citealp[31]{shillingsburg_orientations_2015}), into ``what is the cause of the gloss'',
which implies not only the addresser but also the addressee. Still,
even after deciding that it is an author's note to himself, the editor
of a scholarly edition should convey or discuss the inscription in some
way, while a presumption that it is an author's note to the readers, on
the contrary, compels the editor to present it along with the (reading)
text. Thus, the third task is to find the best way to present a gloss,
which means to modernize the arrangement of textual elements so that we
stick to the original as much as possible, but do not send false signals
about the nature of textual elements.

Susan L. Greenberg, while discussing the changes in the formal
arrangement of texts and, alongside, the nature of editorial activity in
the process of switching from manuscripts to the printed medium,
emphasizes that ``the purpose of editorial intervention has always been
more about sense-making than making rules'' \citep[84]{greenberg_poetics_2018}. A
contemporary reader could recognize the heterogeneity of the gloss
regarding the body text of a literary work if it were an interlinear or
marginalia written in a smaller font. However, such graphic
representation –– especially brackets in the line itself –– is not an
adequate solution for an authorial note, in view of the aesthetic
perception of poetic texts. Editors can have several reasons to avoid
transforming manuscript glosses into footnotes/endnotes in a printed
publication, for example because they wish
to retain a merely graphic alternative of short, one-word and longer
notes known from the case of Maironis' translations or functional
``heterogeneity of Empson's notes, which freely combine glosses and
comments on disparate levels'' \citep[212]{dimitrova_decorus_2012}. In those cases, probably the most neutral solution from the viewpoint
of reading habits is arranging authorial notes as marginalia next to the
corresponding lines, as is often done with the presentation of modern
equivalents in the editions of texts containing numerous lexical and
orthographic archaisms. More generally, the challenge of editorial
\emph{pruning} of glosses reveals that the notion of relative textual
(in)stability also encompasses the tension caused by the longstanding
graphic patterns of the creative writing, which hardly fits in with the
print culture.

\section*{Acknowledgements}
This research was funded by the grants (No. S-MOD-17-7 \& S-LIP-21-11) from the Research Council of Lithuania (LMTLT).

\begin{flushleft}
    % use smallcaps for author names
    \renewcommand*{\mkbibnamefamily}[1]{\textsc{#1}}
    \renewcommand*{\mkbibnamegiven}[1]{\textsc{#1}} 
\printbibliography
\end{flushleft}

\end{paper}