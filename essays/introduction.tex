\title{Introduction}
\contribution{Introduction}
\shortcontribution{Introduction}
\author{}
\begin{preface}
\maketitle
\thispagestyle{contributiontitlepage}
\section*{Writers' Libraries, Creative {Revision}, and Histories of the Holograph}
\begin{flushright}\emph{Olga Beloborodova and Dirk Van Hulle}\end{flushright}
\label{introduction}
\vskip 1em

\section*{}
\textsc{Building bridges has been an important task} of the European Society for
Textual Scholarship in the past decade. In 2012, ESTS co-organised its
international conference in conjunction with the
Arbeitsgemeinschaft für germanistische Edition in Bern; in 2013, in
Paris, together with the Institut des textes et manuscrits modernes
(ITEM). To continue this tradition, we tried to strengthen the ties with
the international consortium GENESIS, working on genetic criticism
around the globe in various disciplines. In March 2022, we therefore
organised the conferences of GENESIS and ESTS back-to-back, in Oxford.
They were preceded by a workshop of the Beckett Digital Manuscript
Project's editorial team and an international colloquium on ``Writers'
Libraries'', with a keynote by Daniel Ferrer.

One challenge that both genetic criticism and textual scholarship face
is the issue of marginalia. In the past, they have often been published
separately from the text they relate to. The affordances of digital
scholarly editing enable new ways of highlighting the interaction
between text and marginalia. But this also requires coordination in
terms of methodology, lest we reinvent the wheel and build bespoke
editions that may work well on their own but do not speak to each other
/ are mutually incompatible. In literary studies, there is a growing
awareness of the importance of studying authors' libraries, since
authors are also –– and primarily –– readers, and their reading affects
their writing in various ways. In addition to the ``extant'' library, a
topic of special interest is the ``virtual'' library –– a reconstruction
of books the writer read, but which have not been preserved, either
because the writer made their reading notes based on a borrowed copy, or
because the copies they used were lost or sold. It is often possible to,
at least partially, reconstruct such a virtual library, but this raises
even more methodological issues than the edition of an extant library,
notably in terms of criteria for the inclusion (or exclusion) of books
depending on the degree of certainty of uncertainty with which the exact
edition of a particular source text can be established.

Marginalia and other reading notes are a central research object in
exogenetic studies. How they are processed in a writer's drafts is a
question that relates to endo- and epigenesis, characterized by layers
of revision in successive draft versions, proofs or even editions, if
the author gets the chance to revise their text (as in the case of the
six editions of \emph{The Origin of Species} published in Darwin's
lifetime). This type of creative revision was the theme of the GENESIS
conference. As the subtitle ``Exercises in Comparative Genetic Criticism''
indicates, it was conceived as an explicit attempt to explore the
possibility of studying creative processes, both literary and
non-literary, from a comparativist point of view. Revising, reworking or
revamping is a crucial element of any artistic process. In a narrow
sense, this revision can be the late phase in the genesis shortly before
the work of art is made public. In a broader sense, it may also include
other forms of re-vision, of looking again.

For instance, a 2018 exhibition of Rubens' ``intertextual''
appropriations in Frankfurt (Städel museum) showed how the artist
gradually processed his own impression and sketch of a sculpture of a
Centaur and then used it to give shape to a painting of a religious
theme, Pilate presenting Christ to the crowd (``Ecce Homo''). From this
``intertextual'' perspective, the painting acquires several layers of
complexity. Not only does it show Christ with a Centaur's torso, but the
whole transformation of a Greek mythological theme into a Christian
topic is a complex process, in which the artist's drafts or sketches
play a crucial intermediary role. Similarly, in literary studies, a
writer's notebooks and drafts play a pivotal role in the analysis of
intertextual relations.

In this context, Michael Baxandall's criticism of the term ``influence''
(and implicitly of the term ``source'') is still relevant, to all types of
artistic research, from architecture to musicology, from art history to
literary studies. ``Influence'', he writes, is ``a curse of art criticism
primarily because of its wrong-headed grammatical prejudice about who is
the agent and who the patient'' \parencite[58]{baxandall_patterns_1985}.

Another ``agent'' in this transformation is the audience –– listeners,
spectators, readers –– adding ``a dimension of experienced meaning'', to
borrow Paul Eggert's phrase \parencite[173]{eggert_work_2019}. The work of art is as much a process as it
is a product and the audience is a crucial participant in this process.
Researchers interested in the genesis of art works, or genetic critics,
are ``readers'' in their own right. So are composers who listen to music,
filmmakers who watch movies, writers who read books.

Genetic criticism is therefore not limited to a focus on the production
process, but also involves a work's reception. The GENESIS –– OXFORD
2022 conference invited scholars from various disciplines (art history,
architecture, musicology, literature, \ldots) to reflect on the dictum
``\emph{ex nihilo nihil fit}'' (nothing comes from nothing), comparing forms of
artistic metamorphosis across various art forms.

Adding a historical dimension to this comparative approach, the ESTS
conference was dedicated to ``Histories of the Holograph'' and took place
against the backdrop of an exciting historiographical book project,
called \emph{A Comparative History of the Literary Draft in Europe}.
Reading notes and literary manuscripts are a constant in literatures of
all ages and linguistic areas, and yet their role in writing processes
in various traditions has seldom been the subject of systematic scrutiny
in comparative literary studies. Likewise, literary histories usually do
not take the writing process into account, leaving this domain largely
within the purview of either palaeography or genetic criticism. Tackling this double (comparative and historiographic) gap is the idea behind this book project, run under the
auspices of the International Comparative Literature Association's
Coordinating Committee for the Comparative History of Literatures in
European Languages (ICLA/CHLEL) and expected to be published in the fall of 2024.
The project focuses on literary drafts in European languages but
transcends the continent's geographical borders by including draft
material from North American and postcolonial literatures. In
conceptualising the project's reach, we opted to focus primarily on
aspects relating to the literary draft (i.e., the textual aspects of
creative processes), but other, non-textual elements such as different
kinds of material bearers (including born-digital texts) and
non-literary drafts (i.e., aspects of creative processes that are not
necessarily textual) are also included, in order to show that creativity
at work crosses genre and media boundaries, and that comparing text
production to other art forms can generate valuable insights.

A comparative history of the draft is an important missing link in
today's literary historiography. One of the reasons of this gap is the
disparate nature of literary manuscripts. In order to be able to produce
an effective and comprehensive comparison of such a disparate cohort of
manuscripts, the project works with eight categories of comparison,
five of which (temporal, spatial, linguistic, generic, and editorial)
relate mainly to the text of the manuscripts and three of which are
non-textual in nature (material, intermedial, and conceptual).

The ESTS conference focused in the first instance on the temporal
comparison, for which the mid-eighteenth century is often used in
genetic criticism as a watershed to distinguish modern manuscripts (i.e., private documents) from earlier (e.g., ancient, medieval) manuscripts,
which are often scribal copies, i.e., meant for public dissemination. But
even though holographs are rarer in the pre-modern period, they do
exist: for example, Petrarch revised the scribal copies of his poems
\emph{Rerum vulgarium fragmenta} in the so-called ``codex of sketches''
(\emph{Codice degli abbozzi}) of the \emph{Canzoniere}. Similarly, the
era of the ``modern manuscript'' needs to be confronted on the other end
of its temporal spectrum with the era of the ``digital manuscript'' –– \label{qtVanhulle3}the
born-digital holograph, so to speak. By comparing drafts across periods
(from early medieval documents to twenty-first century
born-digital texts), the conference aimed to establish a dialogue
between what are traditionally considered separate disciplines. The
question to be asked and answered here is whether the function of the
literary draft changes over time.

In any historiographic investigation, the temporal component is
inextricably connected to the spatial one. In Europe, different
traditions of textual scholarship are based on different archival
situations. For instance, the German tradition is to a large extent
shaped by editions of Goethe's works and the relative abundance of his
extant manuscripts, whereas Shakespeare, whose works are marked by the
relative lack of holograph manuscripts, has served as a paradigm for the
long-standing Anglo-American tradition of bibliography and copy-text
editing. In Italy, the concept and practice of \emph{filologia d'autore}
took root and blossomed due to the abundance of well-preserved
manuscripts, some of which date back to the thirteenth
century. A comprehensive comparative survey of these different
traditions shows how particular influential archival situations have had
an impact on various attitudes to manuscript studies in different
regions in Europe and on postcolonial literatures.

Along with geography comes linguistic variety. If –– as many linguistic
paradigms claim –– language has an impact on our cognitive and creative
processes, a comparative perspective on writing processes in different
languages should uncover notable differences in the creative process
depending on the language in which a work is written. The most inviting
cases for such a comparison are works by translingual authors, such as
Samuel Beckett and Vladimir Nabokov. Because in most cases these authors
are also translators of their work, the process of literary translation
itself deserves a closer look, which leads us beyond translingual
authors and into translators' archives more generally. Very few
translators actually preserve their manuscripts, but the few extant
translators' archives are treasure troves for a genetic investigation,
as the bourgeoning field of genetic translation studies testifies. A
comparative analysis of the translation process generates valuable
insights into the cognitive processes of writing and helps reassess the
somewhat undervalued status of the translator in literary
historiography. Besides, translators' drafts reveal to which degree the
translation process is a collaborative affair between the author and the
translator.

While collaboration for translation purposes may seem logical and
therefore not unexpected, the nature of authorship itself is (perhaps
too easily) branded as solitary. At the same time, there are obvious
differences in conceptualization depending on the literary genre: for
instance, creating a theatre play is universally acknowledged as a
collaborative effort that involves many different agents. By contrast,
writing poetry and prose are generally considered to be the fruits of an
author's solitary labour –– no doubt the legacy of the deeply entrenched
Romantic notion of ``genius''. A genetic comparison across genres, i.e., a
comparative investigation into the writing processes involved in poetry,
fiction and drama, furnishes us with tangible evidence that helps refine
these stereotypical assumptions on the nature of authorship.

One such refinement is the discussion of the role of the editor in the
creative process. Contrary to popular opinion, the writing process does
not end at the moment the manuscript is dispatched to the publisher but
continues in \label{qtVanhulle2}the no-man's-land between the author's desk and the
publishing house, involving editors and, consequently, editorial
intervention. Despite Stephen King's famous claim that ``The editor is
always right'' \parencite[13]{king_writing_2000}, the role of editors in the writing
process is often underrated and rarely studied. The editorial comparison
allows us to revisit the editor's contribution to creativity and
foreground the sociology of writing that marks the creation of every
text. Apart from the collaborative nature of authorship, such a
comparison fleshes out a wide range of editorial interventions into the
writing process: from seemingly more mundane matters of orthography and
punctuation to drastic pruning, to such a degree that it ends up
defining the author's typically curt style.

As mentioned above, \label{qtVanhulle1}creativity is not an exclusively textual matter, as
it is influenced to a large degree by non-textual, material and
medium-specific affordances and constraints. Both the temporal and
spatial comparisons uncover a great variety of materials, from stylus
tablets to tablet computers, from parchment to paper, from papyrus to
pc. When Nietzsche started using an early precursor of the typewriter,
he famously suggested in a letter to Heinrich Köselitz that writing
tools have a cognitive impact: ``unser Schreibzeug arbeitet mit an
unseren Gedanken'' \parencite[``our writing tools assist in our
thought processes''; qtd. in][15]{van_hulle_towards_2019} More
recently, research conducted in writing studies revealed that college
students deploy different planning and revision strategies depending on
whether they are using pen and paper or a word processor \parencite{hayes_new_1996,van_waes_writing_2003}. If it is true that our writing tools have
an impact on our thought processes, the material comparison of literary
drafts may offer us a rare insight into the workings of the creative
mind, and a comparative history might uncover cognitive patterns as we
are moving from paper-based to born-digital literatures.

Similarly,  a genetic comparison across different media gives us a more
precise idea of how medium-specific properties affect the genesis of a
work of art. The intermedial comparison covers a wide range of the
audiovisual spectrum: architecture, film and television on
the one hand, and music and radio on the other. For many of these media,
the notion of adaptation will be of importance: as Ian McEwan reflected
on his role as screenplay writer for the film version of his novel
\emph{On Chesil Beach}, ``it's a sort of demotion from God to a little
cherub, or General to Corporal. {[}\ldots{]} Writing a novel is to
address yourself to a finished literary form. The screenplay is not a
finished form, it's part of the recipe, it's not the meal. It's very,
very different''\parencite[qtd. in][]{noauthor_ian_2018}.
Other forms, such as music and architecture, furnish us with useful
models for comparative genetic criticism.

Among this wealth of perspectives and angles, one should not lose sight
of one of the textual agents at the core of any genetic investigation,
namely the author. How do authors see their own writing methods, and how
do they convey their views? Authors often express themselves in
conceptual metaphors, which can be broadly divided into organic (e.g., 
the ``embryo'', ``birth'', ``growth'', ``germ'' of a work) versus
constructivist metaphors (e.g., writing as ``a factory'' or as
``bricolage''). A more detailed comparative study of such metaphors
fleshes out new insights that not only further shape the current models
of genetic criticism, but –– perhaps more importantly –– give us a sense
of how these metaphors change over time, which is illuminating for
discerning future trends from past and present experiences. This glimpse
into the future of authorship ties in with the two central questions of
the historiographic theme.

First of all: \emph{How does the inclusion of drafts in literary studies
change the idea of literary history?} The study of writing processes
(genetic criticism) focuses less on literary successes and finished
products than on conceptual hesitation, writer's block, creative undoing
and revision processes. Many literary drafts never led to a publication.
A history of writing processes will therefore imply a welcome
alternative to traditional literary histories, which tend to focus on
(the reception of) successfully published works.

And secondly: \emph{What are the consequences of this new comparative
historiographic perspective (on the literary draft) for the future of
manuscript research and literary criticism more generally?} Are
born-digital texts fundamentally different from `analogue' texts in
terms of their writing processes? What is the impact of digital
technologies on the creative process? When a poet sends drafts of their
new poem to friends on their smartphone; when a novelist like Jeroen
Olyslaegers shares all of his preparatory research for his new novel
online, how does that affect the concept of authorship?\footnote{\url{https://wildevrouw.blogspot.com/p/welkom.html}}
What is a draft in the case of a born-digital text? Recovering earlier
versions can become more challenging as it is so easy to erase text,
seemingly irrevocably, in a digital environment. Fortunately, we have
new technologies (digital forensics and keystroke logging software) to
help us shape the future approaches and methodologies of genetic
criticism, armed with the insights from a historiographic perspective.


\begin{flushleft}
    % use smallcaps for author names
    \renewcommand*{\mkbibnamefamily}[1]{\textsc{#1}}
    \renewcommand*{\mkbibnamegiven}[1]{\textsc{#1}} 
\printbibliography
\end{flushleft}
\end{preface}