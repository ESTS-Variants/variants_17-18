\documentclass{article}
%%%% CLASS OPTIONS 

\KOMAoptions{
    fontsize=10pt,              % set default font size
    DIV=calc,
    titlepage=false,
    paper=150mm:220mm,
    twoside=true, 
    twocolumn=false,
    toc=chapterentryfill,       % for dots: chapterentrydotfill
    parskip=false,              % space between paragraphs. "full" gives more space; "false" uses indentation instead
    headings=small,
    bibliography=leveldown,     % turns the Bibliography into a \section rather than a \chapter (so it appears on the same page)
}

%%%% PAGE SIZE

\usepackage[
    top=23mm,
    left=20mm,
    height=173mm,
    width=109mm,
    ]{geometry}

\setlength{\marginparwidth}{1.25cm} % sets up acceptable margin for \todonotes package (see preamble/packages.tex).

%%%% PACKAGES

\usepackage[dvipsnames]{xcolor}
\usepackage[unicode]{hyperref}  % hyperlinks
\usepackage{booktabs}           % professional-quality tables
\usepackage{nicefrac}           % compact symbols for 1/2, etc.
%\usepackage{microtype}          % microtypography
\usepackage{lipsum}             % lorem ipsum at the ready
\usepackage{graphicx}           % for figures
\usepackage{footmisc}           % makes symbol footnotes possible
\usepackage{ragged2e}
\usepackage{changepage}         % detect odd/even pages
\usepackage{array}
\usepackage{float}              % get figures etc. to stay where they are with [H]
\usepackage{subfigure}          % \subfigures witin a \begin{figure}
\usepackage{longtable}          % allows for tables that stretch over multiple pages
\setlength{\marginparwidth}{2cm}
\usepackage[textsize=footnotesize]{todonotes} % enables \todo's for editors
\usepackage{etoolbox}           % supplies commands like \AtBeginEnvironment and \atEndEnvironment
\usepackage{ifdraft}            % switches on proofreading options in the draft mode
\usepackage{rotating}           % provides sidewaysfigure environment
\usepackage{media9}             % allows for video in the pdf
\usepackage{xurl}               % allows URLs to (line)break ANYWHERE

%%%% ENCODING

\usepackage[full]{textcomp}                   % allows \textrightarrow etc.

% LANGUAGES

\usepackage{polyglossia}
\setmainlanguage{english} % Continue using english for rest of the document

% If necessary, the following lets you use \texthindi. Note, however, that BibLaTeX does not support it and will report a 'warning'.
 \setotherlanguages{hindi} 
 \newfontfamily\hindifont{Noto Sans Devanagari}[Script=Devanagari]

% biblatex
\usepackage[
    authordate,
    backend=biber,
    natbib=true,
    maxcitenames=2,
    ]{biblatex-chicago}
\usepackage{csquotes}

% special characters  
\usepackage{textalpha}                  % allows for greek characters in text 

%%%% FONTS

% Palatino font options
\usepackage{unicode-math}
\setmainfont{TeX Gyre Pagella}
\let\circ\undefined
\let\diamond\undefined
\let\bullet\undefined
\let\emptyset\undefined
\let\owns\undefined
\setmathfont{TeX Gyre Pagella Math}
\let\ocirc\undefined
\let\widecheck\undefined

\addtokomafont{disposition}{\rmfamily}  % Palatino for titles etc.
\setkomafont{descriptionlabel}{         % font for description lists    
\usekomafont{captionlabel}\bfseries     % Palatino bold
}
\setkomafont{caption}{\footnotesize}    % smaller font size for captions


\usepackage{mathabx}                    % allows for nicer looking \cup, \curvearrowbotright, etc. !!IMPORTANT!! These are math symbols and should be surrounded by $dollar signs$
\usepackage[normalem]{ulem}                       % allows for strikethrough with \sout etc.
\usepackage{anyfontsize}                          % fixes font scaling issue

%%%% ToC

% No (sub)sections in TOC
\setcounter{tocdepth}{0}                

% Redefines chapter title formatting
\makeatletter                               
\def\@makechapterhead#1{
  \vspace*{50\p@}%
  {\parindent \z@ \normalfont
    \interlinepenalty\@M
    \Large\raggedright #1\par\nobreak%
    \vskip 40\p@%
  }}
\makeatother
% a bit more space between titles and page numbers in TOC

\makeatletter   
\renewcommand\@pnumwidth{2.5em} 
\makeatother

%%%% CONTRIBUTOR

% Title and Author of individual contributions
\makeatletter
% paper/review author = contributor
\newcommand\contributor[1]{\renewcommand\@contributor{#1}}
\newcommand\@contributor{}
\newcommand\thecontributor{\@contributor} 
% paper/review title = contribution
\newcommand\contribution[1]{\renewcommand\@contribution{#1}}
\newcommand\@contribution{}
\newcommand\thecontribution{\@contribution}
% short contributor for running header
\newcommand\shortcontributor[1]{\renewcommand\@shortcontributor{#1}}
\newcommand\@shortcontributor{}
\newcommand\theshortcontributor{\@shortcontributor} 
% short title for running header
\newcommand\shortcontribution[1]{\renewcommand\@shortcontribution{#1}}
\newcommand\@shortcontribution{}
\newcommand\theshortcontribution{\@shortcontribution}
\makeatother

%%%% COPYRIGHT

% choose copyright license
\usepackage[               
    type={CC},
    modifier={by},
    version={4.0},
]{doclicense}

% define \copyrightstatement for ease of use
\newcommand{\copyrightstatement}{
         \doclicenseIcon \ \theyear. 
         \doclicenseLongText            % includes a link
}

%%%% ENVIRONMENTS
% Environments
\AtBeginEnvironment{quote}{\footnotesize\vskip 1em}
\AtEndEnvironment{quote}{\vskip 1em}

\setkomafont{caption}{\footnotesize}

% Preface
\newenvironment{preface}{
    \newrefsection
    \phantomsection
    \cleardoublepage
    \addcontentsline{toc}{part}{\thecontribution}
    % enable running title
    \pagestyle{preface}
    % \chapter*{Editors' Preface}    
    % reset the section counter for each paper
    \setcounter{section}{0}  
    % no running title on first page, page number center bottom instead
    \thispagestyle{chaptertitlepage}
}{}
\AtEndEnvironment{preface}{%
    % safeguard section numbering
    \renewcommand{\thesubsection}{\thesection.\arabic{subsection}}  
    %last page running header fix
    \protect\thispagestyle{preface}
}
% Essays
\newenvironment{paper}{
    \newrefsection
    \phantomsection
    % start every new paper on an uneven page 
    \cleardoublepage
    % enable running title
    \pagestyle{fancy}
    % change section numbering FROM [\chapter].[\section].[\subsection] TO [\section].[\subsection] ETC.
    \renewcommand{\thesection}{\arabic{section}}
    % mark chapter % add author + title to the TOC
    \chapter[\normalfont\textbf{\emph{\thecontributor}}: \thecontribution]{\vspace{-4em}\Large\normalfont\thecontribution\linebreak\normalsize\begin{flushright}\emph{\thecontributor}\end{flushright}}    
    % reset the section counter for each paper
    \setcounter{section}{0}  
    % reset the figure counter for each paper
    \renewcommand\thefigure{\arabic{figure}}    
    % reset the table counter for each paper
    \renewcommand\thetable{\arabic{table}} 
    % no running title on first page, page number center bottom instead, include copyright statement
    \thispagestyle{contributiontitlepage}
    % formatting for the bibliography

}{}
\AtBeginEnvironment{paper}{
    % keeps running title from the first page:
    \renewcommand*{\pagemark}{}%                            
}
\AtEndEnvironment{paper}{
    % safeguard section numbering
    \renewcommand{\thesubsection}{\thesection.\arabic{subsection}}  
    % last page running header fix
    \protect\thispagestyle{fancy}%                              
}
% Reviews
\newenvironment{review}{
    \newrefsection
    \phantomsection
    % start every new paper on an uneven page 
    \cleardoublepage
    % enable running title
    \pagestyle{reviews}
    % change section numbering FROM [\chapter].[\section].[\subsection] TO [\section].[\subsection] ETC.
    \renewcommand{\thesection}{\arabic{section}} 
    % mark chapter % add author + title to the TOC
    \chapter[\normalfont\textbf{\emph{\thecontributor}}: \thecontribution]{}    % reset the section counter for each paper
    \setcounter{section}{0}  
    % no running title on first page, page number center bottom instead, include copyright statement
    \thispagestyle{contributiontitlepage}
    % formatting for the bibliography
}{}
\AtBeginEnvironment{review}{
% keeps running title from the first page
    \renewcommand*{\pagemark}{}%                                   
}
\AtEndEnvironment{review}{
    % author name(s)
    \begin{flushright}\emph{\thecontributor}\end{flushright}
    % safeguard section numbering
    \renewcommand{\thesubsection}{\thesection.\arabic{subsection}} 
    % last page running header fix
    \protect\thispagestyle{reviews}                           
}

% Abstract
\newenvironment{abstract}{% 
\setlength{\parindent}{0pt} \begin{adjustwidth}{2em}{2em}\footnotesize\emph{\abstractname}: }{%
\vskip 1em\end{adjustwidth}
}{}

% Keywords
\newenvironment{keywords}{
\setlength{\parindent}{0pt} \begin{adjustwidth}{2em}{2em}\footnotesize\emph{Keywords}: }{%
\vskip 1em\end{adjustwidth}
}{}

% Review Abstract
\newenvironment{reviewed}{% 
\setlength{\parindent}{0pt}
    \begin{adjustwidth}{2em}{2em}\footnotesize}{%
\vskip 1em\end{adjustwidth}
}{}

% Motto
\newenvironment{motto}{% 
\setlength{\parindent}{0pt} \small\raggedleft}{%
\vskip 2em
}{}

% Example
\newcounter{example}[chapter]
\newenvironment{example}[1][]{\refstepcounter{example}\begin{quote} \rmfamily}{\begin{flushright}(Example~\theexample)\end{flushright}\end{quote}}

%%%% SECTIONOPTIONS

% command for centering section headings
\newcommand{\centerheading}[1]{   
    \hspace*{\fill}#1\hspace*{\fill}
}

% Remove "Part #." from \part titles
% KOMA default: \newcommand*{\partformat}{\partname~\thepart\autodot}
\renewcommand*{\partformat}{} 

% No dots after figure or table numbers
\renewcommand*{\figureformat}{\figurename~\thefigure}
\renewcommand*{\tableformat}{\tablename~\thetable}

% paragraph handling
\setparsizes%
    {1em}% indent
    {0pt}% maximum space between paragraphs
    {0pt plus 1fil}% last line not justified
    

% In the "Authors" section, author names are put in the \paragraph{} headings. To reduce the space after these  headings, the default {-1em} has been changed to {-.4em} below.
\makeatletter
\renewcommand\paragraph{\@startsection {paragraph}{4}{\z@ }{3.25ex \@plus 1ex \@minus .2ex}{-.4em}{\normalfont \normalsize \bfseries }
}
\makeatother

% add the following (uncommented) in environments where you want to count paragraph numbers in the margin
%    \renewcommand*{\paragraphformat}{%
%    \makebox[-4pt][r]{\footnotesize\theparagraph\autodot\enskip}
%    }
%    \renewcommand{\theparagraph}{\arabic{paragraph}}
%    \setcounter{paragraph}{0}
%    \setcounter{secnumdepth}{4}
    
%%%% HEADERFOOTER

% running title
\RequirePackage{fancyhdr}
% cuts off running titles that are too long
%\RequirePackage{truncate}
% makes header as wide as geometry (SET SAME AS \TEXTWIDTH!)
\setlength{\headwidth}{109mm} 
% LO = Left Odd
\fancyhead[LO]{\small\emph{\theshortcontributor} \hspace*{.5em} \theshortcontribution} 
% RE = Right Even
\fancyhead[RE]{\scshape{\small\theissue}}
% LE = Left Even
\fancyhead[LE]{\small\thepage}            
% RE = Right Odd
\fancyhead[RO]{\small\thepage}    
\fancyfoot{}
% no line under running title; cannot be \@z but needs to be 0pt
\renewcommand{\headrulewidth}{0 pt} 

% special style for authors pages
\fancypagestyle{authors}{
    \fancyhead[LO]{\small\textit{Authors}} 
    \fancyhead[LE]{\small\thepage}            
    \fancyhead[RE]{\scshape{\small\theissue}}
    \fancyhead[RO]{\small\thepage}            
    \fancyfoot{}
}

% special style for book reviews
\fancypagestyle{reviews}{
    \fancyhead[LO]{\small\textit{Book Reviews}} 
    \fancyhead[LE]{\small\thepage}            
    \fancyhead[RE]{\scshape{\small\theissue}}
    \fancyhead[RO]{\small\thepage}            
    \fancyfoot{}
}

% special style for Editors' preface.
\fancypagestyle{preface}{
    \fancyhead[LO]{\small\textit{\theshortcontribution}} 
    \fancyhead[LE]{\small\thepage}            
    \fancyhead[RE]{\scshape{\small\theissue}}
    \fancyhead[RO]{\small\thepage}            
    \fancyfoot{}
}
% special style for first pages of contributions etc.
% DOES include copyright statement
\fancypagestyle{contributiontitlepage}{
    \fancyhead[C]{\scriptsize\centering\copyrightstatement}
    \fancyhead[L,R]{}
    \fancyfoot[CE,CO]{\small\thepage}
}
% special style for first pages of other \chapters.
% DOES NOT include copyright statement
\fancypagestyle{chaptertitlepage}{
    \fancyhead[C,L,R]{}
    \fancyfoot[CE,CO]{\small\thepage}
}
% no page numbers on \part pages 
\renewcommand*{\partpagestyle}{empty}

%%%% FOOTNOTEFORMAT

% footnotes
\renewcommand{\footnoterule}{%
    \kern .5em  % call this kerna
    \hrule height 0.4pt width .2\columnwidth    % the .2 value made the footnote ruler (horizontal line) smaller (was at .4)
    \kern .5em % call this kernb
}
\usepackage{footmisc}               
\renewcommand{\footnotelayout}{
    \hspace{1.5em}    % space between footnote mark and footnote text
}    
\newcommand{\mytodo}[1]{\textcolor{red}{#1}}

%%%% CODESNIPPETS

% colours for code notations
\usepackage{listings}       
	\renewcommand\lstlistingname{Quelltext} 
	\lstset{                    % basic formatting (bash etc.)
	       basicstyle=\ttfamily,
 	       showstringspaces=false,
	       commentstyle=\color{BrickRed},
	       keywordstyle=\color{RoyalBlue}
	}
	\lstdefinelanguage{XML}{     % specific XML formatting overrides
		  basicstyle=\ttfamily,
		  morestring=[s]{"}{"},
		  morecomment=[s]{?}{?},
		  morecomment=[s]{!--}{--},
		  commentstyle=\color{OliveGreen},
		  moredelim=[s][\color{Black}]{>}{<},
		  moredelim=[s][\color{RawSienna}]{\ }{=},
		  stringstyle=\color{RoyalBlue},
 		  identifierstyle=\color{Plum}
	}
    % HOW TO USE? BASH EXAMPLE
    %   \begin{lstlisting}[language=bash]
    %   #some comment
    %   cd Documents
    %   \end{lstlisting}
\author{Maria Carolina Escobar Vargas}
\title{Holographs and their Copies: the Writing and the Transmission of
Historical Writing in the Twelfth Century, the Case of Leiden,
University Library, MS BPL 20}

\contributor{
% Add all authors
Maria Carolina Escobar Vargas
}

\begin{document}
\maketitle

\begin{abstract}
% write your abstract here
This paper examines the practice of chronicle writing in the twelfth century, challenging the notion of the holographic manuscript. It challenges the established notion of holographs by exploring how these manuscripts, although ostensibly collections of factual historical information, were influenced by multi-layered institutional and political agendas. This phenomenon was particularly prominent in the Anglo-Norman territories, where chronicles were instrumental in appropriating history and forging identities. Through the case study of Leiden, University Library, MS BPL 20 –– an alleged authorial copy of Robert of Torigni’s \emph{Gesta Normannorum Ducum} –– the article explores whether holograph manuscripts held more authority than their copies. By analyzing the manuscript's codicological details, provenance, and the collaborative efforts in its production, this article highlights the intricate relationship between historical writing, authorial authority, and the evolving nature of medieval historiography. The findings raise pertinent questions about the status and authenticity of holographs and their role in the transmission of historical narratives.
\end{abstract}

\section*{} 
\textsc{This article will argue} that the practice of chronicle writing in the
twelfth century defies the established notion of the holographic
manuscript. Purportedly written as collections of factual historical
information, chronicles in this period were written following
multi-layered agendas. They claimed authority through their basis in
accredited sources whilst also expressing newer institutional or
political agendas. This exercise of historical writing gained pace in
the territories ruled by the Anglo-Norman kings as an important means of
both appropriating the past and building up identity. They are complex
narratives, defying modern perceptions of the factual and the fictional
in the medieval past. They also raise challenging issues around
authorial figures and their roles. This article will consider whether
holograph manuscripts of such chronicles should be granted more status
than other copies of the same texts. Is it possible to understand how
far holographs constituted ``official'' versions of this type of
historical writing when other manuscripts or copies of other chronicles
were not? If holographs were well cared for and kept in the monasteries
where they were written, were they particularly significant compared to
later copies? Do copies bear testimony to the actual significance of the
holograph? And does collaborative production alter the concept of the
holograph?

In order to explore these questions, this article will examine the
manuscript now Leiden, University Library, BPL 20, an alleged authorial
copy of Robert of Torigni's \emph{Gesta Normannorum Ducum}, written at
the abbey of Le Bec \emph{c. 1139}. The manuscript is presumed to have
travelled to England early, with probable links to at least two other
manuscript copies, all datable to the twelfth century: London, British
Library, Cotton Ms Nero D VII \citep[xcvi]{van_houts_gesta_1992}; and a Reading
Abbey manuscript now divided between Cambridge, Gonville and Caius
College, Ms 177/20 and London, British Library, Cotton Ms Vitellius A
VIII \citep[cxii]{van_houts_gesta_1992}.

Leiden, University Library, BPL 20 has been the subject of much recent
analysis.\footnote{BPL 20 is available in digitised format at the
  digital collections of the University of Leiden:
  \url{https://digitalcollections.universiteitleiden.nl/view/item/1611194\#page/63/mode/1up}.
  For a discussion on this manuscript see: \cite[cix]{van_houts_gesta_1992}. \cite[23--25]{crick_historia_1991}. \cite[58--64]{pohl_abbas_2014}. \cite[101--102]{pohl_robert_2018}. \cite[94--95]{cleaver_autograph_2018}. \cite[150]{weston_manuscripts_2017}. \cite[289--298]{avril_notes_1964}. \cite[2--6]{dumville_early_1985}.
  \cite[80--81]{hermans_history_1983}.} Thanks to this, it is possible to
establish the basic codicological information of the manuscript with a
substantial degree of detail. It is dated to the middle of the twelfth
century, with a precise \emph{terminus ante quem} date of \emph{c.} 1163,
and its production can be placed at the Benedictine abbey of Bec in
Normandy (\cite[211]{avril_notes_1964}; \cite[80--81]{hermans_history_1983}; \cite[4]{dumville_early_1985}). It consists of two different parts. Part I, fos. 2r--59v, can
be dated more precisely to \emph{c.} 1139 thanks to the autograph work
of Robert of Torigni and his version of the \emph{Gesta Normannorum
Ducum} (\cite[cix]{van_houts_gesta_1992}; \cite[23--25]{crick_historia_1991}). Part II, including fos.
60r--106v can be dated on palaeographical grounds to \emph{c.} 1160.
Notably, both sections of the manuscript were bound together by 1164
when they appear as such in an entry in a Bec catalogue of that date,
now Avranches, Bibliothèque Municipale, MS 159 fols. 1v--3r (\cite[190--205]{pohl_monastic_2017}; \cite[2--4]{dumville_early_1985}).\footnote{In the Le Bec catalogue the
  manuscript appears as item 115: ``In uno volumine historie
  normannorum, libri VIII, videlicet ab adventu hastingi in regnum
  Francorum usque ad mortem primi Henrici regis anglorum et ducis
  normannorum. Item vita caroli magni imperatoris Romanorum et regis
  Francorum. Item vita alexandri magni regis macedonum. Item epistola
  eiusdem de situ indie ad aristotilem magistrum suum. Item abbreviation
  gestorum regum francie ab egressione eorum a sicanbria usque ad
  principium regni Ludovici iunioris regis francorum. Item historiarum
  de regibus maioris britannie usque ad adventum anglorum in eandem
  insulam, libri XII. In quorum septimo continentur prophetie meslini
  non silvestris sed alterius id est mellini ambrosii. Item exceptiones
  ex libro gilde sapientis historiographi britonum, quem composuit
  devastation sue gentis et de mirabilibus britannie'' [The history of the Normans, in eight books, namely from the arrival of Hastings in the kingdom of the Franks up to the death of the first Henry, king of the English and duke of the Normans. Also, the life of Charlemagne, emperor of the Romans and king of the Franks. Also, the life of Alexander the Great, king of the Macedonians. Also, his letter about the situation in India to Aristotle, his master. Also, an abridgment of the deeds of the kings of France from their departure from Sicambria up to the beginning of the reign of Louis the Younger, king of the Franks. Also, the histories of the kings of Britain up to the arrival of the English in the same island, in twelve books. In the seventh of these are contained the prophecies of Merlin, not the woodland one but the other, that is, Merlin Ambrosius. Also, excerpts from the book of Gildas the Wise, the historiographer of the Britons, who wrote about the devastation of his people and the wonders of Britain] \citep[203; my translation]{pohl_monastic_2017}. The same list of contents appears in the flyleaf of BPL 20,
  which identifies the manuscript as having belonged to Le Bec. Dumville
  places the date of the catalogue in the 1150s (\cite[3]{dumville_early_1985}; see
  also \cite[45--47]{rouse_potens_1991}).} The manuscript contains Einhard,
\emph{Vita Karoli Magni} (fos. 38v--47r), \emph{Vita Alexandri Magni}
(fos. 38v--47r), \emph{Epistola Alexandri Magni ad Aristotelem} (fos.
47r--51v), \emph{Abbreviatio gestorum regum Franciae} (fos. 52r--59r),
\emph{Genealogia comitum Flandriae} (fo. 59v), Geoffrey of Monmouth's
\emph{Historia regum Britanniae} (fos. 60r--101v), \emph{Historia
Britonum} (fos. 101v--106r), and a fragment of Orderic Vitalis'
\emph{Historia Ecclesiastica} (fo. 106v). This is a historical volume
dealing with the histories of France, Normandy and Britain, including
biographies, chronicles, and genealogical information relating to the
counts of Flanders. As previously mentioned, the \emph{Gesta Normannorum
Ducum} in this manuscript is considered an autograph (\cite[cix]{van_houts_gesta_1992}; \cite[24]{crick_historia_1991}; \cite[69]{lieftinck_manuscrits_1964}; \cite[80]{hermans_history_1983}).\footnote{For more on medieval autographs see: \cite{overgaauw_comment_2013}.}
Robert had been a monk of Bec before becoming abbot of Mont St
Michel, and his work on the \emph{Gesta Normannorum Ducum} was done
whilst he was at Bec. Its binding, together with such important works of
secular history, thus suggests that Bec viewed him as an authority who
could place the dukes of Normandy on the same level.

According to its editor, Elizabeth Van Houts, the \emph{Gesta
Normannorum Ducum} constitutes the first post-Carolingian national
history of a Frankish principality \citep[xix]{van_houts_gesta_1992}. It was first
written by Dudo of Saint-Quentin under the heading \emph{De moribus et
actis primorum Normannorum ducum}, between \emph{c}. 996 and \emph{c}.
1015. Dudo's work was continued and revised by William of Jumièges. His
revised version, known as Redaction C, brought the history of the
Normans up to \emph{c.} 1070, thus including the history of the conquest
of England by William the Conqueror. William's text was, in turn,
revised by multiple authors, many of whom remain anonymous. Their work
constitutes the Redactions α, A, B and D of the \emph{Gesta Normannorum
Ducum}. Two other authors would contribute further to the redaction of
the text, Orderic Vitalis and then Robert of Torigni himself. Orderic
worked on the \emph{Gesta Normannorum Ducum} from \emph{c.} 1109
\emph{to c}. 1113 and produced what is known as Redaction E. Apart from
revising William's style, Orderic included extensive interpolations to
books VI and VII dealing with Dukes Robert I and William the Conqueror.
However, he did not expand the narrative beyond the chronological
framework already established by William.

The last Latin author to contribute to the \emph{Gesta Normannorum
Ducum} was Robert of Torigni, who worked on the text in the late 1130s.
He added substantial work to books II, VI and VII. Above all, he added a
whole new book, VIII, on King Henry I, bringing the narrative up to
1135. His redaction is known as Redaction F. Thus, the writing and
rewriting of the \emph{Gesta Normannorum Ducum} can be seen both as a
long-drawn-out process of collaboration and as a challenge to standard
concepts of authorship. It involved a careful process of revisions,
additions, subtractions and manipulation of the text by several
individuals who were in constant dialogue with the work of their
predecessors. It evinces some of the mechanisms under which history was
written in the Central Middle Ages and, importantly, poses interesting
questions to our current understanding of the holograph. If an
historical text could be altered through a considerable timespan, with
the intervention of various authors, some of whom were anonymous, how
does this shift our understanding of ``the Author''? Consequently, if
there is not a single authoritative version of a text, is it still
possible to think about the existence of an authorial holograph? These
questions make it important to examine the surviving copies of Robert's
own work.

There are forty-seven surviving medieval manuscripts of the work
identified by modern editors as the \emph{Gesta Normannorum Ducum} and
fourteen of Dudo's ``original'' text, showing the popularity of the work
and its availability throughout the Middle Ages. That a possible
continuation to the work was considered in the form of a biography of
Geoffrey of Anjou, the husband of the Empress Matilda, is shown by a
letter sent by Robert of Torigni to Prior Gervase of Saint-Céneri in
1151 \citep[132]{van_houts_writing_2018}. No continuation was to be written; however,
the \emph{Gesta Normannorum Ducum} was later used as a major source for
both the vernacular \emph{Roman de Rou} by Wace, written in the last
quarter of the twelfth century, and for the \emph{Chronique des ducs de
Normandie} by Benoît de Sainte-Maure. This provides further proof of the
ongoing status accorded to the work, and to Robert's ``concluding''
version in particular.

The version of the \emph{Gesta Normannorum Ducum} in BPL 20 has been
identified by scholars as Robert's personal copy (\cite[cix, cxxvi]{van_houts_gesta_1992}; \cite[95]{cleaver_autograph_2018}; \cite[59]{pohl_abbas_2014}). This raises important questions
around the concepts of both an authorial holograph and a ``reference''
copy intended for ongoing institutional ownership. The text is accepted
as being in its latest stages of composition, representing a version
that Robert himself thought ready for ``publication'' \emph{c.} 1139, but
which he continued to amend during 1139--1159.\footnote{Benjamin Pohl has
  argued for continuous amendments to the text up to the year 1165,
  \citep[52, 59, 63]{pohl_abbas_2014}.} Parts of the work are missing from this
manuscript, with losses from the beginning of the text and book VIII,
mainly due to the absence of the first two quires. This has the
unfortunate effect that both title page (if any) and incipit are
missing. The work of several scribes, apparently directed by Robert of
Torigni, can be seen in the manuscript and constitute another layer of
collaborative work. The skeleton of the text, which was copied first,
corresponds to that of Redaction E by Orderic Vitalis, with blank spaces
originally left for Robert's additions. Robert's interpolations were
added later in a variety of hands. This team of scribes were most likely
working under the close supervision of Robert himself. The blank spaces
left in the manuscript were probably set before Robert's composition of
his work, as some spaces are too small to fit comfortably (fol. 15v; see
Fig. \ref{fig:escobar1}) and others are too large, so the text appears stretched (fol.
15r; see Fig. \ref{fig:escobar2}).

Benjamin Pohl has identified one of these hands as that of Robert,
citing its presence throughout the manuscript as scribe, annotator and
corrector \citep[58--64]{pohl_abbas_2014}.\footnote{For a counterargument see:
  \cite{bisson_scripts_2019}.} According to Pohl, the specific hand, also present in
Avranches 159, stands out because it "can be seen repeatedly to amend the
work not only of virtually all previous scribes but also of previous
correctors" and because "while most of the hands in Leiden BPL 20 are
confined to certain parts of the manuscript, this one hand features
throughout the entirety of the \emph{Gesta Normannorum Ducum}", \citep[60]{pohl_abbas_2014}. In fact its corrections occur throughout the whole of BPL 20.
As noted by Pohl, this amending hand usually concerns itself with
grammatical corrections that sometimes require erasure and
replacement.\footnote{See fols. 17r, 47v, 52r, 65r, 99r \citep[60]{pohl_abbas_2014}.}
Sometimes it adds more substantial revisions. An excellent example is
the explanatory notes attached to personal names or place names,
betraying a solid interest in Anglo-Norman affairs. They serve to
"render the texts to which they are attached more intelligible and
accessible to Norman audiences".\footnote{See fols. 2v, 4r, 6v, 14v,
  19r, 20v, 25r, 29v, 30r, 30v, 105 (\cite[60]{pohl_abbas_2014}; \cite[217]{van_houts_robert_1989}.} This hand stands out as particularly angular, less regular and
"informal", not the product of a trained scribe (see Fig. \ref{fig:escobar3}).

\begin{figure}[H]
  \centering
    \includegraphics[width=\textwidth]{media/escobar1.jpg}
    \caption{Figure \ref{fig:escobar1}: (BPL 20, fol. 15v).}
    \label{fig:escobar1}
  \end{figure}

 Image released in the public domain by the Leiden University Libraries 


  \begin{figure}
    \includegraphics[width=\textwidth]{media/escobar2.jpg}
    \caption{Figure \ref{fig:escobar1}: (BPL, 20, fol. 15r).}
    \label{fig:escobar2}
\end{figure}

 Image released in the public domain by the Leiden University Libraries 


\begin{figure}[H]
  \centering
  \includegraphics[width=1\textwidth]{media/escobar3.png}
    \caption{Figure \ref{fig:escobar1}: (Fol. 2v).}
    \label{fig:escobar3}
  \end{figure}

 Image released in the public domain by the Leiden University Libraries 


This hand stands out, too, as it shares few characteristics with the
hands that wrote the main body of the text in BPL 20 (see Fig.
\ref{fig:escobar4}).\footnote{See Pohl's remarks for the palaeographical characteristics
  of Robert's hand: ``the overall impression created by the general
  ductus of Robert's hand and the shape of its individual distinctive
  features is that of a person who, despite being fully literate, was
  much more used to writing with a stylus than with a quill'' \citep[76--77]{pohl_abbas_2014}. For more on the use of wax tablets, see: \cite{rouse_vocabulary_1990}.} As Van Houts has argued, there is no question about Robert's
over-riding role in the execution of BPL 20 in all of its stages \citep[cx]{van_houts_gesta_1992}.\footnote{For Robert's interest in genealogy, see: \cite{van_houts_robert_1989}.} His interventions worked to create the ``final word'' on
the text written out by a team of scribes whom he directed. Thus, Robert
stands out as as an author who relied on a team of scribes working under
his instruction. He was increasingly gaining responsibility as a
monastic administrator but still keenly interested in the composition of
history, the compilation of books, and their acquisition.\footnote{Very
  little evidence of Robert's activities as a scribe at Bec survive, for
  this see: \cite[105]{pohl_robert_2018}. For more on Robert's role as an
  administrator, even before his election as prior, see especially pages \cite[117--119]{pohl_robert_2018}.} It is possible that Robert's reliance on his scribal team
was the result of this situation and was thus unique to him. However
this does not seem to have been the case since, as noted by Jenny
Weston, "For the many authors of Le Bec, the scriptorium (and its
scribes) were crucial to the development and presentation of their
ideas. Without diligent scribes recording the text, and making suitable
adjustments at the authors' request, many of the innovative ideas
developed by the community at Le Bec would not have survived" \citep[159--160]{weston_manuscripts_2017}. If this is correct then, at least in the case of an
historical work, both the contents of the text and the hand(s) producing
fair copies were open to constant change. Can Robert's direct
involvement in the production and writing of this manuscript then count
as sufficient evidence for it being a holograph? Do we not need to
account for the efforts of his team of scribes, who, as suggested by
Weston, gave voice not only to an author but to a whole community at
Bec? Would this type of collaborative work in situ not somewhat alter
our definition of the authorial process in the composition of this
narrative? This raises the related question of whether Robert himself
came to be considered as an authority on Norman history.

\begin{figure}[H]
  \centering
  \includegraphics[width=0.8\textwidth]{media/escobar4.png}
    \caption{Figure \ref{fig:escobar1}: (Fol. 20v).}
    \label{fig:escobar4}
  \end{figure}

 Image released in the public domain by the Leiden University Libraries 

  
Robert of Torigny was born in Western Normandy in the early twelfth
century.\footnote{Benjamin Pohl places his date of birth at around 1106 
  \citep[112]{pohl_robert_2018}. For more on Robert of Torigni and his links to the
  Anglo-Normans see: \cite{bates_robert_2012}.} He entered the monastery of Bec in
Normandy in 1128 as a young adult, and twenty years later, \emph{c}.
1149, he became prior; an office he held until he was elected
abbot of Mont-Saint-Michel in 1154. He died in June 1186. At Bec, Robert worked
on the \emph{Gesta Normannorum Ducum} and a \emph{Chronica,} describing
events from 318 to 1111 and elaborating upon the \emph{World Chronicle}
of Sigebert of Gembloux. Robert's knowledge of recent history was
celebrated by Henry of Huntingdon, in an addition to his own
\emph{Historia Anglorum}. This took the form of a letter to one Warin
the Breton, narrating Henry's encounter with Robert at Bec in 1139.
Robert is depicted as having copies of important, new historical works,
even though he is quoted as describing himself as ``a man most devoted to
the investigation and collection of both divine and secular writings'',
(\cite[lxxviii]{van_houts_gesta_1992}; \cite[98]{pohl_robert_2018}). Robert is thus frequently
credited as having been vital to the increased production and collection
of manuscripts at Bec during the middle of the twelfth century. He
carried on with this role at Mont-Saint-Michel, where he completed his
\emph{Chronica} and potentially continued to revise the \emph{Gesta
Normannorum Ducum}, showing a continuous interest in the writing of
history and his ongoing use of the library holdings at Bec. Thanks to
this interest, two book lists detailing the manuscripts held at Bec in
the second half of the twelfth century survive (\cite{pohl_monastic_2017}; \cite[51]{pohl_abbas_2014}). They are included in the flyleaves of the manuscript now Avranches,
Bibliothèque Municipale, Ms 159, which includes a copy of Robert's
\emph{Chronica,} its oldest surviving witness.\footnote{The first list
  includes the donation of books given to Le Bec by Philip de Harcourt,
  Bishop of Bayeaux while the second and longer one is thought to
  include the contents of Le Bec's \emph{Almarii}. The most recent
  edited version of both lists can be now found in \cite{pohl_monastic_2017}.}
Their contents suggest that he influenced the choice of books to be
acquired by Bec. Robert stands out as a man who gradually came into his
own, rising to the challenges and opportunities presented to him while
at Bec, and taking on both greater responsibilities and greater power.
Being an influential member of the community at Bec allowed Robert to
compose the \emph{Gesta Normannorum Ducum} in the manner previously
described, clearly impacting the outcome of his endeavour. Henry of
Huntingdon's testimony confirms Robert as both collector and critic of
historical works.

For his version of the \emph{Gesta Normannorum Ducum}, Robert of Torigni
relied heavily on the version written by Orderic Vitalis and not on the
earlier text of William of Jumièges. He preserved most of the dedicatory
letter and the seven books that constituted Orderic's text's main body.
However, while Orderic had not intervened in the early parts of the
\emph{Gesta Normannorum Ducum}, leaving William of Jumièges' text mostly
unchanged, Robert of Torigni felt free to make more interventions. He
made extensive interpolations to books I and II, primarily based on Dudo
of Saint-Quentin's, \emph{De moribus et actis primorum} \emph{Normanniae
ducum}. He inserted further interpolations throughout, substantially
adding to the text again in book VII, concerning William the Conqueror
\citep[lxxxii]{van_houts_gesta_1992} and again, adding an entire book celebrating
the deeds of King Henry I, instead of including the shorter epilogue
which already appeared in Orderic Vitalis' text. This redaction is the
most popular of all the redactions of the Gesta Normannorum Ducum,
accounting for 23 of the surviving forty-seven medieval manuscripts.
However, does this mean we can think about Robert's version, as
portrayed in BPL 20, as the most authoritative or '`official'' version of
the text? It certainly portrays a moment of composition where after ample
manipulation by several authors and hands, the narrative is reaching its
latest stage of development. Still, does this inherently mean that Robert's
version carries more authority than the one produced by Orderic Vitalis,
William of Jumièges or any of the anonymous authors who worked on the
text's other redactions? Is it possible to think about BPL 20 as an
authoritative copy of the text?

Let us now consider the physical characteristics of BPL 20, for it will
allow us to contextualise the status of the work in terms of its
production and possibly of its intended reception while providing clues
as to the potential authority the work might have carried. BPL 20 is
relatively large, measuring approximately 32.5 by 22.5 cm. The text is
set out in two columns, with double vertical boundary lines and a
divided central margin (see Figs. \ref{fig:escobar5} and \ref{fig:escobar6}). It has 47 lines of text, and
it is rubricated (see Fig. \ref{fig:escobar7}). The Gesta Normannorum Ducum also includes
line-fillers in red in the first six books of the text, particularly
after the number of chapters (see Fig. \ref{fig:escobar8}). It is written in Protogothic
miniscule in various hands, with longish ascenders and
descenders.\footnote{Hermans and Van Houts have identified up to nine
  different hands intervening in the manuscript \citep[80]{hermans_history_1983}.} The script is laterally compact. It has initials of
various sizes, in green, red and blue (see Figs. \ref{fig:escobar9}, \ref{fig:escobar10} and \ref{fig:escobar11}). Some
minor initials have been filled out in ochre, particularly of lists of
chapters, as in fol. 9v (see Fig. \ref{fig:escobar12}). There are occasional more
elaborate examples of initials, as the blue initial in fol. 24r (see
Fig. \ref{fig:escobar13}) or when a curious face looking through a letter O appears in
the same fol. (see Fig. \ref{fig:escobar14}). Larger and more elaborate initials, such as
the ones in fos. 5r (see Fig. \ref{fig:escobar15}), 10v (see Fig. \ref{fig:escobar16}) and 22v (see Fig.
\ref{fig:escobar17}) mark the start of new books. Here a more elaborate setting with
curling lines and penwork is present.

Unfortunately, the loss of the first quire of text means it is not
possible to know whether more elaborate treatment was given to the
opening initial of the volume. Surviving initials at the beginning of
the other works contained in part I of this manuscript vary in size and
level of elaboration (see Figs. \ref{fig:escobar18}, \ref{fig:escobar19}, \ref{fig:escobar20}, \ref{fig:escobar21}). However, they are generally
less complex than the initials opening new books in the \emph{Gesta
Normannorum Ducum}, except perhaps for the 9-line ``A'' found in fol. 52r
opening the \emph{Adbreviatio Gestorum Regnum Franciae} \ref{fig:escobar22}.
This particularly large initial displays three colours, red, green, and
ochre, and it includes foliate pen-worked interlace. This highlights
that even though the main initial has been lost, those remaining show
more elaboration than opening initials for other texts in the volume,
except for the elaborate `'A'' in fol. 52r. As such, it can be argued that
the opening initial of Robert's work probably matched that of other
significant historical texts in this same volume, like the 9-line ``C''
opening Geoffrey of Monmouth's \emph{Historia Regum Britanniae} (see
Fig. \ref{fig:escobar23}), a striking example of the treatment such works could merit.

The initial to Geoffrey's text depicts the image of a mounted knight
leaping over an elaborate interlaced ``C'' on coloured ground (see Fig.
\ref{fig:escobar23}). It comes in part II of BPL 20 and is thus later than the other
initials discussed so far. It evinces a slightly later style but still
uses the same colour palette and plain background as the one observed in
part I of the manuscript. Furthermore, it indicates the treatment
historical works could receive at the twelfth-century Bec scriptorium
and may therefore indicate the level of display given to the opening of
the \emph{Gesta Normannorum Ducum}.

\begin{figure}[H]
  \centering
    \includegraphics[width=\textwidth]{media/escobar5.png}
    \caption{Figure \ref{fig:escobar1}: (Fol. 2r).}
    \label{fig:escobar5}
  \end{figure}

 Image released in the public domain by the Leiden University Libraries 


  \begin{figure}
    \includegraphics[width=\textwidth]{media/escobar6.jpg}
    \caption{Figure \ref{fig:escobar1}: (Fol. 2v).}
    \label{fig:escobar6}
\end{figure}

 Image released in the public domain by the Leiden University Libraries 


\begin{figure}[H]
  \centering
  \includegraphics[width=0.76\textwidth]{media/escobar7.jpg}
    \caption{Figure \ref{fig:escobar1}: (Fol. 10v).}
    \label{fig:escobar7}
  \end{figure}

 Image released in the public domain by the Leiden University Libraries 


\begin{figure}[H]
  \centering
  \includegraphics[width=0.76\textwidth]{media/escobar8.jpg}
    \caption{Figure \ref{fig:escobar1}: (Fol. 12r).}
    \label{fig:escobar8}
  \end{figure}

 Image released in the public domain by the Leiden University Libraries 


\begin{figure}[H]
  \centering
    \includegraphics[width=\textwidth]{media/escobar9.png}
    \caption{Figure \ref{fig:escobar1}: (Fol. 23r).}
    \label{fig:escobar9}
  \end{figure}

 Image released in the public domain by the Leiden University Libraries 


  \begin{figure}
    \includegraphics[width=\textwidth]{media/escobar10.png}
    \caption{Figure \ref{fig:escobar1}: (Fol. 21r).}
    \label{fig:escobar10}
  \end{figure}

 Image released in the public domain by the Leiden University Libraries 


  \begin{figure}
    \includegraphics[width=\textwidth]{media/escobar11.png}
    \caption{Figure \ref{fig:escobar1}: (Fol. 11r).}
    \label{fig:escobar11}
\end{figure}

 Image released in the public domain by the Leiden University Libraries 


\begin{figure}[H]
  \centering
  \includegraphics[width=0.8\textwidth]{media/escobar12.png}
    \caption{Figure \ref{fig:escobar1}: (Fol. 9v).}
    \label{fig:escobar12}
  \end{figure}

 Image released in the public domain by the Leiden University Libraries 


\begin{figure}[H]
  \centering
    \includegraphics[width=\textwidth]{media/escobar13.png}
    \caption{Figure \ref{fig:escobar1}: (Fol. 24r).}
    \label{fig:escobar13}
  \end{figure}

 Image released in the public domain by the Leiden University Libraries 


  \begin{figure}
    \includegraphics[width=\textwidth]{media/escobar14.png}
    \caption{Figure \ref{fig:escobar1}: (Fol. 24r).}
    \label{fig:escobar14}
  \end{figure}

 Image released in the public domain by the Leiden University Libraries 


  \begin{figure}[H]
  \centering
    \includegraphics[width=\textwidth]{media/escobar15.png}
    \caption{Figure \ref{fig:escobar1}: (Fol. 5r).}
    \label{fig:escobar15}
  \end{figure}

 Image released in the public domain by the Leiden University Libraries 


  \begin{figure}
    \includegraphics[width=\textwidth]{media/escobar16.jpg}
    \caption{Figure \ref{fig:escobar1}: (Fol. 10v).}
    \label{fig:escobar16}
  \end{figure}

 Image released in the public domain by the Leiden University Libraries 


   \begin{figure}[H]
  \centering
    \includegraphics[width=\textwidth]{media/escobar17.jpg}
    \caption{Figure \ref{fig:escobar1}: (Fol. 22v).}
    \label{fig:escobar17}
  \end{figure}

 Image released in the public domain by the Leiden University Libraries 


   \begin{figure}
    \includegraphics[width=\textwidth]{media/escobar18.png}
    \caption{Figure \ref{fig:escobar1}: (Fol. 33r, Vita Karoli Magni).}
    \label{fig:escobar18}
\end{figure}

 Image released in the public domain by the Leiden University Libraries 


\begin{figure}[H]
  \centering
    \includegraphics[width=0.89\textwidth]{media/escobar19.png}
    \caption{Figure \ref{fig:escobar1}: (Fol. 38v, Vita Alexandri Magni).}
    \label{fig:escobar19}
  \end{figure}

 Image released in the public domain by the Leiden University Libraries 


  \begin{figure}
    \includegraphics[width=0.7\textwidth]{media/escobar20.png}
    \caption{Figure \ref{fig:escobar1}: (Fol. 42r, Epistola ad Aristotelem).}
    \label{fig:escobar20}
\end{figure}

 Image released in the public domain by the Leiden University Libraries 


\begin{figure}[H]
  \centering
  \includegraphics[width=0.73\textwidth]{media/escobar21.png}
    \caption{Figure \ref{fig:escobar1}: (Fol. 59v, Genealogia comitum flandrie).}
    \label{fig:escobar21}
  \end{figure}

 Image released in the public domain by the Leiden University Libraries 


\begin{figure}[H]
  \centering
    \includegraphics[width=\textwidth]{media/escobar22.png}
    \caption{Figure \ref{fig:escobar1}: (Fol. 52r Adbreviatio Gestorum Regnum \\Franciae).}
    \label{fig:escobar22}
  \end{figure}

 Image released in the public domain by the Leiden University Libraries 

  \hfill
  \begin{figure}
    \includegraphics[width=\textwidth]{media/escobar23.png}
    \caption{Figure \ref{fig:escobar1}: (Fol. 60r, Historia regum Britanniae).\\ \\}
    \label{fig:escobar23}
\end{figure}

 Image released in the public domain by the Leiden University Libraries 


This is all the more true when considering the relative care placed in
the initials at the start of books throughout this copy of Robert of
Torigni. It is of further interest when considering that this copy
consumed considerable resources and can thus be seen as a relatively
high-quality product. At Bec, the willingness of the community to
support the production of a manuscript like BPL 20 is shown by the
resources invested in its creation: parchment, ink, money for the
acquisition of books that would serve as exemplars, and the human
resources involved in both the writing and copying of the historical
narratives included in it.\footnote{For more on monastic book production
  see: \cite{thomson_monastic_2008}; \cite{webber_monastic_2008}.} It shows that Robert's
endeavours were valued enough so that a significant investment was made
in them. Robert's holograph was not luxurious, but it was considered an
important product to make, hold, and maybe even lend as an exemplar of a
valued work. This brings the enquiry back to the issue of whether BPL 20
was indeed treated in this way.

The answer is that early copies were produced from BPL 20 for monastic
communities in England. One is a manuscript produced at Henry I's new
foundation of Reading, now Cambridge, Gonville and Caius College, MS
177/210 and London, British Library Cotton Ms Vitellius A VIII (fols.
5---100). Another is associated with Colchester, and is now London,
British Library Cotton Ms Nero D. VIII. The Reading copy is late twelfth
century. All the relevant folios {[}fos. 5--100{]} of the Cotton portion,
constituting the later part of the original volume, are badly damaged by
fire. It was both written and corrected at Reading Abbey \citep[154]{coates_english_1999}. There is also an ex-libris in the flyleaf of the Cambridge
manuscript, associating it with the monastery at Reading. The original
volume can be identified with an entry in the late twelfth-century
library Catalogue of Reading, where it appears as \emph{Vita Karoli}
\emph{et Alexandri, et gesta Normannorum, et alia}.\footnote{See:
  London, British Library, \cite[fol. 9v]{noauthor_reading_nodate}} The Cambridge
manuscript consists of 59 folios, which include the \emph{Vita Karoli
Magni} (pp. 1--30), \emph{Vita Alexandri Magni} (pp. 30--67),
\emph{Epistola Alexandri Magni ad Aristotelem} (pp. 67--89), and the
\emph{Abbreviatio gestorum regum Franciae} (pp. 89--115). These are
similar to the contents in the first part of BPL 20. A possible third
section containing the \emph{Historia Britonum} is missing. Robert's
version of the \emph{Gesta Normannorum Ducum} begins on fol. 5r of the
Cotton Ms Vitellius A VIII, it is rubricated but only partially legible.
This is one of only two copies of this version which include the
additions dated to c. 1139--59 entered into BPL 20. This evidence
strongly suggests that BPL 20 itself, or a ``loan copy'' produced from it,
was chosen as an exemplar by Reading and that Robert's authority as a
historian was important in this choice. If that argument is accepted
then it follows that the cooperative work on BPL 20 did not contradict
its status as the ``original'' copy of Robert's version.

London, British Library Cotton Ms Nero D VIII is a composite manuscript
containing a variety of texts that can be grouped into three main parts:
fos. 3--175 include a collection of historical works, including the
\emph{Historia Regum Britanniae} and \emph{the Gesta Normannorum Ducum},
and dates to the last quarter of the twelfth century; fos. 176--344
include historical and scientific texts, including the
\emph{Polychronicon}, dating from the last quarter of the fourteenth
century, and fos. 345--347 include a history of the foundation of
Colchester Abbey dating to the sixteenth century. Because of this last
section, a possible link between the volume and Colchester has been
suggested (\cite[xcvi]{van_houts_gesta_1992}; \cite[149--152]{crick_historia_1989}). The first part of
the manuscript, fos. 1--175 is comprised of two different units, written
in an English hand dating from the end of the twelfth century, which
exhibit separate sets of quire signatures. Despite this, they can be
seen as products of the same scriptorium, most likely bound together
early and perhaps meant to be a single whole, as they both contain texts
related to BPL 20.\footnote{Dumville has speculated on whether the
  Cotton manuscript might be a copy of the Reading Abbey manuscript 
  \citep[4]{dumville_early_1985}.} The first part of the manuscript includes
Geoffrey's \emph{Historia Regum Britanniae} (fos. 3r--63r), followed by
Gildas' \emph{Historia Britonum} (fos. 63r--71r), as in BPL 20. Dudo
of Saint-Quentin's \emph{De moribus} (fos. 72r--135v) is followed by the
\emph{Gesta Normannorum Ducum} in Redaction F (fos. 147r--159v), the
\emph{Vita Alexandri Magni} (fos. 160r--169r) and the \emph{Epistola
Alexandri Magni ad Aristotelem} (fos. 169r--174v). The volume ends with a
list of works by Bede (fos. 174v--175r). This portion of Cotton Nero D
VIII is in effect a selection from BPL 20, to which the list of Bedan
texts has been added.

However, the version of Redaction F in this codex is incomplete,
consisting only of books VII, VIII and the so-called \emph{Additamenta.}
The earlier part of the narrative follows Dudo's history and Redaction A
of the \emph{Gesta Normannorum} Ducum (fos. 135v--146v), beginning in
book V and including the short epilogue. In this instance, Redaction A
is part of Dudo's narrative. A rubric closing Dudo's text and running
through folios 146v and 147r of the manuscript states: ``Explicit
historia Dudonis decani sancti Quintini de ducibus Northmannie que uero
sequuntur addita sunt de sexto et septimo libro historiarum Willelmi
Gemmeticensis monachi'' [Here ends the history of Dudo, dean of Saint-Quentin, about the dukes of Normandy. What follows is added from the sixth and seventh books of the histories by the monk William of Jumièges]\footnote{See: \cite{noauthor_digitised_nodate}} This is followed by extracts from
Redaction F, associated here to William of Jumièges, and without any
further separation. It is clear that whoever was running the copying
project was keenly aware of the fact that William's work was
complementing Dudo's version of the narrative. Unlike the Reading Abbey
volume, which might have served as an exemplar to this Cotton manuscript
if it was not BPL 20 itself, the copyists here preferred to collate the
work of the \emph{Gesta Normannorum Ducum}, including Dudo's text,
Redaction A and Redaction F. Here, Robert of Torigni's authorial
interventions in BPL 20 do not seem to have conferred enough authority
to his redaction of the text to grant an inclusion of his whole work in
the manuscript; the copyists of the Colchester MS preferring earlier
redactions to Robert's own. As a holograph, BPL 20 seems to have
influenced the choice of content in the Reading Abbey volume, but this
does not seem to have been the case with this Colchester codex. 
Even though it is unlikely that its status as an autograph copy was unknown to its makers, 
it is not entirely impossible.

The writing of the \emph{Gesta Normannorum Ducum} is revealed by this
analysis as a double collective enterprise involving the efforts,
firstly, of various authors through time. From when Dudo of
Saint-Quentin first began to write his \emph{De moribus}, to when Robert
of Torigni finished the latest touches to his corrections, 150 years had
passed. It is possible to establish with a moderate degree of certainty
that at least four of the authors were monks, except for Dudo of
Saint-Quentin, who had been a canon and chaplain to dukes Richard I and
Richard II. As authors, those who engaged in writing the \emph{Gesta
Normannorum Ducum} depended upon their communities and varying degrees
of lay patronage. Some were anonymous, but in the case of William of
Jumièges, Orderic Vitalis, and Robert of Torigni, it is possible to trace
a joint exercise that exceeded the boundaries of time. This mode of
composition is vastly dissimilar to most modern equivalents, despite
being reasonably typical for medieval writers. It challenges modern
notions of authorship and thus impacts our understanding of what
constitutes a holograph in the medieval period. Does it, therefore,
impact the value of the concept? I would argue that it does not, quite
the contrary, for it illustrates the complex process by which medieval
texts both came to be and achieved authority.

Something quite similar happens when considering the other main factor
evinced by the production of BPL 20, that of its collective scribal
production \emph{in situ}, for it is also not unusual for medieval texts
to be written by a team of scribes that would work in aid of authors for
the composition of their texts, as was the case with Robert of Torigni.
However, neither is it universal, as can be shown by Orderic Vitalis'
authorial copy of this same narrative \parencite[ciii]{van_houts_gesta_1992}.
In the case of monastic authors, this second mode of collective creation
required the resources of a well-doted scriptorium and the acquiescence
of monastic authorities. Robert of Torigni counted upon a team of
scribes who prepared a working copy of the \emph{Gesta Normannorum
Ducum} into which he could insert his interpolations and later
corrections. However, as shown by the Colchester copy of the text,
medieval scribes and their directors remained free to make choices as to
how a text was to be assembled and presented. In the case of Robert's
text, the two English manuscripts discussed both display knowledge of
his authorial version as presented in BPL 20, but both made different
choices. However, this does not in itself challenge the status of BPL 20
as a potential holograph. If such is to be considered a manuscript
handwritten by the person named as its author, as described by the OED,
then the involvement of a team of scribes would contradict this basic
definition of what constitutes a holograph. But I would argue that this
would not be the case for medieval works. Once again, BPL 20 illustrates
the complexities involved in the composition and copying of medieval
texts, highlighting the value of not taking modern conceptions and
definitions at face value when working with narratives written in the
remote past. This type of ``serial'' authorship and team-writing of texts,
which now tend to be discussed as 'the work' of individual, named
authors, challenges not only authorship concepts, but also the status of
a holograph and autograph manuscript in the medieval past. This paper
has presented evidence that, even if a monastic
compiler/editor/continuator started with a copy of an older work but
then revised, interpolated and updated it, as well as having the result
copied by a team of scribes, both the text and the manuscript could
meaningfully be considered as his work. If that argument is accepted
then the concept of the authorial holograph remains important but needs
to be expanded.



\section*{Acknowledgements}
I am grateful to the Universidad  Nacional de Colombia, Sede Medellin, where I am an associate professor in Medieval History, for the generous grant which allowed me to conduct this research.

\begin{flushleft}
    % use smallcaps for author names
    \renewcommand*{\mkbibnamefamily}[1]{\textsc{#1}}
    \renewcommand*{\mkbibnamegiven}[1]{\textsc{#1}} 
\printbibliography
\end{flushleft}

\end{document}