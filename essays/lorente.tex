\documentclass{article}
%%%% CLASS OPTIONS 

\KOMAoptions{
    fontsize=10pt,              % set default font size
    DIV=calc,
    titlepage=false,
    paper=150mm:220mm,
    twoside=true, 
    twocolumn=false,
    toc=chapterentryfill,       % for dots: chapterentrydotfill
    parskip=false,              % space between paragraphs. "full" gives more space; "false" uses indentation instead
    headings=small,
    bibliography=leveldown,     % turns the Bibliography into a \section rather than a \chapter (so it appears on the same page)
}

%%%% PAGE SIZE

\usepackage[
    top=23mm,
    left=20mm,
    height=173mm,
    width=109mm,
    ]{geometry}

\setlength{\marginparwidth}{1.25cm} % sets up acceptable margin for \todonotes package (see preamble/packages.tex).

%%%% PACKAGES

\usepackage[dvipsnames]{xcolor}
\usepackage[unicode]{hyperref}  % hyperlinks
\usepackage{booktabs}           % professional-quality tables
\usepackage{nicefrac}           % compact symbols for 1/2, etc.
%\usepackage{microtype}          % microtypography
\usepackage{lipsum}             % lorem ipsum at the ready
\usepackage{graphicx}           % for figures
\usepackage{footmisc}           % makes symbol footnotes possible
\usepackage{ragged2e}
\usepackage{changepage}         % detect odd/even pages
\usepackage{array}
\usepackage{float}              % get figures etc. to stay where they are with [H]
\usepackage{subfigure}          % \subfigures witin a \begin{figure}
\usepackage{longtable}          % allows for tables that stretch over multiple pages
\setlength{\marginparwidth}{2cm}
\usepackage[textsize=footnotesize]{todonotes} % enables \todo's for editors
\usepackage{etoolbox}           % supplies commands like \AtBeginEnvironment and \atEndEnvironment
\usepackage{ifdraft}            % switches on proofreading options in the draft mode
\usepackage{rotating}           % provides sidewaysfigure environment
\usepackage{media9}             % allows for video in the pdf
\usepackage{xurl}               % allows URLs to (line)break ANYWHERE

%%%% ENCODING

\usepackage[full]{textcomp}                   % allows \textrightarrow etc.

% LANGUAGES

\usepackage{polyglossia}
\setmainlanguage{english} % Continue using english for rest of the document

% If necessary, the following lets you use \texthindi. Note, however, that BibLaTeX does not support it and will report a 'warning'.
 \setotherlanguages{hindi} 
 \newfontfamily\hindifont{Noto Sans Devanagari}[Script=Devanagari]

% biblatex
\usepackage[
    authordate,
    backend=biber,
    natbib=true,
    maxcitenames=2,
    ]{biblatex-chicago}
\usepackage{csquotes}

% special characters  
\usepackage{textalpha}                  % allows for greek characters in text 

%%%% FONTS

% Palatino font options
\usepackage{unicode-math}
\setmainfont{TeX Gyre Pagella}
\let\circ\undefined
\let\diamond\undefined
\let\bullet\undefined
\let\emptyset\undefined
\let\owns\undefined
\setmathfont{TeX Gyre Pagella Math}
\let\ocirc\undefined
\let\widecheck\undefined

\addtokomafont{disposition}{\rmfamily}  % Palatino for titles etc.
\setkomafont{descriptionlabel}{         % font for description lists    
\usekomafont{captionlabel}\bfseries     % Palatino bold
}
\setkomafont{caption}{\footnotesize}    % smaller font size for captions


\usepackage{mathabx}                    % allows for nicer looking \cup, \curvearrowbotright, etc. !!IMPORTANT!! These are math symbols and should be surrounded by $dollar signs$
\usepackage[normalem]{ulem}                       % allows for strikethrough with \sout etc.
\usepackage{anyfontsize}                          % fixes font scaling issue

%%%% ToC

% No (sub)sections in TOC
\setcounter{tocdepth}{0}                

% Redefines chapter title formatting
\makeatletter                               
\def\@makechapterhead#1{
  \vspace*{50\p@}%
  {\parindent \z@ \normalfont
    \interlinepenalty\@M
    \Large\raggedright #1\par\nobreak%
    \vskip 40\p@%
  }}
\makeatother
% a bit more space between titles and page numbers in TOC

\makeatletter   
\renewcommand\@pnumwidth{2.5em} 
\makeatother

%%%% CONTRIBUTOR

% Title and Author of individual contributions
\makeatletter
% paper/review author = contributor
\newcommand\contributor[1]{\renewcommand\@contributor{#1}}
\newcommand\@contributor{}
\newcommand\thecontributor{\@contributor} 
% paper/review title = contribution
\newcommand\contribution[1]{\renewcommand\@contribution{#1}}
\newcommand\@contribution{}
\newcommand\thecontribution{\@contribution}
% short contributor for running header
\newcommand\shortcontributor[1]{\renewcommand\@shortcontributor{#1}}
\newcommand\@shortcontributor{}
\newcommand\theshortcontributor{\@shortcontributor} 
% short title for running header
\newcommand\shortcontribution[1]{\renewcommand\@shortcontribution{#1}}
\newcommand\@shortcontribution{}
\newcommand\theshortcontribution{\@shortcontribution}
\makeatother

%%%% COPYRIGHT

% choose copyright license
\usepackage[               
    type={CC},
    modifier={by},
    version={4.0},
]{doclicense}

% define \copyrightstatement for ease of use
\newcommand{\copyrightstatement}{
         \doclicenseIcon \ \theyear. 
         \doclicenseLongText            % includes a link
}

%%%% ENVIRONMENTS
% Environments
\AtBeginEnvironment{quote}{\footnotesize\vskip 1em}
\AtEndEnvironment{quote}{\vskip 1em}

\setkomafont{caption}{\footnotesize}

% Preface
\newenvironment{preface}{
    \newrefsection
    \phantomsection
    \cleardoublepage
    \addcontentsline{toc}{part}{\thecontribution}
    % enable running title
    \pagestyle{preface}
    % \chapter*{Editors' Preface}    
    % reset the section counter for each paper
    \setcounter{section}{0}  
    % no running title on first page, page number center bottom instead
    \thispagestyle{chaptertitlepage}
}{}
\AtEndEnvironment{preface}{%
    % safeguard section numbering
    \renewcommand{\thesubsection}{\thesection.\arabic{subsection}}  
    %last page running header fix
    \protect\thispagestyle{preface}
}
% Essays
\newenvironment{paper}{
    \newrefsection
    \phantomsection
    % start every new paper on an uneven page 
    \cleardoublepage
    % enable running title
    \pagestyle{fancy}
    % change section numbering FROM [\chapter].[\section].[\subsection] TO [\section].[\subsection] ETC.
    \renewcommand{\thesection}{\arabic{section}}
    % mark chapter % add author + title to the TOC
    \chapter[\normalfont\textbf{\emph{\thecontributor}}: \thecontribution]{\vspace{-4em}\Large\normalfont\thecontribution\linebreak\normalsize\begin{flushright}\emph{\thecontributor}\end{flushright}}    
    % reset the section counter for each paper
    \setcounter{section}{0}  
    % reset the figure counter for each paper
    \renewcommand\thefigure{\arabic{figure}}    
    % reset the table counter for each paper
    \renewcommand\thetable{\arabic{table}} 
    % no running title on first page, page number center bottom instead, include copyright statement
    \thispagestyle{contributiontitlepage}
    % formatting for the bibliography

}{}
\AtBeginEnvironment{paper}{
    % keeps running title from the first page:
    \renewcommand*{\pagemark}{}%                            
}
\AtEndEnvironment{paper}{
    % safeguard section numbering
    \renewcommand{\thesubsection}{\thesection.\arabic{subsection}}  
    % last page running header fix
    \protect\thispagestyle{fancy}%                              
}
% Reviews
\newenvironment{review}{
    \newrefsection
    \phantomsection
    % start every new paper on an uneven page 
    \cleardoublepage
    % enable running title
    \pagestyle{reviews}
    % change section numbering FROM [\chapter].[\section].[\subsection] TO [\section].[\subsection] ETC.
    \renewcommand{\thesection}{\arabic{section}} 
    % mark chapter % add author + title to the TOC
    \chapter[\normalfont\textbf{\emph{\thecontributor}}: \thecontribution]{}    % reset the section counter for each paper
    \setcounter{section}{0}  
    % no running title on first page, page number center bottom instead, include copyright statement
    \thispagestyle{contributiontitlepage}
    % formatting for the bibliography
}{}
\AtBeginEnvironment{review}{
% keeps running title from the first page
    \renewcommand*{\pagemark}{}%                                   
}
\AtEndEnvironment{review}{
    % author name(s)
    \begin{flushright}\emph{\thecontributor}\end{flushright}
    % safeguard section numbering
    \renewcommand{\thesubsection}{\thesection.\arabic{subsection}} 
    % last page running header fix
    \protect\thispagestyle{reviews}                           
}

% Abstract
\newenvironment{abstract}{% 
\setlength{\parindent}{0pt} \begin{adjustwidth}{2em}{2em}\footnotesize\emph{\abstractname}: }{%
\vskip 1em\end{adjustwidth}
}{}

% Keywords
\newenvironment{keywords}{
\setlength{\parindent}{0pt} \begin{adjustwidth}{2em}{2em}\footnotesize\emph{Keywords}: }{%
\vskip 1em\end{adjustwidth}
}{}

% Review Abstract
\newenvironment{reviewed}{% 
\setlength{\parindent}{0pt}
    \begin{adjustwidth}{2em}{2em}\footnotesize}{%
\vskip 1em\end{adjustwidth}
}{}

% Motto
\newenvironment{motto}{% 
\setlength{\parindent}{0pt} \small\raggedleft}{%
\vskip 2em
}{}

% Example
\newcounter{example}[chapter]
\newenvironment{example}[1][]{\refstepcounter{example}\begin{quote} \rmfamily}{\begin{flushright}(Example~\theexample)\end{flushright}\end{quote}}

%%%% SECTIONOPTIONS

% command for centering section headings
\newcommand{\centerheading}[1]{   
    \hspace*{\fill}#1\hspace*{\fill}
}

% Remove "Part #." from \part titles
% KOMA default: \newcommand*{\partformat}{\partname~\thepart\autodot}
\renewcommand*{\partformat}{} 

% No dots after figure or table numbers
\renewcommand*{\figureformat}{\figurename~\thefigure}
\renewcommand*{\tableformat}{\tablename~\thetable}

% paragraph handling
\setparsizes%
    {1em}% indent
    {0pt}% maximum space between paragraphs
    {0pt plus 1fil}% last line not justified
    

% In the "Authors" section, author names are put in the \paragraph{} headings. To reduce the space after these  headings, the default {-1em} has been changed to {-.4em} below.
\makeatletter
\renewcommand\paragraph{\@startsection {paragraph}{4}{\z@ }{3.25ex \@plus 1ex \@minus .2ex}{-.4em}{\normalfont \normalsize \bfseries }
}
\makeatother

% add the following (uncommented) in environments where you want to count paragraph numbers in the margin
%    \renewcommand*{\paragraphformat}{%
%    \makebox[-4pt][r]{\footnotesize\theparagraph\autodot\enskip}
%    }
%    \renewcommand{\theparagraph}{\arabic{paragraph}}
%    \setcounter{paragraph}{0}
%    \setcounter{secnumdepth}{4}
    
%%%% HEADERFOOTER

% running title
\RequirePackage{fancyhdr}
% cuts off running titles that are too long
%\RequirePackage{truncate}
% makes header as wide as geometry (SET SAME AS \TEXTWIDTH!)
\setlength{\headwidth}{109mm} 
% LO = Left Odd
\fancyhead[LO]{\small\emph{\theshortcontributor} \hspace*{.5em} \theshortcontribution} 
% RE = Right Even
\fancyhead[RE]{\scshape{\small\theissue}}
% LE = Left Even
\fancyhead[LE]{\small\thepage}            
% RE = Right Odd
\fancyhead[RO]{\small\thepage}    
\fancyfoot{}
% no line under running title; cannot be \@z but needs to be 0pt
\renewcommand{\headrulewidth}{0 pt} 

% special style for authors pages
\fancypagestyle{authors}{
    \fancyhead[LO]{\small\textit{Authors}} 
    \fancyhead[LE]{\small\thepage}            
    \fancyhead[RE]{\scshape{\small\theissue}}
    \fancyhead[RO]{\small\thepage}            
    \fancyfoot{}
}

% special style for book reviews
\fancypagestyle{reviews}{
    \fancyhead[LO]{\small\textit{Book Reviews}} 
    \fancyhead[LE]{\small\thepage}            
    \fancyhead[RE]{\scshape{\small\theissue}}
    \fancyhead[RO]{\small\thepage}            
    \fancyfoot{}
}

% special style for Editors' preface.
\fancypagestyle{preface}{
    \fancyhead[LO]{\small\textit{\theshortcontribution}} 
    \fancyhead[LE]{\small\thepage}            
    \fancyhead[RE]{\scshape{\small\theissue}}
    \fancyhead[RO]{\small\thepage}            
    \fancyfoot{}
}
% special style for first pages of contributions etc.
% DOES include copyright statement
\fancypagestyle{contributiontitlepage}{
    \fancyhead[C]{\scriptsize\centering\copyrightstatement}
    \fancyhead[L,R]{}
    \fancyfoot[CE,CO]{\small\thepage}
}
% special style for first pages of other \chapters.
% DOES NOT include copyright statement
\fancypagestyle{chaptertitlepage}{
    \fancyhead[C,L,R]{}
    \fancyfoot[CE,CO]{\small\thepage}
}
% no page numbers on \part pages 
\renewcommand*{\partpagestyle}{empty}

%%%% FOOTNOTEFORMAT

% footnotes
\renewcommand{\footnoterule}{%
    \kern .5em  % call this kerna
    \hrule height 0.4pt width .2\columnwidth    % the .2 value made the footnote ruler (horizontal line) smaller (was at .4)
    \kern .5em % call this kernb
}
\usepackage{footmisc}               
\renewcommand{\footnotelayout}{
    \hspace{1.5em}    % space between footnote mark and footnote text
}    
\newcommand{\mytodo}[1]{\textcolor{red}{#1}}

%%%% CODESNIPPETS

% colours for code notations
\usepackage{listings}       
	\renewcommand\lstlistingname{Quelltext} 
	\lstset{                    % basic formatting (bash etc.)
	       basicstyle=\ttfamily,
 	       showstringspaces=false,
	       commentstyle=\color{BrickRed},
	       keywordstyle=\color{RoyalBlue}
	}
	\lstdefinelanguage{XML}{     % specific XML formatting overrides
		  basicstyle=\ttfamily,
		  morestring=[s]{"}{"},
		  morecomment=[s]{?}{?},
		  morecomment=[s]{!--}{--},
		  commentstyle=\color{OliveGreen},
		  moredelim=[s][\color{Black}]{>}{<},
		  moredelim=[s][\color{RawSienna}]{\ }{=},
		  stringstyle=\color{RoyalBlue},
 		  identifierstyle=\color{Plum}
	}
    % HOW TO USE? BASH EXAMPLE
    %   \begin{lstlisting}[language=bash]
    %   #some comment
    %   cd Documents
    %   \end{lstlisting}
\author{Juan Lorente Sánchez}
\title{Dodoens' Herbal in Glasgow University Library MS Ferguson 7 (ff.
23r--48v; 59r): A Physical Description}

\begin{document}
\maketitle

\begin{abstract}
% write your abstract here
This essay analyses the physical characteristics of an unexplored handwritten version of Rembert Dodoens' \emph{A Niewe Herball or Historie of Plants} kept in Glasgow University Library MS Ferguson 7 (ff. 23r--48v; 59r), which also stands, to my knowledge, as the only manuscript copy encompassing contents included in the English printed edition. A number of its codicological and palaeographic features are described, including, among others, the material, ink, foliation, scripts, abbreviations and marginalia. Like most papers of this class, the essay aims to suggest its most possible date of writing, permitting the research community to use it as a reliable source of evidence for both synchronic and diachronic studies inside an academic context. From a methodological viewpoint, the description is informed by three sources: 1) the computerised photographs of the manuscript; 2) certain information from the volume's electronic catalogue entry; and 3) some data kindly offered via email by various staff members of the Archives and Special Collections Section of Glasgow University. All in all, the examination of the text eventually leads to propose the early seventeenth century –– presumably around 1600--1630 –– as date of writing.
\end{abstract}

%%%%%%%%%%%%%%%%%%%%%%%%%%%%
%% YOUR ESSAY STARTS HERE %%
%%%%%%%%%%%%%%%%%%%%%%%%%%%%

% remove asterisk (*) if you want to number your sections
% add a title for your section in between the {curly brackets} if you need one
\section{Introduction} 
Extending back into at least the epochs of Theophrastus, Dioscorides and
Galen, among others (see \cite[xxxvii-xliii]{hunt_plant_1989}; also \cite{moreno_olalla_lelamour_2018}), a herbal may be defined as a manual on medicinal plants and herbs
generally designed for an experienced audience of doctors and
apothecaries, written to ``enable them to know which plants to use for
medical purposes, and how to identify them in the field'' (\cite[24]{elliott_world_2011}; also \cite{britannica_herbal_2012}). Even though the study of plants has
been traditionally regarded as an essential part of natural
philosophy,\footnote{According to Arber, the philosophical approach to
  botany is said to have owed its commencement in the western world ``to
  the unparalleled mental activity of the finest period of Greek
  culture''. For instance, Aristotle's writings on different
  philosophical topics ``include many references to plants, from which
  we can gather some of his general ideas'' \citep[1--2]{arber_herbals_1953}.} different
varieties of herbs have also been used as healing agents throughout history and, as such, it has been necessary to analyse them in detail so
as to discriminate the types employed for several utilitarian purposes
(\cite[1--6]{arber_herbals_1953}; also \cite[19--21]{neville_early_2022}).

Most likely boosted by the significance of the printing press in Europe,
the utilitarian use of herbals reached its height during the sixteenth
and the seventeenth centuries, when most botanists aimed at recovering
the authentic descriptive essence of the classical works, whilst
releasing them from the stored accumulation of periods of mythology. As
a successor of Leonard Fuchs, founder of the initial standards for plant
description which most later authors attempted to meet, Rembert Dodoens
is one of the most renowned herbalists from the perspective of
descriptive botany as, apart from writing a herbal as conventionally
assumed, he contributed to lay the foundations of the modern regional
flora due to the terrific popularity of his work \citep[24--26]{elliott_world_2011}.

To compose and illustrate his famous piece, Dodoens acquired the use of
the woodblocks employed in the octavo edition of Fuchs' \emph{De
historia stirpium}, to which he included some new engravings. Under the
title \emph{Cruydtboeck}, the herbal was originally published in Flemish
and printed in Antwerp in 1554 by Jean van der Loë. The text had a
tremendous impact on many medicinal botanists of the period insomuch
that, only three years after its publication, a French translation
effectuated by Charles de l'Écluse (commonly known as Clusius) came out
with the abbreviated title of \emph{Histoire des Plantes}, and six years
later another Flemish edition, for which Dodoens himself took the
opportunity to provide some additional information included in the
French issue, was released again by van der Loë (\cite[82]{arber_herbals_1953}; also \cite{barlow_old_1913}; \cite{elliott_world_2011}).

In English, the \emph{Cruydtboeck} came to light in 1578, twenty-four
years after the production of the original text, by means of a
translation of de l'Écluse's French edition. Entitled succinctly as
\emph{A Niewe Herball or Historie of Plants}, this translation was
carried out by the herbalist and antiquary, Henry Lyte, ``the thirteenth
lineal descendant of that name and family –– a family which still
flourishes, and can count among its modern representatives the Rev. H.
F. Lyte, author of `Abide with me', and Sir Henry Maxwell-Lyte, the
historian'' (\cite[125]{arber_herbals_1953}; also \cite{boulger_lyte_2004}). Much of
Dodoens' remembrance in the history of English botany is owed to Lyte,
since his translation was well-accepted in the country. It was deemed a
better-quality piece in terms of arrangement and popularity compared to the works
of William Turner and Matthias de L'Obel, among other eminent botanists
\citep[138]{barlow_old_1913}.

Although Lyte's translation has at times incorrectly been assumed to having been published in London
by Gerard Dewes (presumably on account of the inscription ``at London, by
me Gerard Dewes'' that appears on the title page), it was in fact
first printed in Antwerp. This was done so that he could make use of the woodcut
figures which were in possession of the Antwerp printer and, more
importantly, so that Dodoens could keep in touch with the edition at
first-hand. The imprint was
performed by Henry Loë, whilst Dewes' involvement with the text was
constrained to selling, a task he accomplished at London in Powels
Churchyard \citep[139--140]{barlow_old_1913}. Given the outstanding popularity of this translation, three additional
English editions without illustrations were thereafter issued in London
in 1586, 1595 and 1619 by the printers Ninian Newton, Edmund Bollifant,
and Edward Griffin, respectively. At the same time, other printed
versions are thought to have been published in 1589, 1600, and 1678,
albeit no vestige of these have been hitherto found \citep[141]{barlow_old_1913}.

Despite its high popularity in printing, the tradition of the piece is
virtually absent in manuscripts since, to my knowledge, Glasgow
University Library MS Ferguson 7 (ff. 23r--48v; 59r) (hereafter FER7)
stands as the only handwritten copy encompassing descriptions of a
selection of the plants included in the English printed edition. This
considered, the present contribution aims to analyse the physical
characteristics of this unexplored witness, paying attention to a number
of codicological and palaeographic elements. Similar to most
papers of this class, the essay has been conceived in order to
suggest its most possible date of writing and, if accomplished, a
scholar interested in this text may find it a reliable piece of data for
both synchronic and diachronic analyses inside a research context. In
addition to this, the article also attempts to present a minute
account of all those constituents under review. Where possible, the specific elements are illustrated with a figure,\footnote{The images reproduced in this essay are a part of MS Ferguson 7 (front cover, back cover and ff. 23r--48v; 59r), held by the University of Glasgow Archives and Special Collections. The University of Glasgow Library, which owns the rights to the facsimile images of this manuscript, has kindly granted the author the permission to include them here.} so
that they can eventually be used as input for various types of inquiry
within manuscript studies, or any other interrelated academic field
such as historical linguistics.

In terms of structure, the essay has been organised into three different
sections. Section 2 introduces the object of study,
providing a succinct overview of its origin coupled with a description
of how the scribe amasses and personalises it, both in terms of the
amount of the contents and regarding the nature of these. Section 3 focuses on the codicological description of the volume, addressing
some of its external characteristics. Section 4 analyses the
text from a palaeographic viewpoint, inspecting
elements that are normally tackled in the majority of manuals and
guidebooks on the field (i.e., scripts, abbreviations, numerals, and
marginalia). Finally, section 5 offers the conclusions of the
entire examination, followed first by some final remarks on how the
proofs may be used as source of evidence for various forms of investigation, and secondly by outlining the philological research
of the present work that I intend to develop in the future.

\section{Dodoens' \emph{A Niewe Herball or Historie of Plants} in FER7
(ff. 23r--48v; 59r)}

With reference number GB 247 MS Ferguson 7, FER7 forms part of the
collection of manuscripts from the private library of Professor John
Ferguson (1838, Alloa, Scotland –– 1916, Glasgow, Scotland) \citep[1--3]{weston_ferguson_2004}, a Scottish chemist and bibliographer author of the
well-known \emph{Bibliotheca Chemica}, a catalogue of the
alchemical, chemical, and pharmaceutical books in the collection of the
late James Young \citep{ferguson_bibliotheca_1906}. During his lifetime, he collected a large personal
library of around 18,000 volumes. The principal part of this library was purchased
by the Special Collections Department of the University of Glasgow in
1921. The collection contains 118 incunabula and 317 manuscripts of
alchemical interest, many of them including copies of works by authors
such as Constantinus Africanus, Albert Magnus or Roger Bacon (\cite{noauthor_biography_nodate}; \cite[3]{weston_ferguson_2004}; \cite{noauthor_ferguson_nodate}).

As far as the contents of the herbal are concerned, all plant accounts
are taken from Lyte's original 1578 translation, as suggested in the
title of the treatise (`Taken out of D\emph{octor}. Rembert dodoens
phisitian to \th\emph{e} Emp\emph{erour}. his herball made an\emph{n}o
d\emph{om}in\emph{i}. \uline{1578.}', f. 23r). Labelled as \emph{Niewe
Herball} in the online catalogue of Glasgow University
Library,\footnote{According to the online catalogue
  (see \url{http://web.archive.org/web/20241006114320/https://www.gla.ac.uk/collections/\#/details?irn=265620&catType=C&gdcEvent=hierarchy_item_view}), FER7 is
  only composed of the manuscript version of Dodoens' herbal (ff.
  23r--64v) and handwritten passages of a well-known printed collection
  of medical recipes entitled \emph{The Secrets of Alexis of Piedmont}
  (ff. 1r--22v), which is supposed to have been authored by Girolamo
  Ruscelli in 1555 \citep[see][]{bela_authorship_2016} and first translated into English in
  1558 by William Ward. A reading of the full manuscript, however,
  reveals that the material comprised in some of the folios does not
  correspond to the texts contained in the original printed pieces. For
  instance, the contents held in ff. 48v--58v include a short-handwritten
  version of William Ram's \emph{Rams Little Dodeon}, an unconventional
  sort of remedy-book issued in 1606 which was purposed to be an
  abridgment of the printed version of \emph{Niewe Herball} \citep[see][141]{barlow_old_1913}.} the manuscript copy displays a small segment of the
printed pieces insofar as, out of the 571 plants with their diverse
subspecies described in the original, only 198 are recorded in FER7,
that is, approximately 35\% of the total held in the English printed
edition. Together with the number of omitted species, the scribe also
ignores all the illustrations, figures and the table of contents
provided in Lyte's translation, as well as other information beyond the
scope of medicinal botany. This material refers to i) a dedication to
Queen Elizabeth I; ii) an address to ``the friendly and indifferent
Reader''; iii) some commendatory verses addressed to Lyte by W. B.,
Thomas Newton, Wm. Clowes and John Harding; iv) a portrait of Dodoens;
v) the preface and epistle to the reader, and the appendix; and vi) the
colophon \citealp[see][139]{barlow_old_1913}.

The printed translation, in a similar fashion to the primary Flemish
edition, is organised into six books, each of them covering a specific
topic of explanation. The scribe of \emph{Niewe Herball} provides
descriptions from all of these, though he seems to be more interested in
those of the first book, as more than a quarter of the total
incorporated in the manuscript are taken from it. Table 1 illustrates
the topic of every part of the printed text and the number of species
described in them, along with the figure included in the handwritten
copy.

\textbf{Table 1: Number of plant descriptions in the six books of the printed edition and in the manuscript}

\begin{figure}
  \centering
  \includegraphics[width=\textwidth]{media/lorente0.png}
  \label{fig:lorente0}
\end{figure}


 Image reproduced with permission from the University of Glasgow Library 


 
In addition to the number of medicinal herbs selected, the scribe
customises the text of the copied contents idiosyncratically in order to
suit his own needs (see \cite{de_la_cruz_cabanillas_secrets_2020}; also \cite{barbierato_writing_2011}). The printed book systematically offers detailed descriptions of
the plants in different paragraphs, including miscellaneous information such as their physical characteristics, the sub-kinds existing
within the same species, their names in diverse languages, their nature,
the place where they may be found, the time when they grow, and their
medical virtues (see \cite[137--138]{dodoens_niewe_1578}), whereas in the manuscript
the scribe summarises this information and systematically shapes it when
copying, as shown in (\ref{lorente:numlist:1}). Therefore, depending on the plant described,
the scribe just takes those parts of the original in which he is
interested and adjusts them in short paragraphs.


\begin{enumerate}
\def\labelenumi{(\arabic{enumi})}
\item\label{lorente:numlist:1}
  ``Burnett or pimpinell. Pimpinella maior wild burnet. side-\\
  ritis altera. pimpinella. minor, garden burnett. bipennula. pampi-\\
  nula. sanguisorba. sic\emph{cus}. 3º. g\emph{radu}. et fri\emph{gidus}.
2º. \emph{and} astringente.'' 
\end{enumerate}
\begin{flushright}
    (f. 31r)
\end{flushright}


The following two sections, as mentioned above, discuss the physical
description of the handwritten text, taking note of a
sequence of both codicological and palaeographic attributes. From a
methodological standpoint, the physical description is informed by three sources: i) the computerised
photographs of FER7, currently constitutive elements of the \emph{Málaga
Corpus of Early Modern English Scientific Prose} \citep{calle-martin_malaga_2016}, from which the images in the figures have
been taken and adapted;\footnote{\label{lorente:fn:4}The photographs were uploaded to the
  official website of the \emph{Málaga Corpus} after having paid the
  corresponding rate and obtained the subsequent authorisation for their
  online publication. The project encompasses on the whole 20 Early
  Modern English manuscripts of medical kind from five different
  collections, including the Hunter, Ferguson and General Collections at
  Glasgow University Library, the Wellcome Collection at the Wellcome
  Library in London, and the Rylands Collection at the University of
  Manchester Library. Methodologically speaking, the compilation process
  is divided into four sequential stages: i) the manuscript selection
  and digitisation in high resolution images by the institution where
  the witnesses are held; ii) the semi-diplomatic edition and
  description of the treatises \citep[34--35]{petti_english_1977}; iii) the linguistic
  analysis and corpus annotation of these with the tools VARD (\cite{rayson_vard_2005}; \cite{baron_vard2_2008}) and CLAWS (\cite{garside_claws_1987}; \cite{garside_hybrid_1997}); and iv) the digital publication of the editions on the
  website (see \url{https://modernmss.uma.es/theproject/}).} ii) certain
information from the volume's electronic catalogue entry (see footnote
\ref{lorente:fn:4}); and iii) some evidence kindly offered via email by various staff
members of the Archives and Special Collections Section of Glasgow
University.

\section{Codicology}

Codicology refers to the study of manuscripts in its diverse external
aspects, i.e., the materials from which they were constructed, their
structural composition, their quiring, binding, decoration, and the form
in which the texts were laid out on the pages, to name but a few (\cite[78--79]{beal_dictionary_2008}; \cite[141]{mathisen_palaeography_2008}; \cite[112]{nystrom_codicological_2014}).\footnote{In the
  most limited sense of the term, codicology does not encompass a
  general survey of handwritten documents, but it addresses the analysis
  of their physical characteristics, along with their
  interrelationships, so as to offer a combination of features for
  historical research (\cite[102]{gruijs_codicology_1972}; also \cite[34]{beal_dictionary_2008}; \cite[112]{nystrom_codicological_2014}).} The present section, therefore, analyses the
codicological aspects of the text, including the
material, dimension, ink, quiring, binding, ruling, and foliation.

\subsection{Material and dimension}

FER7 is a manuscript consisting of 64 folios and two paper flyleaves with a
dimension of approximately 22.2 cm in length and 15.2 cm in width, the
\emph{Niewe Herball} treatise comprising almost half of them as it
consists of 26 folios (about 40.63\% of the total). Besides parchment, paper already began to be in
use among scribes as a material for books in
the thirteenth century, although it was not until the second half of the
fifteenth century that volumes made of this material became relatively
numerous \citep[31]{derolez_palaeography_2003}. According to Da Rold, by the end of that
century, paper was regarded as a merchandise that could be used either
in the more private spheres or in professional contexts, ``such as an
artist's workshop as a tool to dry pigments as an alternative to
parchment'' \citep[56]{da_rold_paper_2020}.
Even though parchment was more durable and resistant under varying
conditions of humidity and heat, the significant cheapness of paper,
together with the need for bulk quantities of writing material after the
establishment of printing, caused that parchment was largely replaced by
paper, ``remaining in use only for high-quality books or other
specialized uses, such as modern `sheepskins', a word still applied to
school diplomas'' (\cite[144]{mathisen_palaeography_2008}; also \cite[5]{petti_english_1977}). Excepting
ff. 47v and 48r, slightly stained with red ink (Figs. \ref{fig:lorente2} and \ref{fig:lorente3}), the
document is in an almost excellent state of preservation
insofar as it presents minor deterioration caused by the course of time,
especially significant considering the aforementioned low quality of paper in comparison with parchment.

\begin{figure}[H]
  \centering
    \includegraphics[width=\textwidth]{media/lorente2.jpg}
    \caption{Figure \ref{fig:lorente2}: 47v stained with red ink.}
    \label{fig:lorente2}
  \end{figure}
  
  
  \begin{figure}
    \includegraphics[width=\textwidth]{media/lorente3.jpg}
    \caption{Figure \ref{fig:lorente3}: 48r stained with red ink.}
    \label{fig:lorente3}
  \end{figure}


 Image reproduced with permission from the University of Glasgow Library 


 
\subsection{Ink}

A sort of dark brown ink is somewhat erratically employed in the
reproduction of \emph{Niewe Herball}, and some degrees of variation in
the tint and thickness may be noticed throughout the folios as well as
throughout the lines of a same page (Figs. \ref{fig:lorente4} and \ref{fig:lorente5}), a fact leading us to
deduce –– tentatively –– that the ink used to write the
treatise was black iron-gall ink which might have discoloured by both
the scribal irregular application of it and, more importantly, the
passing of time. On technical grounds, the production of this sort of
ink involved the combination of the tannin acids obtained from oak-galls
with a ferrous sulphate, known as copperas, which may require added gum
as a thickener. This permanent black iron-gall ink, whose fading over
the centuries to a rust brown colour is a consequence of the chemical
reactions produced in ink, ``became the standard European black ink
until the nineteenth century'' (\cite[149--150]{mathisen_palaeography_2008}; also \cite[7]{petti_english_1977}; \cite[32]{de_hamel_scribes_1992}).

\begin{figure}[H]
  \centering
    \includegraphics[width=\textwidth]{media/lorente4.jpg}
    \caption{Figure \ref{fig:lorente4}: Variation in tint and thickness in \emph{Niewe Herball} (24r).}
    \label{fig:lorente4}
  \end{figure}


 Image reproduced with permission from the University of Glasgow Library 


 
  \begin{figure}
    \includegraphics[width=\textwidth]{media/lorente5.jpg}
    \caption{Figure \ref{fig:lorente5}: Variation in tint and thickness in \emph{Niewe Herball} (33v).}
    \label{fig:lorente5}
\end{figure}


 Image reproduced with permission from the University of Glasgow Library 


 
\subsection{Quiring and binding}

Once the task of copying a text was accomplished, the next step in the
manufacturing of a volume was ``the sewing of the quires into the sewing
supports in the spine of the book'', either by the scribe himself or by
a specialist of the scriptorium (\cite[285]{romero-barranco_early_2017}; also \cite[6]{petti_english_1977}; \cite[49]{clemens_introduction_2007}). FER7 presents a quarto size
(hereafter 4to) made up of a first quire of 16 leaves, followed by a
second one of 12 leaves, a third one of 24, and a final quire of 12
leaves. The collational formula can be then summarised as follows: 4to:
1\textsuperscript{16} 2\textsuperscript{12} 3\textsuperscript{24}
4\textsuperscript{12}. However, as shown in Figure \ref{fig:lorente6}, the last of these
gatherings comprised initially 16 leaves but its four final leaves are
now lost, which means that the manuscript would have comprised of 68 folios
if the final quire were intact.

\begin{figure}[H]
    \centering
    \includegraphics[width=0.5\textwidth]{media/lorente6.jpg}
    \caption{Figure \ref{fig:lorente6}: Lost folios at the end of FER7.}
    \label{fig:lorente6}
\end{figure}


 Image reproduced with permission from the University of Glasgow Library 


 
\vfill

\begin{figure}[H]
  \centering
    \includegraphics[width=\textwidth]{media/lorente7.jpg}
    \caption{Figure \ref{fig:lorente7}: Front binding of FER7.}
    \label{fig:lorente7}
  \end{figure}


 Image reproduced with permission from the University of Glasgow Library 


 
  \begin{figure}
    \includegraphics[width=\textwidth]{media/lorente8.jpg}
    \caption{Figure \ref{fig:lorente8}: Back binding of FER7.}
    \label{fig:lorente8}
\end{figure}


 Image reproduced with permission from the University of Glasgow Library 


 

Next, the quires were bound together, normally with boards on the front
and back for protective purposes (\cite[310]{johnston_writing_1944}; \cite[159]{mathisen_palaeography_2008}). In Middle English, the binding of a manuscript normally involved a
series of bands in cord or leather, to which the quires were sewn,
attached to two wooden boards covered with a piece of leather in natural
colour or stained \citep[44]{derolez_palaeography_2003}. The Early Modern English binding,
while less luxurious, was more elegant and made in a larger variety of
material, although leather was the most common. This period also witnessed
a significant development in decoration with gold tooling, ``which was
practised in Italy in the 15th century and introduced in England just
before the middle of the 16th century'' \citep[7]{petti_english_1977}. FER7 stands as
a prototypical example of this sort of decoration as the title is
provided in gold on the spine. Notwithstanding this, the binding of the
manuscript is not the primary one insofar as the original card boards,
which were produced at some point in the eighteenth century from layered
print waste coated by blue paper, are included in boards of quarter
straight-grained goatskin with comb-marbled paper sides and decorated with maroon leather spine produced in the late nineteenth
century.\footnote{This has been adapted from the data available at the
  electronic catalogue and some information given by Mr Robert MacLean,
  Assistant Librarian of the Archives and Special Collections Section of
  Glasgow University Library.}



\subsection{Ruling}

Scribal methods for supplying a frame for the writing area and ruling
the pages to maintain straight lines have varied over time \mancite\citep[6]{petti_english_1977}. Until the eleventh century, most manuscripts were ruled in drypoint
(also hard-point), a technique consisting in pressing into the page with
the back of a knife or a stylus made of metal or bone, where just one
side of a page needed to be ruled. In the twelfth century, drypoint
ruling was progressively replaced by leadpoint ruling, which was
distinguished by its grey and reddish-brown colour and required the
ruling of both sides of a page. This technique was in vogue until the
end of the fourteenth century, when it was replaced by ink ruling, often
of the same colour as that used for the text, though this practice
became less fashionable in the late fifteenth century and, from then on,
only frame ruling remained (\cite[23]{de_hamel_scribes_1992}; \cite[16--17]{clemens_introduction_2007}; \cite{calle-martin_exploring_nodate}).
\begin{figure}
    \centering
    \includegraphics[width=0.80\textwidth]{media/lorente9.png}
    \caption{Figure \ref{fig:lorente9}: Ruling in \emph{Niewe Herball} (f. 30v).}
    \label{fig:lorente9}
\end{figure}
\emph{Niewe Herball} presents traces of frame and line ruling, albeit
the second of these methods has almost vanished through the course of
the centuries. As for frame ruling, a vertical greyish
frame is systematically applied to both sides of the written area so as
to delimit both the inner and the outer margins of the pages.

\subsection{Foliation}\label{sec:foliation}

The foliation of bound volumes was an erratic practice until the end of
the Middle Ages, yet other forms of numbering, such as quire-marks,
signatures, and catchwords, were commonly witnessed (\cite[6]{petti_english_1977};
\cite[33--34]{derolez_palaeography_2003}; \cite[5]{romero-barranco_late_2015}). The first type consists
of a Roman numeral, sometimes surrounded by some sort of decoration,
written in the lower margin of the first or the last page of a quire.
The second is a numbering system that indicates the order of
the quires and of the \emph{bifolia} within each quire. The third
involves the writing of the word(s) with which the first page of the next quire began in the bottom margin of the last page of a quire
 (\cite[33--34]{derolez_palaeography_2003}; \cite[49]{clemens_introduction_2007}).

 \begin{figure}[H]
    \centering
    \includegraphics[width=1\textwidth]{media/lorente10.jpg}
    \caption{Figure \ref{fig:lorente10}: Foliation in \emph{Niewe Herball} (f. 44r).}
    \label{fig:lorente10}
\end{figure}


 Image reproduced with permission from the University of Glasgow Library 


 
In \emph{Niewe Herball}, excepting f. 59r, foliation is observed in the
top right corner of all folios rectos using Arabic
numerals (Fig. \ref{fig:lorente10}), conceived as a mechanism for their ordering (see
\cite[286]{romero-barranco_early_2017}). This task might have been carried out either by the
scribe or by a later owner of the text since, according to Petti, ``much
of the foliation and pagination found in manuscripts were added much
latter, the 19th-century librarians being especially fond of foliating''
(\cite[38]{petti_english_1977}; also \cite{denholm-young_handwriting_1954}). The form of the numbers in the
manuscript, however, suggests that the foliation process has been performed
by the scribe himself, as they are identical to those in the text (see
section \ref{sec:numerals}).


\section{Palaeography}

Palaeography focuses on the analysis of historical manuscripts to read their
contents, identify their scripts and determine their dates of
composition \citep[140]{mathisen_palaeography_2008}. The field is thoroughly associated
with codicology, being even considered as its subdiscipline, even though some scholars regard them as separate fields of study. This is due
to the fact that palaeography may also provide some information about
those aspects of manuscripts that do not directly belong to the text
itself, such as its layout on the pages, the marginal annotations, and
the glosses, among others (\cite[93]{gruijs_codicology_1972}; \cite[9--10]{derolez_palaeography_2003}). This
section is devoted to the palaeographic analysis of \emph{Niewe
Herball} in order to examine its scripts and to propose an approximate
date of writing. Abbreviations, numerals, and marginalia are also
considered.

\subsection{Scripts}

The history of English handwriting in Early Modern English is
characterised by the coexistence of two cursive scripts, which arose from the need for a widely acceptable hand incorporating aesthetic
appeal and simplicity with smoothness and facility in execution \citep[16]{petti_english_1977}. The first, known as \emph{secretary} hand, developed from the
traditional gothic script and soon became the mundane hand among English
scribes of the sixteenth century and the first half of the seventeenth
century (\cite[8]{dawson_elizabethan_1966}; also \cite[16]{petti_english_1977}). The
second, the \emph{italic} hand, emerged as a revival of the
Carolingian tradition and marked a new trend in the Tudor age that
eventually resulted in the establishment of a completely new style of
handwriting (\cite[20]{fairbank_renaissance_1960}; \cite[viii]{preston_english_1999}; also \cite[18]{petti_english_1977}).

The \emph{secretary} script came from Italy by way of
France and was in use in England for governmental and private
businesses, diverse forms of records, correspondence, literary pieces,
and scientific compositions, among others (\cite[8--9]{dawson_elizabethan_1966}; \cite[viii]{preston_english_1999}). This handwriting was
considered the major script of the Elizabethan period and ``its gradual
evolution from the medieval `court hand' may be seen in any series of
deeds covering the period from 1450 to 1600'' (\cite[414]{schulz_teaching_1943}; also
\cite{jenkinson_later_1927}). In this respect, Petti identifies three phases of
development of this script to indicate its features and to differentiate
them from the fifteenth-century \emph{secretary}. These are the
\emph{early Tudor secretary}, from 1485 to the later years of the rule
of Henry VIII; the \emph{mid-Tudor secretary}, from the mid-1530s to the
beginning of Elizabeth's reign; and the \emph{Elizabethan secretary},
from about 1560 to the first half of the seventeenth century \citep[16--18]{petti_english_1977}.

The \emph{italic} script, then, paved its way to English
documents towards the second half of the fifteenth century, where it
experienced a slow process in establishing a position and existed
thenceforth side by side with the \emph{secretary} hand (\cite[28]{fairbank_renaissance_1960}; \cite[viii]{preston_english_1999}). Its origin derives from
the dissatisfaction of some Florentine scholars with the gothic book
hands that caused an attempt to create a new simple script shaped on the
Carolingian minuscule. The new handwriting was known as \emph{littera
antiqua} or \emph{lettera anticha}, used in three basic categories, i.e.,
\emph{formata}, \emph{corsiva}, and \emph{corrente} (\cite[6]{morison_early_1943};
\cite[18]{petti_english_1977}). After the second half of the sixteenth century,
\emph{italic} gained some popularity due to its simplicity and
legibility, and its grace and beauty, ``which from the beginning
attracted calligraphers'' \citep[9]{dawson_elizabethan_1966}.

These two scripts were sometimes blended depending on the scribes'
preferences. Both in the late sixteenth and the early seventeenth
century, scriveners of the \emph{secretary} hand usually used the
\emph{italic} script so as to highlight certain elements, such as titles
and quotations, or to indicate emphasis (\cite[9]{dawson_elizabethan_1966}; also \cite[29]{fairbank_renaissance_1960}).

\begin{figure}[H]
  \centering
    \includegraphics[width=\textwidth]{media/lorente11.png}
    \caption{Figure \ref{fig:lorente11}: `galen' (f. 45r).}
    \label{fig:lorente11}
  \end{figure}


 Image reproduced with permission from the University of Glasgow Library 


 
  \begin{figure}
    \includegraphics[width=\textwidth]{media/lorente12.jpg}
    \caption{Figure \ref{fig:lorente12}: `Sengrene' (f. 29r).}
    \label{fig:lorente12}
\end{figure}


 Image reproduced with permission from the University of Glasgow Library 


 
In this same vein, \emph{Niewe Herball} seems to have been written in
the early seventeenth-century –– possibly at some point in the period
circa 1600--1630 –– as it is reproduced with a cursive mixed script made up of
a neat \emph{Elizabethan secretary} hand and some scattered features of
the \emph{italic} script,\footnote{The combination of these two scripts as can be seen in the manuscript is a common practice in
  the subperiod proposed. Taking this into account, the estimated date
  has been suggested following Petti's notice on dating an ancient
  manuscript, arguing that ``where no evidence is available to provide a
  \emph{terminus quo} or \emph{terminus quem}, it is rash to expect to
  date a manuscript more closely than within about fifty years in the
  medieval period and about thirty or forty in the Renaissance, when
  changes are more easily chronicled'' \citep[33]{petti_english_1977}. Later substages than
  the one put forward here have also been disregarded, because the
  \emph{italic} hand gains substantial momentum to the detriment of the
  \emph{secretary}, which fades away in almost all handwriting
  environments.} the latter restricted to indicate certain proper nouns
and the names of some plants, as shown in Figures \ref{fig:lorente11} and \ref{fig:lorente12},
respectively. If compared with other scripts used in previous stages of
the history of English handwriting (see \cite{calle-martin_through_2011}), these two
hands were distinguished by their higher degree of thickness, angularity
and, above all, cursiveness, leaving overall more writing space for
running words per page while permitting scribes to carry out their
duties with a significantly greater velocity.


\begin{figure}[H]
    \centering
    \includegraphics[width=1\textwidth]{media/lorente13.png}
    \caption{Figure \ref{fig:lorente13}: \emph{Elizabethan secretary} letterforms in \emph{Niewe
Herball}.\protect\footnotemark}
    \label{fig:lorente13}
\end{figure}
\footnotetext{I gathered and edited the letterforms in Figure \ref{fig:lorente13}, Figure
  \ref{fig:lorente14}, and in section \ref{sec:numerals} in order to help any potential reader of the essay
  identify the letterforms –– as well as the numerals –– used by the
  author of \emph{Niewe Herball}.}

\begin{figure}[H]
    \centering
    \includegraphics[width=1\textwidth]{media/lorente14.png}
    \caption{Figure \ref{fig:lorente14}: \emph{Secretary} capitals in \emph{Niewe Herball}.}
    \label{fig:lorente14}
\end{figure}


 Image reproduced with permission from the University of Glasgow Library 


 

Figure \ref{fig:lorente13} shows the inventory of \emph{Elizabethan secretary}
letterforms employed in the text, where the following stand out (\cite[202--205]{byrne_elizabethan_1925}; \cite[27--91]{tannenbaum_handwriting_1930}; \cite[13--16]{dawson_elizabethan_1966}; \cite[17]{petti_english_1977}; also \cite{preston_english_1999}; \cite{beal_dictionary_2008}): the
letter \textless a\textgreater{} either preceded by a diagonal
descending flourish over it (2) or appearing open-topped with an
inverted semioval or an oblique stroke preceding it (3); the letter
\textless e\textgreater{} reproduced in three different ways: the open
reversed \textless e\textgreater{} (8), the epsilon variety of
\textless e\textgreater{} (9), and the `two-stroke'
\textless e\textgreater{} (10), the third easily confused with the
letter \textless r\textgreater; the letter \textless h\textgreater{}
written similarly to our present-day form (14), or with a looped body,
the supralinear loop always reduced to a simple bow and the first
semioval either fully straight or dispensed with (15 and 16,
respectively); the letter \textless k\textgreater{} rendered with a curl
resembling a \emph{2} or a lower-case \textless z\textgreater{} to the
right of its middle part (18), or with the \emph{2}-like stroke
converted into an indistinct flourish which may be easily mistaken for a
\textless b\textgreater{} (19); the letters \textless m\textgreater{}
and \textless n\textgreater{} written in multiple forms: with slightly
rounded traces in initial and medial position (21 and 25), as a simple
wavy line in final position (22 and 26), with an upward and backward
stroke word-finally (23 and 27), and with a rounded terminal minim
carried below the line at the end of some words (24 and 28); the
traditional \emph{secretary}, the left-shouldered version, and the
\emph{2}-like form of \textless r\textgreater{} (34, 35 and 36,
respectively); the letter \textless s\textgreater{} with a twofold
representation depending on the position: the hooked
\textless s\textgreater{} both at the beginning and the middle of words
(37), and the \emph{sigma} \textless s\textgreater{} in final position
(38); the \emph{n}-form of \textless u\textgreater{} (42); the letters
\textless v\textgreater{} and \textless w\textgreater{} written in their
traditional shape (43 and 46) and with a left-handed curl over the line
(44 and 47); the letter \textless x\textgreater{} resembling a sort of
\emph{secretary} \textless p\textgreater{} (48); and the letter
\textless y\textgreater{} reproduced with a rounded head and a curved
tail to its right (49 and 50).\footnote{These letterforms are also used
  to represent the letter \emph{thorn} as this rune became
  indistinguishable from \textless y\textgreater{} by the fourteenth
  century. According to Tannenbaum, ``{[}a{]}fter 1400 the thorn fell
  more and more out of use, but was represented in some scripts by the
  letter \emph{y} in the compendia \emph{y\textsuperscript{e}},
  y\textsuperscript{t}, y\textsuperscript{is}, y\textsuperscript{ei},
  y\textsuperscript{m}, y\textsuperscript{an}, y\textsuperscript{r}, for
  `the', `that', `this', `they', `them', `than' and `their'.
  {[}\ldots{]} Many of these abbreviations continued in common use in
  the seventeenth and eighteenth centuries {[}\ldots{]} and is still
  being written for archaic and comic effects'' \citep[88]{tannenbaum_handwriting_1930}.}

Apart from minuscules, capital letters also deserve some deal of
attention, not only for their high incidence (563 cases of majuscules),
but also because of their textual importance in \emph{Niewe Herball},
given that they are employed to mark the beginning of the different
passages or sections.\footnote{Possibly used for decorative reasons,
  capital letters are likewise witnessed in the title of the treatise,
  rendered with a somewhat embroidered script and slightly bigger in
  dimensions than the rest of majuscules.} Based on the
fifteenth-century \emph{hybrid secretary} script (see \cite[17--18]{petti_english_1977}), these capitals employed cursive embellished forms and included,
among others: a rounded variety of
\textless C\textgreater{} with a curved cross-stroke (3); the letter
\textless E\textgreater{} resembling a contemporary
\textless C\textgreater{} with an Arabic \emph{2} set within it (5); the
double \textless F\textgreater{} (6); the letter
\textless H\textgreater{} rendered with an initial left-handed curve
with an ornamental dot below it (8); the letter
\textless K\textgreater{} represented as an \textless L\textgreater{}
with a \emph{2}-like curl across the middle (10); the letter
\textless P\textgreater{} generally represented with a downstroke, a
horizontal stroke, and a right-handed bow (15); the letter
\textless Q\textgreater{} rendered as a capital
\textless O\textgreater{} complemented with a curl converging the right
side of the base and some vertical crossbars (16); and the letter
\textless S\textgreater{} written as the double curve of the roman
equivalent, with the lower part continued round to form a sort of cover
over the top (18) (\cite[31--36]{mckerrow_capital_1927}; \cite[95--118]{tannenbaum_handwriting_1930};
\cite[13--16]{dawson_elizabethan_1966}; \cite[17--18]{petti_english_1977}).




\subsection{Abbreviations}

\begin{figure}[H]
    \centering
    \includegraphics[width=0.82\textwidth]{media/lorente15.png}
    \caption{Figure \ref{fig:lorente15}: Distribution of abbreviations in \emph{Niewe Herball}.}
    \label{fig:lorente15}
\end{figure}


 Image reproduced with permission from the University of Glasgow Library 


 
Incorporated from Latin documents, adopting both their rules and
symbols, the main functions of abbreviations in English writings were to
save time and to conserve the maximum space ``in writing on expensive
materials like vellum (coupled with the necessity of aligning or
`justifying' the right-hand margins by varying the length of words)''
(\cite[218]{greetham_textual_1994}; also \cite[119]{tannenbaum_handwriting_1930}; \cite[22]{petti_english_1977};
\cite[115]{calle-martin_corpus-based_2021}). Chronologically speaking, however, the adoption
was not straightforward, as some variations may have resulted from changing practices over time. The early Middle Ages, for
instance, displayed an abbreviation system that could differ in terms of
the place where they were produced, the typology of the text, and, more
importantly, the idiosyncratic practices of scribes (\cite[22]{petti_english_1977};
\cite[187]{derolez_palaeography_2003}; \cite[89]{clemens_introduction_2007}). The mechanisms of Late
Middle English and Early Modern English, by contrast, were fairly
uniform and constrained to reasonable boundaries, ``presenting few
problems other than whether or not a symbol which usually signifies
omission is, in context, simply a meaningless flourish'' \citep[22]{petti_english_1977}. According to most sources on historical
palaeography, abbreviations can be divided into five different categories: contraction, suspension,
brevigraphs, superscript letters, and elision. \emph{Niewe Herball}
contains instances of all of them; the following subsections give examples for each category. For reasons of clarity, abbreviations are
  expanded with the supplied letter(s) italicised.

\subsubsection{4.2.1. Contraction}

Contraction involves the omission of one or more letters word-medially.
Its use in the text is confined to a tittle or tilde (see \cite[119--120]{tannenbaum_handwriting_1930}; \cite[118]{calle-martin_corpus-based_2021}), which in most cases represents
the letters \textless m\textgreater{} or \textless n\textgreater{} (279
and 400 instances, respectively), as illustrated in Figure \ref{fig:lorente16}. In 18 cases the tilde stands for other letters such as
\textless c\textgreater{} or \textless t\textgreater, as in Figure \ref{fig:lorente17}.

\begin{figure}[H]
  \centering
    \includegraphics[width=\textwidth]{media/lorente16.jpg}
    \caption{Figure \ref{fig:lorente16}: `so\emph{m}mer' (f. 36r).}
    \label{fig:lorente16}
  \end{figure}


 Image reproduced with permission from the University of Glasgow Library 


 
  \begin{figure}
    \includegraphics[width=\textwidth]{media/lorente17.jpg}
    \caption{Figure \ref{fig:lorente17}: `bened\emph{i}c\emph{t}a'(f. 36v).}
    \label{fig:lorente17}
\end{figure}


 Image reproduced with permission from the University of Glasgow Library 


 
\subsubsection{4.2.2. Suspension}

Suspension, in turn, involves the omission of the letter(s) at the end
of a word \citep[124]{tannenbaum_handwriting_1930}. It is indicated in \emph{Niewe
Herball} by means of a tilde over the final letter (Fig. \ref{fig:lorente18}), by a
curved stroke first turning up and then inclining leftwards (Fig. \ref{fig:lorente19}),
and by a period next to the abbreviated unit (Fig. \ref{fig:lorente20}). The first two
of these techniques only denote the consonants \textless m\textgreater{}
or \textless n\textgreater{} (181 and 88 occurrences in the treatise),
whilst the third is used to represent a higher number of letters (143
occurrences).

\begin{figure}[H]
  \centering
    \includegraphics[width=1\textwidth]{media/lorente18.jpg}
    \caption{Figure \ref{fig:lorente18}: `whervpo\emph{n}' (f. 47r).}
    \label{fig:lorente18}
  \end{figure}


 Image reproduced with permission from the University of Glasgow Library 


 
  \begin{figure}
    \includegraphics[width=1\textwidth]{media/lorente19.jpg}
    \caption{Figure \ref{fig:lorente19}: `tertia\emph{m}' \\ (f. 27v).}
    \label{fig:lorente19}
  \end{figure}


 Image reproduced with permission from the University of Glasgow Library 


 
  \begin{figure}
    \includegraphics[width=1\textwidth]{media/lorente20.png}
    \caption{Figure \ref{fig:lorente20}: `D\emph{octor}' \\  (f. 23r).}
    \label{fig:lorente20}
\end{figure}


 Image reproduced with permission from the University of Glasgow Library 


 
\subsubsection{4.2.3. Brevigraphs}

Brevigraphs refer to special sings that represent either two
letters or an entire syllable, depending on the letter to which they are
attached and their word-position \citep[23]{petti_english_1977}. This variation in
significance is strongly associated with the fact that some of these
signs ``derived from the shorthand system known as Tironian notes,
believed to have been developed by Cicero's secretary, Tiro, to record
his master's speeches, {[}whilst{]} others originated in the Roman
system of legal shorthand known as \emph{notae iuris}'' \citep[89]{clemens_introduction_2007}. \emph{Niewe Herball} contains a significant amount of
brevigraphs, most of them comprising different combinations of
\textless r\textgreater- and \textless s\textgreater-related clusters
(see \cite[172]{de_la_cruz-cabanillas_abbreviations_2018}).

As far as \textless r\textgreater-related clusters are concerned, the
following stand out:

\begin{itemize}
\item
  The cluster \emph{`er}' indicated by means of ``an ascending
  flourished stroke, curved leftwards and placed over the preceding
  letter'' \citep[119]{calle-martin_corpus-based_2021} in those cases where the character
  going immediately before it is a \textless \th\textgreater,
  \textless u\textgreater{} or the pair \textless th\textgreater{} (88
  occurrences in total) (Fig. \ref{fig:lorente21}), by a diagonal stroke through the
  letter \textless v\textgreater{} (10 instances) \citep[131]{tannenbaum_handwriting_1930}
  (Fig. \ref{fig:lorente22}), or by a superscript graph resembling a specific variant
  of \textless r\textgreater{} (127 occurrences) (Fig. \ref{fig:lorente23}), the latter
  also standing for the group `\emph{ur}' (82 instances), as shown in
  Figure \ref{fig:lorente24}.
\item
  The clusters `p\emph{ar}', `p\emph{er}' and `p\emph{ur}' (which occur respectivly 37, 15 and 6 times in \emph{Niewe Herball}) represented by a mark below
  the letter \textless p\textgreater{} going from left to right with a
  convex flourish across its stem, as in Figures \ref{fig:lorente25}, \ref{fig:lorente26} and \ref{fig:lorente27},
  respectively. The last of these, however, is also observed as a curved stroke after the \textless p\textgreater{} which
  shifts to the left and terminates in a short descender over it, as in
  Figure \ref{fig:lorente28} (see \cite[128--129]{tannenbaum_handwriting_1930}).
\item
  The clusters `p\emph{ro}', `p\emph{re}' and `p\emph{ri}' (35, 9 and 9
  instances), the first represented with a hollow flourish through the
  stem of the letter \textless p\textgreater{} (Fig. \ref{fig:lorente29}), whereas the
  other two are rendered with a curved stroke going up to the left from
  the lower part of the consonant and ending in a vertical descender
  above it (Figs. \ref{fig:lorente30} and \ref{fig:lorente31}).
\end{itemize}

The \textless s\textgreater-related brevigraphs
comprise of abbreviations of letters either in initial, middle, or
final position. Both word-initially and word-medially, when rendered
with a \emph{2}-like trace appended to its stem, the traditional long
hooked \textless s\textgreater{} is employed 4 times in \emph{Niewe
Herball} to represent the cluster `s\emph{er}', as shown in Figure \ref{fig:lorente32}.
This consonant, however, also appears in a single instance with a stroke
over the last letter of a word in order to represent `\emph{es}', as in
Figure \ref{fig:lorente33}. Word-finally, the most common forms involve a variety of
\textless s\textgreater{} akin to an elongated sort of
\textless c\textgreater{} with a closed arc at its top as an equivalent
of the group `\emph{es}' (596 occurrences) (Fig. \ref{fig:lorente34}), and a
\emph{q}-like graph as indicator of the cluster `\emph{us}' (22
examples), as in Figure \ref{fig:lorente35}.

Apart from the previous symbols, other brevigraphs and the letters they
represent in the text include `\emph{and}' (1,006 occurrences),
`\emph{Christ}' (3 occurrences), `\emph{etc}.' (2 instances) and
`\emph{half}' (7 occurrences), as illustrated in Figures \ref{fig:lorente36}-\ref{fig:lorente39} below.

\begin{figure}[H]
  \centering
    \includegraphics[width=\textwidth]{media/lorente21.jpg}
    \caption{Figure \ref{fig:lorente21}: `togi\th\emph{er}' (f. 26r).}
    \label{fig:lorente21}
  \end{figure}


 Image reproduced with permission from the University of Glasgow Library 


 
  \begin{figure}
    \includegraphics[width=\textwidth]{media/lorente22.png}
    \caption{Figure \ref{fig:lorente22}: `v\emph{er}tues' (f.
26v).}
    \label{fig:lorente22}
\end{figure}


 Image reproduced with permission from the University of Glasgow Library 


 
\begin{figure}[H]
  \centering
    \includegraphics[width=\textwidth]{media/lorente23.jpg}
    \caption{Figure \ref{fig:lorente23}: `rou\emph{n}d\emph{er}' (f. 36r).}
    \label{fig:lorente23}
  \end{figure}


 Image reproduced with permission from the University of Glasgow Library 


 
  \begin{figure}
    \includegraphics[width=\textwidth]{media/lorente24.png}
    \caption{Figure \ref{fig:lorente24}: `colo\emph{ur}' (f. 25r).}
    \label{fig:lorente24}
\end{figure}


 Image reproduced with permission from the University of Glasgow Library 


 
\begin{figure}[H]
  \centering
    \includegraphics[width=1.55in,height=0.72083in]{media/lorente25.jpg}
    \caption{Figure \ref{fig:lorente25}: `p\emph{ar}ched' (f. 41r).\\}
    \label{fig:lorente25}
  \end{figure}


 Image reproduced with permission from the University of Glasgow Library 


 
  \begin{figure}
    \includegraphics[width=1.55in,height=0.72083in]{media/lorente26.jpg}
    \caption{Figure \ref{fig:lorente26}: `temp\emph{er}ate' (f.
26v).\\}
    \label{fig:lorente26}
  \end{figure}


 Image reproduced with permission from the University of Glasgow Library 


 
  \begin{figure}
    \includegraphics[width=1.55in,height=0.72083in]{media/lorente27.jpg}
    \caption{Figure \ref{fig:lorente27}: `p\emph{ur}slane' (f. 46v).}
    \label{fig:lorente27}
  \end{figure}


 Image reproduced with permission from the University of Glasgow Library 


 
  \begin{figure}
    \includegraphics[width=1.55in,height=0.72083in]{media/lorente28.png}
    \caption{Figure \ref{fig:lorente28}: `p\emph{ur}geth' (f.
23v).}
    \label{fig:lorente28}
\end{figure}


 Image reproduced with permission from the University of Glasgow Library 


 
\begin{figure}[H]
  \centering
    \includegraphics[width=1\textwidth]{media/lorente29.jpg}
    \caption{Figure \ref{fig:lorente29}: `p\emph{ro}uoketh \\ (f. 24v).}
    \label{fig:lorente29}
  \end{figure}


 Image reproduced with permission from the University of Glasgow Library 


 
  \begin{figure}
    \includegraphics[width=1\textwidth]{media/lorente30.jpg}
    \caption{Figure \ref{fig:lorente30}: `p\emph{ri}mrose' \\ (f.
30r).}
    \label{fig:lorente30}
  \end{figure}


 Image reproduced with permission from the University of Glasgow Library 


 
  \begin{figure}
    \includegraphics[width=1\textwidth]{media/lorente31.jpg}
    \caption{Figure \ref{fig:lorente31}: `p\emph{re}serueth' \\ (f. 25r).}
    \label{fig:lorente31}
\end{figure}


 Image reproduced with permission from the University of Glasgow Library 


 
\begin{figure}[H]
  \centering
    \includegraphics[width=\textwidth]{media/lorente32.jpg}
    \caption{Figure \ref{fig:lorente32}: `cons\emph{er}ue' (f. 23r).\\}
    \label{fig:lorente32}
  \end{figure}


 Image reproduced with permission from the University of Glasgow Library 


 
  \begin{figure}
    \includegraphics[width=\textwidth]{media/lorente33.jpg}
    \caption{Figure \ref{fig:lorente33}: `ros\emph{es}' (f. 23r).\\}
    \label{fig:lorente33}
  \end{figure}


 Image reproduced with permission from the University of Glasgow Library 


 
  \begin{figure}
    \includegraphics[width=\textwidth]{media/lorente34.jpg}
    \caption{Figure \ref{fig:lorente34}: `bursting\emph{es}' (f. 26r).}
    \label{fig:lorente34}
  \end{figure}


 Image reproduced with permission from the University of Glasgow Library 


 
  \begin{figure}
    \includegraphics[width=\textwidth]{media/lorente35.jpg}
    \caption{Figure \ref{fig:lorente35}: `venimo\emph{us}' (f.
25v).}
    \label{fig:lorente35}
\end{figure}


 Image reproduced with permission from the University of Glasgow Library 


 
\begin{figure}[H]
  \centering
    \includegraphics[width=1.4in,height=0.78083in]{media/lorente36.png}
    \caption{Figure \ref{fig:lorente36}: `\emph{and}' (f. 37r).\\}
    \label{fig:lorente36}
  \end{figure}


 Image reproduced with permission from the University of Glasgow Library 


 
  \begin{figure}
    \includegraphics[width=1.4in,height=0.78083in]{media/lorente37.png}
    \caption{Figure \ref{fig:lorente37}: `\emph{Christ}i' (f. 30v).\\}
    \label{fig:lorente37}
  \end{figure}


 Image reproduced with permission from the University of Glasgow Library 


 
  \begin{figure}
    \includegraphics[width=1.4in,height=0.78083in]{media/lorente38.png}
    \caption{Figure \ref{fig:lorente38}: `\emph{etc.}' (f. 41v).}
    \label{fig:lorente38}
  \end{figure}


 Image reproduced with permission from the University of Glasgow Library 


 
  \begin{figure}
  \includegraphics[width=1.4in,height=0.78083in]{media/lorente39.png}
    \caption{Figure \ref{fig:lorente39}: `\emph{half}' (f. 39r).}
    \label{fig:lorente39}
\end{figure}


 Image reproduced with permission from the University of Glasgow Library 


 
\subsubsection{4.2.4. Superscript letters}

Superscript letters consist of the addition –– or the raising –– of one
or more letters above any part of a word so as to indicate
the omission of some letter(s) within that same unit. This abbreviation
method was à la mode in the sixteenth century, becoming relatively
common for modes of address, numerals, relative and possessive pronouns,
adjectives, and prepositions (\cite[134]{tannenbaum_handwriting_1930}; \cite[24]{petti_english_1977}). In
the text, as well as in most English documents (see \cite[119]{calle-martin_corpus-based_2021}), the superscript letters for the words \emph{\th}\emph{at}, \emph{with} and
\emph{which} (Figs. \ref{fig:lorente40}, \ref{fig:lorente41}, and \ref{fig:lorente42}) present the highest rate, with 38,
157 and 95 occurrences, respectively. In addition, there are other 49
instances where a different word is affected, as in Figure \ref{fig:lorente43}.

\begin{figure}[H]
  \centering
    \includegraphics[width=1.3in,height=0.82083in]{media/lorente40.png}
    \caption{Figure \ref{fig:lorente40}: `\th\emph{a}t' (f. 34r).}
    \label{fig:lorente40}
  \end{figure}


 Image reproduced with permission from the University of Glasgow Library 


 
  \begin{figure}
    \includegraphics[width=1.3in,height=0.82083in]{media/lorente41.jpg}
    \caption{Figure \ref{fig:lorente41}: `w\emph{i}th' (f. 34v).}
    \label{fig:lorente41}
\end{figure}


 Image reproduced with permission from the University of Glasgow Library 


 
\begin{figure}[H]
  \centering
    \includegraphics[width=1.3in,height=0.82083in]{media/lorente42.jpg}
    \caption{Figure \ref{fig:lorente42}: `w\emph{hi}ch' (f. 26v).}
    \label{fig:lorente42}
  \end{figure}


 Image reproduced with permission from the University of Glasgow Library 


 
  \begin{figure}
    \includegraphics[width=1.3in,height=0.82083in]{media/lorente43.jpg}
    \caption{Figure \ref{fig:lorente43}: `g\emph{ra}du' (f.
26v).\protect \footnotemark}
    \label{fig:lorente43}
\end{figure}
\footnotetext{Tannenbaum states that a number of superscript letters, as
  it is the case of Figure \ref{fig:lorente43}, ``might themselves be abbreviated (thus
  sometimes giving us three-story words) and might include the special
  signs {[}\ldots{]} termed brevigraphs'' \citep[134]{tannenbaum_handwriting_1930}.}

\subsubsection{4.2.5. Elision}

Finally, elision differed from traditional mechanisms of abbreviation
insofar as it was not generally used to save time or space, but it
was ``usually made for metrical and linguistic reasons and affects the
pronunciation of the words concerned'' (\cite[121]{tannenbaum_handwriting_1930}; also \cite[25]{petti_english_1977}). Elision may take place in any part of a word. In initial
position, the abbreviation entailed the loss of a syllable (\emph{aphaeresis}) or vowel (\emph{aphesis}), and was primarily
intended to link the word to the preceding or succeeding word. Medial
elision (\emph{syncopation}), in turn, was outstandingly common in
poetry, especially where the letters \textless e\textgreater{} and
\textless v\textgreater{} were concerned. Finally, elision at the end of
a word (\emph{apocope}) enabled it to run on to the following unit
when this began with a vowel or aspirate (\cite[121--124]{tannenbaum_handwriting_1930};
\cite[25]{petti_english_1977}). \emph{Niewe Herball} only includes \emph{apocopes} (18
instances), occurring at all times with the definite article, as
reproduced in Figure \ref{fig:lorente44}.

\begin{figure}[H]
  \centering
    \includegraphics[width=1.9in,height=0.77083in]{media/lorente44.jpg}
    \caption{Figure \ref{fig:lorente44}: `thabouesaid' (f. 29v).}
    \label{fig:lorente44}
    \end{figure}


 Image reproduced with permission from the University of Glasgow Library 


 
\subsection{Numerals}\label{sec:numerals}

As a piece of Early Modern English writing, \emph{Niewe Herball}
includes the two mechanisms of numeration employed in the period, that
is, Roman numbers and Arabic numbers. The former predominated in English
texts until the beginning of the sixteenth century, with an array of
variations and extensions in their forms, while the latter did not fully
pave their way into English documents up to the beginning of the
sixteenth century, despite being comparatively popular abroad long
before (\cite[266]{jenkinson_use_1926}; \citep[28]{petti_english_1977}). Arabic numerals, however,
already began to gradually surpass the Roman system for the marginalia, the
quires and the numeration of the folios in the fifteenth
century, and soon after, they became as frequent as Roman numbers for
general purposes (\cite[153]{tannenbaum_handwriting_1930}; \cite[28]{petti_english_1977}).

Amounting up to 70 occurrences in the text, Roman numerals are used to
show the quantity of stalks, flowers, leaves, or stems that a particular
plant possesses, as in (\ref{lorente:numlist:2}), and to reveal the diverse types existing
within the same species, as in (\ref{lorente:numlist:3}). In relation to their forms, the
equivalents for numbers 2 and 3 stand out over the rest of numerals
since they are twofoldly represented (see Fig. \ref{fig:lorente45}): i) made up of
\emph{i}'s and \emph{j}'s only (i.e., ij and iij, as in A and C); and ii)
accompanied by a sort of superscript \textless o\textgreater{} (i.e., ijº
and iijº, as in B and D). This superscript unit, however, does not have a
specific meaning with Roman numerals in the treatise; in all cases it denotes a cardinal number.

\begin{enumerate}
\def\labelenumi{(\arabic{enumi})}
\setcounter{enumi}{1}
\item\label{lorente:numlist:2}
{[}\ldots{]} it hath great brode leaues, not rou\emph{n}d, but
w\emph{i}t\emph{h} manye corners, or indented angles, w\emph{i}t\emph{h}
manye vaines, like to a horse foote, \uline{vj} or \uline{vij} leaues
springing out of one rote {[}\ldots{]} 
\begin{flushright}
  (f. 23v)  
\end{flushright}

\vspace{1em}

\item\label{lorente:numlist:3}
Iuye, is of \uline{iijº} sort\emph{es}. white yvie, vnknown to us.
black yvie, hedera nigra: dyonisia. hedera mas. 
\begin{flushright}
    (f. 34r)
\end{flushright}
\end{enumerate}

\begin{figure}[H]
  \centering
    \includegraphics[width=.6\textwidth]{media/lorente45.png}
    \caption{Figure \ref{fig:lorente45}: Roman numbers in \emph{Niewe Herball}}
    \label{fig:lorente45}
    \end{figure}


 Image reproduced with permission from the University of Glasgow Library 


 
\begin{figure}[H]
  \centering
    \includegraphics[width=.6\textwidth]{media/lorente46.png}
    \caption{Figure \ref{fig:lorente46}: Arabic numbers in \emph{Niewe Herball}}
    \label{fig:lorente46}
    \end{figure}


 Image reproduced with permission from the University of Glasgow Library 


 
Arabic numbers, in turn, are more regularly witnessed (105 occurrences). Besides being used for the same purposes as Roman numerals, this
system of numeration is likewise employed for the marginal notes (see
section \ref{sec:marginalia}), to number some of the folios (see section \ref{sec:foliation}), to
indicate dates, as in (4), and to specify the degrees in which plants
are boiled, dried, cold or moist, as in (5). When it comes to their
shape (see Fig. \ref{fig:lorente46}), Arabic numerals are rendered in almost the same form as at
the present days. An exception is formed by the number 1, which is reproduced either in almost
the same manner as our lower-case \textless i\textgreater{} (A) or
undotted (B), ``with its head to the left and foot to the right,
{[}which{]} may easily be confused with {[}number{]} 2'' \citep[268]{jenkinson_use_1926} .

\begin{enumerate}
\def\labelenumi{(\arabic{enumi})}
\setcounter{enumi}{3}
\item\label{lorente:numlist:4}
Taken out of D\emph{octor}. Rembert Dodoens phisitian to \th\emph{e}
Emp\emph{erour}. his herball made an\emph{n}o d\emph{om}in\emph{i}.
\uline{1578}. 
\begin{flushright}
    (f. 23r)
\end{flushright}

\item\label{lorente:numlist:5}
{[}\ldots{]} the great \emph{and} small sengrene, frig\emph{idum}.
et sic\emph{cum}. \uline{3º} g\emph{ra}d\emph{u}. the great \emph{and}
small stone crop, ca\emph{lidum}. et sic\emph{cum}. fere \uline{4º}.
\begin{flushright}
    (f. 29v)\footnote{Note that, as opposed to Roman numbers, Arabic
  numerals with a supralinear \textless o\textgreater{} are always used
  to indicate an ordinal number in the text.}
\end{flushright}

\end{enumerate}

\subsection{Marginalia}\label{sec:marginalia}

Marginalia are manual annotations in the
margins of manuscripts, added either by the scribes themselves or by latter
owners. In early English, marginalia served different purposes: they could be used, among others, to point to relevant
information and to refer back to previous information, or they could
merely act as simple adornments of a text (\cite[35]{criado_pena_elizabeth_2019}; also
\cite[247]{beal_dictionary_2008}). The \emph{Niewe Herball} contains eight
instances of marginalia, serving three different purposes: 
\begin{itemize}
    \item to enumerate the typology of the plants
within several of the species described in ff. 29v, 30r, 34v, 36v, 37r
and 47v (Fig. \ref{fig:lorente47});
    \item to show that specific virtues of the herb
radish have been tested and proved true by means of symbols resembling
crosses (Fig. \ref{fig:lorente48});
    \item to provide the name of a likely former
owner of the volume (Fig. \ref{fig:lorente49}).
\end{itemize}

As for the second function, this type of mark ``was frequently used in the margins of
accounts as a sort of check mark, perhaps as the equivalent of our
`O.K.'\,'' \citep[134]{tannenbaum_handwriting_1930}. Thus, the utilisation of these symbols in \emph{Niewe
Herball} means that either the scribe himself or a later user of
the treatise might have checked that certain medical benefits and
qualities of the plant in question were real, and so they could have
employed the cross signs to leave an explicit record of it. The use of
these marks also proves, as Criado-Peña claims, that ``the {[}text{]}
was certainly consulted and written for practical purposes'' \citep[35]{criado_pena_elizabeth_2019}.

Apropos the last usage, in accordance with the data offered in the
online catalogue of the University of Glasgow, the possible early holder
of the manuscript refers to an individual called John Lewis Babrac,
better known as Bawbrac, whose name and alias can also be read clearly
in a manuscript engraving of nine lines inside the original back board of
the volume,\footnote{His biographical information appears to be
  non-existent in both printed and electronic biographical mediums, the
  inscription at the end of the manuscript being the only source where
  some knowledge about him may be viewed.} together with the short
personal handwritten notes found in the margins of f. 32r. In this
inscription, the author addresses the eventual reader as a traveller and
requests him/her to explicitly direct attention to the words appearing
in it, where special emphasis is placed on the figure of Bawbrac and on
his remarkable personality. This suggests that either he
wrote the text at the end of the volume in his own hand –– considering the abovementioned marginal annotations –– 
or that another individual composed it as a tribute to
Bawbrac's memory.

\begin{figure}[H]
  \centering
    \includegraphics[width=.8\textwidth]{media/lorente47.jpg}
    \caption{Figure \ref{fig:lorente47}: Marginal notes in f. 30r.}
    \label{fig:lorente47}
    \end{figure}


 Image reproduced with permission from the University of Glasgow Library 


 
\begin{figure}[H]
  \centering
    \includegraphics[width=.8\textwidth]{media/lorente48.jpg}
    \caption{Figure \ref{fig:lorente48}: Marginal notes in f. 45v.}
    \label{fig:lorente48}
    \end{figure}


 Image reproduced with permission from the University of Glasgow Library 


 
\begin{figure}[H]
  \centering
    \includegraphics[width=.8\textwidth]{media/lorente49.jpg}
    \caption{Figure \ref{fig:lorente49}: Marginal notes in f. 32r.}
    \label{fig:lorente49}
    \end{figure}


 Image reproduced with permission from the University of Glasgow Library 


 
\section{Conclusions, final remarks, and future research}
The present paper carried out a physical description of Glasgow
University Library, MS Ferguson 7 (ff. 23r--48v; 59r), which is, up until
today, the only known handwritten volume comprising a portion of the
English translation of Rembert Dodoens' \emph{A Niewe Herball or
Historie of Plants}, a highly appreciated herbal in the sixteenth and
the seventeenth century. The paper paid attention to some of its
codicological and palaeographic features, including, among others, the
material, ink, foliation, scripts, abbreviations, and marginalia. Based on an examination of the combination of these elements, this essay is able to suggest the most probable date of composition, permitting the research community to use it as a faithful input for distinct
manners of analysis, either synchronic or diachronic. In doing so, a
scholar may consider it appropriate to use the present description as a
reference to evaluate whether the appearance of a particular feature is
a recurrent procedure of a period or context, for instance. On the
whole, the examination of the manuscript enabled me to propose the early seventeenth century –– presumably around 1600--1630 –– as date of writing, as the treatise is
rendered with a rather readable cursive mixed hand, arising from the
blend of the \emph{Elizabethan secretary} script, together with some
characteristics of the \emph{italic} hand.

Furthermore, although certain points deserve further looking into –– for instance regarding the
actual ownership of the treatise –– this work has opened the
door for possible forthcoming analyses, both quantitative and
qualitative, as it has, where possible, provided the specific occurrence
of the text's scrutinised elements. To name but one example, the
detailed figures could be used to test, if any, the variation between different text-types and/or
registers with regard to the function of two kinds of numeration of the period.

From a personal perspective, the examination of the physical
characteristics of this unique piece of Early Modern English handwriting serves as a guide, and hence a starting point, to create a scholarly edition of the text. This
edition will allow for a fine-grained examination of the
scribal usage of English at different linguistic levels, as well as for
an analysis of the similarities and differences of the
language used in this treatise and the printed edition from which it derives.
All in all, future research on these issues will
contribute to several philological subareas, such as manuscript studies and
historical linguistics, and cast new light on the
early use of English from an interdisciplinary standpoint.


\section*{Acknowledgements}
The present essay has
  been funded by the Spanish Ministry of Economy, Industry and
  Competitiveness (grant number FFI217-88060-P). This grant is hereby
  gratefully acknowledged. I would like to thank the editors and the two
  anonymous reviewers of \emph{Variants} for their feedback and remarks,
  which have definitively helped me develop an improved version of the
  essay. Finally, I would also like to express my appreciation to Professor
  Javier Calle-Martín and Dr Jesús Romero-Barranco for their support and valuable
  comments on previous drafts of the paper, as well as to Miss Fiona
  Neale and Mr Robert MacLean (Glasgow University Library) because of
  their generosity and kindness in providing me with helpful information
  about the volume when I requested it.
  
\begin{flushleft}
    % use smallcaps for author names
    \renewcommand*{\mkbibnamefamily}[1]{\textsc{#1}}
    \renewcommand*{\mkbibnamegiven}[1]{\textsc{#1}} 
\printbibliography
\end{flushleft}

\end{document}