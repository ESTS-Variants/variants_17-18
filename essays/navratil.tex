\contributor{Martin Navrátil}
\contribution{The Creative Role of the Editor During the Reconstruction of a Lost Collection}
\shortcontributor{Martin Navrátil}
\shortcontribution{The Creative Role of the Editor}

\begin{paper}
\renewcommand*{\pagemark}{}

\begin{abstract}
In 1946, the Slovak poet Vojtech Mihálik (1926--2001) participated in
a literary competition organized by the Tranoscius publishing house, in
which he won second place with his collection of poems \emph{Ruža} [The Rose]. However, the collection was never published. There is only one
surviving bit of evidence: a tentative list of poem titles recorded in one of the author's manuscript notebooks. What can a scholarly
editor do in a situation like this, other than give up all hope of
discovering the collection? A possible solution would be to try and reconstruct
the collection in order to create a complete version. Since we cannot
rely on the historical records of the collection, we will try to
reconstruct it according to this list of poems and the author's
manuscripts from this period, which contain dated poems arranged in
chronological order. This approach requires the use of methods that are
uncommon in the context of the conventions of Slovak scholarly editing.
A different solution would to attempt to present the alternatives
that may be considered with regard to the instructions provided by the
list of poem titles: i.e. presenting a dynamic reconstruction process.
Individual goals may also inform the choice of a suitable presentation
medium (a print edition vs. a digital edition).
\end{abstract}


\section*{Introduction}
\textsc{Just like every other part of society,} post-war literature in the former
Czechoslovakia was affected by the political situation, the specifics of
which were determined by the Soviet sphere of influence. After the
Communist Party took over the government in 1948 and made an effort to
completely seize control of all areas of society, the state's editorial
policy was gradually reorganized. After 1948, the 69 publishing houses
that had existed in Slovakia were reduced to only 17, and they were
assigned different segments of the publishing industry. Publishing was
subject to the strict oversight of the Communist Party. The situation
also had an impact on religious publishing houses, where only two
remained out of the original eight –– one Catholic (Spolok svätého
Vojtecha) and one Lutheran (Tranoscius). Moreover, their activities were
reduced to the publication of liturgical books and educational religious
literature (see: \cite{sutovec_k_2011}). As the government was nationalizing
publishing houses and taking control over them, the situation became
difficult to navigate. Many documents were destroyed, while others were
``saved'' by employees of the publishing houses, who took them for
safekeeping to prevent them from being destroyed. As a result, the
archives from this period are very fragmentary.

Editorial activities focusing on this period are also faced with another
phenomenon: the censorship and self-censorship of authors. With the
exception of writers who had already been leftist intellectuals in the
interwar period, several authors shifted their thinking ``to the left''
after the onset of communism. This was because some of them saw
socialism as having the potential to address the pressing social issues
faced by post-war society, as well as being a counterweight to fascist
trends. Some were drawn to socialist writing by opportunism, others were
pressured to change their attitudes by censorship and publishing bans,
but there were also authors who decided to give up the option to publish
their literature in an official capacity rather than being forced to
change the character of their writing.

Prominent Slovak poet Vojtech Mihálik's unpublished collection of poems
\emph{Ruža} [The Rose], the reconstruction of which is presented in this paper,
bears all of the previously mentioned marks of this period. Our search
for this collection in the archive of the Tranoscius publishing house
was unsuccessful, even though its manuscript had received recognition in
a literary contest that the Tranoscius publishing house had organized in
1946. Another factor that comes into play is the fact that the poet
gradually became closer to the communist regime. In the early period of
his poetic writing, Mihálik considered himself a member of so-called
Catholic modernism. But after his permanent shift towards atheist
communism, he decided to hide, marginalize and even denounce his early
Catholic writing, so not even he had an interest in making his early
works known.\footnote{\label{foot:navratil1}Despite the fact that he did not support the
  research of his early, spiritual work and he would later even attempt
  to hide it, he took great care to preserve these materials. Most of
  Vojtech Mihálik's literary estate is still well preserved in the
  private archive of his second wife, Drahomíra Miháliková. Unless noted otherwise, all transcripts from and references to Mihálik's unpublished documents were found in this personal archive.} Since
Mihálik is one of the most important Slovak poets of the second half of
the 20th century, naturally, this sparks an interest in discovering the
collection.

\section{Searching for sources for the \emph{Ruža}
collection}

Two newspaper reports –– the first one from late 1946, the second one
from early 1947 –– informed that the up-and-coming Slovak poet Vojtech
Mihálik placed second in a literary contest organized by the Tranoscius
publishing house with his collection of poems titled \emph{Ruža,} and that he
received the prize money associated with this placement.\footnote{See: \cite[4]{noauthor_vysledok_1946}; \cite[169]{noauthor_radost_1947}.}. This
is where the trail ends for the \emph{Ruža} collection.\footnote{The report
  published in the \emph{Plameň} [Flame] magazine, where Mihálik was
  a contributor, was published together with Mihálik's poem
  \emph{Rovnováha} [Balance], which could indirectly suggest that this
  poem was present in the collection awarded in the contest \mancite\parencite[169]{noauthor_radost_1947}.} The
Tranoscius publishing house only published the work that finished first
in the contest –– nadrealist poet Július Lenko's collection
\emph{Hviezdy ukrutnice} [Cruel Stars] (1947).\footnote{Nadrealism is, succinctly said, the Slovak modification of surrealism. For more information, see \parencite[167--168]{vlasin_slovnik_1977}.} In the following year,
1947, Mihálik debuted with his collection \emph{Anjeli} [Angels] and the
collection \emph{Ruža} fell into oblivion.

Of course, the first step was trying to find the contest documents in
the archive of the Tranoscius publishing house, where wide-scale
shredding had taken place in the 1950s. However, this search was
unsuccessful. In addition to this, steps were taken to search Mihálik's
literary estate for the documents used for his contest submission. The
core of the estate consists of manuscript notebooks where poems are
ordered chronologically (a large part of them also include dates).
However, our search in the estate did not yield any material records
capturing a complete version of the \emph{Ruža} collection, and neither
is there a fragment that would stabilize a specific continuous part of
the collection. The only preserved clue that serves as an indicator of
the collection's form is a list of poem titles for the \emph{Ruža}
collection found at the end of one of these manuscript
notebooks (see Fig. \ref{fig:navratil1}). The notebook is titled \emph{Básne} [Poems] and includes
poems written between 28 September 1944 and 30 March 1946. It can be
assumed that these poems, which are connected to the collection with
certain degrees of likelihood, were scattered among Mihálik's manuscript
notebooks. Moreover, the fact that the list of poems was written at the
end of a notebook, ending with a poem written on 30 March 1946, indicates
that the poems in question are either present in this notebook or in
other notebooks that preceded it chronologically.
  
If the existing historical records of the texts cannot be relied upon, two
options remain: either give up all hope of discovering the collection,
or –– especially considering the aforementioned facts, such as the list
and poems found in the manuscript notebooks –– try to reconstruct it
experimentally despite the risk that it will not be completely identical
to its historical version. At this point, it is relevant to state that
if the preliminary research of the source materials had not offered such
promising opportunities to reconstruct a complete version of the
collection, the editor would not have embarked on such an expensive
project. Without the option to reconstruct the complete collection,
perhaps it would have been simpler to just record its known torso.

So, what can a scholarly editor do in a situation like this, other than
give up all hope of discovering the collection? One can either attempt
to reconstruct the collection in order to create a complete version, or
attempt to present the alternatives that may be considered with regard
to the instructions provided by the list of poem titles, i.e. presenting
a potential reconstruction process. A book is the optimal medium for the
reconstructed, stabilized version of \emph{Ruža}. If the text is to be
offered to general readers, it needs to be stabilized and presented as
a stable structure. The purpose of a book edition of the collection
stabilized by the editor is to offer readers an undisturbed aesthetic
experience of the poetry. The readers are presented with a stabilized
text so that they can perceive the rhythmic, euphonic, imaginative,
idea-related and other characteristics of the work of art in question,
without having to deal with the unstabilized form of the text, its
layers, grouping, etc. This step was taken with regard to the fact that
``{[}t{]}exts as texts depend for readability, and indeed enjoyment, on
their presence and simple, since culturally ingrained, availability in
the materiality of the book'' \citep[XIV]{gabler_foreword_2016}. On the other hand, the
digital environment would be suitable for efforts to present the dynamic
process of reconstruction, indicate connections in the composition,
visualize the editor's individual actions, and enable textual variations.

\section{The constructive role of the editor when reconstructing a lost
collection}

The goal of offering a complete version of \emph{Ruža} raises the
question of what role its editor plays as a textual critic, i.e. how the
editor's role is viewed.\footnote{This is an attempt to formulate an
  answer to the question asked by Kathryn Sutherland during the panel
  session \emph{Reconstructing the text using manuscripts}, 18 March
  2022 (at the ESTS 2022 Conference. Histories of the Holograph: From
  Ancient to Modern Manuscripts and Beyond, 17--19 March 2022, Oxford).}
After the printing press was invented, texts no longer had to be copied
by hand, but they could be typeset and replicated several times. This
invention changed the way texts were distributed and preserved, what
types of mistakes were common, etc. When preparing the text of a modern
work of literature for publication, the editor is first expected to
collect all textual documents representing the work, which will then be
subject to textual criticism. These documents are then used to stabilize
the text. Peter Shillingsburg mentions two significant shifts that have
influenced the approach to contemporary critical editing: ``The first
shift replaced a quest for lost archetypes and unachieved ideals with
practical materiality and a search for order in extant documents. The
second replaced the ideal of original authenticity and of aesthetic
sensitivity with social awareness'' \citep[167]{shillingsburg_how_2010}. However,
in the case of \emph{Ruža,} the option to rely on practical materiality
and social awareness is limited. In the discussed case of Vojtech
Mihálik's collection, records in the form of a complete text of the
collection cannot be used simply due to the fact that they have not been
preserved. The only way of discovering the collection's form at least
partially is to ``assemble'' it using the preserved list. Of course,
this may seem very suspicious to scholars as the editor might be seen as
having excessive influence over the form of the text.

The reconstruction of the \emph{Ruža} collection of poems is a process,
where the collection is assembled from fragments –– in this case the
poems themselves. The sources used are originals, even though they may
contain unintended or non-authorial elements due to transmission. Unlike
textual criticism in classical philology, the goal here is not to find
the hypothetical oldest text, but a hypothetical text from 1946. This
means that there is an attempt to ``freeze'' the text at a certain point
of the poem's development. The documents that ``surround'' the
hypothetical version of the collection are used to create a text that
could be presented to readers from the public. Of course, this approach
is not in line with the common approaches to scholarly editing in modern
literature. However, in every moment of the reconstruction, the editor
relies on the textual records of individual poems. In this case, the
whole collection is an eclectic version. The reconstruction of
\emph{Ruža} is a special case where the editor's intervention in the
stabilization of the text takes on an extreme form, his input in the
stabilization of the text is significant and his role is exceptionally
creative. However, this is not an example of limitless editorial
creativity –– the editorial construct is created within certain limits.

Through this scholarly gesture, the stabilized collection would launch
the social presence of the collection among the public. In addition to
classical and medieval works, literary history includes many modern
works of literature whose reception is based on editorial intervention.
Throughout literary history, it is not uncommon for society to accept an
editorial construct that has become a living part of the literary
tradition. Notable examples include the posthumous editorial efforts to
transform the unorganized notes of Blaise Pascal with the goal of
creating a compact work, as well as an example from the Czech literary
tradition, where poet Jan Neruda's book \emph{Zpěvy páteční} [Friday
Songs] was first stabilized by Jaroslav Vrchlický in 1896, after
Neruda's death. After the death of Franz Kafka, his close friend Max
Brod did not handle Kafka's literary estate in line with the provided
instructions, but rather he assumed the right for the editorial
interpretation of Kafka's works, which influenced the reception and
interpretation of Kafka's works for a long time. In 1977, Christopher
Tolkien –– with the help of Canadian writer Guy Gavriel Kay –– used the
unfinished projects of his father John Ronald Reuel Tolkien to prepare
\emph{Silmarillion} for print.

But what should be done if the researcher only has a list of poems
available rather than the collection in question, and historical records
cannot be used as a starting point? First of all, it is necessary to
assemble the most complete collection of textual documents relating to
the literary work (relevant poem variants, correspondence, etc.),
determine their mutual relationships, find potential violations in the
texts and then attempt to draw all the available information from them.
In each moment of this process, the preserved textual documents need to
be approached critically. Through gradual accumulation, the information
gained this way can be influenced by other, related information, and the
textual corpus of the collection is gradually completed up to a point
when it is no longer possible to rely on evidence. But what then? When
all available reliable clues about the collection have been taken into
consideration, it becomes necessary to do a reconstruction \emph{sui
generis}, which could provide a plausible version of the collection. In
reality, this involves the consideration of all possible options and
then moving within their boundaries.

\section{The reconstruction of \emph{Ruža} (a case
study)}

The reconstruction process consisted of two general steps:
reconstructing the composition of \emph{Ruža} and reconstructing the
poems themselves. In the first step, the corpus of manuscripts was used
to pick the poems that would make up this edition of \emph{Ruža.} The
first decision that needed to be made was how to approach the author's
instructions in the list of poems that was recorded in his manuscript, such as when he crossed out the names
of certain poems, or introduced a sense of uncertainty by adding one or more
question marks behind their titles (as can be seen in Fig. \ref{fig:navratil1}). Since one poem with question
marks is crossed out and other poems with question marks are not, it can
be inferred that the poet first used a question mark to express his
uncertainty about whether a poem should be included in the collection,
then –– as the final step –– he removed it by crossing it out.
Based on this rationale, crossed-out poems are not included in the reconstructed
collection. This includes the poems \emph{Štyri dni exercícií} [Four
Days of Spiritual Exercises], \emph{Ťarcha myšlienky} [The Weight of an
Idea], \emph{Motív helénsky} [A Hellenic Motif] and \emph{Karambol}
[Carom], which are included in the \emph{Appendix}, along with other
poems that were considered at first but later removed during the
reconstruction process. On the other hand, poems marked with a question
mark have their place in the collection.\footnote{Until 1950, Mihálik
  diligently recorded his poems one after another according to the date
  of their creation in notebooks. From these notebooks, he would then
  select and compile collections of poems based on the available
  selection of poems (such as \emph{Ruža}, \emph{Anjeli},
  \emph{Plebejská košeľa} [Plebeian Shirt], etc.), as if the young poet
  were choosing suitable poems for the planned collection from the
  entropy of texts in the early notebooks. From the 1950s onwards, he no
  longer assembled such poetic ``diaries'' from which he would select
  poems and subsequently compose a collection. Instead, he composed the
  collections more consciously, with a relatively clear intention from
  the initial idea, or through its gradual realization, "crystallizing"
  in the creative process towards a specific collection
  (\emph{Spievajúce srdce} [Singing Heart], 1952; \emph{Vzbúrený Jób}
  [Rebellious Job], 1960; \emph{Tŕpky} [She-Sufferers], 1963;
  \emph{Appassionata}, 1964, etc.). Occasionally, he incorporated some
  of his early poems into new poetic projects. In the case of the
  collection \emph{Anjeli}, such a list of poems did not survive in the
  handwritten notebooks. In the case of the collection \emph{Plebejská
  košeľa}, two lists were preserved: the first one unrealized and the
  second one serving as the basis for the published collection. However,
  in the published collection, two poems (\emph{Svedok sa stále vracia}
  [Witness Keeps Coming Back], \emph{Ukrutná pochodeň} [Cruel Torch])
  were omitted, and some poems were rearranged, but these changes were
  not reflected in the manuscript list of poems. The elimination of
  poems and their reordering occurred during the communication process
  with the publishing house, as evidenced by the preserved typescript of
  the collection.}

\begin{figure}
\small
\begin{multicols}{2}
\begin{center}CESTY\end{center}

Údel ??

Kohosi mi to pripomína ?

\sout{Štyri dni exercícií}

\sout{Ťarcha myšlienky ???}

Plánky ?

Hmla

Póly ??

\sout{Vojenská}

Tak nečujne

Či vôbec ?

Mlčanie pod slnkom

Rispetti

Fragment

\sout{Motív helénsky}

\sout{Karambol}

Trýzeň

Sonet (20. marca 1945)

Haničkina cesta do neba ?

Úzkosť

Sonet (23.apríla 1945)

Podvod

Údel

Sonet (6. júla 1945)

Angeloi

Chladná smrť

Úľ

Koncentrácia

Zmätená opera

Žije? Žije

\vspace{1em}

\begin{center}STUHA\end{center}

I, II, III

\vspace{1em}

\begin{center}NÁMESAČNÁ CHLOÉ\end{center}

\newpage

\begin{center}ZADUMANÍ\end{center}

Poklad

Exkluzívna 

Hviezdica

Mlčanie

Sama

Mámenie

Kvitnúca

Nevysvetlila 

Ako vlastne mizla

Spiaca

Melancholická

Katarzis

Svídanie 

Kruh

Ľúbiaci tratí zrak

Teplúčko 

Priznanie

\vspace{1em} 

\begin{center}AJ MOLOCHOVI POD ZUBY\end{center}

Bez milosti

Dezertér

Vojenská

Rovnováha

Chvála zeme slovenskej ???

Mobilizácia

Glorifikácia 

Vojna

Povojnová

Pod rumami klíči

\end{multicols}
      \caption{A transcription of the list of poems recorded in a manuscript notebook in Drahomíra Miháliková's private archive of Mihàlik's unpublished documents. See also footnote \ref{foot:navratil1}.}
    \label{fig:navratil1}
\end{figure}

This narrowed-down list was then used for looking up poems in Mihálik's
manuscript notebooks, where the poet wrote poems chronologically and
included dates. Afterwards, the selected poems were used to form the
content of the collection. Problems arose when working with the author's
poetry cycle titled \emph{Cesty} [Roads]. Most of the poems were easy to
identify because the poems in the handwritten notebooks consistently
have titles. However, there were cases where several different poems in
the notebooks shared the same titles (for example, in the handwritten
notebooks containing Mihálik\textquotesingle s early works, there were
two poems titled \emph{Úzkosť} [Anxiety], ten poems titled
\emph{Rispetto} or \emph{Rispetti}, etc.). If it was impossible to
determine with certainty which of the poems with identical names
belonged in the collection, it was necessary to support the choice with
arguments. For example, it was simple to rule out the poems whose names appeared in
the list, but which were written after 31 July 1946 –– the submission
deadline for the literary contest, serving as the \emph{terminus ante
quem.} These poems included \emph{Úzkosť} written between 16
and 19 February 1948, \emph{Úľ} [The Beehive] from 11 June 1947 and
\emph{Rovnováha} [Balance] from 28 May 1948. The poems \emph{Úzkosť}
written on March 22, 1945, \emph{Úľ} from December 7, 1945, and
\emph{Rovnováha} from April 13, 1945 were therefore included in the collection instead. Three poems
titled \emph{Sonet} [Sonnet] were easy to identify since the list also
provided a date for each of them (and there is no evidence to suggest that Mihálik did wrote any other
sonnet on the given dates). In addition to all this, the \emph{Cesty}
poetry cycle includes as many as two poems with the title \emph{Údel}
[Fate]. Since it can be assumed that Mihálik had read the list of poems
several times before submitting it for the contest (as evidenced by the
crossed-out poems and those with question marks), this duplication
cannot be considered a mistake. The manuscript notebooks include two
poems titled \emph{Údel} (written on 15 August 1943 and 5 March 1945).
However, their order in the collection is not clear from the list.
  
When the list was completed with all the unambiguously identifiable
poems from the \emph{Cesty} poetry cycle, three names in the list
remained uncertain: \emph{Hmla} [Fog], \emph{Rispetti} and \emph{Trýzeň}
[Torment]. Reading the various poems that may be considered for each of
the titles given in the list and doing a stylistic, motivic or thematic
analysis does not provide a safe answer as to which of the uncertain
poems belongs in the collection. The range of motifs and themes
presented in the poetry cycle is too wide, so this criterion is neither
sufficiently relevant nor sufficiently decisive. However, it also turned
out that all the added poems in the cycle were ordered chronologically,
including those that had been crossed out or moved in the original manuscript, such as
\emph{Vojenská} [A Military Poem]. Adding all this information (and ambiguity) to Mihálik's notes from Figure \ref{fig:navratil1} produces the following annotated version of his list of poems:


\begin{quote}
CESTY

Údel, 15 August 1943

Kohosi mi to pripomína, 28 February 1944

Plánky, 23 April 1944

\emph{Hmla, 1943 / Hmla, 3 June 1944}

Póly, July 1944

Tak nečujne, 23 July 1944

Či vôbec, 24 July 1944

Mlčanie pod slnkom, 22 September 1944

\emph{Rispetti}

\setlength\parindent{14pt} \emph{Rispetto}, 28 September 1944 /

\emph{Rispetto}, 29 September 1944 /

\emph{Rispetto}, 1 October 1944 /

\emph{Rispetto}, 6 October 1944 /

\emph{Rispetti}, 21 February 1945 /

\emph{Rispetti}, 21 February 1945 /

\emph{Rispetti}, 21 April 1945 /

\emph{Rispetto}, 6 July 1945 /

\emph{Rispetto}, 26 August 1945
\setlength\parindent{0pt}
Fragment, 9 November 1944

\emph{Trýzeň, 4 March 1945 / Trýzeň, 23 May 1945}

Sonet, 20 March 1945

Haničkina cesta do neba, 21 March 1945

Úzkosť, 22 March 1945

Sonet, 23 April 1945

Podvod, 4 July 1945

Údel, 5 March 1945

Sonet, 6 July 1945

Angeloi, 16 November 1945

Chladná smrť, 21 November 1945

Úľ, 7 December 1945

Koncentrácia, 14 December 1945

Zmätená opera, 25 December 1945

Žije? Žije, 27 December 1945

\hfill

STUHA

I., 17 January 1945

II., 22 January 1945

III., 24 March 1945

\hfill

NÁMESAČNÁ CHLOÉ

30 August –– 7 October 1945

\hfill

ZADUMANÍ

Poklad, 22 November 1944

Exkluzívna, 1 October 1944

Hviezdica, 27 December 1944

Mlčanie, 9 November 1944

Sama, 14 November 1944

Mámenie, 16 December 1944

Kvitnúca, 23 December 1944

Nevysvetlila, 24 December 1944

Ako vlastne mizla, 1 January 1945

Spiaca, 3 January 1945

Melancholická, 5 January 1945

Katarzis, 8 January 1945

Svídanie, 9 January 1945

Kruh, 11 January 1945

Ľúbiaci tratí zrak, 12 January 1945

Teplúčko, 12 January 1945

Priznanie, 13 January 1945

\hfill 

AJ MOLOCHOVI POD ZUBY

Bez milosti, 21 August 1945

Dezertér, 6 June 1944

Vojenská, 22 July 1944

Rovnováha, 13 April 1945

Chvála zeme slovenskej, 24 April 1944

Mobilizácia, 4 May 1945

Glorifikácia, 7 May 1945

Vojna, 21 May 1945

Povojnová, 31 May 1945

Pod rumami klíči, 26 December 1945\footnote{According to the list (after
  the first search for poems and excluding the uncertain ones), the core
  of the \emph{Cesty} poetry cycle as well as the whole collection was
  created between 15 August 1943 and 27 December 1945. The introductory
  poem –– \emph{Údel} [Fate] significantly widens this interval. Without
  it, the core of the \emph{Cesty} poetry cycle would be delimited by
  the poem \emph{Kohosi mi to pripomína} [It Reminds Me of Somebody], written on 28 February 1944.}
\end{quote}


This led to an attempt to apply this chronological principle when making
decisions about poems whose presence in the collection was uncertain. In
the case of \emph{Hmla} and \emph{Trýzeň}, both titles appeared twice.
In each case, however, only one of the poems fit in the time interval
between when the two surrounding poems had been written. For instance,
in the case of the poem \emph{Trýzeň}, it was necessary to choose
between two poems with the same title. Taking a look at the poem that
directly precedes it in the list (\emph{Fragment}) and the poem that
comes directly after it (\emph{Sonet} from 20 March 1945) results in the
following fragment of three poems that appear after one another in the above
list:
\begin{quote}
\emph{Fragment}, 9 November 1944

\emph{Trýzeň, 4 March 1945 / Trýzeň, 23 May 1945}

Sonet, 20 March 1945
\end{quote}
This specific rule applied in the \emph{Cesty} poetry cycle is satisfied
by the poem \emph{Trýzeň} from 4 March 1945, which fulfills the
chronological order of poems (9 November 1944, 4 March 1945, 20 March
1945).

To start off the collection, the chronological approach used in the
\emph{Cesty} poetry cycle led to the selection of the poem \emph{Údel},
which was written half a year earlier than the second poem in the list,
\emph{Kohosi mi to pripomína} [It Reminds Me of Somebody]. The other
poem with the title \emph{Údel,} written on 5 March 1945 and appearing
later in the chronology, is also the proverbial exception to the rule,
disrupting the chronological order of the poems that follow one another
(\emph{Podvod} [Deception], 4 July 1945 –– \emph{Údel}, 5 March 1945 --
\emph{Sonet}, 6 July 1945). With the exception of this poem\emph{,} even
if the list was expanded in both ways, the chronology would remain
intact.

The last remaining issue arose with the title \emph{Rispetti}. It is
uncertain whether Mihálik distinguished that \emph{rispetti} is the
plural form of the Italian word \emph{rispetto} when he wrote poems
named after this Tuscan genre of poetry. On the one hand, his
manuscripts use both forms, which might indicate that he did indeed
distinguish between the singular and plural forms of the word. On the
other hand, his manuscripts include a single rispetto titled
\emph{Najtichšie rispetti} [The Most Silent Rispetti], which makes it
seem that he either did not notice this difference or he did not
distinguish between the two forms at the time of writing this poem.
Although in Slovak poetry it has become a convention to use the term
\emph{rispetti} for a single rispetto as well, this does not
sufficiently explain Mihálik's alternating use of the two forms.
However, if the poet knew about this difference when he was composing
the collection\emph{,} it can be expected that he included several
rispetti in the collection. By the submission deadline for the literary
contest, Mihálik had written nine rispetti titled \emph{Rispetto} or
\emph{Rispetti} (one of the poems titled \emph{Rispetto} was created
after the deadline of the competition).

Selecting only some of the rispetti for the reconstruction would come
with a risk that the final selection would not include any of the
original ones. On the other hand, including all nine rispetti in the
collection might result in the presence of redundant ones, but the
collection would include all rispetti that were present in the original
collection. It could be argued whether it is better to remove integral
parts of the collection or expand it with content that does not belong
in it. This depends on how much it would alter the structure of the
collection to add a poem that was not supposed to be included in it or,
conversely, fail to include a poem that was supposed to be in it. For
now, these questions remain impossible to answer.

One alternative was to include three rispetti titled \emph{Rispetti}, as
their common use of the plural form might indicate that they are
connected. However, the final selection was determined by the principle
of chronological order; The interval between 22 September 1944 and
9 November 1944, which is delimited by the poems \emph{Mlčanie pod
slnkom} [Silence Under the Sun] and \emph{Fragment}, now includes four
rispetti titled \emph{Rispetto} written between 28 September and
6 October 1944. One more thing that connects these four poems is their
appearance in a single issue of the \emph{Akademik} [Academician] magazine. After the reconstruction, the \emph{Cesty} poetry cycle
includes 26 poems. All poems included in the collection originate from
two manuscript notebooks.

\begin{quote}
\emph{Básne} [Poems]

15 August 1943 –– 26 September 1944

116 pages; 10 x 16.5 cm; graph papier

126 poems (out of which 31 are translations)

\hfill

\emph{Básne} [Poems]

28 September 1944 –– 30 March 1946

150 pages; 10 x 16.5 cm; graph papier

143 poems (out of which 34 are translations)

The list for \emph{Ruža}.\footnote{Mihálik\textquotesingle s early works
  from 1939 to 1950 are represented by eighteen notebooks or notepads of
  various sizes, as well as several loose sheets with manuscript poems.
  Nine of these notebooks form a continuous, chronologically arranged
  series based on the creation of the poems (the remaining ones consist
  of dedicatory manuscript collections by the emerging poet or served
  for jotting down sketches, ideas, etc.). The search for poems was
  conducted from this corpus. For a more detailed description of
  Mihálik's manuscript notebooks, see: \cite[17--21]{navratil_pramene_2019}.}
\end{quote}

\noindent The second step was to stabilize the wording of each poem that had
a certain number of variants with alternative wordings, aiming to get
closer to the hypothetical wording of the poem as it was when submitted
for the contest in 1946. It needs to be highlighted that even
a stabilized text in the manuscript is not in itself a guarantee of
a wording identical to the one used for the contest. When the author
copied the poem from the manuscript notebook to the contest submission,
he might have reopened the writing process and chosen a different
wording (even shortly before submitting the poem) without implementing
these changes in the manuscript notebook.\footnote{The poems preserved
  from the editorial offices of the 1940s were in manuscript form. The
  only exception is the composition \emph{Brána} [Gate] from the end of
  1947, which has been preserved in typescript form in the Literary
  Archive of the Slovak National Library in Martin. See: \cite{mihalik_brana_nodate}, \citetitle{mihalik_brana_nodate}.} The editing process for the reconstructed
edition of \emph{Ruža} is based on invariants, as stabilized in the
manuscript. The base text of the poem is selected from a specific
genetic layer of poems with transformations present, or from
a subsequent clean copy of the poem or a variant published in a magazine
before the \emph{terminus ante quem}. Therefore, since the poet only
came out with his book debut in 1947, the book versions of these poems could
not be used as base variants. These could potentially be used for the
verification, correction, and stratification of the genetic layers.

There is no universal determining criterium that could be applied to all
the poems: while some are available as clean copies, some have new
genetic layers incorporated into them, some are only available in
a single (manuscript) variant, others have several variants and the date
of publication comes into play. Not even the clever decision to
automatically use the manuscript version or mechanically restore its
last genetic layer would result in a simple solution: when there are new
genetic layers in the manuscript, it can often be reasonably assumed
that these may have been added after several years, when the poem was
being incorporated in a different collection or anthology, perhaps even
on several different occasions.

Therefore, there sometimes arose a need to directively determine
principles according to which the stabilization of poem wordings was
organized, yet the formulation of individual rules was based on certain
criteria. This way, certain principles gradually arose during the
reconstruction process, aiming to minimize the number of subjective
decisions (for instance: the last version of each poem before the
submission deadline for the literary contest should serve as the base
text, even though this does not automatically mean the variant is
identical to the one that was actually submitted). Some of the general
principles of stabilizing the text were as follows: ``The names of the
poems are accepted in advance as those presented in the list of poems
for \emph{Ruža}, even if they had different names in specific parts of
the manuscript notebook'',\footnote{For instance, the poem recorded in
  the list as ``Sonet (23 April 45)'' was originally written in the
  notebook with the title \emph{Trocha svetla nad vodou} [A Bit of Light
  Above the Water''], later rewritten to \emph{Nesmelý sonet} [A Shy
  Sonnet].} ``If the only available variant is a clean copy from
a manuscript notebook, this will be accepted as the base text without
any issues'', ``If the only available variant comes from a manuscript
notebook, but the text includes new genetic layers, the last genetic
layer will be extracted'', ``If several variants are available
(published in a magazine, as a clean copy, in a book), those that were
verifiably created or published after 31 July 1946 will not be
considered for the base text'', ``If a clean copy or a magazine copy of
a poem is available, and it appears to be the last variant in its chronology
before the contest's submission deadline and it shows a new stage in the
poem's development (or confirms a previous genetic layer in the
autograph), it will be accepted as the base text'', etc. This way, the set of rules
gradually grew throughout the reconstruction process, to cover anything from simple
cases to more complex ones.

A different approach to stabilizing the wording of the poems resulted in
several different groups of poems. The solution was clear if the poem
was only available in one clean-copy variant in the manuscript notebook
(group A) or if the poem was available in a clean copy while the other
variants came from a time period after the submission deadline for the
literary contest (B). A more difficult situation arose when stabilizing
the texts of poems where the original clean copy in the manuscript
notebook included new genetic layers whose date of origin was difficult
to determine and/or the poem had several variants (whose date of origin
was sometimes difficult to determine too).

Therefore, group C consisted of poems whose only available variant came from the manuscript notebook, which included further genetic layers whose date of origin was difficult to determine. Group D consisted of poems with variants from the manuscript notebook, which included further genetic layers whose date of origin was difficult to determine, and other variants that verifiably came from a time period after the submission deadline for the literary contest. In the case of groups C and D, the last genetic layer of the ``notebook'' poem was extracted.

In the case of the poem \emph{Exkluzívna} [Exclusive] (the only poem
in group E), the new genetic layer was removed because it was most
likely added to the manuscript notebook after the contest deadline. In
other cases, copies of the poems from magazines (F) or clean copies
(G)\footnote{Clean copies were mostly preserved in magazine archives.}
were used as the source because they captured the last stage in the
development of these poems before the contest deadline. The gaps caused
by missing sources, which brought about an undecidable situation (H),
were then either filled by the editor's subjective decisions –– as was
the case with the poems \emph{Koncentrácia} [Concentration]
and \emph{Rovnováha} [Balance] –– or the editor applied the unverifiable
assumption that the transformations were added to the manuscript after
a copy of the poem had been sent to the magazine (which probably
occurred after the submission deadline). In these two cases, the wording
of the poems was ``cleared'' of the new genetic layers that had been
added to the clean copy. Finally, after a complete form of the
\emph{Ruža} collection was stabilized, the texts underwent linguistic
proofreading, balancing the pros and cons of Slovak editorial tradition.

Naturally, this editorial construct has its pitfalls. These primarily
arise in those places on which the previous parts of this text have
focused: from a composition perspective, poems from the \emph{Cesty}
poetry cycle were problematic (\emph{Údel}, \emph{Hmla}, \emph{Rispetti}
and \emph{Trýzeň}), and from the perspective of stabilization, most
issues arose with those poems that either contained new genetic layers
in the manuscript notebook or they had several variants (especially if
it was difficult to determine the date when they were written). It
should also be noted that the preserved poem variants do not necessarily
reflect the variants that were actually submitted for the literary
contest since the author may have reopened the writing process when
copying the poems to the contest submission.

Certain questions from the reconstruction process can be asked again:
Why do the first and second poems in the \emph{Cesty} poetry cycle
(\emph{Údel} –– \emph{Kohosi mi to pripomína}) have such a long time gap
between each other (15 August 1943 –– 28 February 1944), while a similar
time gap is not present between any other poems in the cycle? Why does
the second poem titled \emph{Údel} break the chronological order, which
is in conflict with the rest of the whole? Is the chronological
principle in the \emph{Cesty} poetry cycle a sufficiently important
cardinal rule for it to be used for the selection of the poems
\emph{Hmla}, \emph{Rispetti} and \emph{Trýzeň} (as well as the order of
both poems titled \emph{Údel})? Is it possible that the \emph{Ruža}
collection, as submitted for the competition, contained a different poem
titled \emph{Údel} that was not preserved in the manuscript notebooks
for reasons unknown (a question which could also be applied to the other
poem names)? Questions could also be raised in relation to the poems
included in the \emph{Appendix}. Is it possible that some of them are
more suitable for the collection than the ones that have been chosen?
And based on what criterion? Was a suitable approach chosen for the
stabilization of the poems? A problem that remains unresolved is the
unclear reading of two poems: \emph{Teplúčko} [Warmth] and
\emph{Glorifikácia} [Glorification].\footnote{The manuscript of the poem
  \emph{Povojnová} [A Post-War Poem] also included an unclear reading,
  but a clean copy of the poem was preserved in the Literary Archive of
  the Slovak National Library, so this was used as the base text. In
  addition to its difficult/uncertain reading, this poem also had
  mutually competing alternatives where the author either put off his
  decision or did not write it down clearly. Pierre-Marc de Biasi refers
  to this phenomenon as an \emph{unresolved alternative} \citep[77]{de_biasi_textova_2018}.}

If we accept this reconstruction of the collection \emph{Ruža}, we can
relate it to his subsequent work. Surprisingly, not a single poem from
\emph{Ruža} appeared in Mihálik\textquotesingle s debut collection
\emph{Anjeli}. However, there are connections in \emph{Ruža} to the
forthcoming \emph{Anjeli} through angelic motifs inspired by Rainer
Maria Rilke. In the collection \emph{Ruža}, there are poems that later,
often after significant revisions related to Mihálik\textquotesingle s
"conversion" from Catholicism to communism, appeared in other poetry
collections. In his second collection \emph{Plebejská košeľa} (1950), he
included three poems, one poem in \emph{Neumriem na slame} [I Will Not
Die on Straw] (1955), twelve poems in \emph{Sonety pre tvoju samotu}
[Sonnets for Your Loneliness] (1966), and one poem in \emph{Rodisko}
[Birthplace] (1996). Eleven more poems were gradually published in
selected writings or collected works. The remaining twenty-nine poems
from \emph{Ruža} either remained scattered in magazines (thirteen poems
were published in magazines before the competition deadline) or simply
stayed in manuscript form and have not been published yet (twenty-three
poems have not been published so far).

In the collection \emph{Ruža}, Vojtech Mihálik presented himself in a
wide range of poetic positions, both in terms of form (sonnet, rispetto,
regular stanzaic structures, unrhymed poems, Claudelian verse, verse
with relaxed syllabic structure, etc.) and themes (love, anti-war
sentiments, religious themes, social emphasis, reflection on artistic
creation, motifs of death, and more). The subsequent collection
\emph{Anjeli} is characterized by a greater thematic concentration on
the spiritual mastery of the world, as well as the tension between the
transcendent and earthly realms. In this sense, it appears as a
carefully composed selection from a large corpus of
Mihálik\textquotesingle s early poems.

The shift from predominantly spiritual perception of the world to a
focus on social and anti-war themes in the second collection,
\emph{Plebejská košeľa}, was often mistakenly interpreted as
Mihálik\textquotesingle s transition to socialist realism. This deep
discrepancy arose from the absolutization of the spiritual pole of
Mihálik\textquotesingle s work in light of his debut collection
\emph{Anjeli}, which was a deliberately assembled selection that
represents only a fraction of his rich corpus of early poems. Thus, the
collection \emph{Ruža} provides significant testimony to
Mihálik\textquotesingle s poetic development, the diverse nature of his
early poetic works, and the different strategies employed in composing
collections (thematic dispersion vs. thematic concentration).

\section{How to present the dynamic process of
reconstruction}

In the book edition of the work, its diachronic dimension plays a lesser
role. In a way, the editorial construct of Vojtech Mihálik's collection
\emph{Ruža} is an openly editorial interpretation of the documents
comprising or ``surrounding'' the work. However, as an interpretation,
it relies on detailed knowledge of the relevant textual documents. Of
course, the demands on the editor's research activities are by no means
lowered. However, due to the nature of these activities, at certain
moments during the reconstruction, the lack of evidence meant that
subjective editorial decisions were unavoidable. A suitable solution in
cases like this is to create a digital edition that offers an
opportunity to study the editor's process or even compose the collection
anew. Therefore, when selecting a suitable publication platform and
finding a suitable structure for the materials associated with the
\emph{Ruža} collection, it first needs to be answered how to present the
collection's reconstruction to readers so that they are not just relying
on its metatextual scholarly description. Digital technology makes it
possible to provide readers with all the preserved base sources
``surrounding'' the collection, allowing them to follow the dynamics of
individual texts and verify the editor's work.

A digital edition of \emph{Ruža} would present the processes leading up
to the collection's composition, and, for the period after 1946, it would
present the scattered ``fates'' of individual poems after Vojtech
Mihálik removed the collection from his plans. There is tension between
the print and digital editions since they have opposing goals: the goal
of the reconstruction is to use this dynamic material to create a static
form, so the reconstruction goes against the tendency of capturing
a dynamic text. It actually strives to ``freeze'' this movement at
a specific moment so that the readers of the book edition can be
presented with a stable composition of \emph{Ruža}. On the other hand,
the digital edition can present the dynamic constitutive process of
reconstructing the collection. The things that are fixed in the book
edition are set into motion by the digital one, which questions
non-authoritative elements.

The shift to the digital environment is also associated with a paradigm
shift: the reader –– to a certain degree passive against textual
reliability and textual variability –– becomes an active user (whether
as an expert or a layperson). As a result of his scholarly activity, the
editor presents readers with the proposed form of the collection, which
is, naturally, justified through scholarly research. However, readers
have the option to inspect it and ultimately reconstruct it anew. It is
up to the readers to decide whether they will accept the presented form
of the collection or not. They have the option to review the work
presented by the editor since they have almost all the relevant
materials at their disposal –– they need not rely on mere references to
the materials or bibliographic citations; the availability of texts
means that readers are not dependent on blindly trusting the editor's
narrative.

Making the collection available in a digital environment enables users
to work with its form with flexibility. In a way, this allows users to
assume the editor's perspective: it reveals the process of stabilizing
the collection, it offers a choice between several poem alternatives, it
allows users to explore poem variants and genetic layers, while also
enabling them to follow the history of the poems reaching beyond the
\emph{Ruža} collection. The material that the editor had used to
structure the print edition, including the inevitable losses, can now be
organized in the user interface of the digital environment in a new and
more extensive fashion.

The digital edition of \emph{Vojtech Mihálik:
Ruža} is structured around two principal items in its main menu.\footnote{This edition
  meets the attributes that Patrick Sahle lists in his
  definition of a scholarly digital edition \parencite[38]{sahle_what_2016}. The digital edition of
  \emph{Vojtech Mihálik: Ruža} is on a USB flash drive and serves as a
  supplement to the monograph \emph{Neznáme dielo Vojtecha Mihálika:
  Rekonštrukcia zbierky Ruža} [Unknown Work of Vojtech Mihálik: A
  Reconstruction of the Rose Collection] \parencite{navratil_nezname_2021}.} The main item is titled
\emph{EDITION} and the other is titled \emph{The reconstruction of the
collection}. After choosing the latter item of the main menu, the
application gradually guides users through the reconstruction of the
\emph{Ruža} collection (see Fig. \ref{fig:navratil3}). By doing so, the application challenges users to
be active. In an interactive way, the application offers users the
option to try an alternative reconstruction of the \emph{Ruža}
collection. First, the users are provided with a global description of
the topic in question, in which they acquaint themselves with the
sources used for the reconstruction, the historical contours of the
period when the collection was created, the method used to determine the
form of the collection, the two levels of reconstruction (the selection
of texts, the selection of wordings), and the above-mentioned principles of
text stabilization, as well as the user manual. Users can then use this
information to form their own approach to the reconstruction
process.\footnote{Here it needs to be noted that the editor's
  description of his approach to the reconstruction, together with the
  justifications of his decisions may –– despite an effort to
  present the user with the reconstruction process objectively and in an
  incomplete, open manner –– lead the user to simply confirm the
  editor's choices. However, this is an inevitable consequence of the
  editor exploring the topic at hand and his effort to present it to the
  user.}

\begin{figure}[t]
  \centering
  \includegraphics[width=1\textwidth]{media/navratil6.jpeg}
    \caption{Digital edition of \emph{Ruža}.}
    \label{fig:navratil3}
  \end{figure}

In the following step, the user enters a two-step selection mode, where
they select poems using the list and then choose their wordings. The
instructions in the application guide the user through activities that
follow the process through which the editor stabilized the wordings of
the texts in the collection. The user has access to all the information
that the editor had at his disposal, which they can use to make their
own choices.

On this level, the user first gets acquainted with the list of poems and
its disputed elements. For each title in the list, the user gets
a relevant description that should help them make a choice (the
description is partially formalized to make it easier for the user to
navigate it). The user can decide to act in accordance with the list and
reject a poem, but they may also decide to include it in the
reconstruction even though it is crossed out in the list (the option to
``Include despite being crossed out''). Similarly, poems made uncertain
by one or several question marks can be kept in the collection or
omitted from it (the option to ``Omit from the \emph{Ruža}
collection''). The user can also make a variety of decisions in regard
to the other problematic poems (\emph{Údel}, \emph{Hmla},
\emph{Rispetti}, \emph{Trýzeň}): they can choose a different approach
than the editor when determining the order of the two poems titled
\emph{Údel}, they can select any number of the nine available rispetti
(0--9), etc. The description indicates relationships between individual
poems and relevant data that the user needs to make their decision.

After a specific poem has been chosen, the user can explore the
different variants that the editor worked with before settling on
a wording for the edition, as well as the options that the editor
considered when composing the collection (see Fig. \ref{fig:navratil4}). Thus, for each poem, the user
is provided with a description of its editorial stabilization, its
relevant variants and a parallel view function. In addition to the
information provided by the editor, the user can also support their
decision through an exploration of the poem's genesis and the provided
facsimile. Since the application also serves an archival purpose, the
presented poems are diplomatic copies. For each variant that is based on
an autograph, the edition also includes a facsimile or a linear
transcription (a transcription that preserves the original grammar and
spelling and interprets the chronological order of the crossed-out and
newly written text, but does not record the position of this
transformative input or newly written segment); the deleted and added
genetic layers are transcribed using special symbols. For their own
reconstruction, the user can select one of the variants available in the
transcription (including potential modifications) or diplomatic copy
(without any changes), or they can make use of the ``Edit'' option to
modify the text at their own discretion. The user can make
interventions, manually extract a certain layer, or do a linguistic
update of the text (variations in the spelling, a punctuation update,
etc.).

\begin{figure}[H]
  \centering
  \includegraphics[width=1\textwidth]{media/navratil5.jpeg}
    \caption{The poem \emph{Vojna} [War] in the digital edition of \emph{Ruža}.}
    \label{fig:navratil4}
  \end{figure}

The user can explore all the possibilities of the \emph{Ruža} collection
and even combine various wordings. They can go through the individual
levels of a poem's reconstruction (macrocomposition, microcomposition)
until the poem reaches a certain form and is saved into the database
used for the user's own reconstruction of the collection. If at some
point the user wishes to review their selection, they can either clear
the entire database or delete their choice of a specific poem variant.
The user can generate a PDF document of the reconstructed collection at
any point, either while the reconstruction process is still open or at
the end, when a choice has been made from all the available options and
the reconstruction process has finished.

\hypertarget{conclusion}{%
\section*{Conclusion}\label{conclusion}}

Even though this paper presents two ways of presenting Vojtech Mihálik's
lost and reconstructed collection \emph{Ruža}, it needs to be noted once
again that even if all possible combinations of the reconstruction are
exhausted, it is still possible that the process will not yield the
correct combination –– a version of the collection identical to the one
that the author submitted for the literary contest organized by the
Tranoscius publishing house. When copying the text of the collection for
the contest submission, the author may have transformed it on several
levels, including both its microcomposition and its macrocomposition. It
is possible that these changes were not retroactively incorporated in
either the list of poems (which appears less likely) or the wordings of
poems as captured in the author's manuscript notebooks. Therefore, it is
possible that some changes are not captured in any of the preserved
sources. There are several pieces of evidence for this.\footnote{For
  instance, the clean-copy version of the poem \emph{Povojnová} that
  Vojtech Mihálik likely submitted to the \emph{Elán} magazine in Autumn
  1945 (not focusing on the order in which the letters were sent or an
  interpretation thereof) is different from the version preserved in his
  manuscript. Signs of the reopened writing process, which resulted in
  differences between the two poem variants, are not solidified in the
  manuscript. See: \cite{mihalik_skapinajuci_nodate}, \citetitle{mihalik_skapinajuci_nodate}.}

First of all, it is important to acknowledge the editor's share in the
end product, as well as the transparency of his work. When comparing the
reconstructed collection to its lost historical version, \emph{Ruža}, in
its reconstructed form, can serve as an eclectic edition of the
collection. The degree to which the editor's construct ``overlaps'' with
its historical records could only be verified if the version submitted
for the contest was found. Despite these shortcomings, the digital
edition provides perhaps the best answer to the unclear form of the
\emph{Ruža} collection, although so far it can only be presented as an
editorial construct. The construct itself is a compromise made with the
intention of presenting a stabilized version of the collection to
general readers. The reconstruction of this ``lost'' collection and the
literary and historical interpretation of Vojtech Mihálik's work
associated with it, bring new possibilities for interpreting the poet's
personality in relation to the further genesis of his work.

\section*{A note on the text}

The submitted contribution was prepared for the ESTS 2020 conference.
Due to the Covid-19 pandemic, the conference was postponed by two years,
and the time I gained as a result of the pandemic allowed me to expand
the original contribution into a more extensive monograph titled
\emph{Neznáme dielo Vojtecha Mihálika: Rekonštrukcia zbierky Ruža} [The Unknown Work by Vojtech Mihálik: A Reconstruction of the Rose Collection]. The
monograph was published in late 2021 in the Slovak language. The primary
target audience of the monograph is Slovak readers, but the submitted
text condenses the issues discussed in the monograph and is enriched
with necessary contextual information for international readership. I
view the publication of this text as an opportunity to share some of my
findings with the broader international research community.
\\ \\
\textbf{This work was supported by the Slovak Research and Development
Agency under Contract No. APVV-20-0414.}
\\ \\


\begin{flushleft}
    % use smallcaps for author names
    \renewcommand*{\mkbibnamefamily}[1]{\textsc{#1}}
    \renewcommand*{\mkbibnamegiven}[1]{\textsc{#1}} 
\printbibliography
\end{flushleft}

\end{paper}
