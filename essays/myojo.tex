%%%%%%%%%%%%%%
%% METADATA %%
%%%%%%%%%%%%%%

\contributor{
% Add all authors
Kiyoko Myojo
}

\contribution{
% Add full title
A New Approach to Editing Kafka's \emph{Der Process}: Creating a Base Text for Japanese Translation
}

\shortcontributor{
% short version of authors for running header
Kiyoko Myojo
}

\shortcontribution{
% short version of title for running header
A New Approach to Editing Kafka's \emph{Der Process}
}

\begin{paper}
\renewcommand*{\pagemark}{}

\begin{abstract}
% write your abstract here
The existing translations of Kafka's \emph{Der Process} in Japan present a simplified reading text, even when they claim to provide something more extensive. One reason why Japanese translations of the scholarly editions have not been attempted in earnest is that it is not possible to adequately translate some of the variants of the German text into Japanese.

In this article I present some features of a new edition of \emph{Der Process} that will give greater insight into Kafka's writing process. More specifically, I propose an edition that attempts to convey Kafka's écriture, highlighting the fragmentary and unfinished nature of his texts. Additionally, this edition will be translatable into Japanese and be aimed at a non-specialist audience. 

I address some of the difficulties that such an edition would encounter. The principal difficulty is to decide what material should be included. Previous editions have been guided by different ideals, either the ideal of a finished narrative work or the ideal of fidelity to the manuscripts. Neither of these ideals are sufficient for an edition focused on the writing process itself. I argue that material that has traditionally been considered extraneous should be included. 

Finally, I outline a new approach to editing, I call 'third generation editing', that does not aim to establish a canonical edition, but rather to produce alternative ways of editing Kafka.
\end{abstract}

%%%%%%%%%%%%%%%%%%%%%%%%%%%%
%% YOUR ESSAY STARTS HERE %%
%%%%%%%%%%%%%%%%%%%%%%%%%%%%

% remove asterisk (*) if you want to number your sections
% add a title for your section in between the {curly brackets} if you need one
\section{A Japanese Translation of the
Historical-critical Edition?}\label{a-japanese-translation-of-the-historical-critical-edition}

\textsc{In Japan}, a small paperback book has been sold as the official
translation of the historical-critical edition of \emph{Der Process}
{[}\emph{The Trial}{]} since 2009 \citep{kafka_sosho_2009}. But how is it possible
that the official translation of the historical-critical edition is such
a small paperback book? Textual scholars will immediately realize how
strange and impossible that is. However, in Japan, few people have any
conception of the true state of affairs. The back cover of the book
states the following: ``This is a radical new translation that delivers
Kafka's text as it is, based on the latest historical-critical edition
that is fully faithful to the manuscripts''.\footnote{In this article,
  all the translations of Japanese and German texts into English are my
  own, unless otherwise noted.} As any Kafka researcher knows, the
historical-critical edition of \emph{Der Process}, published in 1997,
as part of \emph{Franz Kafka-Ausgabe}, or \emph{FKA}, is a large,
heavy box containing seventeen booklets \citep{kafka_process_1997}. Its appearance
bears no resemblance to this little paperback book at all. What's more,
this little book contains a simple, easy-to-read text without any
textual notes, while the \emph{FKA} provides no reading text, but only
facsimiles of the manuscripts and their diplomatic transcription. The
simple Japanese text is just a translation of the reading text of the
critical edition, not the historical-critical edition. So, to put it
bluntly, the blurb of the paperback book uses the academic term
``historical-critical edition'' for hype alone.

The official translation of the critical edition of \emph{Der Process}
published in 1990 \citep{kafka_proces_1990-1,kafka_proces_1990} has also been sold in Japan
since 2001 \citep{kafka_shinpan_2001}. This book has been accompanied by the following
description: ``People have long awaited a critical edition presenting Kafka's
manuscripts faithfully. Finally, the epoch-making version has emerged
that is as close as possible to what Kafka himself wrote, eliminating
all the parts that were modified by people other than
Kafka''.\footnote{See the advertisement of the end of the book \citep{kafka_shinpan_2001}.} Here once again you can see that the academic term ``critical
edition'' is used as a promotional phrase. It is again dubious as to
whether the book can live up to the blurb, and really be called a
translation of the critical edition. Kafka's critical edition, known as
\emph{Kritische Kafka-Ausgabe}, or \emph{KKA}, is a two-volume set
comprising the ``Textband'' and the ``Apparatband''. Nevertheless, the
Japanese version forms just one single volume in which only the reading
text of the ``Textband'' is translated. Almost none of the variants
found in the ``Apparatband'' are presented. These Japanese translations
of the scholarly editions indicate that Japanese scholars are not
familiar with Western textual theories. This is a very serious
structural issue in the world of Japanese literary research, but I would
prefer to leave this issue for a future article.

In Japan, Kafka is a very popular writer. Besides the previously
described versions, there are also at least eight different translations
of the edition of \emph{Der Process}, edited by Max Brod. Two of these
eight \citep{kafka_shinpan_1953,kafka_shinpan_1966} are still selling as well as the translations
of the two new editions mentioned above. Now let me emphasize the
following: the Japanese translations of \emph{Der Process} have been
giving the reader the false impression that the issue of the reading
text is largely resolved.

All these translations, not just the translations of the Brod edition (which
was first published in 1925) provide a simple, easy-to-read text, which
further strengthens this impression. However, how to correctly determine
Kafka's text is, so to speak, an aporia, for which no definitive
resolution has been found. That is why \emph{FKA}, which goes to the
other extreme, has finally emerged. As mentioned above, \emph{FKA}
presents only facsimiles of the manuscripts and their diplomatic
transcriptions containing no definitive text. That is why this supposedly
official Japanese translation of \emph{FKA} is so egregious in its
simplicity. And this is the reason why there is such a large perception
gap between scholars of Kafka and general readers in Japan about what
Kafka actually wrote.\footnote{I have already discussed the details of
  the Japanese translation situation of Kafka's texts and pointed out
  the issues in the several papers in Japanese (e.g., \cite{myojo_kyokaisen_2012,myojo_honyakukano_2019}).}

\section{The Need for an Alternative
Translation}\label{the-need-for-an-alternative-translation}

Kafka is a very important writer in modern literature. Many people
working in the humanities, not just literary scholars, refer to his
work. In Japan, those scholars who cannot read German, and students who
study the humanities, all learn from Japanese translations of Kafka's
work. But are they really reading what Kafka wrote? What is the essence
of Kafka literature? It is already well recognized that it is
significant to read the deletions, additions, and corrections in his
manuscript, in order to fully comprehend and appreciate the Kafka
text.\footnote{See e.g., \cite{pasley_schrift_1995,schutterle_franz_2002,battegay_schrift_2010,kleinwort_spate_2013}.} One should read Kafka's mysterious works not as
something static and complete, but rather as a dynamic ``écriture'',
which provides an insight as to why most of his works are incomplete,
and as a hint for unlocking the mystery behind their incompleteness.
That is why, in the critical edition, all ``Entstehungsvarianten'' {[}genetic variants{]} are
shown in the ``Apparatband'' \citep[see][161--350]{kafka_proces_1990}. For the same
reason, the facsimile edition has also been published. If so, it is no
exaggeration to say that Kafka's texts cannot be fully understood
without reading the revision process on Kafka's manuscripts.

So how can we bridge that perception gap between scholars of Kafka and
other readers? While I have been trying for years to achieve this goal by pointing
out the significance of the fragmentary, dynamic aspects of Kafka's
texts in my books and essays (e.g., \cite{myojo_atarashii_2002}), only literary scholars
have taken note of these observations, not general readers. Despite
being disappointed with this situation, I could not help but ask myself
the following: if they do realize the discrepancy, but cannot read
German, what should they read? And gradually, I began to think that I
should translate it myself in order to directly present the essence of
Kafka's fragmented writings and reach a wider audience. You may have noticed that I have subtly shifted my discussion of the target audience of my future book from ``general reader'' to ``a wider audience''. That is because I realized that it does not
make sense to create a comprehensive Japanese text for a very general
audience. I mentioned earlier that Kafka is popular in Japan, but he is
not so famous that everyone knows about him. Furthermore, in terms of
expanding the readership base for Kafka, the existing, simple Japanese
translations work well as introductory, gateway-like books that
direct people's attention to him. My potential book should target those
who have already been attracted to Kafka's texts by these existing books
and want to go one step further and understand them better. So by the
term ``a wider audience'', I do not wish to imply those general readers,
but rather the intellectually curious readers with an interest in the
humanities, and readers who have already read the popular translations
that are currently available. In other words, I wish to address those
who endeavor to delve further into Kafka's writing and thought process,
but who are not equipped to jump straight into the complexity of the
scholarly editions.

Thus, I started thinking about editing \emph{Der Process} in order to
produce an alternative translation. Why should I work on editing first,
and not translating? This is because there is no appropriate base text
that can be translated into Japanese. The two existing scholarly
editions cannot be the basis of translation for several reasons. First, given the major differences between the Japanese and German
writing systems, it is next to impossible to faithfully replace
\emph{FKA}'s diplomatic transcription with Japanese. Secondly, it is
also impossible to faithfully reproduce the German variants,
lemmatically presented in the separate volume, in Japanese. Furthermore, even
if it were possible linguistically, I think it would be better not to
adopt a method like \emph{KKA}, that separates text and variants, as I
believe that this makes the book very challenging to read and
appreciate, and thus may exclude that ``wider audience'' that my edition
is aiming for. Unless you are a researcher, you rarely read two books
simultaneously –– such as comparing the ``master'' volume to the
alternative variants presented in the other. So, if I want to show the
work-in-progress aspect of Kafka's writing, one of the problems I have
to consider is how to incorporate the ``Entstehungsvarianten'' into the
text. This is a big hurdle that is very difficult to overcome, which
requires complicated work. First, from the huge
number of variants presented in \emph{KKA} and \emph{FKA}, I have to
select the ones that have been recognized as important in terms of
interpretation and/or the ones that are reproducible in Japanese from
the viewpoint of the linguistic system. Of course, the requirements to
define the selection criteria of which variants to include, and
which to exclude, is another hornet's nest entirely.

And, once that minefield is cleared, I then have to decide on a
presentation method for these variants. That is, how should I illustrate
the functional differences in these amendments (i.e., which words have
been deleted and which words have been added) when incorporating them
into the base text? What kind of expression method should be used?

This is also an important problem, but in the case of \emph{Der
Process}, there is a more primary and fundamental issue than those
regarding the presentation of the revision process, namely that of
compressing the materials into one book. As mentioned earlier, the
facsimile edition of \emph{Der Process} no longer forms a single book,
but rather sixteen booklets containing facsimiles and transcripts of the
manuscripts and one of commentary in a single box. This mode of
presentation reveals the consensus that there is no single ``correct'' way
to order the parts of \emph{Der Process}. Kafka left behind more than a
dozen bundles of loose-leaf notes, each demarcated in their own covers,
but with little sense of order beyond that. Brod, when editing the
unfinished work after Kafka's death, considered each of those bundles as
a ``chapter'' and arranged them in the order of the story he inferred.
As is well known to Kafka researchers, the chapter arrangement of
\emph{Der Process} has been repeatedly discussed and disputed since the
1950s. The chapter arrangement used in the critical edition, which was
issued about forty years after the discussion began, was severely
criticized for its lack of differentiation from the Brod edition. It
seemed we had eschewed any progress in this regard and returned full
circle. Thus, the solution provided by \emph{FKA} to give up
establishing the chapter arrangement was regarded as, at least, an
objectively correct one.

\section{The Third Methodology of The Parts
Arrangement}\label{the-third-methodology-of-the-parts-arrangement}

It should be noted here that, with the exception of the facsimile
edition, all past discussions of the chapter arrangement, so far, have
been made in terms of maintaining the consistency of the story. The Brod
edition and the critical edition follow the narrative principle. They
prioritize the construction of a resolved story. On the other hand, the
facsimile edition follows another policy, the so-called
``Schriftträgerprinzip''. This principle privileges fidelity to the
source materials, and foregoes any assumptions pertaining to order. In
terms of materiality, those stacks of paper should be regarded as parts,
not chapters, with no definitive order.

So, what is an alternative to the existing two methodologies? The
conclusion I came up with, after much reflection, was to put these
``parts'' in the chronological order in which Kafka wrote them, i.e., in
the order in which the texts were generated. As mentioned earlier, the
importance of understanding Kafka's writing process is well recognized,
and is part of the reason we delve so deeply into his works. And so, by
presenting the work in this form, we can provide a great illustration
of his unique ``écriture''.

Forty years ago, Gerhard Neumann summarized the legitimacy of any
editing policy of Kafka's work as follows:

\begin{quote}
Die editorische Entscheidung kann in jenem Fall nur einer der beiden
Möglichkeiten gerecht werden. \emph{Entweder} wird sie sich auf die
Wiedergabe und Rekonstruktion der ``écriture'' als eines potentiell
unendlichen \mbox{Selbstverfertigungsprozesses} eines sich im Schreiben
erarbeitenden Subjekts konzentrieren; \emph{oder aber} sie muß sich, als
zweite Möglichkeit editorischer Entscheidung, die Herauspräparierung
eines definitiven ``Werks'' im Sinne der ästhetischen Kanonbildung zur
Aufgabe machen, eines Werks, das einen Trümmerhaufen mißratener Ansätze
und verworfener Varianten unter sich begräbt. 

\begin{flushright}
    \parencite[161; emphasis in original]{neumann_werk_1981}
\end{flushright}

{[}In that case, the editorial decision can only do justice to one of
the two possibilities. \emph{Either} it will focus on the reproduction
and reconstruction of ``écriture'' as a potentially endless process of
self-production by a subject establishing himself through writing;
\emph{or}, as a second possibility for an editorial decision, it must
set itself the task of preparing a definitive ``œuvre'' in the sense of
the formation of an aesthetic canon –– an œuvre that buries under itself
a heap of ruins of failed approaches and discarded variants.{]}
\end{quote}

Neumann talks about the case of Kafka's ``Oktavhefte'', but this view
applies to all of Kafka's writing. Both the Brod and critical editions
are based on the latter kind of editorial judgment. The facsimile edition, on the other hand, seems to be based on the former principle.
However, it is a faithful reproduction of the physical material, not the
``écriture''. So, one could say that the editorial possibilities of this
former type have yet to be explored. Thus I have chosen, as my elusive
third option, to delve into that very exploration.

If the policy of the arrangement is to reproduce the writing process,
the first thing to work on is to determine, to the best of our ability,
the order in which the texts are written.

Both the \emph{KKA} editor Malcolm Pasley, and the \emph{FKA} editor
Roland Reuß share the same view on the following point: Kafka firstly
wrote the ``Verhaftung'' {[}Arrest{]} part and the ``Ende''
{[}End{]} part, which are, according to Pasley, ``in großer zeitlicher Nähe
zueinander'' {[}close to each other in terms of time{]} \citep[111]{kafka_zur_1990}.

This makes it difficult to arrange the parts and to reconstruct the
chronology of his writing. Which part was written first, ``Verhaftung'' or ``Ende''?
Pasley indicates that ``Verhaftung'' would have been written earlier by citing some features on the manuscript as evidence of this. Reuß
questions Pasley's view, and speculates that those two parts would have
been written ``gleichzeitig'' {[}at the same time{]} \citep[6]{kafka_zur_1997}.
As those two parts were written in separate notebooks, it makes sense to
infer as such. Pasley probably thinks so as well, as far as I can ascertain
from his writing, although he does not explicitly mention it. As Reuß
himself notes when using the expression ``gleichzeitig'', it is
impossible for these parts to have been literally written
simultaneously. More precisely, we should think of it as a zigzag
movement back and forth between the two pieces: writing some sentences
of one, then switching to the other. So there is a need to infer the
order of this zigzag writing from the state of the handwritten
manuscript.

I have already considered this issue and published the results in a
Japanese paper \citep[see][]{myojo_taiho_2021}. In this paper I've provided some
speculation about how the zigzag writing was done, and presented in what
order those parts would be arranged in my future edition. Note that the
``parts'' I have used at this time have different boundaries from those
bundles in the facsimile edition. And they are also demarcated
differently from the chapters in the critical edition. I will continue
to provide further updates on my progress in this matter,\footnote{\label{myojo:fn:ai}At the time of writing this article (spring of 2022), I was still focusing on traditional methods for establishing chronology, which amounts to educated guesses. But I am currently investigating the use of generative AI to more accurately determine chronology. This new research is being undertaken with Yasuhiro Sakamoto, a researcher in Bildwissenschaft and computer science.} but will leave
that for a future paper, as there is a more important and fundamental
issue to address: the beginning of the writing process.

\section{ What is \emph{Der
Process}?}\label{what-is-der-process}

It's not just the chapter unit that is newly demarcated in my edition.
What constitutes the entire story of \emph{Der Process} is also
demarcated. Like Pasley and Reuß, I also infer that ``Verhaftung'' and
``Ende'' were written at the same time, but probably ``Verhaftung''
first. If so, will the arrest scene be the beginning of my edition of
\emph{Der Process}? No. A better question would be: should my
edition limit itself to the texts that are currently subsumed under the
title of \emph{Der Process}, or should it include other texts as well?

I ask this question for the following reason: in the past, when
specialists have discussed chapter arrangement, they have also discussed
a text that is not included in the Brod edition, \emph{KKA}, or
\emph{FKA}. This text is a work known by the title of ``Ein Traum''
{[}A Dream{]}. It is a small piece published in the collection of
short stories \emph{Ein Landarzt} {[}\emph{A Country Doctor}{]} while
Kafka was still alive. The name of the main character in the work is
``Josef K.'' Some specialists have argued that ``Ein Traum'' should be
considered as part of \emph{Der Process} (e.g., \cite{binder_kafka-kommentar_1976}), and, as
such, this differs from the traditional view proposed by the editors
mentioned above.

It is clear why the facsimile edition, \emph{FKA}, doesn't include this
text: it was edited using the ``Schriftträgerprinzip''. The extant
``Schriftträger'' of \emph{Der Process} is a set of folios kept in the
German Literature Archive (DLA) in Marbach. It is the group of folios
that Kafka himself handed over to Brod before his death. This set does
not include the manuscript of ``Ein Traum'', as it had already been lost
in the process of its previous publication.

So why doesn't the Brod edition include the text of ``Ein Traum''? Brod
himself claims that ``Ein Traum'' is not essentially related to the plot
of \emph{Der Process} in the postscript of the first edition of the
work.

\begin{quote}
Die unvollendeten {[}Kapitel{]} lasse ich für den Schlußband der
Nachlaßausgabe zurück, sie enthalten nichts für den Gang der Handlung
Wesentliches. Eines dieser Fragmente wurde vom Dichter selbst unter dem
Titel ``Ein Traum'' in den Band ``Ein Landarzt'' aufgenommen.

\begin{flushright}
    \parencite[323]{kafka_prozess_1946}
\end{flushright}

{[}I leave the unfinished {[}chapters{]} for the final volume of the
posthumous edition, they contain nothing essential for the course of the
plot. One of these fragments was included by the poet himself in the
volume ``A Country Doctor'' under the title ``A Dream''.{]}
\end{quote}

The reason why \emph{KKA} doesn't include that text is similar. Pasley
claims in the ``Apparatband'' of the critical edition that Kafka himself
does not associate the two works.

\begin{quote}
Auch der handschriftlich nicht überlieferte Text ``Ein Traum'', den Kafka
1916 veröffentlicht hat, hat einen Protagonisten mit dem Namen ``Josef
K.'' . Kafka bringt das kleine Prosastück –– wie er es in einem Brief an
Felice Bauer [\ldots] nennt –– jedoch weder hier noch irgendwo sonst
mit dem ``Proceß''-Roman in Verbindung. 

\begin{flushright}
    \parencite[73]{kafka_zur_1990}
\end{flushright}

{[}The text that has not survived in manuscript, ``Ein Traum'', which
Kafka published in 1916, also has a protagonist named ``Josef K.''
However, Kafka does not link the small piece of prose –– as he calls it
in a letter to Felice Bauer [\ldots] –– either here or anywhere else
with the ``Proceß'' novel.{]}
\end{quote}

But their claims are highly contestable. Whether or not the texts are
related to each other should be judged by interpretation. And the past
debate about the chapter arrangement shows that interpretations
differing from Brod and Pasley can also be valid.

I believe that ``Ein Traum'' is related to \emph{Der Process} and thus I
will include it in my future edition.\footnote{I have touched upon the subject of the relationship between ``Ein Traum'' and \emph{Der Process} in several of my previous Japanese publications, most recently in \citet{myojo_kafka_2015}, highlighting the significance of the connection.} This means that I will redefine
the work of \emph{Der Process} itself.

What is \emph{Der Process}? As mentioned above, I am trying to edit it
not as ``Werk'' {[}work{]}, but ``Schrift'' {[}writing{]}. In other words, I am working on what
is, again according to Neumann, ``die Wiedergabe und Rekonstruktion der
`écriture' als eines potentiell unendlichen Selbstverfertigungsprozesses
eines sich im Schreiben erarbeitenden Subjekts'' {[}the reproduction
and reconstruction of ``écriture'' as a potentially endless process of
self-production by a subject establishing himself through writing{]}
\citep[161]{neumann_werk_1981}.

If I consider \emph{Der Process} to be ``Schrift'' rather than ``Werk'',
then the most important factor in determining its frame is the
timeframe. This means that I first have to tackle the following issue:
Which period of ``écriture'' that is ``potentiell unendlich'', in Neumann's
reasonable expression, can be regarded as the \emph{Der Process} period?
That is, where is the beginning of my edition of \emph{Der Process} to
be located without sticking to the conventional frame? It is well known
that the name ``Josef K.'' appears for the first time in a fragment
written on July 29, 1914. This fragment is found in the volume of
\emph{Tagebücher}, not in \emph{Der Process}, in the
critical edition. Also, of course, it is not included in the facsimile
edition of \emph{Der Process}. The text is as follows:

\begin{quote}
Josef K., der Sohn eines reichen Kaufmanns, ging eines abends nach einem
großen Streit den er mit seinem Vater gehabt hatte –– der Vater hatte
ihm sein liederliches Leben vorgeworfen und dessen sofortige Einstellung
verlangt –– ohne eine bestimmte Absicht nur in vollständiger
Unsicherheit und Müdigkeit in das Haus der Kaufmannschaft, das von allen
Seiten frei in der Nähe des Hafens stand. Der Türhüter verneigte sich
tief. Josef sah ihn ohne Gruß flüchtig an. ``Diese stummen
untergeordneten Personen machen alles, was man von ihnen vorausetzt''
dachte er. ``Denke ich, daß er mich mit unpassenden Blicken beobachtet so
tut er es wirklich.'' Und er drehte sich nochmals wieder ohne Gruß nach
dem Türhüter um; dieser wandte sich zur Straße und sah zum
wolkenbedeckten Himmel auf. 

\begin{flushright}
    \parencite[666--667]{kafka_tagebucher_1990}
\end{flushright}

{[}Joseph K., the son of a rich merchant, one evening after a violent
quarrel with his father –– his father had reproached him for his
dissipated life and demanded that he put an immediate stop to it ––
went, with no definite purpose but only because he was tired and
completely at a loss, to the house of the corporation of merchants which
stood all by itself near the harbour. The doorkeeper made a deep bow,
Joseph looked casually at him without a word of greeting. ``These silent
underlings do everything one supposes them to be doing,'' he thought. ``If
I imagine that he is looking at me insolently, then he really is.'' And
he once more turned to the doorkeeper, again without a word of greeting;
the latter turned towards the street and looked up at the overcast
sky.{]} 

\begin{flushright}
    \parencite[257]{kafka_diaries_1975}
\end{flushright}

\end{quote}

In this fragment, a man named Josef K. appears as the main character,
which is clearly regarded as the beginning of a story. The appearance of
this man is quite different from that of Josef K. of \emph{Der Process}.
The Josef K. of this fragment is an idle son who lives with his father, a wealthy merchant, while Josef K. of \emph{Der Process} is a man
living alone who works at a bank. However, the fragment also features
the ``Türhüter'' {[}gatekeeper{]}, an important figure that appears
in \emph{Der Process}. In the ``Apparatband'' of the critical edition,
you can find the following passage by Pasley:

\begin{quote}
Im gleichen Tagebuchheft [\ldots] findet sich unter dem Datum des 29.
Juli, als erstes Zeugnis der neuen Schaffensperiode, das Erzählfragment
``Josef K. [\ldots] Himmel auf'' (KKAT 666f.), das abgesehen von der
Namenidentität auch gewisse thematische Beziehungen zum Anfang des
Romans aufweist. Der Name ``Josef K.'' wird hier durch die Korrektur
(Hans Gorre \textgreater{} Josef K.) zum erstenmal eingeführt.

\begin{flushright}
    \parencite[73]{kafka_zur_1990}
\end{flushright}

{[}In the same diary notebook [\ldots] dated July 29, the first
testimony of the new creative period is found in the narrative fragment
``Josef K. [\ldots] Himmel auf'' (KKAT 666f.), which apart from the identity of
the name also contains certain themes related to the beginning of the
novel. The name ``Josef K.'' is introduced here for the first time
through a correction of names (Hans Gorre \textgreater{} Josef K.).{]}
\end{quote}

Pasley regards the date on which that fragment was written as the
beginning of the ``neue Schaffensperiode'', and considered it to be
related to \emph{Der Process}. The \emph{FKA} editor Reuß also pays
attention to the fragment:

\begin{quote}
Am 29. Juli führte Kafka in ihm in einem Erzählansatz den Namen ``Josef
K.'' (als Änderung eines zuerst erwogenen ``Hans Gorre'') ein.
Thematische Überschneidungen dieses Entwurfs mit dem ``Process''-Kontext
sind allenfalls vage vorgezeichnet, so daß man diesen Erzählansatz kaum
als Vorarbeit zum ``Process'' auffassen kann. 

\begin{flushright}
    \parencite[4]{kafka_zur_1997}
\end{flushright}

{[}On July 29, Kafka introduced the name ``Josef K.'' (as a modification
of an initially considered ``Hans Gorre'') in a narrative approach. The
thematic overlaps of this draft with the ``Process'' context are at most
vaguely sketched out, so that this narrative sentence can hardly be
understood as a preparatory work for the ``Process''.{]}
\end{quote}

Reuß also acknowledges the possibility of ``Thematische
Überschneidungen'', although he considers the relevance to be
``allenfalls vage''. I believe this fragment has a strong connection to
\emph{Der Process} and, like Pasley, regard it as a sign of the start of
Kafka's new creative period. Therefore, I am going to make it the
beginning of my future edition. This conviction has been further
strengthened since I saw the handwritten manuscript of this fragment
stored in the Bodleian Library at Oxford University. This fragment
starts from the top left of folio 16r in ``MS Kafka 9'', that is, from
the top of the new page. At the bottom of the page just before that,
namely, at the bottom of folio 15v, a thick and dark horizontal line is
drawn, which makes us feel the writer's intention to make a clear break.
From such an aspect of the manuscript, it seems to me that Kafka decided
on something and then, pausing his writing so far, tried to
start a new story.

\section{Appearance of the Name Josef
K.}\label{appearance-of-the-name-josef-k.}

This alternative brings a new approach to the reading comprehension of
the work. Immediately after that fragment in Kafka's manuscript, we find
another fragment that I believe is also connected to \emph{Der Process}.
This fragment as a printed version can be read in the \emph{Tagebücher}
volume of \emph{KKA}.

\begin{quote}
Ich war ganz ratlos. Noch vor einem Weilchen hatte ich gewußt, was zu
tun war. Der Chef hatte mich mit ausgestreckter Hand bis zur Tür des
Geschäftes gedrängt. Hinter den zwei Pulten standen meine Kollegen,
angebliche Freunde, die grauen Gesichter ins Dunkel gesenkt, um den
Gesichtsausdruck zu verbergen. ``Hinaus'' rief der Chef, ``Dieb! Hinaus!
Ich sage: Hinaus!{}'' ``Es ist nicht wahr'' rief ich zum hundertsten mal
``ich habe nicht gestohlen! Es ist ein Irrtum oder eine Verläumdung!
Rühren Sie mich nicht an! Ich werde Sie klagen! Es gibt noch Gerichte!
Ich gehe nicht! Fünf Jahre habe ich Ihnen gedient wie ein Sohn und jetzt
werde ich als Dieb behandelt. Ich habe nicht gestohlen, ich habe nicht
gestohlen, hören Sie doch um Himmelswillen, ich habe nicht gestohlen.''

\begin{flushright}
    \parencite[667]{kafka_tagebucher_1990}
\end{flushright}

{[}I was in great perplexity. Only a moment ago I had known what to do.
With his arm held out before him the boss had pushed me to the door of
the store. Behind the two counters stood my fellow clerks, supposedly my
friends, their grey faces lowered in the darkness to conceal their
expressions. ``Get out!'' the boss shouted. ``Thief! Get out! Get out, I
say!'' ``It's not true,'' I shouted for the hundredth time; ``I didn't
steal! It's a mistake or a slander! Don't you touch me! I'll sue you!
There are still courts here! I won't go! For five years, I slaved for
you like a son and now you treat me like a thief. I didn't steal; for
God's sake, listen to me, I didn't steal.''{]} 

\begin{flushright}
    \parencite[298]{kafka_diaries_1975}
\end{flushright}

\end{quote}

In this fragment, the narration is from the first-person perspective, so
we don't have a character name to help us identify this character, as it
is the case with Josef K. in \emph{Der Process}. However, the action has
an affinity to a key theme in \emph{Der Process}. In the fragment, the
narrator recalls an event where he is accused of stealing money by his
boss, despite his pleas to the contrary. This fragment then, however,
regresses through the protagonist's state of denial and automated
response mechanisms, to the acknowledgement of his own wrong doing, and
it ends in the middle of his inner perplexity as follows:

\begin{quote}
Und nun war ich ratlos. Ich hatte gestohlen, hatte aus der Ladenkasse
einen Fünf-Gulden-schein gezogen, um abends mit Sophie ins Teater gehen
zu können. Sie wollte gar nicht ins Teater gehen, in 3 Tagen war
Gehaltauszahlung, dann hätte ich eigenes Geld gehabt, außerdem hatte ich
den Diebstahl unsinnig ausgeführt, bei hellem Tag, neben dem Glasfenster
des Kontors, hinter dem der Chef saß und mir zusah. ``Dieb!{}'' schrie er
und sprang aus dem Kontor. ``Ich habe nicht gestohlen'' war mein erstes
Wort, aber die Fünfguldennote war in meiner Hand und die Kassa war
offen. 

\begin{flushright}
    \parencite[668]{kafka_tagebucher_1990}
\end{flushright}

{[}And now I was in a quandary. I had stolen, had slipped a five-gulden
bill out of the till to take Sophie to the theatre that evening. But she
didn't even want to go to the theatre; payday was three days off, at
that time I should have my own money; besides, I had committed the theft
stupidly, in broad daylight, near the glass window of the office in
which the boss sat looking at me. ``Thief!'' he shouted, and sprang out of
the office. ``I didn't steal,'' was the first thing I said, but the
five-gulden bill was in my hand and the till open.{]} 

\begin{flushright}
    \parencite[298]{kafka_diaries_1975}
\end{flushright}

\end{quote}

This fragment was also written on July 29 1914. I will include it in my
edition of \emph{Der Process}, too, because it also has a strong
connection with \emph{Der Process}. This fragment, like \emph{Der
Process}, deals with the issue of sin, and awareness of sin. And they
also have something in common in that the hero's sin is committed
misguidedly from the relationship with the woman he is dating. Elias
Canetti has proposed the now widely accepted interpretation that the
hero's sin depicted in \emph{Der Process} can be closely related to the
author's own abandonment of his engagement with Felice Bauer at the time
of writing \citep[see][]{canetti_andere_1969}. Kafka was engaged to her on June 1, 1914,
after they had been dating (mainly by letter) for about two years. But
on July 12, he ended their engagement.

It is well known that the protagonist's name –– Josef K. –– bears a
strong resemblance to that of the author (\cite[183]{sokel_franz_1976}; see also \cite[195]{binder_kafka-kommentar_1976}). The suggestion of that identity was
deliberately brought about by Kafka, as evidenced by the change of the
name used in the manuscript. Both Pasley and Reuß brought attention
to the fact that the name ``Josef K.'' was originally written as ``Hans Gorre''. As has
already been pointed out, Hans Gorre is also a possible pseudonym of the
author's \citep[33]{kurze_kafkas_2016}. The name Gorre is a cypher of Kafka in terms
of the number of characters, and the arrangement of consonants and
vowels. It's the same kind of encryption that Kafka uses for the names of the protagonists in
his previous novels: Samsa in \emph{Die
Verwandlung} {[}\emph{The Metamorphosis}{]} and Bendemann in \emph{Das
Urteil} {[}\emph{The Judgement}{]}. However, he changed Gorre to Josef K., a name
more directly reminiscent of Kafka's own name. And Josef K. is
also strongly associated with Kaiser Franz Josef,
the then-emperor of the Austro-Hungarian empire where Kafka lived.
Again, the name Josef K. first appeared in the fragment written on July
29th. This date is important, because July 29, 1914 was a fateful day
for the Habsburg Empire. On July 28 the empire declared war on Serbia,
and on the 29th, Emperor Franz Josef issued a notice of the start of the
war. The timing of when Franz Kafka decided to start writing the story
of Josef K.'s sin correlates perfectly with the timing of Kaiser Franz
Josef's cataclysmic declaration –– one that would change the course of
world history. I do not think this is just a coincidence.

If we read Kafka's writings starting from this fragment of Josef K., we
will read the scene of the arrest in a completely different light. Thus,
the fragment of Josef K., the son of a wealthy merchant, is immediately
followed by the fragment that is narrated in the first-person (where he
steals but protests innocence). Readers of my future edition of
\emph{Der Process} will read the following famous sentence after reading the
fragments discussed above.\footnote{However, the arrest scene was not written
  immediately after those fragments. Estimating when it was written is
  an issue that should be reconsidered, and this issue is related to how
  many texts were written between those fragments and that famous
  sentence, and which ones to select from. I will eventually tackle the
  issue.}

\begin{quote}
Jemand musste Josef K. verleumdet haben, denn ohne dass er etwas Böses
getan hätte, wurde er eines Morgens verhaftet. 

\begin{flushright}
    \parencite[7]{kafka_proces_1990-1}
\end{flushright}

{[}Someone must have slandered Josef K., for one morning, without having
done anything wrong, he was arrested.{]} 

\begin{flushright}
    \parencite[5]{kafka_trial_2009}
\end{flushright}

\end{quote}

Is the narrator telling the truth?\footnote{The unreliability of Kafka's narrator has been explored in the literature, and is a significant issue that is relevant, not just to this novel, but to many of Kafka's works \citep[see][]{myojo_kafka_2014}.}

\section{The Third Generation of
Editing}\label{the-third-generation-of-editing}

As you may have noticed, my judgment regarding the inclusion of texts
means opening Pandora's box. If the fragments that are currently
classified as diary entries are included, the demarcation process will
be almost out of control. My proposed criterion for inclusion is whether
a text is related to \emph{Der Process}. If we follow this extremely
vague criterion, it could pertain to a broad range of Kafka's writings.

Many researchers have already pointed out that a lot of Kafka's works,
which he wrote in parallel while he was writing the story of Josef K.,
are closely related to \emph{Der Process} and to each other. They
include the short story \emph{In der Strafkolonie} {[}\emph{In the Penal
Colony}{]}, which Kafka published in his lifetime, and several novel
fragments, like ``Teater von Oklahama [sic]'' {[}``Theater of
Oklahama''{]}, which is clearly regarded as a chapter for the unfinished novel
\emph{Der Verschollene} {[}\emph{The Man who Disappeared}{]},
``Erinnerungen an die Kaldabahn'' {[}``Memories of the Kalda
Railway''{]}, which is presented in the \emph{Tagebücher} volume in
\emph{KKA}, and ``Der Unterstaatsanwalt'' {[}``The Assistant
Prosecutor''{]}, which is presented in the \emph{Nachgelassene
Schriften und Fragmente I} {[}\emph{Posthumous Writings and Fragments
I}{]} volume in \emph{KKA}. Christian Kurze aptly calls them
``Satellitentexte'' {[}satellite texts{]} of \emph{Der Process} and
makes a legitimate claim that they reveal what Jost Schillemeit calls
``eine innere Biographie von Kafkas Schreiben'' {[}an inner biography
of Kafka's writing{]} (\cite{kurze_kafkas_2016}; see also \cite[224]{schillemeit_unterbrochene_2004}). In
this regard my editing attempt can be considered an attempt to make that
biography readable. However, needless to say, the realization of this
attempt is extremely difficult. In addition to some of the major hurdles
already mentioned above, there is an even more fundamental challenge:
which ones to select from the many texts related to \emph{Der Process}
and how to present them? If I aim at a wider audience, this
choice and presentation is very important. And all the decisions
concerning them have to be entirely based on my interpretation.

You might think that what I am proposing is an edition that eschews the
standard approach to scholarly editing, and you would be correct. I am
trying to create a version full of my interpretations in every detail and I believe that only this can bring the essence of Kafka's
writings to Japanese readers as \emph{KKA} and \emph{FKA} try to convey.
The text I edit is by no means academically orthodox: it is just an
alternative text. However, it is a text that can only be produced by a
scholar.

I call this alternative form of scholarly editing ``the third generation
of editing''.\footnote{I have already published a Japanese paper on the
  concept of third-generation editing, but here I will only give the
  outline. For more details of the concept, see \cite{myojo_daisansedai_2021}. A related paper in German will be published in the near future \citep[see][]{myojo_editionspraxis_forthcoming}. What I did
  not mention in those articles is that third-generation editing is
  similar to copy-text editing, specifically in terms of eclectic
  composition. However, the former differs from the latter in that it
  aims only to create alternative texts, not authoritative texts. I will
  not expand on this point here because it lies outside the scope of the
  current essay.} The first generation of editing is the phase when
texts aim to popularize the writer's work in order to establish it among
the general readership. In the case of Kafka, it was the time when Brod
was editing his texts. The second generation is when scholarly
editing produces a text as the basis of research, and in the case of
Kafka, the time when KKA and FKA were produced. Third-generation editing
is editing that can be performed on the assumption that texts produced
in those two generations already exist. There are many possible forms of
text produced in this third generation. The new type of scholarly
editing that I am currently working on, with the aim of creating a base
text for translation into Japanese, is just one of these possibilities.

Last but not least, my editing project is still in its infancy. I have
invited young researchers to form a team to work on this project and we
have already published four papers in Japanese on this project,
including co-authored papers (see e.g., \cite{myojo_honyakukano_2019,myojo_taiho_2021}).
There is still a long way to go.\footnote{After the submission of this article, another co-authored paper was published \citep[see][]{myojo_grubach_2022}. In addition, as mentioned in footnote \ref{myojo:fn:ai}, another young researcher specialising in information science has recently joined the team. As noted there, recent developments in AI technology have transformed this project.}

The reason why I am now writing about this project at such an early
stage is that I am too keenly aware of the tremendous difficulty of
achieving it. We will continue to strive towards our lofty ambition, but
to be honest, in the face of the huge challenges ahead, we are afraid
that it may end up as a thought experiment. Even so, I do not
consider the thought experiment to be worthless. The dilemma of
translating any scholarly edited ``untranslatable'' texts, such as
\emph{Der Process}, is not limited to Japan.\footnote{One example of this
  might be the Italian translation of Hölderlin by Luigi Reitani
  \citep{holderlin_tutte_2001}.} I believe that creating a better framework and methodology for resolving these
matters, in a way that can transcend national borders in an increasingly
globalized world, is an urgent issue for us. Furthermore, I think that tackling this issue may be
an opportunity to rejuvenate literary research, whose decline is currently
regarded as a serious problem. Of course, no scholar can solve
this task by herself. By raising this issue, I hope to invite the whole
community of textual scholars to put their collective minds to the
task.\footnote{In Japan, a team of 11 researchers consisting of experts
  on Plato, Dante, Chaucer, Shakespeare, Pascal, Hölderlin, Musil,
  Beckett and others has already been formed in response to this call.
  Our project ``The Third Generation of Editing: Textual Scholarship
  Research with the Aim to Regenerate the Classics and to Revitalize
  Literary Studies'' (PI: K. Myojo, FY: 2022--2027) is supported by MEXT
  / JSPS, Grant-in-Aid for Scientific Research (A) (grant number JP
  22H00008).}

\section*{Acknowledgements} This study is a part of the research project
``The Practices of Textual Scholarship: Towards Application for Cultural
Preservation and Literary Education'' (PI: K. Myojo, FY: 2016--2022),
which is supported by MEXT / JSPS, Grant-in-Aid for Scientific Research
(A) (grant number JP16H01921). This article is a revised version of a paper given at the 17th Annual Conference of the European Society for Textual Scholarship, ``Histories of the Holograph from Ancient to Modern Manuscripts and Beyond'', University of Oxford, U.K., March 2022.


\begin{flushleft}
    % use smallcaps for author names
    \renewcommand*{\mkbibnamefamily}[1]{\textsc{#1}}
    \renewcommand*{\mkbibnamegiven}[1]{\textsc{#1}} 
\printbibliography
\end{flushleft}

\end{paper}